% Generated by Sphinx.
\def\sphinxdocclass{report}
\newif\ifsphinxKeepOldNames \sphinxKeepOldNamestrue
\documentclass[letterpaper,10pt,brazil]{sphinxmanual}
\usepackage{iftex}

\ifPDFTeX
  \usepackage[utf8]{inputenc}
\fi
\ifdefined\DeclareUnicodeCharacter
  \DeclareUnicodeCharacter{00A0}{\nobreakspace}
\fi
\usepackage{cmap}
\usepackage[T1]{fontenc}
\usepackage{amsmath,amssymb,amstext}
\usepackage{babel}
\usepackage{times}
\usepackage[Sonny]{fncychap}
\usepackage{longtable}
\usepackage{sphinx}
\usepackage{multirow}
\usepackage{eqparbox}


\addto\captionsbrazil{\renewcommand{\figurename}{Fig.\@ }}
\addto\captionsbrazil{\renewcommand{\tablename}{Tabela }}
\SetupFloatingEnvironment{literal-block}{name=Listagem }

\addto\extrasbrazil{\def\pageautorefname{página}}




\title{Documentação - MAME}
\date{out 28, 2018}
\release{0.194}
\author{Equipe de Desenvolvimento do MAME, MAMEdev \\Tradução e Revisão Wellington T. Uemura \\Português do Brasil}
\newcommand{\sphinxlogo}{}
\renewcommand{\releasename}{Release}
\makeindex

\makeatletter
\def\PYG@reset{\let\PYG@it=\relax \let\PYG@bf=\relax%
    \let\PYG@ul=\relax \let\PYG@tc=\relax%
    \let\PYG@bc=\relax \let\PYG@ff=\relax}
\def\PYG@tok#1{\csname PYG@tok@#1\endcsname}
\def\PYG@toks#1+{\ifx\relax#1\empty\else%
    \PYG@tok{#1}\expandafter\PYG@toks\fi}
\def\PYG@do#1{\PYG@bc{\PYG@tc{\PYG@ul{%
    \PYG@it{\PYG@bf{\PYG@ff{#1}}}}}}}
\def\PYG#1#2{\PYG@reset\PYG@toks#1+\relax+\PYG@do{#2}}

\expandafter\def\csname PYG@tok@gd\endcsname{\def\PYG@tc##1{\textcolor[rgb]{0.63,0.00,0.00}{##1}}}
\expandafter\def\csname PYG@tok@gu\endcsname{\let\PYG@bf=\textbf\def\PYG@tc##1{\textcolor[rgb]{0.50,0.00,0.50}{##1}}}
\expandafter\def\csname PYG@tok@gt\endcsname{\def\PYG@tc##1{\textcolor[rgb]{0.00,0.27,0.87}{##1}}}
\expandafter\def\csname PYG@tok@gs\endcsname{\let\PYG@bf=\textbf}
\expandafter\def\csname PYG@tok@gr\endcsname{\def\PYG@tc##1{\textcolor[rgb]{1.00,0.00,0.00}{##1}}}
\expandafter\def\csname PYG@tok@cm\endcsname{\let\PYG@it=\textit\def\PYG@tc##1{\textcolor[rgb]{0.25,0.50,0.56}{##1}}}
\expandafter\def\csname PYG@tok@vg\endcsname{\def\PYG@tc##1{\textcolor[rgb]{0.73,0.38,0.84}{##1}}}
\expandafter\def\csname PYG@tok@vi\endcsname{\def\PYG@tc##1{\textcolor[rgb]{0.73,0.38,0.84}{##1}}}
\expandafter\def\csname PYG@tok@vm\endcsname{\def\PYG@tc##1{\textcolor[rgb]{0.73,0.38,0.84}{##1}}}
\expandafter\def\csname PYG@tok@mh\endcsname{\def\PYG@tc##1{\textcolor[rgb]{0.13,0.50,0.31}{##1}}}
\expandafter\def\csname PYG@tok@cs\endcsname{\def\PYG@tc##1{\textcolor[rgb]{0.25,0.50,0.56}{##1}}\def\PYG@bc##1{\setlength{\fboxsep}{0pt}\colorbox[rgb]{1.00,0.94,0.94}{\strut ##1}}}
\expandafter\def\csname PYG@tok@ge\endcsname{\let\PYG@it=\textit}
\expandafter\def\csname PYG@tok@vc\endcsname{\def\PYG@tc##1{\textcolor[rgb]{0.73,0.38,0.84}{##1}}}
\expandafter\def\csname PYG@tok@il\endcsname{\def\PYG@tc##1{\textcolor[rgb]{0.13,0.50,0.31}{##1}}}
\expandafter\def\csname PYG@tok@go\endcsname{\def\PYG@tc##1{\textcolor[rgb]{0.20,0.20,0.20}{##1}}}
\expandafter\def\csname PYG@tok@cp\endcsname{\def\PYG@tc##1{\textcolor[rgb]{0.00,0.44,0.13}{##1}}}
\expandafter\def\csname PYG@tok@gi\endcsname{\def\PYG@tc##1{\textcolor[rgb]{0.00,0.63,0.00}{##1}}}
\expandafter\def\csname PYG@tok@gh\endcsname{\let\PYG@bf=\textbf\def\PYG@tc##1{\textcolor[rgb]{0.00,0.00,0.50}{##1}}}
\expandafter\def\csname PYG@tok@ni\endcsname{\let\PYG@bf=\textbf\def\PYG@tc##1{\textcolor[rgb]{0.84,0.33,0.22}{##1}}}
\expandafter\def\csname PYG@tok@nl\endcsname{\let\PYG@bf=\textbf\def\PYG@tc##1{\textcolor[rgb]{0.00,0.13,0.44}{##1}}}
\expandafter\def\csname PYG@tok@nn\endcsname{\let\PYG@bf=\textbf\def\PYG@tc##1{\textcolor[rgb]{0.05,0.52,0.71}{##1}}}
\expandafter\def\csname PYG@tok@no\endcsname{\def\PYG@tc##1{\textcolor[rgb]{0.38,0.68,0.84}{##1}}}
\expandafter\def\csname PYG@tok@na\endcsname{\def\PYG@tc##1{\textcolor[rgb]{0.25,0.44,0.63}{##1}}}
\expandafter\def\csname PYG@tok@nb\endcsname{\def\PYG@tc##1{\textcolor[rgb]{0.00,0.44,0.13}{##1}}}
\expandafter\def\csname PYG@tok@nc\endcsname{\let\PYG@bf=\textbf\def\PYG@tc##1{\textcolor[rgb]{0.05,0.52,0.71}{##1}}}
\expandafter\def\csname PYG@tok@nd\endcsname{\let\PYG@bf=\textbf\def\PYG@tc##1{\textcolor[rgb]{0.33,0.33,0.33}{##1}}}
\expandafter\def\csname PYG@tok@ne\endcsname{\def\PYG@tc##1{\textcolor[rgb]{0.00,0.44,0.13}{##1}}}
\expandafter\def\csname PYG@tok@nf\endcsname{\def\PYG@tc##1{\textcolor[rgb]{0.02,0.16,0.49}{##1}}}
\expandafter\def\csname PYG@tok@si\endcsname{\let\PYG@it=\textit\def\PYG@tc##1{\textcolor[rgb]{0.44,0.63,0.82}{##1}}}
\expandafter\def\csname PYG@tok@s2\endcsname{\def\PYG@tc##1{\textcolor[rgb]{0.25,0.44,0.63}{##1}}}
\expandafter\def\csname PYG@tok@nt\endcsname{\let\PYG@bf=\textbf\def\PYG@tc##1{\textcolor[rgb]{0.02,0.16,0.45}{##1}}}
\expandafter\def\csname PYG@tok@nv\endcsname{\def\PYG@tc##1{\textcolor[rgb]{0.73,0.38,0.84}{##1}}}
\expandafter\def\csname PYG@tok@s1\endcsname{\def\PYG@tc##1{\textcolor[rgb]{0.25,0.44,0.63}{##1}}}
\expandafter\def\csname PYG@tok@dl\endcsname{\def\PYG@tc##1{\textcolor[rgb]{0.25,0.44,0.63}{##1}}}
\expandafter\def\csname PYG@tok@ch\endcsname{\let\PYG@it=\textit\def\PYG@tc##1{\textcolor[rgb]{0.25,0.50,0.56}{##1}}}
\expandafter\def\csname PYG@tok@m\endcsname{\def\PYG@tc##1{\textcolor[rgb]{0.13,0.50,0.31}{##1}}}
\expandafter\def\csname PYG@tok@gp\endcsname{\let\PYG@bf=\textbf\def\PYG@tc##1{\textcolor[rgb]{0.78,0.36,0.04}{##1}}}
\expandafter\def\csname PYG@tok@sh\endcsname{\def\PYG@tc##1{\textcolor[rgb]{0.25,0.44,0.63}{##1}}}
\expandafter\def\csname PYG@tok@ow\endcsname{\let\PYG@bf=\textbf\def\PYG@tc##1{\textcolor[rgb]{0.00,0.44,0.13}{##1}}}
\expandafter\def\csname PYG@tok@sx\endcsname{\def\PYG@tc##1{\textcolor[rgb]{0.78,0.36,0.04}{##1}}}
\expandafter\def\csname PYG@tok@bp\endcsname{\def\PYG@tc##1{\textcolor[rgb]{0.00,0.44,0.13}{##1}}}
\expandafter\def\csname PYG@tok@c1\endcsname{\let\PYG@it=\textit\def\PYG@tc##1{\textcolor[rgb]{0.25,0.50,0.56}{##1}}}
\expandafter\def\csname PYG@tok@fm\endcsname{\def\PYG@tc##1{\textcolor[rgb]{0.02,0.16,0.49}{##1}}}
\expandafter\def\csname PYG@tok@o\endcsname{\def\PYG@tc##1{\textcolor[rgb]{0.40,0.40,0.40}{##1}}}
\expandafter\def\csname PYG@tok@kc\endcsname{\let\PYG@bf=\textbf\def\PYG@tc##1{\textcolor[rgb]{0.00,0.44,0.13}{##1}}}
\expandafter\def\csname PYG@tok@c\endcsname{\let\PYG@it=\textit\def\PYG@tc##1{\textcolor[rgb]{0.25,0.50,0.56}{##1}}}
\expandafter\def\csname PYG@tok@mf\endcsname{\def\PYG@tc##1{\textcolor[rgb]{0.13,0.50,0.31}{##1}}}
\expandafter\def\csname PYG@tok@err\endcsname{\def\PYG@bc##1{\setlength{\fboxsep}{0pt}\fcolorbox[rgb]{1.00,0.00,0.00}{1,1,1}{\strut ##1}}}
\expandafter\def\csname PYG@tok@mb\endcsname{\def\PYG@tc##1{\textcolor[rgb]{0.13,0.50,0.31}{##1}}}
\expandafter\def\csname PYG@tok@ss\endcsname{\def\PYG@tc##1{\textcolor[rgb]{0.32,0.47,0.09}{##1}}}
\expandafter\def\csname PYG@tok@sr\endcsname{\def\PYG@tc##1{\textcolor[rgb]{0.14,0.33,0.53}{##1}}}
\expandafter\def\csname PYG@tok@mo\endcsname{\def\PYG@tc##1{\textcolor[rgb]{0.13,0.50,0.31}{##1}}}
\expandafter\def\csname PYG@tok@kd\endcsname{\let\PYG@bf=\textbf\def\PYG@tc##1{\textcolor[rgb]{0.00,0.44,0.13}{##1}}}
\expandafter\def\csname PYG@tok@mi\endcsname{\def\PYG@tc##1{\textcolor[rgb]{0.13,0.50,0.31}{##1}}}
\expandafter\def\csname PYG@tok@kn\endcsname{\let\PYG@bf=\textbf\def\PYG@tc##1{\textcolor[rgb]{0.00,0.44,0.13}{##1}}}
\expandafter\def\csname PYG@tok@cpf\endcsname{\let\PYG@it=\textit\def\PYG@tc##1{\textcolor[rgb]{0.25,0.50,0.56}{##1}}}
\expandafter\def\csname PYG@tok@kr\endcsname{\let\PYG@bf=\textbf\def\PYG@tc##1{\textcolor[rgb]{0.00,0.44,0.13}{##1}}}
\expandafter\def\csname PYG@tok@s\endcsname{\def\PYG@tc##1{\textcolor[rgb]{0.25,0.44,0.63}{##1}}}
\expandafter\def\csname PYG@tok@kp\endcsname{\def\PYG@tc##1{\textcolor[rgb]{0.00,0.44,0.13}{##1}}}
\expandafter\def\csname PYG@tok@w\endcsname{\def\PYG@tc##1{\textcolor[rgb]{0.73,0.73,0.73}{##1}}}
\expandafter\def\csname PYG@tok@kt\endcsname{\def\PYG@tc##1{\textcolor[rgb]{0.56,0.13,0.00}{##1}}}
\expandafter\def\csname PYG@tok@sc\endcsname{\def\PYG@tc##1{\textcolor[rgb]{0.25,0.44,0.63}{##1}}}
\expandafter\def\csname PYG@tok@sb\endcsname{\def\PYG@tc##1{\textcolor[rgb]{0.25,0.44,0.63}{##1}}}
\expandafter\def\csname PYG@tok@sa\endcsname{\def\PYG@tc##1{\textcolor[rgb]{0.25,0.44,0.63}{##1}}}
\expandafter\def\csname PYG@tok@k\endcsname{\let\PYG@bf=\textbf\def\PYG@tc##1{\textcolor[rgb]{0.00,0.44,0.13}{##1}}}
\expandafter\def\csname PYG@tok@se\endcsname{\let\PYG@bf=\textbf\def\PYG@tc##1{\textcolor[rgb]{0.25,0.44,0.63}{##1}}}
\expandafter\def\csname PYG@tok@sd\endcsname{\let\PYG@it=\textit\def\PYG@tc##1{\textcolor[rgb]{0.25,0.44,0.63}{##1}}}

\def\PYGZbs{\char`\\}
\def\PYGZus{\char`\_}
\def\PYGZob{\char`\{}
\def\PYGZcb{\char`\}}
\def\PYGZca{\char`\^}
\def\PYGZam{\char`\&}
\def\PYGZlt{\char`\<}
\def\PYGZgt{\char`\>}
\def\PYGZsh{\char`\#}
\def\PYGZpc{\char`\%}
\def\PYGZdl{\char`\$}
\def\PYGZhy{\char`\-}
\def\PYGZsq{\char`\'}
\def\PYGZdq{\char`\"}
\def\PYGZti{\char`\~}
% for compatibility with earlier versions
\def\PYGZat{@}
\def\PYGZlb{[}
\def\PYGZrb{]}
\makeatother

\renewcommand\PYGZsq{\textquotesingle}

\begin{document}

\maketitle
\tableofcontents
\phantomsection\label{index::doc}


\begin{notice}{note}{Nota:}
Essa publicação é o resultado de um trabalho em contante evolução.
Você pode acompanhar se há alguma atualização dos tópicos e assuntos
aqui tratados acessando o
\href{https://github.com/mamedev/mame/issues}{issue tracker}.
Veja como você pode contribuir na seção \href{https://github.com/mamedev/mame/blob/master/docs/CONTRIBUTING.md}{contribute} no
site do GitHub.
\end{notice}


\chapter{O QUE É O MAME?}
\label{whatis:o-que-e-o-mame}\label{whatis:documentacao-mame}\label{whatis::doc}
O MAME é uma estrutura multiúso de trabalho voltado para a emulação.

O objetivo do MAME é preservar décadas de história de software,
impedindo que com a evolução da tecnologia, este importante software
``vintage'' seja esquecido e se perca ao longo tempo. Isso se torna
possível usando o próprio código fonte do MAME para documentar o
funcionamento do hardware. O fato do software funcionar, serve como
validação primária de como a documentação é precisa (de que outra forma
você poderia provar que recriou fielmente o hardware?).
O MAME com o tempo (originalmente significava \emph{Multiple Arcade Machine
Emulator} ou em tradução livre, \emph{Multi Emulador de Máquinas Arcade})
absorveu o projeto irmão MESS (Multi Emulator Super System).
O MAME agora documenta também uma grande variedade de computadores
(principalmente aqueles bem mais antigos), consoles de videogame,
calculadoras, indo muito além do seu foco inicial.

\begin{DUlineblock}{0em}
\item[] 
\item[] \textbf{MAME®}
\item[] \textbf{Copyright © 1997-2018 by Nicola Salmoria and the MAME team}
\item[] \textbf{MAME é uma marca registrada e propriedade de Gregory Ember}
\item[] 
\end{DUlineblock}


\section{I. Objetivo}
\label{whatis:i-objetivo}
O objetivo principal do MAME é ser uma referência ao funcionamento
interno das máquinas emuladas. Tanto para fins educacionais como para
fins de preservação histórica impedindo que o software desapareça para
sempre quando o hardware original em que ele roda parar de funcionar.
Uma vez preservando o software e demonstrando que seu comportamento
emulado corresponde ao original, ele também deve ser capaz de utilizar o
hardware em si. Apesar de ser considerado um efeito colateral muito bem
vindo, este não é o foco principal do MAME.

Não é nossa intenção infringir quaisquer direitos autorais assim como as
patentes dos jogos originais. Todo o código-fonte do MAME é de criação
própria e disponível gratuitamente. O emulador requer imagens ROMs
originais, CD, disco rígido e outras imagens de mídia usada pela máquinas
para operar e que devem ser providenciadas pelo usuário. Nenhuma parte
do código fonte do jogo original está inclusa no executável.
\clearpage

\section{II. Custo}
\label{whatis:ii-custo}
O MAME é gratuito.

Seu código-fonte é gratuito. O projeto como um todo é distribuído
através da Licença Pública Geral GNU, versão 2 ou mais recente
(GPL-2.0+), mas a maior parte do código (incluindo a funcionalidade
principal) também estão disponíveis através da clausula Nº 3 da licença
BSD (3-Clause BSD License).


\section{III. Software em formato de imagem}
\label{whatis:iii-software-em-formato-de-imagem}
Os formatos de imagem de mídias como ROM, CD, disco rígido dentre outros
formatos, são materiais protegidos por direitos autorais.
Eles não podem ser distribuídos sem a permissão \textbf{explícita} dos
seu responsáveis que detém a propriedade intelectual destes direitos,
tão pouco são ``abandonware'' \footnote[1]{\sphinxAtStartFootnote%
Abandonware é um software que foi descontinuado e não é mais
desenvolvido, mantido, comercializado seja porque ficou
obsoleto, a empresa não existe mais ou qualquer outro motivo que
ao olhar do usuário o software foi abandonado, daí o termo
\emph{abandonware} ou \textbf{abandoned software}. (Nota do Tradutor)
} e qualquer dos software compatíveis com o
MAME jamais perdem os seus direitos autorais.

O MAME não se destina a ser usado como uma ferramenta de pirataria ou de
violação de direitos autorais em massa. Portanto é veementemente contra,
assim como o desejo do autor de que a sua propriedade autoral seja
comercializada anunciada ou vinculada a qualquer tipo de recursos que
forneçam cópias ilegais de ROM, CD, disco rígido ou outras imagens de
mídia.


\section{IV. Obras derivativas}
\label{whatis:iv-obras-derivativas}
Caso você queira usar o nome MAME como parte do seu trabalho ou derivado
dele, há regras a serem seguidas, pois MAME é uma marca registrada.
Em geral, isso significa que você deve pedir permissão assim como requer
que você siga as diretrizes acima.

Qualquer trabalho derivativo do MAME deve refletir o número da versão
usada.


\section{V. Mais informações e contato legal}
\label{whatis:v-mais-informacoes-e-contato-legal}
Para questões relacionadas à licença do MAME, marca comercial ou
qualquer outra utilização, acesse:

\url{https://www.mamedev.org/}


\chapter{CONFIGURANDO O MAME PARA SER USADO PELA PRIMEIRA VEZ}
\label{initialsetup/index:configurando-o-mame-para-ser-usado-pela-primeira-vez}\label{initialsetup/index::doc}
Essa seção lida com os primeiros passos a serem dados por aqueles que
vão usar o MAME pela primeira vez, incluindo o download assim como
a compilação da sua versão customizada do MAME.


\section{Uma introdução ao MAME}
\label{initialsetup/mameintro:uma-introducao-ao-mame}\label{initialsetup/mameintro::doc}
O MAME antigamente era conhecido pelo seu acrônimo \emph{Multi Arcade Machine
Emulator} ou \emph{Emulador Múltiplo de Máquinas Arcade} numa tradução livre,
que foi desenvolvido para documentar e reproduzir, através de emulação,
a mesma funcionalidade dos componentes internos das máquinas arcade,
computadores, consoles de videogame, calculadoras e outros tipos de
máquinas eletrônicas voltada ao entretenimento. Programas e jogos estes
que foram originalmente desenvolvidos para rodarem apenas nestes
sistemas, agora o MAME permite que você os rode novamente através de
emulação usando um PC moderno.

Em um determinado momento, haviam dois projetos separados, o MAME e o
MESS. O MAME lidava apenas máquinas arcade enquanto o MESS lidava com
todo o resto. Ambos agora trabalham em conjunto dentro do MAME.

A maior parte do MAME é programado em C++, alguns componentes principais
em C e outras linguagens auxiliares. Atualmente o MAME consegue emular
mais de 3200 sistemas independentes das últimas 5 décadas.


\section{O objetivo do MAME}
\label{initialsetup/mameintro:o-objetivo-do-mame}
O principal objetivo do MAME é a preservação de décadas de história dos
arcades, computadores e consoles. À medida que a tecnologia continua
avançando, o MAME impede que esses importantes sistemas “vintage” se
percam e sejam esquecidos.


\section{Os sistemas emulados pelo MAME}
\label{initialsetup/mameintro:os-sistemas-emulados-pelo-mame}
O ProjectMESS contém uma lista completa dos sistemas atualmente
emulados. Você irá notar que ter um sistema emulado, não significa que
a emulação dele está perfeita. Você pode querer:
\begin{enumerate}
\item {} 
verificar o status da emulação nas páginas wiki de cada sistema,
acessível a partir da página de drivers (por exemplo, para o Apple
Macintosh, olhe o arquivo de driver mac.cpp, você pode acessar também
as páginas do \textbf{macplus} e \textbf{macse}),

\item {} 
assim como ler também os registros correspondentes no arquivo
\textbf{sysinfo.dat} para entender melhor quais problemas você pode
encontrar durante a execução de um sistema no MAME. (para o Apple
Macintosh Plus, também é necessário verificar esse arquivo).

\end{enumerate}

Como alternativa, você também pode ver essa condição por conta própria,
caso haja, prestando atenção na tela de aviso vermelha ou bege que
aparece antes do inicio da emulação. Observe que, se você tiver
informações que podem ajudar a melhorar a emulação de um sistema emulado
ou se você puder contribuir com correções e/ou novas adições ao código
fonte atual, siga as instruções na página de contato ou poste uma
mensagem no Fórum do MAME em \url{https://forum.mamedev.org/}


\section{Os sistemas operacionais compatíveis}
\label{initialsetup/mameintro:os-sistemas-operacionais-compativeis}
O código fonte atual pode ser compilada diretamente nos principais
sistemas operacionais: Microsoft Windows (ambos com suporte nativo para
DirectX/BGFX ou com suporte SDL), Linux, FreeBSD e Max OS X. Além disso
há suporte para ambas as versões de 32 e 64 bits, saiba que a versão
64 bits mostra um improviso significativo na performance se comparado
com a versão de 32 bits.


\section{Requisitos do sistema}
\label{initialsetup/mameintro:requisitos-do-sistema}
O desenvolvimento do MAME gira em torno das linguagens C/C++ e já foi
portado para diferentes plataformas. Com o passar do tempo, à medida que
o hardware do computador vai evoluindo, o código do MAME evolui junto
para aproveitar melhor o maior poder de processamento e os novos
recursos de hardware.

Os binários oficiais do MAME são compilados e projetados para rodar em
qualquer sistema Windows. Os requisitos mínimos são:
\begin{itemize}
\item {} 
Processador Intel Core ou equivalente com pelo menos 2.0 GHz

\item {} 
Sistema Operacional de 32-bit (Windos Vista SP1 ou mais recente, Mac
10.9 ou mais recente)

\item {} 
4 GB de RAM

\item {} 
DirectX 9.0c para Windows

\item {} 
Uma placa gráfica compatível com Direct3D ou OpenGL

\item {} 
Qualquer placa de som compatível com DirectSound

\end{itemize}

Claro, os requisitos mínimos são apenas um pequeno exemplo. Você pode
não obter o melhor desempenho possível usando a configuração acima, mas
o MAME deverá rodar sem maiores problemas. As versões mais recentes do
MAME tendem a exigir mais recursos de hardware do que as suas versões
anteriores, assim, versões mais antigas do MAME poderão ter uma
performance melhor caso você tenha um PC mais fraco, porém ao custo de
perder as melhorias feitas nos sistemas existentes, dos novos sistemas
que foram adicionados assim como as correções posteriores à versão do
MAME que você estiver usando.

O MAME tira vantagem dos recursos de hardware 3D para a exibição das
ilustrações assim como o redimensionamento do software ou jogo para tela
inteira. Para fazer uso destes benefícios, você deve ter uma placa de
vídeo mais recente capaz de lidar com Direct3D 8 e com pelo menos 16 MB
de memória RAM.

Os filtros especiais HLSL ou GLSL assim como o efeito de simulação de
tela de tubo CRT, causam uma sobrecarga extra na emulação, especialmente
em resoluções mais altas. Assim você precisará de uma placa de vídeo
moderna o bastante para aguentar o tranco, poderosa, pois a carga de
processamento sobe exponencialmente à medida que se aumenta também a
resolução. Se HLSL ou GLSL ficar muito pesado, tente reduzir o tamanho
da resolução de vídeo da emulação.

Tenha sempre em mente que, mesmo usando os computadores mais rápidos
disponíveis hoje, o MAME ainda é incapaz de rodar alguns sistemas em
sua velocidade nativa. O principal objetivo do projeto não é fazer com
que todos os sistemas emulados rodem na sua velocidade nativa, seja no
seu computador ou seja lá onde você estiver rodando o MAME; o principal
objetivo é documentar o hardware e reproduzir o seu comportamento
original tão fielmente quanto for possível.


\subsection{Extrações de BIOS e programas}
\label{initialsetup/mameintro:extracoes-de-bios-e-programas}
Para que o MAME consiga emular a maioria destes sistemas, o conteúdo dos
circuitos integrados originais destes aparelhos precisam ser extraídos.
Isso pode ser feito extraindo esses dados do aparelho original você
mesmo, ou procurando por eles na internet por sua conta e risco.

O MAME não fornece, disponibiliza ou vem acompanhado de nenhum deles
justamente pelo fato desses conteúdos estarem protegidos por leis de
direitos autorais. Caso tenha interesse em encontrar algum software que
rode em uma das máquinas já emuladas, lembre-se, o Google e outros sites
de pesquisa são os seus melhores amigos nessas horas.


\section{Instalando o MAME}
\label{initialsetup/installingmame:instalando-o-mame}\label{initialsetup/installingmame::doc}

\subsection{Microsoft Windows}
\label{initialsetup/installingmame:microsoft-windows}
Baixe a versão mais recente disponível em
\href{https://www.mamedev.org/}{www.mamedev.org} e extraia o seu conteúdo.
Dentro da pasta onde ele foi extraído haverá vários arquivos e pastas
(abaixo nós mostraremos para que servem alguns deles), dentre eles o
aquivo mais importante de todos que é o \textbf{MAME.EXE}, o emulador em si.
Ele é um programa que funciona na linha de comando.

O processo de instalação termina aqui, fácil não?


\subsection{Outros sistemas operacionais}
\label{initialsetup/installingmame:outros-sistemas-operacionais}
Neste caso você pode procurar por uma versão pré compilada do executável
do (SDL) MAME que pode ser encontrado em alguns repositórios da
distribuição Linux da sua preferência. Ou então, compile o MAME você
mesmo, baixando e descompactando o arquivo do código fonte em alguma
pasta de sua preferência.

Caso você escolha a opção de compilar você mesmo o MAME, consulte a
seção {\hyperref[initialsetup/compilingmame:compiling\string-mame]{\sphinxcrossref{\DUrole{std,std-ref}{Compilando o MAME}}}} para mais detalhes.


\section{Compilando o MAME}
\label{initialsetup/compilingmame:compilando-o-mame}\label{initialsetup/compilingmame::doc}\label{initialsetup/compilingmame:compiling-mame}

\subsection{Para todas as plataformas}
\label{initialsetup/compilingmame:para-todas-as-plataformas}\begin{itemize}
\item {} 
Sempre que você estiver alterando os parâmetros de construção, (como
alternar entre uma versão baseada em SDL e uma versão nativa do
Windows ou adicionar ferramentas à lista de compilação) você precisa
executar um \textbf{make REGENIE=1} para fazer com que todas as novas
opções adicionais sejam incluídas nos arquivos de configuração
responsável pela construção do executável do MAME. Caso não seja
feito, será muito complicado identificar e localizar possíveis
erros.

\item {} 
Caso você queira incluir as ferramentas adicionais na sua compilação
como por exemplo, o programa \emph{CHDMAN}, adicione a opção \textbf{TOOLS=1}
ao comando make, assim \textbf{make REGENIE=1 TOOLS=1}. Isso fará com que
o make compile todas as outras ferramentas que acompanham o MAME
além do \emph{CHDMAN}.

\item {} 
Você pode customizar a sua compilação escolhendo um driver em
específico caso queira, usando a opção \emph{SOURCES=\textless{}driver\textgreater{}} junto com
o comando make. Por exemplo, caso queira compilar uma versão
customizada do MAME que só rode o jogo \textbf{Pac Man}, você faz assim
\textbf{make SOURCES=src/mame/drivers/pacman.cpp REGENIE=1}, note que é
muito importante não se esquecer da opção obrigatória \textbf{REGENIE}
para que o make recrie todas as configurações necessárias durante a
compilação desta versão customizada do MAME.

\item {} 
É possível usar os núcleos extras do seu processador para ajudar a
reduzir o tempo de compilação. Isso é feito adicionando o parâmetro
\textbf{-j} ao comando make. Observe que a quantidade máxima de núcleos
que você pode usar fica limitado a quantidade de núcleos que o seu
processador tiver mais um. Usando valores acima do limite do seu
processador não faz com que a compilação fique mais rápida, além
disso, a sobrecarga extra de processamento pode fazer com que seu
processador superaqueça, seu sistema pode ficar mais lento e pare de
responder, etc. Logo, a configuração ideal para se obter a melhor
velocidade possível de compilação num processador Quad Core seria
\textbf{make -j5}, por exemplo.

\item {} 
As instruções de depuração também podem ser adicionadas na
compilação usando a opção \emph{SYMBOLS=1}, embora seja totalmente
desnecessária para a grande maioria das pessoas.

\end{itemize}

Aqui alguns exemplos somando tudo o que foi mostrado até agora para
reconstruir o MAME com apenas o driver do jogo \textbf{Pac Man}, com as
ferramentas extras em um computador com processador Quad Core (i5 ou i7
por exemplo):
\begin{quote}

\sphinxcode{make SOURCES=src/mame/drivers/pacman.cpp TOOLS=1 REGENIE=1 -j5}
\end{quote}

Reconstruindo o MAME em um notebook com processadores Dual Core (i3 ou i5 por exemplo):
\begin{quote}

\sphinxcode{make -j3}
\end{quote}


\subsection{Microsoft Windows}
\label{initialsetup/compilingmame:microsoft-windows}
Aqui algumas notas voltadas especificamente para a compilação do MAME no
Windows.
\begin{itemize}
\item {} 
Consulte \href{https://mamedev.org/tools/}{o site do MAME} para obter o
kit completo de ferramentas mais recente para compilar o sua versão do
MAME no Windows.

\item {} 
Você precisará baixar o conjunto de ferramentas desse link para
começar. Esse kit de ferramentas são atualizados periodicamente,
assim é \textbf{obrigatório} usar a versão mais recente deste kit para
que seja possível compilar as novas versões do MAME.

\item {} 
Também é possível compilar o MAME usando o \emph{Visual Studio 2017}
(caso esteja instalado no seu PC) ao usar \textbf{make vs2017}. Esse
comando \emph{sempre} regenera todas as configurações de compilação logo,
a opção \textbf{REGENIE=1} \emph{não é necessário}.

\item {} 
As versões anteriores ao SDL \emph{2 2.0.3} ou \emph{2.0.4} tem problemas,
certifique-se que você tenha a versão mais recente.

\end{itemize}


\subsection{Fedora Linux}
\label{initialsetup/compilingmame:fedora-linux}
Alguns pré-requisitos precisam ser atendidos na sua distro antes de
continuar. As versões anteriores ao SDL \emph{2 2.0.3} ou \emph{2.0.4} tem
problemas, certifique-se que você tenha a versão mais recente.
\begin{quote}

\sphinxcode{sudo dnf install gcc gcc-c++ SDL2-devel SDL2\_ttf-devel
libXinerama-devel qt5-qtbase-devel qt5-qttools expat-devel
fontconfig-devel alsa-lib-devel}
\end{quote}

A compilação é exatamente como descrito acima para todas as Plataformas.


\subsection{Debian e Ubuntu (incluindo dispositivos Raspberry Pi e ODROID)}
\label{initialsetup/compilingmame:compiling-mame-debian}\label{initialsetup/compilingmame:debian-e-ubuntu-incluindo-dispositivos-raspberry-pi-e-odroid}
Alguns pré-requisitos precisam ser atendidos na sua distro antes de
continuar. As versões anteriores ao SDL \emph{2 2.0.3} ou \emph{2.0.4} tem
problemas, certifique-se que você tenha a versão mais recente.
\begin{quote}

\sphinxcode{sudo apt-get install git build-essential libsdl2-dev
libsdl2-ttf-dev libfontconfig-dev qt5-default}
\end{quote}

A compilação é exatamente como descrito acima para todas as Plataformas.


\subsection{Arch Linux}
\label{initialsetup/compilingmame:arch-linux}
Alguns pré-requisitos precisam ser atendidos na sua distro antes de
continuar.
\begin{quote}

\sphinxcode{sudo pacman -S base-devel git sdl2 gconf sdl2\_ttf gcc qt5}
\end{quote}

A compilação é exatamente como descrito acima para Todas as Plataformas.


\subsection{Apple Mac OS X}
\label{initialsetup/compilingmame:apple-mac-os-x}
Você precisará de alguns pré-requisitos para começar. Certifique-se de
estar no \emph{OS X 10.9 Mavericks} ou mais recente.
É \textbf{OBRIGATÓRIO} o uso do SDL 2.0.4 para o OS X.
\begin{itemize}
\item {} 
Instale o \textbf{Xcode} que você encontra no Mac App Store

\item {} 
Inicie o programa \textbf{Xcode}.

\item {} 
Será feito o download de alguns pré-requisitos adicionais.
Deixe rodando antes de continuar.

\item {} 
Ao terminar saia do \textbf{Xcode} e abra uma janela do \textbf{Terminal}

\item {} 
Digite o comando \sphinxcode{xcode-select -{-}install} para instalar o kit
obrigatório de ferramentas para o MAME.

\end{itemize}

Em seguida, é preciso baixar e instalar o SDL 2.
\begin{itemize}
\item {} 
Vá para \href{http://libsdl.org/download-2.0.php}{este site} e baixe o
arquivo .dmg para o \emph{Mac OS X}.

\item {} 
Caso o arquivo .dmg não abra sozinho de forma automática, abra você
mesmo

\item {} 
Clique no `Macintosh HD' (ou seja lá o nome que você estiver usando
no disco rígido do seu Mac), no painel esquerdo onde está localizado
o \textbf{Finder}, abra a pasta \textbf{Biblioteca} e arraste o arquivo
\textbf{SDL2.framework} na pasta \textbf{Frameworks}.

\end{itemize}

Por fim, para começar a compilar, use o Terminal para navegar até onde
você tem o código fonte do MAME (comando \emph{cd}) e siga as instruções
normais de compilação acima para todas as Plataformas.

É possível fazer o MAME funcionar a partir da versão 10.6, porém é um
pouco mais complicado:
\begin{itemize}
\item {} 
Você precisará instalar o \textbf{clang-3.7}, \textbf{ld64}, \textbf{libcxx} e o
\textbf{python27} do MacPorts.

\item {} 
Em seguida, adicione essas opções ao seu comando make ou
useroptions.mak:

\end{itemize}

\begin{DUlineblock}{0em}
\item[] \sphinxcode{OVERRIDE\_CC=/opt/local/bin/clang-mp-3.7}
\item[] \sphinxcode{OVERRIDE\_CXX=/opt/local/bin/clang++-mp-3.7}
\item[] \sphinxcode{PYTHON\_EXECUTABLE=/opt/local/bin/python2.7}
\item[] \sphinxcode{ARCHOPTS=-stdlib=libc++}
\end{DUlineblock}


\subsection{Javascript Emscripten e HTML}
\label{initialsetup/compilingmame:javascript-emscripten-e-html}
Primeiro, baixe e instale o \textbf{Emscripten 1.37.29} ou mais recente
segundo as instruções no \href{https://kripken.github.io/emscripten-site/docs/getting\_started/downloads.html}{site oficial}

Depois de instalar o Emscripten, será possível compilar o MAME direto,
usando a ferramenta `\textbf{emmake}`. O MAME completo é muito grande para
ser carregado numa página web de uma só vez, então é preferível que você
compile versões menores e separadas do MAME usando o parâmetro
\emph{SOURCES}, por exemplo, faça o comando abaixo no mesmo diretório do MAME:
\begin{quote}

\sphinxcode{emmake make SUBTARGET=pacmantest SOURCES=src/mame/drivers/pacman.cpp}
\end{quote}

O parâmetro \emph{SOURCES} deve apontar para pelo menos um arquivo de driver
\emph{.cpp}. O comando make tentará localizar e reunir todas as dependências
para compilar o executável do MAME junto com o driver que você
definiu. No entanto porém, caso ocorra algum erro e o processo não
encontre algum arquivo, é necessário declarar manualmente um ou mais
arquivos que faltam (separados por vírgula). Por exemplo:

\begin{DUlineblock}{0em}
\item[] \sphinxcode{emmake make SUBTARGET=apple2e SOURCES=src/mame/drivers/apple2e.cpp,src/mame/machine/applefdc.cpp}
\end{DUlineblock}

O valor do parâmetro \emph{SUBTARGET} serve apenas para se diferenciar dentre
as várias compilações existente e não precisa ser definido caso não seja
necessário.

O Emscripten oferece suporte à compilação do WebAssembly com um loader
de JavaScript em vez do JavaScript inteiro, esse é o padrão em versões
mais recentes. Para ligar ou desligar o WebAssembly de modo forçado,
adicione \textbf{WEBASSEMBLY=1} ou \textbf{WEBASSEMBLY=0} ao comando make.

Outros comandos make também poderão ser usados como foi o
parâmetro \textbf{-j} que foi usado visando fazer uso da compilação
multitarefa.

Quando a compilação atinge a fase da emcc, talvez você veja uma
certa quantidade de mensagens de aviso do tipo \emph{``unresolved symbol''}.
Até o presente momento, isso é esperado para funções relacionadas com o
OpenGL como a função ``\emph{glPointSize}''. Outros podem também indicar que um
arquivo de dependência adicional precisa ser especificado na lista
\emph{SOURCES}. Infelizmente, este processo não é automatizado e você
precisará localizar e informar o arquivo de código fonte assim como os
arquivos que contém os símbolos que estão faltando. Você também pode
ter a sorte de se safar caso ignore os avisos e continue a compilação,
desde que os códigos ausentes não sejam usados no momento da execução.

Se tudo correr bem, um arquivo. js será criado no diretório. Este
arquivo não pode ser executado sozinho, ele precisa de um loader HTML
para que ele possa ser exibido e que seja possível também passar os
parâmetros de linha de comando para o executável.

O \href{https://github.com/db48x/emularity}{Projeto Emularity} oferece tal
loader.

Existem amostras de arquivos .html nesse repositório que pode ser
editado para refletir as suas configurações pessoais e apontar o caminho
do seu arquivo js recém compilado do MAME. Abaixo está a lista dos
arquivos que você precisa colocar num servidor web:
\begin{itemize}
\item {} 
O arquivo .js compilado do MAME

\item {} 
O arquivo .wasm do MAME caso você o tenha compilado com WebAssembly

\item {} 
Os arquivos .js do pacote Emularity (loader.js, browserfs.js, etc)

\item {} 
Um arquivo .zip com as ROMs do driver que você deseja rodar
(caso haja)

\item {} 
Qualquer outro programa que você quiser rodar com o driver do MAME

\item {} 
Um loader do Emularity .html customizado para utilizar todos os
itens acima.

\end{itemize}

Devido a restrição de segurança dos navegadores atuais, você precisa
usar um servidor web ao invés de tentar rodá-los localmente.

Caso algo dê errado e não funcione, você pode abrir o console Web do seu
navegador principal e ver qual o erro que ele mostra (por exemplo,
faltando alguma coisa, algum arquivo de ROM incorreto, etc).
Um erro do tipo ``\textbf{ReferenceError: foo is not defined}'' pode indicar
que provavelmente faltou informar um arquivo de código fonte na lista da
opção \textbf{SOURCES}.


\section{Compilação cruzada}
\label{initialsetup/crosscompilemame::doc}\label{initialsetup/crosscompilemame:compilacao-cruzada}

\subsection{Definição}
\label{initialsetup/crosscompilemame:definicao}
Compilação cruzada \footnote[1]{\sphinxAtStartFootnote%
Cross compiling no Inglês. (Nota do tradutor)
} é o processo de poder compilar um executável
numa plataforma diferente da qual ela se destina. Como usar o ambiente
Linux para compilar um programa que rode no Windows, Mac ou qualquer
outra plataforma compatível com o MAME.
Nas instruções a seguir iremos  configurar um ambiente de compilação
cruzada em uma plataforma Linux para compilar uma versão do MAME voltada
para o Microsoft Windows, apesar do processo abaixo ser voltado para
Windows, ele pode servir também de modelo para que você possa compilar o
MAME para outras plataformas compatíveis além do Windows.


\subsection{Vantagens}
\label{initialsetup/crosscompilemame:vantagens}
Dentre as várias vantagens é possível citar as mais relevantes:
\begin{itemize}
\item {} 
Transformar o código fonte em linguagem de máquina consome muitos
recursos e em geral a plataforma de destino pode não ter todos os
recursos disponíveis em comparação com computador que está sendo
usando para compilar, como por exemplo, poder de processamento,
memória, etc.

\item {} 
Ainda que você utilize o mesmo computador com dois sistemas
operacionais instalados como o Linux e o Windows no mesmo
computador, o tempo que você leva para compilar uma versão do MAME
para o Linux é muito menor do que compilar uma versão nativa do
MAME no Windows. Compilar uma versão do MAME para Linux leva em
torno de 30 minutos para mais ou para menos dependendo do poder de
processamento do seu computador, compilando o mesmo código fonte do
MAME, na mesma máquina com o Windows, usando a mesma versão do \emph{gcc}
e \emph{g++}, a tarefa pode levar algumas \emph{horas} \footnote[2]{\sphinxAtStartFootnote%
Todo o processo no meu computador leva cerca de 4 horas, AMD FX
tm-8350, 16GiB de memória DDR3. (Nota do tradutor)
} ainda que você tenha um
computador mais recente.

\item {} 
Ao utilizar o processo de compilação cruzada, o tempo final de
compilação leva aproximadamente o mesmo tempo que a versão nativa do
Linux fazendo com que você ganhe tempo e economize recursos, afinal
de contas, manter o processador a 100\% compilando o código fonte por
cerca de 30 minutos é uma coisa, fazer exatamente a mesma coisa
gastando algumas horas além de ser uma perda de tempo, a sua conta
de energia pode ficar um pouco mais cara no final do mês.

\item {} 
Assim o motivo principal para adotar a compilação cruzada é a
economia de tempo e recursos.

\end{itemize}


\subsection{Preparando o ambiente}
\label{initialsetup/crosscompilemame:preparando-o-ambiente}
A plataforma usada neste exemplo foi o \emph{Debian 9}, porém pode ser
qualquer outro, a vantagem do Debian é que ela é uma distribuição muito
estável do sistema operacional Linux, por causa disso, o Debian não
utiliza a última versão de nenhum software como o gcc por exemplo.
Geralmente ela fica alguma versões para trás do último lançamento
encontrado na internet pois o foco é a estabilidade ao invés de
empacotar a última versão do que quer que seja.

Precisamos instalar os pacotes abaixo para compilar binários voltados ao
sistema Windows, para outros sistemas operacionais ou dispositivos, o
procedimento será semelhante bastando que você escolha o conjuntos de
pacotes apropriados para a plataforma que você deseja compilar.
O comando abaixo vai instalar ferramentas adicionais além das quais já
foram descritas na seção {\hyperref[initialsetup/compilingmame:compiling\string-mame\string-debian]{\sphinxcrossref{\DUrole{std,std-ref}{Debian e Ubuntu (incluindo dispositivos Raspberry Pi e ODROID)}}}}, note que o
comando abaixo é formado por uma linha só:
\begin{quote}

\sphinxcode{sudo aptitude install binutils-mingw-w64-x86-64 g++-mingw-w64 g++-mingw-w64-x86-64 gcc-mingw-w64 gcc-mingw-w64-base gcc-mingw-w64-x86-64 gobjc++-mingw-w64 mingw-w64 mingw-w64-common mingw-w64-tools mingw-w64-x86-64-dev win-iconv-mingw-w64-dev}
\end{quote}

Como estamos fazendo uma compilação entre plataformas é necessário
usar a versão POSIX para o \textbf{gcc}, \textbf{ar} e \textbf{g++}, o POSIX vem de
\emph{Interface Portável entre Sistemas Operacionais} que é regida pela
norma \href{https://standards.ieee.org/standard/1003\_1-2017.html}{IEEE 1003} \footnote[3]{\sphinxAtStartFootnote%
IEEE é conhecido no Brasil como \href{https://pt.wikipedia.org/wiki/Instituto\_de\_Engenheiros\_Eletricistas\_e\_Eletrônicos}{Instituto de Engenheiros
Eletricistas e Eletrônicos}. (Nota do tradutor)
}.
Para configurar os atalhos do \textbf{gcc}, \textbf{ar} e \textbf{g++} voltado para
a criação de binários para a plataforma \textbf{64-Bits} faça os comandos
abaixo no terminal, note que \textbf{cada} comando \emph{sudo} é formado por uma
linha só:

\begin{Verbatim}[commandchars=\\\{\}]
\PYG{n}{sudo} \PYG{n}{ln} \PYG{o}{\PYGZhy{}}\PYG{n}{s} \PYG{o}{/}\PYG{n}{usr}\PYG{o}{/}\PYG{n+nb}{bin}\PYG{o}{/}\PYG{n}{x86\PYGZus{}64}\PYG{o}{\PYGZhy{}}\PYG{n}{w64}\PYG{o}{\PYGZhy{}}\PYG{n}{mingw32}\PYG{o}{\PYGZhy{}}\PYG{n}{g}\PYG{o}{+}\PYG{o}{+}\PYG{o}{\PYGZhy{}}\PYG{n}{posix} \PYG{o}{/}\PYG{n}{usr}\PYG{o}{/}\PYG{n}{x86\PYGZus{}64}\PYG{o}{\PYGZhy{}}\PYG{n}{w64}\PYG{o}{\PYGZhy{}}\PYG{n}{mingw32}\PYG{o}{/}\PYG{n+nb}{bin}\PYG{o}{/}\PYG{n}{x86\PYGZus{}64}\PYG{o}{\PYGZhy{}}\PYG{n}{w64}\PYG{o}{\PYGZhy{}}\PYG{n}{mingw32}\PYG{o}{\PYGZhy{}}\PYG{n}{g}\PYG{o}{+}\PYG{o}{+}
\PYG{n}{sudo} \PYG{n}{ln} \PYG{o}{\PYGZhy{}}\PYG{n}{s} \PYG{o}{/}\PYG{n}{usr}\PYG{o}{/}\PYG{n+nb}{bin}\PYG{o}{/}\PYG{n}{x86\PYGZus{}64}\PYG{o}{\PYGZhy{}}\PYG{n}{w64}\PYG{o}{\PYGZhy{}}\PYG{n}{mingw32}\PYG{o}{\PYGZhy{}}\PYG{n}{gcc}\PYG{o}{\PYGZhy{}}\PYG{n}{ar}\PYG{o}{\PYGZhy{}}\PYG{n}{posix} \PYG{o}{/}\PYG{n}{usr}\PYG{o}{/}\PYG{n}{x86\PYGZus{}64}\PYG{o}{\PYGZhy{}}\PYG{n}{w64}\PYG{o}{\PYGZhy{}}\PYG{n}{mingw32}\PYG{o}{/}\PYG{n+nb}{bin}\PYG{o}{/}\PYG{n}{x86\PYGZus{}64}\PYG{o}{\PYGZhy{}}\PYG{n}{w64}\PYG{o}{\PYGZhy{}}\PYG{n}{mingw32}\PYG{o}{\PYGZhy{}}\PYG{n}{gcc}\PYG{o}{\PYGZhy{}}\PYG{n}{ar}
\PYG{n}{sudo} \PYG{n}{ln} \PYG{o}{\PYGZhy{}}\PYG{n}{s} \PYG{o}{/}\PYG{n}{usr}\PYG{o}{/}\PYG{n+nb}{bin}\PYG{o}{/}\PYG{n}{x86\PYGZus{}64}\PYG{o}{\PYGZhy{}}\PYG{n}{w64}\PYG{o}{\PYGZhy{}}\PYG{n}{mingw32}\PYG{o}{\PYGZhy{}}\PYG{n}{gcc}\PYG{o}{\PYGZhy{}}\PYG{n}{posix} \PYG{o}{/}\PYG{n}{usr}\PYG{o}{/}\PYG{n}{x86\PYGZus{}64}\PYG{o}{\PYGZhy{}}\PYG{n}{w64}\PYG{o}{\PYGZhy{}}\PYG{n}{mingw32}\PYG{o}{/}\PYG{n+nb}{bin}\PYG{o}{/}\PYG{n}{x86\PYGZus{}64}\PYG{o}{\PYGZhy{}}\PYG{n}{w64}\PYG{o}{\PYGZhy{}}\PYG{n}{mingw32}\PYG{o}{\PYGZhy{}}\PYG{n}{gcc}
\end{Verbatim}

Já para a plataforma \textbf{32-Bits} faremos estes comandos, note que
\textbf{cada} comando \emph{sudo} é formado por uma linha só:

\begin{Verbatim}[commandchars=\\\{\}]
\PYG{n}{sudo} \PYG{n}{ln} \PYG{o}{\PYGZhy{}}\PYG{n}{s} \PYG{o}{/}\PYG{n}{usr}\PYG{o}{/}\PYG{n+nb}{bin}\PYG{o}{/}\PYG{n}{i686}\PYG{o}{\PYGZhy{}}\PYG{n}{w64}\PYG{o}{\PYGZhy{}}\PYG{n}{mingw32}\PYG{o}{\PYGZhy{}}\PYG{n}{g}\PYG{o}{+}\PYG{o}{+}\PYG{o}{\PYGZhy{}}\PYG{n}{posix} \PYG{o}{/}\PYG{n}{usr}\PYG{o}{/}\PYG{n}{i686}\PYG{o}{\PYGZhy{}}\PYG{n}{w64}\PYG{o}{\PYGZhy{}}\PYG{n}{mingw32}\PYG{o}{/}\PYG{n+nb}{bin}\PYG{o}{/}\PYG{n}{i686}\PYG{o}{\PYGZhy{}}\PYG{n}{w64}\PYG{o}{\PYGZhy{}}\PYG{n}{mingw32}\PYG{o}{\PYGZhy{}}\PYG{n}{g}\PYG{o}{+}\PYG{o}{+}
\PYG{n}{sudo} \PYG{n}{ln} \PYG{o}{\PYGZhy{}}\PYG{n}{s} \PYG{o}{/}\PYG{n}{usr}\PYG{o}{/}\PYG{n+nb}{bin}\PYG{o}{/}\PYG{n}{i686}\PYG{o}{\PYGZhy{}}\PYG{n}{w64}\PYG{o}{\PYGZhy{}}\PYG{n}{mingw32}\PYG{o}{\PYGZhy{}}\PYG{n}{gcc}\PYG{o}{\PYGZhy{}}\PYG{n}{ar}\PYG{o}{\PYGZhy{}}\PYG{n}{posix} \PYG{o}{/}\PYG{n}{usr}\PYG{o}{/}\PYG{n}{i686}\PYG{o}{\PYGZhy{}}\PYG{n}{w64}\PYG{o}{\PYGZhy{}}\PYG{n}{mingw32}\PYG{o}{/}\PYG{n+nb}{bin}\PYG{o}{/}\PYG{n}{i686}\PYG{o}{\PYGZhy{}}\PYG{n}{w64}\PYG{o}{\PYGZhy{}}\PYG{n}{mingw32}\PYG{o}{\PYGZhy{}}\PYG{n}{gcc}\PYG{o}{\PYGZhy{}}\PYG{n}{ar}
\PYG{n}{sudo} \PYG{n}{ln} \PYG{o}{\PYGZhy{}}\PYG{n}{s} \PYG{o}{/}\PYG{n}{usr}\PYG{o}{/}\PYG{n+nb}{bin}\PYG{o}{/}\PYG{n}{i686}\PYG{o}{\PYGZhy{}}\PYG{n}{w64}\PYG{o}{\PYGZhy{}}\PYG{n}{mingw32}\PYG{o}{\PYGZhy{}}\PYG{n}{gcc}\PYG{o}{\PYGZhy{}}\PYG{l+m+mf}{6.3}\PYG{o}{\PYGZhy{}}\PYG{n}{posix} \PYG{o}{/}\PYG{n}{usr}\PYG{o}{/}\PYG{n}{i686}\PYG{o}{\PYGZhy{}}\PYG{n}{w64}\PYG{o}{\PYGZhy{}}\PYG{n}{mingw32}\PYG{o}{/}\PYG{n+nb}{bin}\PYG{o}{/}\PYG{n}{i686}\PYG{o}{\PYGZhy{}}\PYG{n}{w64}\PYG{o}{\PYGZhy{}}\PYG{n}{mingw32}\PYG{o}{\PYGZhy{}}\PYG{n}{gcc}
\end{Verbatim}

Precisamos disponibilizar as variáveis \textbf{MINGW64} e \textbf{MINGW32} no
ambiente, elas são necessárias para que os scripts usados para a
compilação do MAME saibam onde encontrá-los.
Não é necessário usar o \emph{sudo} para o comando abaixo pois você deseja
aplicar a variável no ambiente da sua conta comum, não use uma conta com
poderes administrativos. É mais fácil criar uma conta comum apenas para
ser utilizada para compilar o MAME.

\begin{DUlineblock}{0em}
\item[] \sphinxcode{echo "export MINGW64="/usr/x86\_64-w64-mingw32"" \textgreater{}\textgreater{} \textasciitilde{}/.bashrc}
\item[] \sphinxcode{echo "export MINGW32="/usr/i686-w64-mingw32"" \textgreater{}\textgreater{} \textasciitilde{}/.bashrc}
\end{DUlineblock}

Recarregue as configurações do seu terminal com o comando \sphinxcode{. .bashrc}
(ponto, espaço, ponto bashrc) ou saia e retorne à sua conta. É
necessário aferir a configuração para que se tenha certeza de que as
variáveis estão definidas no ambiente corretamente fazendo o comando
abaixo:

\begin{DUlineblock}{0em}
\item[] \sphinxcode{\$ echo \$MINGW64}
\item[] \sphinxcode{/usr/x86\_64-w64-mingw32}
\item[] \sphinxcode{\$ echo \$MINGW32}
\item[] \sphinxcode{/usr/i686-w64-mingw32}
\end{DUlineblock}

Caso o seu ambiente não tenha retornado nada, tenha certeza de que as
instruções acima foram seguidas corretamente, se a sua distribuição
Linux - ou outra distribuição - utiliza o arquivo \sphinxcode{.bashrc}, caso não
utilize, verifique no manual da sua distribuição qual arquivo de
configuração ela utiliza para armazenar as variáveis do ambiente e onde
ele se localiza.
\clearpage

\subsection{Compilando o MAME para Windows no Linux}
\label{initialsetup/crosscompilemame:compilando-o-mame-para-windows-no-linux}
Para compilar uma versão \emph{64-Bits} do MAME para o \textbf{Windows}, execute o
comando abaixo, lembrando que o comando deve ser executado de dentro da
pasta raiz \footnote[4]{\sphinxAtStartFootnote%
É no mesmo diretório onde existe um arquivo chamado
\textbf{makefile}. (Nota do tradutor)
} do código fonte do MAME:

\begin{Verbatim}[commandchars=\\\{\}]
\PYG{n}{make} \PYG{n}{clean} \PYG{o}{\PYGZam{}}\PYG{o}{\PYGZam{}} \PYG{n}{make} \PYG{n}{TARGETOS}\PYG{o}{=}\PYG{n}{windows} \PYG{n}{CROSS\PYGZus{}BUILD}\PYG{o}{=}\PYG{l+m+mi}{1} \PYG{n}{SYMBOLS}\PYG{o}{=}\PYG{l+m+mi}{1} \PYG{n}{SYMLEVEL}\PYG{o}{=}\PYG{l+m+mi}{1} \PYG{n}{STRIP\PYGZus{}SYMBOLS}\PYG{o}{=}\PYG{l+m+mi}{1} \PYG{n}{SSE2}\PYG{o}{=}\PYG{l+m+mi}{1} \PYG{n}{PTR64}\PYG{o}{=}\PYG{l+m+mi}{1}
\end{Verbatim}

Caso você queira compilar uma versão \emph{32-Bits} do MAME faça o comando
abaixo:

\begin{Verbatim}[commandchars=\\\{\}]
\PYG{n}{make} \PYG{n}{clean} \PYG{o}{\PYGZam{}}\PYG{o}{\PYGZam{}} \PYG{n}{make} \PYG{n}{TARGETOS}\PYG{o}{=}\PYG{n}{windows} \PYG{n}{CROSS\PYGZus{}BUILD}\PYG{o}{=}\PYG{l+m+mi}{1} \PYG{n}{SYMBOLS}\PYG{o}{=}\PYG{l+m+mi}{1} \PYG{n}{SYMLEVEL}\PYG{o}{=}\PYG{l+m+mi}{1} \PYG{n}{STRIP\PYGZus{}SYMBOLS}\PYG{o}{=}\PYG{l+m+mi}{1} \PYG{n}{SSE2}\PYG{o}{=}\PYG{l+m+mi}{1}
\end{Verbatim}

Assim como na compilação nativa, você pode adicionar a opção \textbf{-j} no
final do comando visando acelerar o processo de compilação usando os
núcleos do seu processador como já foi explicado com mais detalhes no
capítulo {\hyperref[initialsetup/compilingmame:compiling\string-mame]{\sphinxcrossref{\DUrole{std,std-ref}{Compilando o MAME}}}}:

\begin{Verbatim}[commandchars=\\\{\}]
\PYG{n}{make} \PYG{n}{clean} \PYG{o}{\PYGZam{}}\PYG{o}{\PYGZam{}} \PYG{n}{make} \PYG{n}{TARGETOS}\PYG{o}{=}\PYG{n}{windows} \PYG{n}{CROSS\PYGZus{}BUILD}\PYG{o}{=}\PYG{l+m+mi}{1} \PYG{n}{SYMBOLS}\PYG{o}{=}\PYG{l+m+mi}{1} \PYG{n}{SYMLEVEL}\PYG{o}{=}\PYG{l+m+mi}{1} \PYG{n}{STRIP\PYGZus{}SYMBOLS}\PYG{o}{=}\PYG{l+m+mi}{1} \PYG{n}{SSE2}\PYG{o}{=}\PYG{l+m+mi}{1} \PYG{n}{PTR64}\PYG{o}{=}\PYG{l+m+mi}{1} \PYG{o}{\PYGZhy{}}\PYG{n}{j5}
\end{Verbatim}

Abaixo algumas descrições resumidas das opções usadas:
\begin{itemize}
\item {} 
\textbf{make}
\begin{quote}

Executa o comando de compilação do código fonte.
\end{quote}

\item {} 
\textbf{clean}
\begin{quote}

Apaga todo o diretório \textbf{build}, é dentro deste diretório onde
qualquer compilação ou configuração prévia fica armazenada.
\end{quote}

\item {} 
\textbf{TARGETOS=windows}
\begin{quote}

Define o Sistema Operacional alvo, Windows.
\end{quote}

\item {} 
\textbf{CROSS\_BUILD=1}
\begin{quote}

Define que é uma compilação cruzada.
\end{quote}

\item {} 
\textbf{SYMBOLS=1}
\begin{quote}

Define que o MAME conterá símbolos de depuração.
\end{quote}

\item {} 
\textbf{SYMLEVEL=1}
\begin{quote}

Define a quantidade de símbolos de depuração que o MAME terá,
valores maiores que \textbf{1} incluirá mais e mais símbolos
deixando o arquivo final maior do que ele já é.
\end{quote}

\item {} 
\textbf{STRIP\_SYMBOLS=1}
\begin{quote}

Define que os símbolos ao invés de ficar embutido no MAME ficará
em um arquivo ``\textbf{.sym}'' separado.
\end{quote}

\item {} 
\textbf{SSE2=1}
\begin{quote}

\textbf{Double Precision Streaming SIMD Extensions}, em resumo, são
instruções que otimizam o desempenho em processadores
compatíveis. O MAME terá uma melhor performance quando essa
opção é utilizada durante a compilação.
Assim informa a \href{https://www.mamedev.org/?p=451}{nota publicada} no site do MAME.
\end{quote}

\item {} 
\textbf{PTR64=1}
\begin{quote}

Quando igual a \textbf{1} irá gerar uma versão 64 Bits do MAME e 32
Bits quando for igual a \textbf{0}.
\end{quote}

\end{itemize}

Caso não haja nenhum problema durante esse processo, você terá um
executável do MAME chamado \textbf{mame64.exe} para a versão \emph{64-Bits} ou
\textbf{mame.exe} caso você tenha compilado uma versão para \emph{32-Bits}.

Junto com estes binários será criado também um arquivo de símbolos,
para a versão \emph{64-Bits} será criado o arquivo \textbf{mame64.sym} ou
\textbf{mame.sym} para a versão \emph{32-Bits}. Estes arquivos devem \textbf{sempre}
estar junto com o executável do MAME, pois em caso de algum erro crítico
durante a emulação, esse arquivo ``\textbf{.sym}'' é usado para traduzir as
referências usadas no código fonte junto com os códigos de erro, muito
útil para os desenvolvedores. Aqui um exemplo de como estes códigos de
erro se parecem:

\begin{Verbatim}[commandchars=\\\{\}]
\PYG{n+ne}{Exception} \PYG{n}{at} \PYG{n}{EIP}\PYG{o}{=}\PYG{l+m+mi}{00000000} \PYG{p}{(}\PYG{n}{something\PYGZus{}state}\PYG{p}{:}\PYG{p}{:}\PYG{n}{something}\PYG{p}{(}\PYG{p}{)}\PYG{o}{+}\PYG{l+m+mh}{0x0000}\PYG{p}{)}\PYG{p}{:} \PYG{n}{ACCESS} \PYG{n}{VIOLATION}
\PYG{n}{While} \PYG{n}{attempting} \PYG{n}{to} \PYG{n}{read} \PYG{n}{memory} \PYG{n}{at} \PYG{l+m+mi}{00000000}
\PYG{o}{\PYGZhy{}}\PYG{o}{\PYGZhy{}}\PYG{o}{\PYGZhy{}}\PYG{o}{\PYGZhy{}}\PYG{o}{\PYGZhy{}}\PYG{o}{\PYGZhy{}}\PYG{o}{\PYGZhy{}}\PYG{o}{\PYGZhy{}}\PYG{o}{\PYGZhy{}}\PYG{o}{\PYGZhy{}}\PYG{o}{\PYGZhy{}}\PYG{o}{\PYGZhy{}}\PYG{o}{\PYGZhy{}}\PYG{o}{\PYGZhy{}}\PYG{o}{\PYGZhy{}}\PYG{o}{\PYGZhy{}}\PYG{o}{\PYGZhy{}}\PYG{o}{\PYGZhy{}}\PYG{o}{\PYGZhy{}}\PYG{o}{\PYGZhy{}}\PYG{o}{\PYGZhy{}}\PYG{o}{\PYGZhy{}}\PYG{o}{\PYGZhy{}}\PYG{o}{\PYGZhy{}}\PYG{o}{\PYGZhy{}}\PYG{o}{\PYGZhy{}}\PYG{o}{\PYGZhy{}}\PYG{o}{\PYGZhy{}}\PYG{o}{\PYGZhy{}}\PYG{o}{\PYGZhy{}}\PYG{o}{\PYGZhy{}}\PYG{o}{\PYGZhy{}}\PYG{o}{\PYGZhy{}}\PYG{o}{\PYGZhy{}}\PYG{o}{\PYGZhy{}}\PYG{o}{\PYGZhy{}}\PYG{o}{\PYGZhy{}}\PYG{o}{\PYGZhy{}}\PYG{o}{\PYGZhy{}}\PYG{o}{\PYGZhy{}}\PYG{o}{\PYGZhy{}}\PYG{o}{\PYGZhy{}}\PYG{o}{\PYGZhy{}}\PYG{o}{\PYGZhy{}}\PYG{o}{\PYGZhy{}}\PYG{o}{\PYGZhy{}}\PYG{o}{\PYGZhy{}}\PYG{o}{\PYGZhy{}}\PYG{o}{\PYGZhy{}}\PYG{o}{\PYGZhy{}}\PYG{o}{\PYGZhy{}}\PYG{o}{\PYGZhy{}}\PYG{o}{\PYGZhy{}}
\PYG{n}{EAX}\PYG{o}{=}\PYG{l+m+mi}{00000000} \PYG{n}{EBX}\PYG{o}{=}\PYG{l+m+mi}{0}\PYG{n}{fffffff} \PYG{n}{ECX}\PYG{o}{=}\PYG{l+m+mi}{0}\PYG{n}{fffffff} \PYG{n}{EDX}\PYG{o}{=}\PYG{l+m+mi}{00000000}
\PYG{n}{ESI}\PYG{o}{=}\PYG{l+m+mi}{00000000} \PYG{n}{EDI}\PYG{o}{=}\PYG{l+m+mi}{00000000} \PYG{n}{EBP}\PYG{o}{=}\PYG{l+m+mi}{00000000} \PYG{n}{ESP}\PYG{o}{=}\PYG{l+m+mi}{00000000}
\PYG{o}{\PYGZhy{}}\PYG{o}{\PYGZhy{}}\PYG{o}{\PYGZhy{}}\PYG{o}{\PYGZhy{}}\PYG{o}{\PYGZhy{}}\PYG{o}{\PYGZhy{}}\PYG{o}{\PYGZhy{}}\PYG{o}{\PYGZhy{}}\PYG{o}{\PYGZhy{}}\PYG{o}{\PYGZhy{}}\PYG{o}{\PYGZhy{}}\PYG{o}{\PYGZhy{}}\PYG{o}{\PYGZhy{}}\PYG{o}{\PYGZhy{}}\PYG{o}{\PYGZhy{}}\PYG{o}{\PYGZhy{}}\PYG{o}{\PYGZhy{}}\PYG{o}{\PYGZhy{}}\PYG{o}{\PYGZhy{}}\PYG{o}{\PYGZhy{}}\PYG{o}{\PYGZhy{}}\PYG{o}{\PYGZhy{}}\PYG{o}{\PYGZhy{}}\PYG{o}{\PYGZhy{}}\PYG{o}{\PYGZhy{}}\PYG{o}{\PYGZhy{}}\PYG{o}{\PYGZhy{}}\PYG{o}{\PYGZhy{}}\PYG{o}{\PYGZhy{}}\PYG{o}{\PYGZhy{}}\PYG{o}{\PYGZhy{}}\PYG{o}{\PYGZhy{}}\PYG{o}{\PYGZhy{}}\PYG{o}{\PYGZhy{}}\PYG{o}{\PYGZhy{}}\PYG{o}{\PYGZhy{}}\PYG{o}{\PYGZhy{}}\PYG{o}{\PYGZhy{}}\PYG{o}{\PYGZhy{}}\PYG{o}{\PYGZhy{}}\PYG{o}{\PYGZhy{}}\PYG{o}{\PYGZhy{}}\PYG{o}{\PYGZhy{}}\PYG{o}{\PYGZhy{}}\PYG{o}{\PYGZhy{}}\PYG{o}{\PYGZhy{}}\PYG{o}{\PYGZhy{}}\PYG{o}{\PYGZhy{}}\PYG{o}{\PYGZhy{}}\PYG{o}{\PYGZhy{}}\PYG{o}{\PYGZhy{}}\PYG{o}{\PYGZhy{}}\PYG{o}{\PYGZhy{}}
\PYG{n}{Stack} \PYG{n}{crawl}\PYG{p}{:}
\PYG{l+m+mi}{0012}\PYG{n}{abcd}\PYG{p}{:} \PYG{l+m+mi}{00123456} \PYG{p}{(}\PYG{n}{something\PYGZus{}state}\PYG{p}{:}\PYG{p}{:}\PYG{n}{something}\PYG{p}{(}\PYG{p}{)}\PYG{o}{+}\PYG{l+m+mh}{0x0000}\PYG{p}{)}
\PYG{l+m+mi}{0034}\PYG{n}{ef01}\PYG{p}{:} \PYG{l+m+mi}{00789}\PYG{n}{abc} \PYG{p}{(}\PYG{n}{something\PYGZus{}state}\PYG{p}{:}\PYG{p}{:}\PYG{n}{something}\PYG{p}{(}\PYG{p}{)}\PYG{o}{+}\PYG{l+m+mh}{0x0000}\PYG{p}{)}
\PYG{n}{E} \PYG{n}{a} \PYG{n}{listagem} \PYG{n}{continua}
\PYG{o}{.}\PYG{o}{.}\PYG{o}{.}
\end{Verbatim}

Caso o MAME trave durante a emulação e esses códigos apareçam na tela,
copie e reporte \footnote[5]{\sphinxAtStartFootnote%
Pedimos a gentileza de relatar os problemas encontrados em
Inglês. (Nota do tradutor)
} o erro no fórum
\href{https://mametesters.org/view\_all\_bug\_page.php/}{MAME testers}.


\subsection{Lidando com alguns problemas comuns}
\label{initialsetup/crosscompilemame:lidando-com-alguns-problemas-comuns}
Algumas vezes o processo de compilação é interrompido antes de chegar ao
fim, os motivos são os mais diversos, pode ser a falta de alguma
biblioteca, erro de configuração em algum lugar, uma atualização do
código fonte onde algum desenvolvedor deixou passar algo desapercebido,
enfim, se você está encarando a tarefa de compilar o seu próprio MAME,
``\emph{problema}'' é algo que você deve estar preparado caso ocorra.

A primeira coisa a se prestar atenção é ver no terminal, console ou
\emph{prompt de comando} que você estiver usando, qual o erro que fez todo o
processo parar, para compilar novamente a partir do ponto que a
compilação parou, tudo o que você precisa fazer é repetir o comando de
compilação sem o \textbf{make clean \&\&} no começo.

Observe que caso você esteja atualizando o código fonte direto do
\href{https://github.com/mamedev/mame}{repositório GIT do MAME}, é
necessário que você SEMPRE faça um \textbf{make clean} antes de compilar
um novo binário, independente da plataforma.

Geralmente o processo continua sem maiores problemas, porém caso o
processo pare novamente no mesmo lugar, pode haver algum outro problema
como a falta de alguma biblioteca, incompatibilidade com alguma coisa,
etc.
Caso esteja usando a versão ``GIT'' ao invés da versão final do MAME,
saiba que a versão ``GIT'' sofre várias atualizações ao longo do dia e por
isso aguarde algumas horas, atualize novamente o código fonte e tente
outra vez.


\chapter{NOÇÕES GERAIS DE USO E CONFIGURAÇÃO DO MAME}
\label{usingmame/index:nocoes-gerais-de-uso-e-configuracao-do-mame}\label{usingmame/index::doc}
Esta seção descreve as informações gerais de uso e informações sobre
o MAME. A intenção é abordar aspectos comuns de uso e da configuração do
MAME que se aplicam a todos os sistemas operacionais.

Para opções adicionais que sejam especificas à um sistema operacional,
veja a seção do documento que seja especifica para a sua plataforma.


\section{Usando o MAME}
\label{usingmame/usingmame:usando-o-mame}\label{usingmame/usingmame::doc}
Caso você queira sair usando sem precisar usar a linha de comando
saiba que você já pode usar o MAME sem precisar baixar e configurar
nenhuma interface gráfica. Inicie o MAME sem parâmetros, lhe será
apresentada a interface gráfica do MAME ao clicar duas vezes no arquivo
\textbf{mame.exe} ou executando-o diretamente da linha de comando.
Caso você esteja interessado em desvendar todo o poder que o MAME pode
te oferecer, continue lendo.

Em plataformas baseadas em Macintosh OS X e plataformas com base *nix,
certifique-se de configurar a fonte do seu sistema para que corresponda
ao seu idioma antes de iniciar, caso contrário você pode não conseguir
ler o texto devido à falta de glifos e outros caracteres.

Caso você seja um novo usuário do MAME, você pode a princípio, achá-lo
um pouco complexo. Vamos falar um pouco sobre as listas de programas
(\emph{softlists}), pois elas podem simplificar bastante as coisas para você.
Caso o conteúdo que você esteja tentando reproduzir já esteja listado no
MAME, iniciar o conteúdo é tão fácil quanto;
\begin{quote}

\textbf{mame.exe} \textless{}\emph{system}\textgreater{} \textless{}\emph{software}\textgreater{}
\end{quote}

Por exemplo:
\begin{quote}

\textbf{mame.exe nes metroidu}
\end{quote}

Isso vai fazer com que o a versão americana do Metroid para o Nintendo
Entertainment System seja carregada.

Alternativamente, você poderia começar MAME com:
\begin{quote}

\textbf{mame.exe nes}
\end{quote}

E escolher numa \emph{lista de jogos} qual deseja iniciar. A partir daí
você pode escolher qualquer jogo compatível com a lista que você tenha,
essa listagem nada mais é do que um conjunto de todas as ROMs que você
tem armazenado na pasta ROMs ou outro lugar que você tenha configurado.
Observe que muitas cópias de ROMs antigas, de fitas e discos que
funcionavam em versões anteriores, podem não mais serem reconhecidas
pelas versões mais novas do MAME, exigindo que você as atualize ou as
renomeie para um nome compatível com a última versão do MAME para que
elas possam voltar a funcionar.

Caso você esteja carregando uma placa de arcade ou outro conteúdo que
não esteja na lista, as coisas ficam um pouco mais complicadas.
\clearpage
A estrutura básica da linha de comando fica assim:
\begin{quote}

\textbf{mame.exe} \textless{}\emph{system}\textgreater{} \textless{}\emph{media}\textgreater{} \textless{}\emph{software}\textgreater{} \textless{}\emph{options}\textgreater{}
\end{quote}

Onde:
\begin{itemize}
\item {} 
\textless{}\emph{system}\textgreater{} é o apelido ou o nome encurtado do sistema que deseja
emular (por exemplo, nes, c64, etc).

\item {} 
\textless{}\emph{media}\textgreater{} é o seletor da mídia que você deseja carregar (se for um
cartucho, tente \textbf{-cart} ou \textbf{-cart1}; caso seja um disquete,
tente \textbf{-flop} or \textbf{-flop1}; caso seja um CD-ROM, tente
\textbf{-cdrom}).

\item {} 
\textless{}\emph{software}\textgreater{} é o programa ou jogo que deseja carregar (também pode
ser usado o caminho completo para o arquivo a ser carregado ou como
o nome abreviado do arquivo que esteja na sua lista de software).

\item {} 
\textless{}\emph{options}\textgreater{} é qualquer opção de linha de comando adicional para
controles, vídeo, áudio, etc.

\end{itemize}

Lembre-se que se você digitar um nome de sistema \textless{}\emph{system}\textgreater{} que não
corresponda a nenhum sistema emulado \footnote[1]{\sphinxAtStartFootnote%
Existe uma diferença entre sistema e máquina, o comando em
questão funciona apenas com sistemas. Arcades são considerados
máquinas como o CPS1, CP2, ZN, etc. O comando ao ser usado com
uma máquina irá retornar um erro ``\emph{Unknown system}''.
(Nota do tradutor)
}, o MAME irá sugerir algumas opções
próximas ao que você digitou. Caso você não saiba quais mídias \textless{}\emph{media}\textgreater{}
estão disponíveis, você sempre poderá iniciar a emulação como mostra
o exemplo abaixo:
\begin{quote}

\textbf{mame.exe} \textless{}\emph{system}\textgreater{} \textbf{-listmedia}
\end{quote}

Caso você não saiba qual opção \textless{}\emph{options}\textgreater{} está disponível, há algumas
coisas que você pode fazer. Primeiro de tudo, você pode verificar a
seção deste manual sobre as opções de linha de comando. Você também pode
tentar alguns citados em {\hyperref[usingmame/frontends:frontends]{\sphinxcrossref{\DUrole{std,std-ref}{Interfaces}}}}, dentre outros disponíveis
para o MAME.

Como alternativa, você também pode usar a opção abaixo para obter ajuda:
\begin{quote}

\textbf{mame.exe -help}
\end{quote}

O comando exibe algumas opções básicas de uso, a versão do MAME e outras
informações.
\begin{quote}

\textbf{mame.exe -showusage}
\end{quote}

Mostra uma lista (bastante longa) das opções de linha de comando
disponíveis para o MAME. As opções principais são descritas na seção
{\hyperref[commandline/commandline\string-index:index\string-commandline]{\sphinxcrossref{\DUrole{std,std-ref}{Indice das opções de linha de comando}}}} deste manual.
\begin{quote}

\textbf{mame.exe -showconfig}
\end{quote}

Mostra uma lista (bastante longa) das opções de configuração que estão
sendo usadas pelo MAME. Essas configurações sempre podem ser modificadas
na linha de comando ou editadas diretamente no arquivo \textbf{mame.ini} que
é o arquivo de configuração primário do MAME. Você pode encontrar uma
descrição de algumas opções de configuração na seção
{\hyperref[commandline/commandline\string-index:index\string-commandline]{\sphinxcrossref{\DUrole{std,std-ref}{Indice das opções de linha de comando}}}} do manual (na maioria dos casos, cada opção de
configuração listada ali, possui uma versão equivalente para a linha de
comando).
\begin{quote}

\textbf{mame.exe -createconfig}
\end{quote}

Cria um novo arquivo \textbf{mame.ini} com as configurações primárias já
predefinidas.
Observe que o \textbf{mame.ini} é basicamente um arquivo de texto simples,
portanto, você pode abri-lo com qualquer editor de texto (como o
Notepad, Geany, Emacs ou TextEdit por exemplo) e alterar todas as opções
conforme a sua necessidade. A principio, não há a necessidade de nenhum
ajuste específico para começar a usar o MAME, então você pode
basicamente deixar a maioria das opções inalteradas.

Caso o MAME venha a ser atualizado, novas opções disponíveis serão
aplicadas ao mame.ini anterior \footnote[2]{\sphinxAtStartFootnote%
Caso você tenha alguma opção customizada neste arquivo, é
recomendável que um backp seja feito antes de executar o
comando. (Nota do tradutor)
} quando o comando for executado
novamente.

Agora que você tem mais confiança, você pode tentar melhorar e
customizar as opções do MAME. Só tenha em mente a ordem em que as opções
são lidas.

Veja {\hyperref[advanced/multiconfig:advanced\string-multi\string-cfg]{\sphinxcrossref{\DUrole{std,std-ref}{A ordem de leitura dos arquivos}}}} para obter mais informações.


\section{Teclas já predefinidas}
\label{usingmame/defaultkeys:default-keys}\label{usingmame/defaultkeys::doc}\label{usingmame/defaultkeys:teclas-ja-predefinidas}
Todas as teclas abaixo podem ser configuradas na interface do usuário.
Esta lista mostra as teclas que já vem pré-configuradas.

\begin{longtable}{|p{0.475\linewidth}|p{0.475\linewidth}|}
\hline
\endfirsthead

\multicolumn{2}{c}%
{{\tablecontinued{\tablename\ \thetable{} -- continuação da página anterior}}} \\
\hline
\endhead

\hline \multicolumn{2}{|r|}{{\tablecontinued{Continuação na próxima página}}} \\ \hline
\endfoot

\endlastfoot


Tecla
&
\begin{DUlineblock}{0em}
\item[] Ação
\end{DUlineblock}
\\
\hline
\textbf{Tab}
&
\begin{DUlineblock}{0em}
\item[] Exibe um cardápio que dá acesso a diferentes configurações.
\end{DUlineblock}
\\
\hline
\textbf{\textasciitilde{}}
&
\begin{DUlineblock}{0em}
\item[] Exibe opções configuráveis na parte de baixo da tela, use as seguintes teclas para controlá-las:
\item[] 
\item[] * \textbf{Cima} - selecione o parâmetro anterior para modificar
\item[] * \textbf{Baixo} - selecione o próximo parâmetro para modificar
\item[] * \textbf{Esquerda} - reduz o valor do parâmetro selecionado
\item[] * \textbf{Direita} - incrementa o valor do parâmetro selecionado
\item[] * \textbf{Enter} - zera o valor do parâmetro para seu valor inicial
\item[] * \textbf{Control+Esquerda} - reduz o valor em passos de \emph{10x}
\item[] * \textbf{Shift+Esquerda} - reduz o valor em passos de \emph{0.1x}
\item[] * \textbf{Alt+Esquerda} - reduz o valor pela menor quantidade
\item[] * \textbf{Control+Direita} - incrementa o valor em passos de \emph{10x}
\item[] * \textbf{Shift+Direita} - incrementa o valor em passos de \emph{0.1x}
\item[] * \textbf{Alt+Right} - incrementa o valor pela menor quantidade
\item[] 
\item[] Se você estiver rodando com a opção \textbf{-debug}, esta tecla envia um `break' para a emulação.
\end{DUlineblock}
\\
\hline
\textbf{P}
&
\begin{DUlineblock}{0em}
\item[] Pausa o jogo.
\end{DUlineblock}
\\
\hline
\textbf{Shift+P}
&
\begin{DUlineblock}{0em}
\item[] Enquanto estiver pausado, avança para o próximo quadro. Caso a opção retroceder
\item[] esteja ativa, será capturado um novo estado de retrocesso, assim como este também será salvo.
\end{DUlineblock}
\\
\hline
\textbf{Shift+\textasciitilde{}}
&
\begin{DUlineblock}{0em}
\item[] Enquanto estiver pausado, carrega o estado de salvamento de retrocesso mais recente.
\end{DUlineblock}
\\
\hline
\textbf{F2}
&
\begin{DUlineblock}{0em}
\item[] Modo de serviço para jogos que seja compatíveis.
\end{DUlineblock}
\\
\hline
\textbf{F3}
&
\begin{DUlineblock}{0em}
\item[] Reinicia o jogo.
\end{DUlineblock}
\\
\hline
\textbf{Shift+F3}
&
\begin{DUlineblock}{0em}
\item[] Executa uma ``reinicialização forçada'', fechando e reiniciando a emulação do zero.
\item[] Este produz um reinicio mais limpo e completo do que pressionar apenas o F3.
\end{DUlineblock}
\\
\hline
\textbf{LCtrl+F3}
&
\begin{DUlineblock}{0em}
\item[] {[}APENAS SDL{]} - Alterna o alongamento irregular.
\end{DUlineblock}
\\
\hline
\textbf{F4}
&
\begin{DUlineblock}{0em}
\item[] Mostra a paleta GFX decodificada dos jogos e tilemaps. A tecla Enter alterna entre os modos \textbf{paleta}, \textbf{gráficos} e \textbf{tilemaps}. Pressione \textbf{F4} novamente para sair. Os controles das teclas variam de um modo para outro.
\item[] 
\item[] Paleta / modo tabela de cores colortable:
\item[]
\begin{DUlineblock}{\DUlineblockindent}
\item[] * \textbf{{[} {]}} - alterna entre os modos paleta e tabela de cores
\item[] * \textbf{Cima/Baixo} - rolar uma linha de cada vez para cima e para baixo
\item[] * \textbf{Page Up/Page Down} - desloca uma página de cada vez para cima e para baixo
\item[] * \textbf{Home/End} - ir para o topo ou final da lista
\item[] * \textbf{-/+} - aumenta ou diminui a quantidade de cores por linha
\item[] * \textbf{Enter} - altera para o visualizador gráfico
\item[] 
\end{DUlineblock}
\item[] Modo gráfico:
\item[]
\begin{DUlineblock}{\DUlineblockindent}
\item[] * \textbf{{[} {]}} - alterna entre diferentes conjuntos de gráficos
\item[] * \textbf{Cima/Baixo} - rolar uma linha de cada vez para cima e para baixo
\item[] * \textbf{Page Up/Page Down} - desloca uma página de cada vez para cima e para baixo
\item[] * \textbf{Home/End} - ir para o topo ou final da lista
\item[] * \textbf{Esquerda/Direita} - altera a cor que está sendo exibida
\item[] * \textbf{R} - rotacionar os blocos em \emph{90º} no sentido horário
\item[] * \textbf{-/+} - aumenta/diminui a quantidade de blocos por linha
\item[] * \textbf{Enter} - alterna para o visualizador de tilemap
\item[] 
\end{DUlineblock}
\item[] Modo Tilemap: \protect\footnotemark[1] \protect\footnotemark[2]
\item[]
\begin{DUlineblock}{\DUlineblockindent}
\item[] * \textbf{{[} {]}} - alterna entre diferentes tilemaps
\item[] * \textbf{Cima/Baixo/Esquerda/Direita} - rolar \emph{8 pixels} de cada vez
\item[] * \textbf{Shift+Cima/Baixo/Esquerda/Direita} - rolar \emph{1 pixel} de cada vez
\item[] * \textbf{Control+Cima/Baixo/Esquerda/Direita} - rolar \emph{64 pixels} de cada vez
\item[] * \textbf{R} - rotacionar o ângulo de visão do tilemap em \emph{90º} no sentido horário
\item[] * \textbf{-/+} - aumenta/diminui o fator de zoom
\item[] * \textbf{Enter} - altera para o modo paleta de cores ou tabela de cores
\item[] 
\end{DUlineblock}
\end{DUlineblock}
\\
\hline
\textbf{LCtrl+F4}
&
\begin{DUlineblock}{0em}
\item[] {[}\textbf{APENAS SDL}{]} - Alterna a relação de aspecto da tela.
\end{DUlineblock}
\\
\hline
\textbf{LCtrl+F5}
&
\begin{DUlineblock}{0em}
\item[] {[}\textbf{APENAS SDL}{]} - Alterna o Filtro.
\end{DUlineblock}
\\
\hline
\textbf{Alt+Ctrl+F5}
&
\begin{DUlineblock}{0em}
\item[] {[}\textbf{NÃO SDL APENAS MS WINDOWS}{]} - Alterna o processamento HLSL final
\end{DUlineblock}
\\
\hline
\textbf{F6}
&
\begin{DUlineblock}{0em}
\item[] Alterna o modo de trapaça (caso o MAME seja iniciado com a opção \textbf{-cheat})
\end{DUlineblock}
\\
\hline
\textbf{LCtrl+F6}
&
\begin{DUlineblock}{0em}
\item[] Diminui a Proporção de Escala Preliminar
\end{DUlineblock}
\\
\hline
\textbf{F7}
&
\begin{DUlineblock}{0em}
\item[] Ler a gravação de estado. Você será solicitado a pressionar uma tecla para determinar qual
\item[] a gravação de estado você deseja carregar.
\item[] 
\item[] \emph{Observe que o recurso de gravação de estado não é compatível com uma grande quantidade}
\item[] \emph{de drivers. Caso não exista a compatibilidade em um determinado driver, você receberá}
\item[] \emph{um alerta ao tentar carregar ou salvar.}
\end{DUlineblock}
\\
\hline
\textbf{LCtrl+F7}
&
\begin{DUlineblock}{0em}
\item[] Aumenta a Proporção de Escala Preliminar
\end{DUlineblock}
\\
\hline
\textbf{Shift+F7}
&
\begin{DUlineblock}{0em}
\item[] Cria uma gravação de estado. Precisa pressionar uma tecla a mais para identificar o estado,
\item[] semelhante à opção de carregamento acima.
\end{DUlineblock}
\\
\hline
\textbf{F8}
&
\begin{DUlineblock}{0em}
\item[] Diminui o pulo de quadro.
\end{DUlineblock}
\\
\hline
\textbf{F9}
&
\begin{DUlineblock}{0em}
\item[] Aumenta o pulo de quadro.
\end{DUlineblock}
\\
\hline
\textbf{F10}
&
\begin{DUlineblock}{0em}
\item[] Alterna o afogador de velocidade.
\end{DUlineblock}
\\
\hline
\textbf{F11}
&
\begin{DUlineblock}{0em}
\item[] Alterna o indicador de velocidade.
\end{DUlineblock}
\\
\hline
\textbf{Shift+F11}
&
\begin{DUlineblock}{0em}
\item[] Alterna indicador interno de perfil (caso tenha sido compilado com).
\end{DUlineblock}
\\
\hline
\textbf{Alt+F11}
&
\begin{DUlineblock}{0em}
\item[] Grava vídeo renderizado com filtros HLSL.
\end{DUlineblock}
\\
\hline
\textbf{F12}
&
\begin{DUlineblock}{0em}
\item[] Salva um instantâneo de tela.
\end{DUlineblock}
\\
\hline
\textbf{Alt+F12}
&
\begin{DUlineblock}{0em}
\item[] Tira um instantâneo de tela renderizado com filtros HLSL.
\end{DUlineblock}
\\
\hline
\textbf{Insert}
&
\begin{DUlineblock}{0em}
\item[] {[}\textbf{APENAS JANELA, NÃO SDL}{]} Avanço rápido. Enquanto a tecla estiver pressionada, roda o jogo com
\item[] o afogador deligado e com o pulo de quadros no máximo.
\end{DUlineblock}
\\
\hline
\textbf{Page DN}
&
\begin{DUlineblock}{0em}
\item[] {[}\textbf{APENAS SDL}{]} Avanço rápido. Enquanto a tecla estiver pressionada, roda o jogo com o afogador de velocidade
\item[] desligado e com o pulo de quadros no máximo.
\end{DUlineblock}
\\
\hline
\textbf{Alt+ENTER}
&
\begin{DUlineblock}{0em}
\item[] Alterna entre o modo janelado e de tela inteira.
\end{DUlineblock}
\\
\hline
\textbf{Scroll Lock}
&
\begin{DUlineblock}{0em}
\item[] Mapeamento padrão para \textbf{-uimodekey}.
\item[] 
\item[] Essa tecla permite que os usuários ativem ou desativem o teclado emulado
\item[] em máquinas que precisam. Todas as emulações que precisam de teclados emulados
\item[] começarão nesse modo e você só poderá acessar
\item[] a IU (pressionando TAB), depois de pressionar essa tecla primeiro. Você pode mudar a condição
\item[] inicial do teclado emulado como demonstrado logo abaixo com mais detalhes
\item[] usando a opção \textbf{-ui\_active}.
\end{DUlineblock}
\\
\hline
\textbf{Escape}
&
\begin{DUlineblock}{0em}
\item[] Sai do emulador.
\end{DUlineblock}
\\
\hline\end{longtable}

\footnotetext[1]{\sphinxAtStartFootnote%
Nem todos os jogos possuem gráficos tilemap decodificados.
}\footnotetext[2]{\sphinxAtStartFootnote%
\textbf{tilemaps} são como pequenos recortes ou pedaços usados para montar uma imagem do jogo.
}

\section{Os menus do MAME}
\label{usingmame/mamemenus:os-menus-do-mame}\label{usingmame/mamemenus::doc}
Caso você inicie o MAME sem nenhum parâmetro na linha de comando ou
rodando ele com o clicar do mouse, um cardápio de opções será exibido,
entre eles a lista de seleção de jogos. Apesar das teclas abaixo
permitirem que você navegue nessa opções, você também pode usar o seu
mouse se desejar.

{[}a fazer: Isso precisa ser expandido URGENTEMENTE. Aguardando respostas
para algumas questões importantes...{]}


\section{Interfaces}
\label{usingmame/frontends:interfaces}\label{usingmame/frontends::doc}\label{usingmame/frontends:frontends}
Existem várias ferramentas de terceiros desenvolvidas para o MAME que
servem para tornar a vida do usuário mais agradável no que tange a
seleção de diferentes sistemas, software, organização da sua lista de
jogos preferidos dentre inúmeras funções extras.

Entre o usuário e o MAME é possível usar ferramentas chamadas
``\emph{Frontends}'' que funcionam como uma interface intermediária. Elas
existem aos milhares e seria inviável tentar listá-las todas aqui.

Já fica o alerta, algumas são gratuitas porém nem todas elas são.
Algumas frontends mais antigas antecedem a fusão de MAME e MESS e não
suportam as novas funcionalidades herdadas do MESS como consoles,
handheld, etc.

Observe que nós da equipe MAME não estamos endossando nenhuma das
interfaces abaixo ou de qualquer outra, elas servem como um bom ponto
de partida já que todas elas são bem conhecidas e gratuitas.

\begin{DUlineblock}{0em}
\item[] QMC2 (multiple platforms)
\item[] Download: \url{http://qmc2.batcom-it.net/}
\item[] 
\item[] IV/Play (Microsoft Windows)
\item[] Download: \url{http://www.mameui.info/}
\item[] 
\item[] EmuLoader (Microsoft Windows)
\item[] Download: \url{http://emuloader.mameworld.info/}
\end{DUlineblock}

A equipe do MAME não oferece qualquer tipo de suporte para qualquer
problema que você venha a ter com elas. Para receber suporte, sugerimos
entrar em contato diretamente com o autor da interface ou buscar ajuda
em qualquer um dos fóruns relacionados com o MAME espalhados pela
Internet.
\clearpage

\section{Sobre ROMs e seus conjuntos}
\label{usingmame/aboutromsets:sobre-roms-e-seus-conjuntos}\label{usingmame/aboutromsets::doc}
O manuseio e atualização de ROMs e seus conjuntos usados no MAME é
provavelmente a maior área de confusão e frustração que os usuários
do MAME enfrentam.
Esta seção tem como objetivo esclarecer muitas das perguntas mais
comuns e abordar detalhes simples que você precisa saber para usar
o MAME de forma mais eficaz.

Vamos começar explicando o que é ROM.


\subsection{O que é uma imagem ROM?}
\label{usingmame/aboutromsets:o-que-e-uma-imagem-rom}
É uma imagem dos dados que estão dentro de um determinado circuito
integrado \footnote[1]{\sphinxAtStartFootnote%
Estes circuitos integrados também são conhecidos pela abreviação
``CI'' (se fala CÊ-Í), assim como é chamado de ``chip'' em Inglês.
(Nota do tradutor)
} na placa-mãe do arcade (ou outro dispositivo eletrônico)
em formato binário.

Para a maioria dos consoles e portáteis, os CIs individuais são
frequentemente (mas nem sempre) mesclados em um único arquivo.
Já as máquinas arcade a coisa é um pouco mais complicada devido ao seu
design, você normalmente precisará de dados encontrados em diferentes
circuitos espalhados pela placa.
Ao agrupar todos os arquivos do Puckman juntos, você obterá um conjunto
de ROMs \footnote[2]{\sphinxAtStartFootnote%
Esse conjunto é chamado de \emph{ROM set} em Inglês.
(Nota do tradutor)
} do jogo Puckman.

Um exemplo de uma imagem ROM seria o arquivo \textbf{pm1\_prg1.6e} que estaria
armazenada em um conjunto de ROM \textbf{Puckman}.


\subsection{Por que ROM e não algum outro nome?}
\label{usingmame/aboutromsets:por-que-rom-e-nao-algum-outro-nome}
ROM é um acrônimo de ``Read-Only Memory'' que significa uma memória que
serve somente para leitura. Alguns dos CIs usados para armazenar dados
não são regraváveis como por exemplo uma PROM, uma vez gravados os dados
se tornam permanentes (contanto que o CI não seja danificado ou
envelheça até a pifar de vez!).
Há outros modelos de CIs como as EPROM que podem ser reprogramados.

Rom dump ou dump é o processo de se extrair o conteúdo existente de
dentro um circuito integrado onde esteja armazenado o programa do que
quer que seja, logo, o nome deste conteúdo tornou-se conhecido como uma
``imagem ROM'' ou apenas ``ROM'' para simplificar.


\subsection{Pais, Clones, Divisão e Mesclagem}
\label{usingmame/aboutromsets:pais-clones-divisao-e-mesclagem}
Enquanto os desenvolvedores do MAME recebiam a sua terceira ou quarta
revisão do Pac Man, com correções de bugs e outras alterações no código
original, eles rapidamente descobriram que quase todas as placas
e integrados quem continham os dados das ROMs eram idênticas as versões
anteriormente copiadas. Para economizar espaço, o MAME foi ajustado para
usar um conjunto de sistema hierárquico de família, onde o \textbf{pai} seria
a ROM principal e seus derivados viriam logo abaixo sendo chamado de
\textbf{filho}.

A última revisão corrigida de um determinado conjunto (World) será
definido como pai dessa família, mas nem sempre.
Todos os conjuntos que em geral usarem os mesmos CIs (por exemplo,
a versão japonesa do Puckman e a versão USA/World do Pac Man) serão
definidos como clones, pois conterá apenas os arquivos que forem
diferentes se comparadas ao conjunto pai.

Caso o usuário tente rodar um jogo clone ou seus conjuntos subsequentes,
sem antes ter o jogo pai disponível, o usuário será informado do
problema. Usando o exemplo anterior, ao tentar jogar a versão Americana
do Pac Man sem antes ter o conjunto pai \textbf{PUCKMAN.ZIP}, aparecerá uma
mensagem de erro informando quais os arquivos estão faltando.

Agora vamos adicionar as últimas peças desse quebra-cabeças:
\textbf{não-mesclados} (\emph{non-merged}), \textbf{dividido} (\emph{split}), e conjuntos
\textbf{mesclados} (\emph{merged}).

O MAME é extremamente versátil sobre onde dados da ROM estão localizados
e é muito inteligente para identificar o que ele precisa. Isso nos
permite fazer algumas mágicas relacionada com a maneira com o qual nós
armazenamos estes conjuntos de ROMs, visando a economia de espaço.

Um \textbf{conjunto não-mesclado} (\emph{non-merged set}) é aquele que contém tudo
o que for necessário para que um determinado jogo rode armazenado dentro
de um único arquivo ZIP. Normalmente isso é ineficiente, muito espaço é
perdido, mas é o melhor caminho a seguir se você tem poucos jogos e
deseja que tudo seja simples e fácil de trabalhar.
Para a maioria dos usuários, este é um modo na qual nós não
recomendamos.

Um \textbf{conjunto dividido} (\emph{split set}) é aquele em que o conjunto pai
contém todos os arquivos de dados que ele precisa, e os conjuntos clones
contêm \emph{apenas} o que foi alterado em comparação com o conjunto pai.
Isso economiza espaço, mas não é tão eficiente quanto um conjunto
mesclado.

Um \textbf{conjunto mesclado} (\emph{merged set}) de ROMs, contêm os arquivos do
conjunto pai e um ou mais conjuntos de clones armazenados dentro de um
mesmo arquivo. Caso o conjunto do Puckman, Midway Pac-Man (USA) seja
combinado juntamente com várias versões piratas (bootleg) em um único
arquivo chamado \textbf{PUCKMAN.ZIP} por exemplo, o resultado final é o
chamado \emph{merged set}. Um conjunto mesclado completo com o pai e todos
os clones usam menos espaço do que um conjunto dividido (\emph{split set}).

Estes são princípios básicos de conjuntos, porém existem dois outros
tipos de conjunto que serão usados no MAME de tempos em tempos.

Primeiro, é o \textbf{conjunto de BIOS} (\emph{BIOS set}).
Algumas máquinas arcade compartilhavam uma plataforma de hardware em
comum, como o hardware de arcade Neo-Geo. Como a placa principal tinham
todos os dados necessários para iniciar e realizar seu auto-teste do
hardware antes de seguir para o cartucho de jogos. Aliás, não é
apropriado colocar os dados do jogo para iniciar junto com a BIOS.
Em vez disso, ele é armazenado separadamente como uma imagem BIOS para o
próprio sistema (por exemplo, \textbf{NEOGEO.ZIP} para jogos Neo-Geo)

Segundo, o \textbf{conjunto de dispositivos} (\emph{device set}).
Frequentemente, os fabricantes de arcade reutilizavam várias partes de
seus projetos várias vezes a fim de economizar tempo e dinheiro. Alguns
desses circuitos menores reapareceriam em novas placas desde que
tivessem um mínimo em comum com as placas anteriores lançadas e que
usavam o mesmo circuito, então você não poderia simplesmente tê-los
compartilhando os dados do circuito/ROM por meio de uma relação normal
de pai/clone. Em vez disso, esses desenhos reutilizados e os dados da
ROM são categorizados como um dispositivo \emph{Device}, com os dados
armazenados como um conjunto de dispositivos \emph{Device set}. Por exemplo,
a Namco utilizou um circuito integrado customizado de entrada e saída
(I/O) \emph{Namco 51xx} para para lidar com os comandos do joystick e as
chaves DIP para o jogo Galaga, assim como para outros jogos, você também
precisará do conjunto de dispositivos armazenado no arquivo
\textbf{NAMCO51.ZIP} e assim também para outros jogos que precisem dele.


\subsection{Solucionando problemas dos seus conjuntos de ROMs e um pouco de história}
\label{usingmame/aboutromsets:solucionando-problemas-dos-seus-conjuntos-de-roms-e-um-pouco-de-historia}
A frustração de muitos usuários do MAME podem estar relacionadas com
mudanças e modificações, julgadas como desnecessárias por muitos, que os
arquivos ROM sofrem ao longo do tempo e que parece que nossa intenção é
fazer da vida de vocês mais difícil. Entender a origem dessas mudanças e
por quê elas são necessárias ajudará você a evitar ser pego de surpresa
quando essas mudanças acontecem e saber o que precisa ser feito para
manter os seus conjuntos atualizados.

Uma grande quantidade de ROMs e seus conjuntos existiam antes da
emulação. Esses conjuntos iniciais foram criados por proprietários das
casas de arcades e usados para reparar as placas quebradas que não
funcionavam mais, e para a substituição de componentes/peças/integrados
danificados. Infelizmente, alguns destes conjuntos não continham todas
as informações necessárias, especialmente as mais críticas. Muitas das
imagens extraídas inicialmente continham falhas, erros, como
por exemplo, a falta de informação responsável pela paleta de cores da
tela.

Os primeiros emuladores simulavam artificialmente
esses dados de cores que faltavam, de maneira mais próxima possível mas
nunca correta, até descobrirem os dados que faltavam em outros circuitos
integrados. Isso resultou na necessidade de voltar, extrair os dados
ausentes e atualizar os conjuntos antigos com novos arquivos conforme
fosse necessário.

Não demoraria muito para descobrir que muitos dos conjuntos existentes
tinham dados ruins para um ou mais circuitos integrados. Os dados desses
também precisariam ser extraídos novamente, talvez de uma máquina
diferente, e muitos outros conjuntos precisariam de revisões completas.

Ocasionalmente, alguns jogos seriam descobertos com sua documentação
feita de forma totalmente incorreta. Alguns jogos considerados originais
eram na verdade, cópias piratas de fabricantes desconhecidos. Alguns
jogos que foram considerados como ``piratas'', eram na verdade a versão
original do jogo. Os dados de alguns jogos estavam bagunçados, de forma
que não se sabia de qual região a placa era como por exemplo, jogos
World misturado com Japão) o que exigiu também ajustes internos e a
correção dos nomes.

Mesmo agora, acontecem achados milagrosos e ocasionais que mudam a nossa
compreensão desses jogos. Como é fundamental que uma documentação seja
precisa para registrar a história dos arcades, o MAME mudará o nome dos
conjuntos sempre que for necessário, visando a precisão e mantendo as
coisas da maneira mais correta possível sempre no limite do conhecimento
que a equipe tem a cada novo lançamento do MAME.

Isso resulta em uma compatibilidade muito irregular para os conjuntos de
ROMs que deixam de funcionar nas versões mais antigas do MAME.
Alguns jogos podem não ter mudado muito entre 20 ou 30 novas versões
do MAME, assim como outros podem ter mudado drasticamente entre as novas
versões lançadas.

Se você encontrar problemas com um determinado conjunto que não funciona
mais, há várias coisas a serem verificadas:
\begin{itemize}
\item {} 
Você está tentando rodar um conjunto de ROMs destinado à uma versão
mais antiga do MAME?

\item {} 
Você tem o conjunto de BIOS necessários ou a ROM dos dispositivos?

\item {} 
Seria este um clone que precisaria ter o pai também?

\end{itemize}

O MAME sempre informará quais os arquivos estão faltando, dentro de
quais conjuntos e onde eles foram procurados.


\subsection{ROMs e CHDs}
\label{usingmame/aboutromsets:roms-e-chds}
Os dados do CI que contém a ROM tendem a ser relativamente pequenos
e são carregados sem maiores problemas na memória do sistema.
Alguns jogos também usavam mídias adicionais de armazenamento, como
discos rígidos, CD-ROMs, DVDs e Laserdiscs. Esses meios de armazenamento
são, por questões técnicas diversas, inadequados para serem armazenados
da mesma forma que os dados da ROM e em alguns casos não caberão por
inteiro na memória.

Assim, um novo formato foi criado para eles, sendo armazenados num
arquivo CHD. \textbf{Compressed Hunks of Data} ou numa tradução literal seria
\textbf{Pedações de Dados Comprimidos} ou CHD para simplificar.
São projetados especificamente em torno das necessidades da mídia de
armazenamento em massa. Alguns jogos de arcade, consoles e PCs
precisarão de um arquivo CHD para rodar.

Como os CHDs já estão comprimidos, eles \textbf{NÃO} devem ser armazenados
dentro de um arquivo ZIP ou 7Z como você faria com os conjuntos de ROM.
\clearpage

\section{Problemas comuns e algumas perguntas frequentes}
\label{usingmame/commonissues:problemas-comuns-e-algumas-perguntas-frequentes}\label{usingmame/commonissues::doc}
\textbf{Aviso Legal: As informações a seguir não tem qualquer fundamento
jurídico e tão pouco foi escrito por um advogado.}
\begin{enumerate}
\item {} 
{\hyperref[usingmame/commonissues:rapid\string-coins]{\sphinxcrossref{\DUrole{std,std-ref}{Por que o meu jogo mostra uma tela de erro quando eu insiro moedas rapidamente?}}}}

\item {} 
{\hyperref[usingmame/commonissues:broken\string-package]{\sphinxcrossref{\DUrole{std,std-ref}{Por quê o meu pacote MAME não oficial (o EmuCR ou qualquer outro por exemplo) não funciona direito? Por quê a minha atualização oficial está quebrada?}}}}

\item {} 
{\hyperref[usingmame/commonissues:faster\string-if\string-x]{\sphinxcrossref{\DUrole{std,std-ref}{Por quê o MAME suporta jogos de console e terminais burros? Não seria mais rápido se o MAME suportasse apenas jogos de arcade? Não usaria menos memória RAM? Não faria com que o MAME ficasse mais rápido por causa de A, B ou C?}}}}

\item {} 
{\hyperref[usingmame/commonissues:neogeo\string-broken]{\sphinxcrossref{\DUrole{std,std-ref}{Por quê a minha ROM de Neo-Geo não funcionam mais? Como eu faço para que o jogo Humble Bundle volte a funcionar?}}}}

\item {} 
{\hyperref[usingmame/commonissues:sega\string-sgmdc]{\sphinxcrossref{\DUrole{std,std-ref}{Como posso usar a coleção para a Steam do Mega Drive Classics collection do Sega Genesis com o MAME?}}}}

\item {} 
{\hyperref[usingmame/commonissues:missing\string-roms]{\sphinxcrossref{\DUrole{std,std-ref}{Por quê o MAME alega que ``faltam arquivos'' sendo que eu tenho essas ROMs?}}}}

\item {} 
{\hyperref[usingmame/commonissues:rom\string-verify]{\sphinxcrossref{\DUrole{std,std-ref}{Como posso ter certeza que tenho as ROMs certas?}}}}

\item {} 
{\hyperref[usingmame/commonissues:parent\string-sets]{\sphinxcrossref{\DUrole{std,std-ref}{Por que alguns jogos têm a versão Americana como a principal, outras têm a Japonesa e outros a versão  Mundo (World)?}}}}

\item {} 
{\hyperref[usingmame/commonissues:legal\string-roms]{\sphinxcrossref{\DUrole{std,std-ref}{Como faço para obter legalmente as ROMs ou as imagens de disco para poder rodar no MAME?}}}}

\item {} 
{\hyperref[usingmame/commonissues:roms\string-grey]{\sphinxcrossref{\DUrole{std,std-ref}{A cópia legal das ROMs não esbarram num possível limiar jurídico?}}}}

\item {} 
{\hyperref[usingmame/commonissues:abandonware]{\sphinxcrossref{\DUrole{std,std-ref}{As ROMs dos jogos não podem ser consideradas abandonadas com o tempo (abandonware)?}}}}

\item {} 
{\hyperref[usingmame/commonissues:old\string-sets]{\sphinxcrossref{\DUrole{std,std-ref}{Eu tinha ROMs que funcionavam com uma versão antiga do MAME e agora não funcionam mais. O que aconteceu?}}}}

\item {} 
{\hyperref[usingmame/commonissues:ebay\string-cabs]{\sphinxcrossref{\DUrole{std,std-ref}{E aqueles gabinetes de arcade vendidos no Mercado Livre, OLX e outros lugares que vêm com todas as ROMs?}}}}

\item {} 
{\hyperref[usingmame/commonissues:rom\string-dvds]{\sphinxcrossref{\DUrole{std,std-ref}{E aqueles caras que gravam DVDs com ROMs e cobram apenas o preço da mídia?}}}}

\item {} 
{\hyperref[usingmame/commonissues:dmca\string-exemption]{\sphinxcrossref{\DUrole{std,std-ref}{Mas não há uma isenção especial do DMCA que torne a cópia de uma ROM legal?}}}}

\item {} 
{\hyperref[usingmame/commonissues:hours]{\sphinxcrossref{\DUrole{std,std-ref}{Há algum problema se eu baixar a ROM e ``experimentar'' por 24 horas?}}}}

\item {} 
{\hyperref[usingmame/commonissues:commercial\string-use]{\sphinxcrossref{\DUrole{std,std-ref}{E se eu comprar um gabinete com ROMs legalizadas, posso disponibilizá-lo em um local público para que eu possa ganhar dinheiro?}}}}

\item {} 
{\hyperref[usingmame/commonissues:ultracade]{\sphinxcrossref{\DUrole{std,std-ref}{Mas eu já vi gabinetes do Ultracade e Global VR Classics montados em lugares públicos? Por quê eles podem?}}}}

\item {} 
{\hyperref[usingmame/commonissues:blackscreen\string-directx]{\sphinxcrossref{\DUrole{std,std-ref}{AJUDA! Eu estou tendo tela preta ou uma mensagem de erro relacionada com o DirectX no Windows!}}}}

\item {} 
{\hyperref[usingmame/commonissues:controllerissues]{\sphinxcrossref{\DUrole{std,std-ref}{Eu tenho um controlador que não quer funcionar com a versão nativa do MAME no Windows, o que posso fazer?}}}}

\item {} 
{\hyperref[usingmame/commonissues:externalopl]{\sphinxcrossref{\DUrole{std,std-ref}{O que aconteceu com o suporte do MAME para placas de som externas com o OPL2 integrado?}}}}

\end{enumerate}


\subsection{Por que o meu jogo mostra uma tela de erro quando eu insiro moedas rapidamente?}
\label{usingmame/commonissues:rapid-coins}\label{usingmame/commonissues:por-que-o-meu-jogo-mostra-uma-tela-de-erro-quando-eu-insiro-moedas-rapidamente}
Isso não é um bug do MAME.
No hardware de arcade original, você simplesmente não poderia inserir
moedas tão rápido quanto você faz apertando um botão. A única maneira
que você pode obter crédito nesse ritmo é se o hardware do mecanismo de
moedas estiver com defeito ou se você estivesse fisicamente tentando
enganar o mecanismo de moeda.

Em ambos os casos, o jogo apresentaria um erro para que o responsável
investigasse a situação, evitando que algum espertinho tirasse vantagem
em cima daquele que trabalha duro para conquistar seu dinheiro.
Mantenha um ritmo lento de inserção de moedas e para que este erro não
ocorra.


\subsection{Por quê o meu pacote MAME não oficial (o EmuCR ou qualquer outro por exemplo) não funciona direito? Por quê a minha atualização oficial está quebrada?}
\label{usingmame/commonissues:broken-package}\label{usingmame/commonissues:por-que-o-meu-pacote-mame-nao-oficial-o-emucr-ou-qualquer-outro-por-exemplo-nao-funciona-direito-por-que-a-minha-atualizacao-oficial-esta-quebrada}
Em muitos casos, as alterações de vários subsistemas tais como plug-ins
Lua, HLSL ou BGFX vem como atualizações para diversos arquivos
diferentes assim como o código fonte principal do MAME.
Infelizmente as versões que vem de terceiros podem vir como apenas um
executável principal do MAME ou com arquivos externos desatualizados,
que podem quebrar a relação entre estes arquivos externos e o código
fonte principal do MAME. Apesar das repetidas tentativas de entrar em
contato com alguns destes terceiros para alertá-los, estes insistem em
distribuir um MAME quebrado e sem as atualizações.

Como não temos qualquer controle sobre como estes terceiros distribuem
essas versões, tudo o que podemos fazer para sites como EmuCR é informar
que não fornecemos suporte para programas que nós não compilamos.
Compile o seu próprio MAME ou use um dos pacotes oficialmente
distribuídos por nós.

Você também pode acabar tendo este problema caso você não tenha
atualizado o conteúdo das pastas HLSL e BGFX com as últimas versões
oficiais do MAME.


\subsection{Por quê o MAME suporta jogos de console e terminais burros? Não seria mais rápido se o MAME suportasse apenas jogos de arcade? Não usaria menos memória RAM? Não faria com que o MAME ficasse mais rápido por causa de A, B ou C?}
\label{usingmame/commonissues:faster-if-x}\label{usingmame/commonissues:por-que-o-mame-suporta-jogos-de-console-e-terminais-burros-nao-seria-mais-rapido-se-o-mame-suportasse-apenas-jogos-de-arcade-nao-usaria-menos-memoria-ram-nao-faria-com-que-o-mame-ficasse-mais-rapido-por-causa-de-a-b-ou-c}
Este é um equívoco comum.
A velocidade da emulação não é regida pelo tamanho final MAME, apenas as
partes mais ativamente usadas são carregadas na memória quando for
necessário.

Para o MAME os dispositivos adicionais são uma coisa boa pois nos
permite realizar testes de estresse em seções dos vários núcleos de CPU
e outras partes da emulação que normalmente não veem uma utilização mais
pesada. Enquanto um computador e uma máquina de arcade podem usar
exatamente o mesmo CPU, a maneira como eles usam este CPU pode diferir
drasticamente.

Nenhuma parte do MAME é descartável, independente de qual seja.
O princípio que o MAME defende que é a preservação e a documentação,
sejam as máquinas de vídeo poker quanto os arcades, não importa.
O MAME é open source, muitas coisas já foram abordadas da melhor maneira
possível, caso você seja um programador habilidoso, há sempre espaço
para melhorias e elas são sempre bem vindas.


\subsection{Por quê a minha ROM de Neo-Geo não funcionam mais? Como eu faço para que o jogo Humble Bundle volte a funcionar?}
\label{usingmame/commonissues:neogeo-broken}\label{usingmame/commonissues:por-que-a-minha-rom-de-neo-geo-nao-funcionam-mais-como-eu-faco-para-que-o-jogo-humble-bundle-volte-a-funcionar}
Recentemente a BIOS do Neo-Geo foi atualizada para adicionar uma nova
versão da BIOS Universal. Isso começou entre as versões 0.171 e 0.172 do
MAME que resultou em um erro ao tentar carregar qualquer jogo de
Neo-Geo com um conjunto \textbf{neogeo.zip} desatualizado.

Isso também afeta o conjunto de pacote do jogo Humble Bundle:
os jogos em si estão corretos e atualizados a partir da versão 0.173 do
MAME (e provavelmente continuará assim) no entanto você mesmo terá que
atualizar estes arquivos que estão dentro dos pacotes .ZIP.
No entanto, o conjunto de BIOS do Neo-Geo (\textbf{neogeo.zip}) incluído no
pacote do jogo Humble Bundle está incompleto até a versão 0.172 do MAME.

Sugerimos que você entre em contato com o fornecedor dos seus jogos
(Humble Bundle e DotEmu) e peça para eles atualizarem o jogo para a
versão mais recente. Se muita gente pedir de forma gentil, pode ser que
eles atualizem para você.


\subsection{Como posso usar a coleção para a Steam do Mega Drive Classics collection do Sega Genesis com o MAME?}
\label{usingmame/commonissues:sega-sgmdc}\label{usingmame/commonissues:como-posso-usar-a-colecao-para-a-steam-do-mega-drive-classics-collection-do-sega-genesis-com-o-mame}
A partir da atualização de Abril de 2016, todas as imagens ROM incluídas
no conjunto são agora 100\% compatíveis com o MAME e outros emuladores
\emph{Genesis/Mega Drive}. As ROMs estão guardadas na pasta
\textbf{steamapps\textbackslash{}Sega Classics\textbackslash{}uncompressed ROMs} como uma série de
extensões em formatos de imagem do tipo \emph{.68K} e \emph{.SGD}, que podem ser
carregadas diretamente no MAME. Os manuais em PDF para os jogos podem
também serem encontrados na pasta \textbf{steamapps\textbackslash{}Sega Classics\textbackslash{}manuals}.


\subsection{Por quê o MAME alega que ``faltam arquivos'' sendo que eu tenho essas ROMs?}
\label{usingmame/commonissues:missing-roms}\label{usingmame/commonissues:por-que-o-mame-alega-que-faltam-arquivos-sendo-que-eu-tenho-essas-roms}
Pode ser causado por várias razões:
\begin{itemize}
\item {} 
Não é incomum as ROMs de um jogo mudarem entre as novas versões do
MAME. Por quê isso aconteceria?
Muitas vezes é feita uma extração melhor do ci que contém a ROM ou
então foi feita uma extração mais completa hoje e que não foi possível
na época, ou até mesmo foi feito uma nova extração para corrigir os
erros detectados nas ROMs anteriores. As primeiras versões do MAME
não eram tão chatas sobre esta questão, porém as versões mais recentes
são.
Além disso, podem haver mais características de um jogo emulado em uma
versão posterior que não havia na versão anterior, o que exige a
execução de mais códigos dentro do MAME para rodar essa nova ROM.

\item {} 
Você pode descobrir que alguns jogos precisam de arquivos CHD.
Um arquivo CHD é uma representação comprimida de uma imagem de um jogo
em disco rígido, CD-ROM ou laserdisc, geralmente não é incluído como
parte das ROMs de um jogo. No entanto, assim como na maioria dos
casos, esses arquivos são necessários para rodar o jogo, e o MAME vai
reclamar se eles não puderem ser encontrados.

\item {} 
Alguns jogos como Neo-Geo, Playchoice-10, Convertible Video System,
Deco Cassette, MegaTech, MegaPlay, ST-V Titan e outros, precisam das
suas ROMs e do conjunto de BIOS. As ROMs da BIOS geralmente contêm um
código da ROM que é usado para inicializar a máquina, o código faz
lista dos jogos em sistema multijogos e o código comum a todos os
jogos no referido sistema. As ROMs da BIOS devem estar nomeadas
corretamente e comprimida em formato .ZIP dentro da pasta ROMs.

\item {} 
Versões mais antigas do MAME precisavam de tabelas de descriptografia
para que fosse possível emular jogos da Capcom Play System 2
(também conhecido como jogos CPS2). Que foram criados pela equipe
CPS2Shock.

\item {} 
Alguns jogos no MAME são considerados ``Clones'' de outros jogos.
Isto é, o jogo em questão é simplesmente uma versão alternativa do
mesmo jogo. As versões alternativas de alguns jogos incluem as versões
com texto em outros idiomas, com diferentes datas de direito autoral,
versões posteriores ou atualizações, versões piratas, etc.
Os jogos ``clonados'' muitas vezes se sobrepõem algum código da ROM do
jogo, como se fosse a versão original. Para verificar se você tem
algum tipo de jogo ``clonado'' digite o comando
``\textbf{MAME -listclones}''. Para rodar um ``jogo clonado'' basta colocar a
ROM pai dentro da pasta ROMs (sempre zipada).

\end{itemize}


\subsection{Como posso ter certeza que tenho as ROMs certas?}
\label{usingmame/commonissues:como-posso-ter-certeza-que-tenho-as-roms-certas}\label{usingmame/commonissues:rom-verify}
O MAME verifica se você tem as ROMs corretas antes de iniciar a
emulação. Caso você vir alguma mensagem de erro, as suas ROMs não são
aquelas testadas e que funcionam corretamente com o MAME. Você precisará
obter as ROMs corretas através de meios legais.

Se você tiver vários jogos e quiser verificar se eles são compatíveis
com a versão atual do MAME, você poderá usar a opção \emph{-verifyroms}.

\begin{DUlineblock}{0em}
\item[] Por Exemplo:
\item[] \textbf{mame -verifyroms robby}
\item[] ...verifica as suas ROMs para o jogo com nome \emph{Robby Roto} e exibe os resultados na tela.
\end{DUlineblock}

\begin{DUlineblock}{0em}
\item[] \textbf{mame -verifyroms * \textgreater{}verify.txt}
\item[] ...verifica a autenticidade de TODAS as ROMs dentro do seu diretório ROMs e grava os resultados dentro de um arquivo de texto chamado \emph{verify.txt}.
\end{DUlineblock}


\subsection{Por que alguns jogos têm a versão Americana como a principal, outras têm a Japonesa e outros a versão  Mundo (World)?}
\label{usingmame/commonissues:por-que-alguns-jogos-tem-a-versao-americana-como-a-principal-outras-tem-a-japonesa-e-outros-a-versao-mundo-world}\label{usingmame/commonissues:parent-sets}
Embora essa regra nem sempre seja verdadeira, normalmente é a maneira na
qual estes conjuntos são organizados. A prioridade normal é usar o
conjunto \textbf{Mundo}, caso esteja disponível, \textbf{Americana}, se não
existir nenhum outro conjunto mundial em Inglês e \textbf{japonês} ou uma
outra região qualquer.

As exceções são aplicadas quando os conjuntos Americanos e Mundo têm
censuras ou alterações significativas da sua versão original.
Por exemplo, o jogo Gals Panic (do conjunto \textbf{galsnew}) usa a versão
Americana como pai porque têm recursos adicionais se comparado com a
versão de exportação mundial (do conjunto \textbf{galsnewa}). Esses são
recursos opcionais censurados, como uma opção de layout de controle
adicional (que não usa nenhum botão) e clipes de voz no idioma Inglês.

Uma outra exceção seria para os jogos que foram licenciados por
terceiros para que fossem exportados e lançados lá fora.
O Pac Man, por exemplo, foi publicado pela Midway nos EUA, embora tenha
sido criado pela Namco do Japão. Como resultado, o conjunto pai é o
conjunto japonês \textbf{puckman}, que mantém os direitos autorais da Namco.

Por último, um desenvolvedor que adiciona um novo conjunto, este pode
optar por usar qualquer esquema de hierarquia e de nomenclatura que
deseje e não fica restrito às regras acima.
No entanto, a maioria seguem essas diretrizes.


\subsection{Como faço para obter legalmente as ROMs ou as imagens de disco para poder rodar no MAME?}
\label{usingmame/commonissues:como-faco-para-obter-legalmente-as-roms-ou-as-imagens-de-disco-para-poder-rodar-no-mame}\label{usingmame/commonissues:legal-roms}
As principais opções são:
\begin{itemize}
\item {} 
Você pode obter uma licença para eles, comprando uma através de um
distribuidor ou fornecedor que tenha a devida autoridade para fazê-lo.

\item {} 
Você pode baixar um dos conjuntos de ROMs que foram disponibilizados
gratuitamente para o público em geral e para o uso não comercial do
mesmo.

\item {} 
Você pode comprar uma PCB de arcade e extrair as ROMs ou discos você
mesmo e usar com o MAME.

\end{itemize}

No mais, você está por sua própria conta e risco.


\subsection{A cópia legal das ROMs não esbarram num possível limiar jurídico?}
\label{usingmame/commonissues:a-copia-legal-das-roms-nao-esbarram-num-possivel-limiar-juridico}\label{usingmame/commonissues:roms-grey}
Não, de forma alguma.
Você não tem permissão para fazer cópias de software sem a permissão do
proprietário que detém estes direitos. A questão é preto no branco,
mais claro que isso, impossível.


\subsection{As ROMs dos jogos não podem ser consideradas abandonadas com o tempo (abandonware)?}
\label{usingmame/commonissues:as-roms-dos-jogos-nao-podem-ser-consideradas-abandonadas-com-o-tempo-abandonware}\label{usingmame/commonissues:abandonware}
Não.

Até mesmo as empresas que faliram tiveram seus ativos comprados por
alguém e esse alguém hoje é o detentor legal desses direitos autorais.


\subsection{Eu tinha ROMs que funcionavam com uma versão antiga do MAME e agora não funcionam mais. O que aconteceu?}
\label{usingmame/commonissues:old-sets}\label{usingmame/commonissues:eu-tinha-roms-que-funcionavam-com-uma-versao-antiga-do-mame-e-agora-nao-funcionam-mais-o-que-aconteceu}
O MAME com o passar do tempo aperfeiçoa a emulação dos jogos antigos,
mesmo quando não pareça óbvio para os usuários. Outras vezes, visando
melhorar a emulação para que o jogo funcione corretamente, é necessário
obter mais dados do jogo original. Dados estes que foram negligenciados
por algum motivo qualquer, às vezes simplesmente não foi possível
extrair o conteúdo do CI de forma apropriada (para se ter uma ideia,
a técnica de ``\emph{decapping}'' \footnote[1]{\sphinxAtStartFootnote%
Decapping é um processo feito no CI para expor seu núcleo, é
possível ver algumas fotos desse processo no blog do \href{http://caps0ff.blogspot.com}{CAPS0ff}. (Nota do tradutor)
} dos circuitos integrados só se tornou
viável recentemente, facilitando muito para aqueles que colaboram com o
projeto e não tem os mesmos recursos que um laboratórios de ponta).
Em outros casos, é muito mais simples.
Mais conjuntos de um determinado jogo foram extraídos e organizados cada
um com a sua versão, região, modelo, tipo, etc.


\subsection{E aqueles gabinetes de arcade vendidos no Mercado Livre, OLX e outros lugares que vêm com todas as ROMs?}
\label{usingmame/commonissues:ebay-cabs}\label{usingmame/commonissues:e-aqueles-gabinetes-de-arcade-vendidos-no-mercado-livre-olx-e-outros-lugares-que-vem-com-todas-as-roms}
Ele poderá estar cometendo um crime caso o vendedor não tenha uma
licença adequada ou permissão para fazer a venda, sem falar nas
devidas permissões legais e licenças para vender um gabinete junto com
essas ROMs. Ele só poderá vendê-las junto com o gabinete quando ele
tiver uma licença ou permissão para vender as ROMs em seu nome, vindas
de um distribuidor ou fornecedor licenciado para tanto.
Caso contrário, estamos falando de pirataria de software.

E para incluir uma versão do MAME nestes gabinetes que eles estão
vendendo junto com as ROMs, seria necessário também assinar um contrato
conosco para obter uma versão licenciada do MAME para rodar apenas as
ROMs que ele adquiriu de forma legal e mais nada.


\subsection{E aqueles caras que gravam DVDs com ROMs e cobram apenas o preço da mídia?}
\label{usingmame/commonissues:rom-dvds}\label{usingmame/commonissues:e-aqueles-caras-que-gravam-dvds-com-roms-e-cobram-apenas-o-preco-da-midia}
O que eles fazem é tão ilegal quanto vender as ROMs de forma direta ou
junto com os gabinetes. Enquanto alguém possuir os direitos autorais
destes jogos, fazer cópias ilegais da maneira que for e
disponibilizá-las para venda é crime e ponto final. Caso alguém vá para
a internet vender cópias piratas do último álbum de um artista qualquer
a preço de banana cobrando apenas o custo da mídia, você acha que eles
conseguiriam sair impunes dessa?

Pior ainda, muitas dessas pessoas gostam de afirmar que elas estão
ajudando o projeto. Para a equipe do MAME, essas pessoas só criam mais
problemas. Nós não estamos associados a essas pessoas de forma
alguma, independentemente de quão ``oficiais'' elas se achem.
Ao comprar pirataria você está incentivando os criminosos a continuar
lucrando com a venda de software pirata na qual eles não possuem direito
algum.

\textbf{Qualquer pessoa que use o nome do MAME e/ou seu logotipo para vender
esses produtos, também está violando direitos autorais e a marca
registrada do MAME.}


\subsection{Mas não há uma isenção especial do DMCA que torne a cópia de uma ROM legal?}
\label{usingmame/commonissues:dmca-exemption}\label{usingmame/commonissues:mas-nao-ha-uma-isencao-especial-do-dmca-que-torne-a-copia-de-uma-rom-legal}
Não.

Você entendeu essas isenções de forma errada. A isenção permite que as
pessoas façam a engenharia reversa para quebrar a criptografia que
protege a cópia de programas de computador obsoletos.

Ela permite que se faça isso para descobrir como esses programas
obsoletos funcionavam, não sendo ilegal de acordo com a DMCA.
Isso nada tem haver com legalidade de violar os direitos autorais dos
programas de computador alheios, que é o que você faz caso faça cópias
ilegais de ROMs.

O DMCA é uma lei Americana, é um acrônimo para \textbf{Digital Millennium
Copyright Act} ou numa tradução literal ficaria ``Lei dos Direitos
Autorais do Milênio Digital''.
No Brasil essa lei não tem validade alguma e tão pouco existe qualquer
lei equivalente no Brasil.


\subsection{Há algum problema se eu baixar a ROM e ``experimentar'' por 24 horas?}
\label{usingmame/commonissues:hours}\label{usingmame/commonissues:ha-algum-problema-se-eu-baixar-a-rom-e-experimentar-por-24-horas}
Esta é uma lenda urbana criada por pessoas que distribuem ROMs para
download em seus sites, tentando justificar o fato deles estarem
infringindo a lei. Não existe nada disso em qualquer lei de direitos
autorais nos EUA e muito menos no Brasil ou em qualquer outro lugar.


\subsection{E se eu comprar um gabinete com ROMs legalizadas, posso disponibilizá-lo em um local público para que eu possa ganhar dinheiro?}
\label{usingmame/commonissues:commercial-use}\label{usingmame/commonissues:e-se-eu-comprar-um-gabinete-com-roms-legalizadas-posso-disponibiliza-lo-em-um-local-publico-para-que-eu-possa-ganhar-dinheiro}
Geralmente não.

Tais ROMs são licenciadas apenas para fins pessoais e de uso não
comercial a não ser que você tenha adquirido uma licença que diga o
contrário e permita tal uso.


\subsection{Mas eu já vi gabinetes do Ultracade e Global VR Classics montados em lugares públicos? Por quê eles podem?}
\label{usingmame/commonissues:ultracade}\label{usingmame/commonissues:mas-eu-ja-vi-gabinetes-do-ultracade-e-global-vr-classics-montados-em-lugares-publicos-por-que-eles-podem}
O Ultracade tinha dois produtos distintos. O produto Ultracade é uma
máquina comercial com licenças comerciais para uso dos jogos.
Estas máquinas foram concebidas para serem colocadas em local público
e gerar renda, como as máquinas de arcade tradicionais. Seus outros
produtos são a série Arcade Legends, elas possuem uma licença voltada
para uso exclusivo em ambiente particular e residencial.
Desde sua aquisição pela empresa Global VR eles só oferecem o gabinete
Global VR Classics, que equivale ao produto Ultracade anterior.


\subsection{AJUDA! Eu estou tendo tela preta ou uma mensagem de erro relacionada com o DirectX no Windows!}
\label{usingmame/commonissues:ajuda-eu-estou-tendo-tela-preta-ou-uma-mensagem-de-erro-relacionada-com-o-directx-no-windows}\label{usingmame/commonissues:blackscreen-directx}
Possivelmente os arquivos Runtimes do DirectX, estejam faltando ou estão
danificados. Você pode baixar a ferramenta do DirectX mais recente
direto do site da Microsoft no endereço abaixo:
\url{https://www.microsoft.com/pt-br/download/details.aspx?displaylang=en\&id=35}

Informações adicionais para a solução de problemas podem ser encontradas
na página da Microsoft em:
\url{https://support.microsoft.com/pt-br/help/179113/how-to-install-the-latest-version-of-directx}


\subsection{Eu tenho um controlador que não quer funcionar com a versão nativa do MAME no Windows, o que posso fazer?}
\label{usingmame/commonissues:controllerissues}\label{usingmame/commonissues:eu-tenho-um-controlador-que-nao-quer-funcionar-com-a-versao-nativa-do-mame-no-windows-o-que-posso-fazer}
O MAME predefine que lerá de forma direta os dados do(s) joystick(s), do
mouse e do(s) teclado(s) no Windows. Isso funciona com a maioria dos
dispositivos fornecendo resultados mais estáveis. No entanto, alguns
dispositivos precisam da instalação de drivers especiais que podem não
funcionar ou não ser compatível com o MAME.

Tente configurar as opções \textbf{keyboardprovider}, \textbf{mouseprovider} ou
\textbf{joystickprovider} (dependendo de qual tipo de dispositivo de entrada
ele seja) vindo da entrada direta para uma das outras opções como o
dinput ou win32. Consulte {\hyperref[commandline/commandline\string-all:osd\string-commandline\string-options]{\sphinxcrossref{\DUrole{std,std-ref}{Opções relacionadas as informações exibidas na tela (OSD)}}}} para obter
detalhes sobre provedores compatíveis.


\subsection{O que aconteceu com o suporte do MAME para placas de som externas com o OPL2 integrado?}
\label{usingmame/commonissues:o-que-aconteceu-com-o-suporte-do-mame-para-placas-de-som-externas-com-o-opl2-integrado}\label{usingmame/commonissues:externalopl}
O MAME ao invés de emular o OPL2 \footnote[2]{\sphinxAtStartFootnote%
OPL é um acrônimo de ``\emph{FM Operator Type-L}'' ou em uma tradução
livre, \emph{Operador de Modulação em Frequência Tipo L}, o 2 é o
número do modelo. (Nota do tradutor)
}, inicialmente adicionou o suporte
para placas de som com o CI YM3212 da Yamaha em sua versão 0.23.
Na versão nativa do MAME nunca houve apoio a essa funcionalidade e foi
completamente eliminada na versão 0.60 do MAME pois a emulação do OPL2
tornou-se avançada o suficiente para ser a melhor solução para a
maioria dos casos naquela época. E hoje as placas de som atuais e mais
modernas, não vem mais com o YM3212 embutido, tornando-se então a única
solução atualmente.

As versões não oficiais do MAME podem também ter mantido esse suporte
por um período de tempo maior.


\chapter{CONFIGURAÇÕES E OPÇÕES DE LINHA DE COMANDO}
\label{commandline/index:configuracoes-e-opcoes-de-linha-de-comando}\label{commandline/index::doc}

\section{Opções universais de linha de comando}
\label{commandline/commandline-all:opcoes-universais-de-linha-de-comando}\label{commandline/commandline-all::doc}
Esta seção contém todas as opções de configuração disponíveis em todas
as versões compiladas do MAME, SDL e Windows.


\subsection{Comandos e verbos}
\label{commandline/commandline-all:comandos-e-verbos}
Os comandos incluem o nome do executável como o \textbf{mame}, bem como
várias ferramentas incluídas na distribuição do MAME, como por exemplo
o \textbf{romcmp} e o \textbf{srcclean}.

Os verbos são as ações a serem tomadas em conjunto com o comando, por
exemplo, \textbf{mame -validate pacman} onde \emph{mame} é o \emph{comando} \footnote[1]{\sphinxAtStartFootnote%
No nosso idioma o \textbf{mame} seria o programa ou aplicativo e o
que vem depois seria o comando. (Nota do tradutor)
} em si,
\emph{-validate} é o verbo e \emph{pacman} a máquina a ser validada.


\subsection{Conjunto de instruções}
\label{commandline/commandline-all:conjunto-de-instrucoes}
Muitos verbos suportam o uso de um \emph{conjunto de instruções} \footnote[2]{\sphinxAtStartFootnote%
\textbf{Pattern}, segundo o \emph{Oxford Dictionary} significa arranjar
algo de forma repetitiva, seguindo um padrão, uma padronagem.
Tradicionalmente ``\emph{pattern}'' é traduzido como ``\emph{padrão}'' porém
fica claro que não estamos falando de algo igual sendo repetido,
mas de um conjunto de instruções ou um conjunto de comandos em
cadência que está informando ao programa as opções que o usuário
deseja usar. (Nota do tradutor)
}, que
podem ser um sistema ou um nome abreviado do dispositivo (por exemplo,
\textbf{a2600}, \textbf{zorba\_kbd}) ou um conjunto de instruções globais que
correspondam a um dos dois (por exemplo, \textbf{zorba\_*}).

Dependendo do comando com o qual você esteja combinando este conjunto de
instruções, a correspondência dessas combinações podem equiparar um
sistema ou sistemas e dispositivos. É aconselhável colocar aspas em
torno dos seus arranjos para evitar que o seu ambiente tente
interpretá-los de forma independente em relação aos nomes dos arquivos
que desejamos usar (por exemplo, \textbf{mame -validate ``pac*''}).


\subsection{Principais verbos}
\label{commandline/commandline-all:principais-verbos}\phantomsection\label{commandline/commandline-all:mame-commandline-help}
\textbf{-help} / \textbf{-h} / \textbf{-?}
\begin{quote}

Exibe a versão atual do MAME e o aviso de direitos autorais.
\end{quote}
\phantomsection\label{commandline/commandline-all:mame-commandline-validate}
\textbf{-validate} / \textbf{-valid} {[}\textless{}\emph{pattern}\textgreater{}{]}
\begin{quote}

Executa validação interna em um ou mais drivers e dispositivos
no sistema. Execute isso antes de enviar qualquer alterações para
nós visando garantir que você não tenha violado qualquer uma das
regras do sistema principal.

Caso um padrão seja definido, ele validará a correspondência padrão
do sistema em questão, caso contrário, validará todos os sistemas e
dispositivos.
\end{quote}


\subsection{Verbos de configuração}
\label{commandline/commandline-all:verbos-de-configuracao}\phantomsection\label{commandline/commandline-all:mame-commandline-createconfig}
\textbf{-createconfig} / \textbf{-cc}
\begin{quote}

Cria um arquivo mame.ini pré-configurado. Todas as opções de
configuração (não verbos) descritos abaixo podem ser permanentemente
alterados, basta editar este arquivo de configuração.
\end{quote}
\phantomsection\label{commandline/commandline-all:mame-commandline-showconfig}
\textbf{-showconfig} / \textbf{-sc}
\begin{quote}

Exibe as configurações atualmente usadas. Caso você direcione isso
para um arquivo, você também pode utilizá-lo como um arquivo INI,
como mostra o exemplo abaixo:
\begin{quote}

\textbf{mame -showconfig \textgreater{}mame.ini}
\end{quote}

É o mesmo que \textbf{-createconfig}.
\end{quote}
\phantomsection\label{commandline/commandline-all:mame-commandline-showusage}
\textbf{-showusage} / \textbf{-su}
\begin{quote}

Exibe um breve resumo de todas as opções da linha de comando.
Para as opções que não são mencionados aqui, o breve resumo dado por
``\emph{mame -showusage}'' geralmente é suficiente para a maioria das
pessoas.
\end{quote}


\subsection{Verbos frontend}
\label{commandline/commandline-all:verbos-frontend}
É predefinido que todos os verbos ``\textbf{-list}'' abaixo escrevam
informações na tela. Se você deseja gravar a informação em um arquivo de
texto, adicione isto ao final do seu comando:
\begin{quote}

\textbf{\textgreater{} nome do arquivo}
\end{quote}

Onde `\emph{nome do arquivo}` é o caminho e o nome do arquivo de texto (por
exemplo, \emph{lista.txt}).
Exemplo:
\begin{quote}

Isso cria (ou sobrescreve se já existir) o arquivo \sphinxcode{lista.txt} e
completa o arquivo com os resultados de \textbf{-listcrc puckman}.
Em outras palavras, a lista de cada ROM usada em Puckman e o CRC
para essa ROM é gravada nesse arquivo.
\end{quote}
\phantomsection\label{commandline/commandline-all:mame-commandline-listxml}
\textbf{-listxml} / \textbf{-lx} {[}\textless{}\emph{pattern}\textgreater{}{]}
\begin{quote}

Lista os detalhes abrangentes de todos os sistemas e drivers
suportados. A saída é bastante longa, então é melhor redirecionar
isso para um arquivo. A saída está em formato XML. É predefinido que
todos os sistemas sejam listados, no entanto, você pode filtrar essa
lista se usar um nome de máquina, jogo ou coringa após o comando
\textbf{-listxml}.
\end{quote}
\phantomsection\label{commandline/commandline-all:mame-commandline-listfull}
\textbf{-listfull} / \textbf{-ll} {[}\textless{}\emph{pattern}\textgreater{}{]}
\begin{quote}

Exibe uma lista dos nomes e descrições dos drivers do sistema.
É predefinido que todos os sistemas sejam listados, no entanto, você
pode filtrar essa lista se usar um nome de máquina, jogo ou coringa
após o comando \textbf{-listfull}.
\end{quote}
\phantomsection\label{commandline/commandline-all:mame-commandline-listsource}
\textbf{-listsource} / \textbf{-ls} {[}\textless{}\emph{pattern}\textgreater{}{]}
\begin{quote}

Exibe uma lista de drivers e os nomes dos arquivos relacionados nos
quais os drivers do sistema estão definidos. Útil para localizar em
qual driver um determinado sistema roda, útil para relatar bugs.
É predefinido que todos os sistemas sejam listados, no entanto, você
pode filtrar essa lista se usar um nome de máquina, jogo ou coringa
após o comando \textbf{-listsource}.
\end{quote}
\phantomsection\label{commandline/commandline-all:mame-commandline-listclones}
\textbf{-listclones} / \textbf{-lc} {[}\textless{}\emph{pattern}\textgreater{}{]}
\begin{quote}

Exibe uma lista de clones. É predefinido que todos os clones sejam
listados, no entanto, você pode filtrar essa lista se usar um nome
de máquina, jogo ou coringa após o comando \textbf{-listclones}.
\end{quote}
\phantomsection\label{commandline/commandline-all:mame-commandline-listbrothers}
\textbf{-listbrothers} / \textbf{-lb} {[}\textless{}\emph{pattern}\textgreater{}{]}
\begin{quote}

Exibe uma lista de `\emph{irmãos}`, ou melhor, outros conjuntos que
compartilham do mesmo driver que o nome do sistema pesquisado.
\end{quote}
\phantomsection\label{commandline/commandline-all:mame-commandline-listcrc}
\textbf{-listcrc} {[}\textless{}\emph{pattern}\textgreater{}{]}
\begin{quote}

Exibe uma lista completa de CRCs de todas as imagens ROM
que compõem uma máquina, nomes de sistema ou dispositivo.
Caso nenhum termo seja usado depois do comando, \emph{todos} os
resultados dos sistemas e dispositivos serão exibidos.
\end{quote}
\phantomsection\label{commandline/commandline-all:mame-commandline-listroms}
\textbf{-listroms} / \textbf{-lr} {[}\textless{}\emph{pattern}\textgreater{}{]}
\begin{quote}

Exibe uma lista de todas as imagens ROM que compõem uma máquina ou
dispositivo. Pode ser filtrado caso seja usado um nome de sistema,
dispositivos ou máquina. Caso nenhum termo seja usado como filtro
depois do comando, \emph{todos} os resultados dos sistemas e dispositivos
serão exibidos.
\end{quote}
\phantomsection\label{commandline/commandline-all:mame-commandline-listsamples}
\textbf{-listsamples} {[}\textless{}\emph{pattern}\textgreater{}{]}
\begin{quote}

Exibe uma lista das amostras que fazem parte de uma determinada
máquina, nomes de sistema ou nome de dispositivos. Caso nenhum termo
seja usado como filtro depois do comando, \emph{todos} os resultados dos
sistemas e dispositivos serão exibidos.
\end{quote}
\phantomsection\label{commandline/commandline-all:mame-commandline-verifyroms}
\textbf{-verifyroms} {[}\textless{}\emph{pattern}\textgreater{}{]}
\begin{quote}

Verifica se há imagens ROM inválidas ou ausentes. É predefinido que
todos os drivers que possuam arquivos ZIP ou diretórios válidos no
rompath (caminho da rom) sejam verificados, no entanto, você pode
limitar essa lista se usar um termo como filtro após o comando
\textbf{-verifyroms}.
\end{quote}
\phantomsection\label{commandline/commandline-all:mame-commandline-verifysamples}
\textbf{-verifysamples} {[}\textless{}\emph{pattern}\textgreater{}{]}
\begin{quote}

Verifica se há amostras inválidas ou ausentes. É predefinido que
todos os drivers que possuem arquivos ZIP ou diretórios válidos no
samplepath sejam verificados no caminho da pasta onde os arquivos de
amostras se encontram, no entanto, você pode filtrar essa lista se
usar um nome de máquina, jogo ou coringa após o comando
\textbf{-verifysamples}.
\end{quote}
\phantomsection\label{commandline/commandline-all:mame-commandline-romident}
\textbf{-romident} {[}\emph{caminho\textbackslash{}completo\textbackslash{}para\textbackslash{}a\textbackslash{}rom\textbackslash{}a\textbackslash{}ser\textbackslash{}conferida.zip}{]}
\begin{quote}

Tenta identificar os arquivos ROM conhecidos pelo MAME e que sejam
compartilhados ou que também sejam usados por outras máquinas no
arquivo ou diretório .zip determinado. Este comando pode ser usado
para tentar identificar conjuntos de ROM retirados de placas
desconhecidas.
Na saída, o nível de erro é retornado como um dos seguintes:
\begin{itemize}
\item {} 
0: significa que todos os arquivos foram identificados

\item {} 
7: significa que todos os arquivos foram identificados, exceto um ou mais arquivos não qualificados como ``não-ROM''

\item {} 
8: significa que alguns arquivos foram identificados

\item {} 
9: significa que nenhum arquivo foi identificado

\end{itemize}
\end{quote}
\phantomsection\label{commandline/commandline-all:mame-commandline-listdevices}
\textbf{-listdevices} / \textbf{-ld} {[}\textless{}\emph{pattern}\textgreater{}{]}
\begin{quote}

Exibe uma lista de todos os dispositivos conhecidos e conectados
em um sistema. O '':'' é considerado o próprio sistema
com a lista de dispositivos sendo anexada para dar ao usuário
uma melhor compreensão do que a emulação está usando. Caso os
slots sejam populados por dispositivos, todos os slots
adicionais que esses dispositivos fornecerem ficarão visíveis
com \textbf{-listdevices} também.
Por exemplo, caso você instale um controlador de disquete em um
PC, este listará os slots da unidade de disco.
\end{quote}
\phantomsection\label{commandline/commandline-all:mame-commandline-listslots}
\textbf{-listslots} / \textbf{-lslot} {[}\textless{}\emph{pattern}\textgreater{}{]}
\begin{quote}

Mostra os slots disponíveis e as opções para cada slot caso
estejam disponíveis. Usado principalmente pelo MAME para
permitir o controle plug-and-play de placas internas, assim
como os PCs que precisam de vídeo, som e outras placas de
expansão.
\begin{quote}

Caso os slots estejam populados com dispositivos, todos os slots
adicionais que esses dispositivos fornecerem ficarão visíveis
com \textbf{-listslots} também. Por exemplo, caso você instale um
controlador de disquete em um PC, este listará os slots da
unidade de disco.

O nome do slot (por exemplo, \textbf{ctrl1}) pode ser usado a partir
da linha de comando (\textbf{-ctrl1} neste caso)
\end{quote}
\end{quote}
\phantomsection\label{commandline/commandline-all:mame-commandline-listmedia}
\textbf{-listmedia} / \textbf{-lm} {[}\textless{}\emph{pattern}\textgreater{}{]}
\begin{quote}

Liste a mídia disponível para uso do sistema. Isso inclui tipos
de mídia como cartucho, cassete, disquete e mais. Extensões de
arquivo comumente conhecidas também são suportadas.
\end{quote}
\phantomsection\label{commandline/commandline-all:mame-commandline-listsoftware}
\textbf{-listsoftware} / \textbf{-lsoft} {[}\textless{}\emph{pattern}\textgreater{}{]}
\begin{quote}

Mostre na tela a lista de software completa que pode ser
usadas através de um determinado termo ou sistema. Observe que
isso é simplesmente um copiar/colar do arquivo .XML que reside
na pasta HASH e que pode ser usada.
\end{quote}
\phantomsection\label{commandline/commandline-all:mame-commandline-verifysoftware}
\textbf{-verifysoftware} / \textbf{-vsoft} {[}\textless{}\emph{pattern}\textgreater{}{]}
\begin{quote}

Verifica se há imagens ROM inválidas ou ausentes na lista de
software. Por predefinição, todos os drivers que possuem arquivos
ZIP ou diretórios válidos no rompath (caminho da rom) serão
verificados, no entanto, você pode limitar essa lista definindo um
nome de driver específico ou \emph{combinações} após o comando
\textbf{-verifysoftware}.
\end{quote}
\phantomsection\label{commandline/commandline-all:mame-commandline-getsoftlist}
\textbf{-getsoftlist} / \textbf{-glist} {[}\textless{}\emph{pattern}\textgreater{}{]}
\begin{quote}

Postagens para exibir na tela uma listas de software específicos
que correspondem ao nome do sistema fornecido.
\end{quote}
\phantomsection\label{commandline/commandline-all:mame-commandline-verifysoftlist}
\textbf{-verifysoftlist} / \textbf{-vlist} {[}\emph{softwarelistname}{]}
\begin{quote}

Verifica ROMs ausentes com base em uma lista de software
predeterminado na pasta \textbf{hash}.
É predefinido que a busca e a verificação será feita em todos os
drivers e arquivos ZIP em diretórios válidos no \emph{rompath} (caminho da
rom), no entanto, você pode limitar essa lista usando um nome que
contenha a lista de software em ``\emph{softwarelistname}'' após o comando
\textbf{-verifysoftlist}. As listas estão na pasta \emph{hash} e devem ser
informadas sem a extensão .XML.
\end{quote}


\subsection{Opções relacionadas as informações exibidas na tela (OSD)}
\label{commandline/commandline-all:osd-commandline-options}\label{commandline/commandline-all:opcoes-relacionadas-as-informacoes-exibidas-na-tela-osd}\phantomsection\label{commandline/commandline-all:mame-commandline-uimodekey}
\textbf{-uimodekey} {[}\emph{keystring}{]}
\begin{quote}

Tecla usada para ativar e desativar o teclado emulado.
A configuração predefinida é \emph{SCRLOCK} no Windows, \emph{Forward Delete}
no Mac (use \emph{FN-Delete} em laptop/teclados compacto).
\end{quote}
\phantomsection\label{commandline/commandline-all:mame-commandline-uifontprovider}
\textbf{-uifontprovider}
\begin{quote}

Escolha a fonte da Interface do Usuário

No Windows, você pode escolher entre: \textbf{win}, \textbf{dwrite}, \textbf{none}
ou \textbf{auto}.
No Mac, você pode escolher entre: \textbf{osx} ou \textbf{auto}
Em outras plataformas, você pode escolher entre: \textbf{sdl} ou
\textbf{auto}.
\begin{quote}

O valor predefinido é \textbf{auto}
\end{quote}
\end{quote}
\phantomsection\label{commandline/commandline-all:mame-commandline-keyboardprovider}
\textbf{-keyboardprovider}
\begin{quote}

Escolhe como o MAME lidará com o teclado.

No Windows, você pode escolher entre: \textbf{auto}, \textbf{rawinput},
\textbf{dinput}, \textbf{win32}, ou \textbf{none}.
No SDL, você pode escolher entre: \textbf{auto}, \textbf{sdl}, \textbf{none}
\begin{quote}

O valor predefinido é \textbf{auto}.
No Windows, \textbf{auto} tentará o \textbf{rawinput}, caso contrário
retornará para \textbf{dinput}. No SDL, o auto será predefinido como
\textbf{sdl}.
\end{quote}
\end{quote}
\phantomsection\label{commandline/commandline-all:mame-commandline-mouseprovider}
\textbf{-mouseprovider}
\begin{quote}

Escolhe como o MAME lidará com o mouse.

No Windows, você pode escolher entre: \textbf{auto}, \textbf{rawinput},
\textbf{dinput}, \textbf{win32}, or \textbf{none}.
No SDL, você pode escolher entre: \textbf{auto}, \textbf{sdl}, \textbf{none}
\begin{quote}

O valor predefinido é \textbf{auto}.
No Windows, \textbf{auto} tentará o \textbf{rawinput}, caso contrário
retornará para \textbf{dinput}. No SDL, o \textbf{auto} será predefinido
como \textbf{sdl}.
\end{quote}
\end{quote}
\phantomsection\label{commandline/commandline-all:mame-commandline-lightgunprovider}
\textbf{-lightgunprovider}
\begin{quote}

Escolhe como o MAME lidará com a arma de luz (\emph{light gun}).

No Windows, você pode escolher entre: \textbf{auto}, \textbf{rawinput},
\textbf{win32}, ou \textbf{none}.
No SDL, você pode escolher entre: \textbf{auto}, \textbf{x11}, \textbf{none}.
\begin{quote}

O valor predefinido é \textbf{auto}.
o Windows, \emph{auto} tentará \textbf{rawinput}, caso contrário retornará
para \textbf{win32} ou \textbf{none} caso não encontre nenhum.
No SDL/Linux, \textbf{auto} é predefinido como \textbf{x11} ou \textbf{none}
caso não encontre nenhum.
Em outro tipo de SDL, \textbf{auto} será predefinido para \textbf{none}.
\end{quote}
\end{quote}
\phantomsection\label{commandline/commandline-all:mame-commandline-joystickprovider}
\textbf{-joystickprovider}
\begin{quote}

Escolhe como o MAME lidará com o joystick.

No Windows, você pode escolher entre: \textbf{auto}, \textbf{winhybrid},
\textbf{dinput}, \textbf{xinput}, ou \textbf{none}.
No SDL, você pode escolher entre: \textbf{auto}, \textbf{sdl}, \textbf{none}.
\begin{quote}

O valor predefinido é \textbf{auto}.
No Windows, o auto será predefinido para dinput.
\end{quote}

Repare que no controle do Microsoft X-Box 360 e X-Box One, eles
funcionarão melhor com \textbf{winhybrid} ou \textbf{xinput}. A opção de
controle \emph{winhybrid} suporta uma mistura de DirectInput e Xinput ao
mesmo tempo.
No SDL, \textbf{auto} será predefinido para \textbf{sdl}.
\end{quote}


\subsection{Opções relacionados ao OSD CLI}
\label{commandline/commandline-all:opcoes-relacionados-ao-osd-cli}\phantomsection\label{commandline/commandline-all:mame-commandline-listmidi}
\textbf{-listmidi}
\begin{quote}

Cria uma lista de dispositivos MIDI I/O disponíveis que possam ser
usados com a emulação.
\end{quote}
\phantomsection\label{commandline/commandline-all:mame-commandline-listnetwork}
\textbf{-listnetwork}
\begin{quote}

Cria uma lista de adaptadores de rede disponíveis que possam ser
usados com a emulação.
\end{quote}


\subsection{Opções de saída do OSD}
\label{commandline/commandline-all:opcoes-de-saida-do-osd}\phantomsection\label{commandline/commandline-all:mame-commandline-output}
\textbf{-output}
\begin{quote}

Escolhe como o MAME lidará com o processamento de notificações de
saída.

Você pode escolher entre: \textbf{auto}, \textbf{none}, \textbf{console} ou
\textbf{network}.
\begin{quote}

O valor predefinido para a porta de rede é \textbf{8000}.
\end{quote}
\end{quote}


\subsection{Opções de configuração}
\label{commandline/commandline-all:opcoes-de-configuracao}\phantomsection\label{commandline/commandline-all:mame-commandline-noreadconfig}
\textbf{-{[}no{]}readconfig} / \textbf{-{[}no{]}rc}
\begin{quote}

Ativa ou desativa a leitura dos arquivos de configuração,
é predefinido que os arquivos de configuração sejam lidos.
O MAME faz a leitura destes arquivos na seguinte ordem:
\begin{itemize}
\item {} 
\textbf{mame.ini}

\item {} 
\textbf{\textless{}meumame\textgreater{}.ini}   (por exemplo, caso o arquivo binário do MAME seja renomeado para mame060.exe, então o MAME carregará o aquivo mame060.ini)

\item {} 
\textbf{debug.ini}       (caso o depurador esteja habilitado)

\item {} 
\textbf{\textless{}driver\textgreater{}.ini}    (com base no nome do arquivo fonte ou driver)

\item {} 
\textbf{vertical.ini}    (para sistemas com orientação vertical do monitor)

\item {} 
\textbf{horizont.ini}    (para sistemas com orientação horizontal do monitor)

\item {} 
\textbf{arcade.ini}      (para sistemas adicionados no código fonte com a macro  GAME() )

\item {} 
\textbf{console.ini}     (para sistemas adicionados no código fonte com a macro CONS() )

\item {} 
\textbf{computer.ini}    (para sistemas adicionados no código fonte com a macro COMP() )

\item {} 
\textbf{othersys.ini}    (para sistemas adicionados no código fonte com a macro SYST() )

\item {} 
\textbf{vector.ini}      (para sistemas com vetores apenas)

\item {} 
\textbf{\textless{}parent\textgreater{}.ini}    (para clones apenas, poderá ser chamado de forma recursiva)

\item {} 
\textbf{\textless{}systemname\textgreater{}.ini}

\end{itemize}

(Veja mais em {\hyperref[advanced/multiconfig:advanced\string-multi\string-cfg]{\sphinxcrossref{\DUrole{std,std-ref}{A ordem de leitura dos arquivos}}}} para maiores detalhes)

As configurações nos INIs posteriores substituem aquelas dos INIs
anteriores.
Então, por exemplo, se você quiser desabilitar os efeitos de
sobreposição nos sistemas vetoriais, você pode criar um arquivo
\textbf{vector.ini} com a linha ``effect none'' nele, ele irá sobrescrever
qualquer valor de efeito que você tenha em seu mame.ini.
\begin{quote}

O valor predefinido é \textbf{Ligado} (\textbf{-readconfig}).
\end{quote}
\end{quote}


\subsection{Principais opções de caminho}
\label{commandline/commandline-all:principais-opcoes-de-caminho}\phantomsection\label{commandline/commandline-all:mame-commandline-homepath}
\textbf{-homepath} \emph{\textless{}path\textgreater{}}
\begin{quote}

Define o caminho onde o diretório base \emph{plugins} deve ser
encontrado.
\begin{quote}

O valor predefinido é `.' (isto é, no diretório base atual).
\end{quote}
\end{quote}
\phantomsection\label{commandline/commandline-all:mame-commandline-rompath}
\textbf{-rompath} / \textbf{-rp} \textless{}\emph{path}\textgreater{}
\begin{quote}

Define o caminho completo para encontrar imagens ROM, disco rígido,
fita cassete, etc. Mais de um caminho pode ser definido separando-os
por ponto e vírgula.
\begin{quote}

O valor predefinido é `roms' (isto é, um diretório chamado
``roms'' criado no mesmo diretório que o executável do MAME).
\end{quote}
\end{quote}
\phantomsection\label{commandline/commandline-all:mame-commandline-hashpath}
\textbf{-hashpath} \textless{}\emph{path}\textgreater{}
\begin{quote}

Define o caminho completo para a pasta com os arquivos \emph{hash} que é
usado pela \emph{lista de software} no gerenciador de arquivos. Mais de
um caminho pode ser definido separando-os por ponto e vírgula.
\begin{quote}

O valor predefinido é `hash' (isto é, um diretório chamado
``hash'' no mesmo diretório que o executável do MAME).
\end{quote}
\end{quote}
\phantomsection\label{commandline/commandline-all:mame-commandline-samplepath}
\textbf{-samplepath} / \textbf{-sp} \textless{}\emph{path}\textgreater{}
\begin{quote}

Define o caminho completo para os arquivos de amostras (samples).
Mais de um caminho pode ser definido separando-os por ponto e
vírgula.
\begin{quote}

O valor predefinido é `samples' (isto é, um diretório chamado
``samples'' no mesmo diretório que o executável do MAME).
\end{quote}
\end{quote}
\phantomsection\label{commandline/commandline-all:mame-commandline-artpath}
\textbf{-artpath} \textless{}\emph{path}\textgreater{} / \textbf{-artwork\_directory} \textless{}\emph{path}\textgreater{}
\begin{quote}

Define o caminho completo para os arquivos de ilustrações
(artworks). Mais de um caminho pode ser definido separando-os por
ponto e vírgula.
\begin{quote}

O valor predefinido é `artwork' (isto é, um diretório chamado
``artwork'' no mesmo diretório que o executável do MAME).
\end{quote}
\end{quote}
\phantomsection\label{commandline/commandline-all:mame-commandline-ctrlrpath}
\textbf{-ctrlrpath} / \textbf{-ctrlr\_directory} \textless{}\emph{path}\textgreater{}
\begin{quote}

Define o caminho completo para os arquivos de configuração
específico para controle. Mais de um caminho pode ser definido
separando-os por ponto e vírgula.
\begin{quote}

O valor predefinido é `ctrlr' (isto é, um diretório chamado
``ctrlr'' no mesmo diretório que o executável do MAME).
\end{quote}
\end{quote}
\phantomsection\label{commandline/commandline-all:mame-commandline-inipath}
\textbf{-inipath} \textless{}\emph{path}\textgreater{}
\begin{quote}

Define o caminho completo para os arquivos \emph{.INI}. Mais de um
caminho pode ser definido separando-os por ponto e vírgula.
\begin{quote}

O valor predefinido é `.;ini' (isto é, procure primeiro no
diretório onde se encontra o executável do MAME, em seguida
dentro do diretório ``\emph{ini}''.
\end{quote}
\end{quote}
\phantomsection\label{commandline/commandline-all:mame-commandline-fontpath}
\textbf{-fontpath} \textless{}\emph{path}\textgreater{}
\begin{quote}

Define o caminho completo para os arquivos de fontes \emph{.BDF}.
Mais de um caminho pode ser definido separando-os por ponto e
vírgula.
\begin{quote}

O valor predefinido é `.' (isto é, no diretório base atual).
\end{quote}
\end{quote}
\phantomsection\label{commandline/commandline-all:mame-commandline-cheatpath}
\textbf{-cheatpath} \textless{}\emph{path}\textgreater{}
\begin{quote}

Define o caminho completo para os arquivos de trapaça em formato
\emph{.XML}.
Mais de um caminho pode ser definido separando-os por ponto e
vírgula.
\begin{quote}

O valor predefinido é ``cheat'' (isto é, uma pasta chamada ``cheat'',
localizada no mesmo diretório que o executável do MAME).
\end{quote}
\end{quote}
\phantomsection\label{commandline/commandline-all:mame-commandline-crosshairpath}
\textbf{-crosshairpath} \textless{}\emph{path}\textgreater{}
\begin{quote}

Define o caminho completo para os arquivos de mira \textbf{crosshair}.
Mais de um caminho pode ser definido separando-os por ponto e
vírgula.
\begin{quote}

O valor predefinido é ``\emph{crosshair}'' (isto é, um diretório
chamado ``\emph{crosshair}'' no mesmo diretório que o executável do
MAME). Caso uma mira seja definida no menu, o MAME procurará por
\emph{nomedosistema\textbackslash{}cross\#.png}, em seguida no ``\emph{crosshairpath}''
especificado onde ``\emph{\#}'' é o número do jogador.
Caso nenhuma mira seja definida, o MAME usará a sua própria.
\end{quote}
\end{quote}
\phantomsection\label{commandline/commandline-all:mame-commandline-pluginspath}
\textbf{-pluginspath} \textless{}\emph{path}\textgreater{}
\begin{quote}

Define o caminho completo para os plug-ins Lua.
\end{quote}
\phantomsection\label{commandline/commandline-all:mame-commandline-languagepath}
\textbf{-languagepath} \textless{}\emph{path}\textgreater{}
\begin{quote}

Define o caminho para os arquivos de idioma da interface.
\end{quote}
\phantomsection\label{commandline/commandline-all:mame-commandline-swpath}
\textbf{-swpath} \emph{\textless{}path\textgreater{}}
\begin{quote}

Define um caminho onde programas avulsos (software) serão
encontrados para uso com o emulador.
\end{quote}


\subsection{Principais opções de caminho final de diretório}
\label{commandline/commandline-all:principais-opcoes-de-caminho-final-de-diretorio}\phantomsection\label{commandline/commandline-all:mame-commandline-cfgdirectory}
\textbf{-cfg\_directory} \textless{}\emph{path}\textgreater{}
\begin{quote}

Define um único diretório onde os arquivos de configuração
são armazenados. Os arquivos de configuração armazenam as
customizações feitas pelo usuário que são lidas na inicialização e
escritas quando o MAME sai.
\begin{quote}

O valor predefinido é `cfg' (isto é, um diretório com o nome
``cfg'' no mesmo diretório que o executável do MAME). Caso este
diretório não exista, ele será criado automaticamente.
\end{quote}
\end{quote}
\phantomsection\label{commandline/commandline-all:mame-commandline-nvramdirectory}
\textbf{-nvram\_directory} \textless{}\emph{path}\textgreater{}
\begin{quote}

Define um único diretório onde os arquivos NVRAM são
armazenados. Arquivos NVRAM armazenam o conteúdo de EEPROM e RAM não
volátil (NVRAM) para sistemas que usavam esse tipo de hardware.
Esses dados são lidos na inicialização e gravados quando o MAME sai.
\begin{quote}

O valor predefinido é \textbf{nvram} (isto é, um diretório com nome
``nvram'' no mesmo diretório que o executável do MAME).
Caso este diretório não exista, ele será criado automaticamente.
\end{quote}
\end{quote}
\phantomsection\label{commandline/commandline-all:mame-commandline-inputdirectory}
\textbf{-input\_directory} \textless{}\emph{path}\textgreater{}
\begin{quote}

Define um único diretório onde os arquivos de gravação de entrada
serão armazenados. As gravações de entrada são criadas através da
opção \textbf{-record} e reproduzidas através da opção \textbf{-playback}.
\begin{quote}

O valor predefinido é \textbf{inp} (ou seja, um diretório de nome
``inp'' no mesmo diretório que o executável do MAME). Caso este
diretório não exista, ele será criado automaticamente.
\end{quote}
\end{quote}
\phantomsection\label{commandline/commandline-all:mame-commandline-statedirectory}
\textbf{-state\_directory} \textless{}\emph{path}\textgreater{}
\begin{quote}

Define um único diretório onde os arquivos de gravação de estado
são armazenados. Os arquivos de estado são gravados, lidos e
escritos mediante a solicitação do utilizador ou ao usar a opção
\textbf{-autosave}.
\begin{quote}

O valor predefinido é `sta' (isto é, um diretório de nome ``sta''
é salvo no mesmo diretório que o executável do MAME).
Caso este diretório não exista, ele será criado automaticamente.
\end{quote}
\end{quote}
\phantomsection\label{commandline/commandline-all:mame-commandline-snapshotdirectory}
\textbf{-snapshot\_directory} \textless{}\emph{path}\textgreater{}
\begin{quote}

Define um único diretório onde serão armazenados os instantâneos de
tela quando solicitado pelo usuário.
\begin{quote}

O valor predefinido é `snap' (isto é, um diretório chamado
``snap'' no mesmo diretório que o executável do MAME). Caso este
diretório não exista, ele será criado automaticamente.
\end{quote}
\end{quote}
\phantomsection\label{commandline/commandline-all:mame-commandline-diffdirectory}
\textbf{-diff\_directory} \textless{}\emph{path}\textgreater{}
\begin{quote}

Define um único diretório onde os arquivos de diferencial do disco
rígido serão armazenados. Os arquivos de diferencial de disco rígido
armazenam qualquer dado que é escrito de volta na imagem do disco
rígido, visando a preservação da imagem original. Os arquivos de
diferencial são criados na inicialização com um sistema que use um
disco rígido.
\begin{quote}

O valor predefinido é `diff' (isto é, um diretório chamado
``diff'' no mesmo diretório que o executável do MAME). Caso este
diretório não exista, ele será criado automaticamente.
\end{quote}
\end{quote}
\phantomsection\label{commandline/commandline-all:mame-commandline-commentdirectory}
\textbf{-comment\_directory} \textless{}\emph{path}\textgreater{}
\begin{quote}

Define um único diretório onde os arquivos de comentário do
depurador são armazenados. Os arquivos de comentário do depurador
são escritos pelo depurador quando comentários são adicionados ao
sistema para a desmontagem.
\begin{quote}

O valor predefinido é `comments' (isto é, um diretório chamado
``comments'' no mesmo diretório que o executável do MAME). Caso este
diretório não exista, ele será criado automaticamente.
\end{quote}
\end{quote}


\subsection{Principais opções de estado e reprodução}
\label{commandline/commandline-all:principais-opcoes-de-estado-e-reproducao}\phantomsection\label{commandline/commandline-all:mame-commandline-norewind}
\textbf{-{[}no{]}rewind}
\begin{quote}

Quando ativo e a emulação for pausada, automaticamente é salvo o
estado da condição da memória toda a vez que um quadro for avançado.
O rebobinamento das condições de estado que foram salvas podem ser
carregadas de forma consecutiva ao pressionar a tecla de atalho para
rebobinar passo único (\emph{Shift Esquerdo + til}) \footnote[3]{\sphinxAtStartFootnote%
Até que o teclado \textbf{ABNT-2} seja mapeado pela equipe do MAMEDev,
essa tecla fica do lado esquerdo da tecla 1, logo abaixo da
tecla ESQ. (Nota do tradutor)
}.
\begin{quote}

O valor predefinido é \textbf{Desligado} (\textbf{-norewind}).
\end{quote}

Caso o depurador esteja no estado `break', a condição de estado
atual é criada a cada `step in', `step over' ou caso ocorra um
`step out'. Nesse modo, rebobinar os estados salvos podem ser
carregados executando o depurador `rewind' ou o comando (`rw').
\end{quote}
\phantomsection\label{commandline/commandline-all:mame-commandline-rewindcapacity}
\textbf{-rewind\_capacity} \textless{}\emph{value}\textgreater{}
\begin{quote}

Define a capacidade de rebobinar em megabytes.
É a quantidade total de memória que será usada para rebobinar
savestates. Quando a capacidade alcança o limite, os antigos
savestates são apagados enquanto novos são capturados. Definindo uma
capacidade menor do que o savestate atual, desabilita o
rebobinamento. Os valores negativos abaixo de zero são
automaticamente fixados em 0.
\end{quote}
\phantomsection\label{commandline/commandline-all:mame-commandline-state}
\textbf{-state} \textless{}\emph{slot}\textgreater{}
\begin{quote}

Depois de iniciar um sistema determinado, fará com que o estado
salvo no \textless{}\emph{slot}\textgreater{} seja carregado imediatamente.
\end{quote}
\phantomsection\label{commandline/commandline-all:mame-commandline-noautosave}
\textbf{-{[}no{]}autosave}
\begin{quote}

Quando ativado, cria automaticamente um arquivo de estado ao sair do
MAME e automaticamente tenta recarregá-lo caso o MAME inicie
novamente com o mesmo sistema. Isso só funciona para sistemas que
habilitaram explicitamente o suporte a estado de salvamento em seu
driver.
\begin{quote}

O valor predefinido é \textbf{Desligado} (\textbf{-noautosave}).
\end{quote}
\end{quote}
\phantomsection\label{commandline/commandline-all:mame-commandline-playback}
\textbf{-playback} / \textbf{-pb} \textless{}\emph{filename}\textgreater{}
\begin{quote}

Faz a reprodução de um arquivo de gravação. Esse recurso não
funciona de maneira confiável com todos os sistemas, mas pode ser
usado para assistir a uma sessão de jogo gravada anteriormente do
início ao fim. Para tornar as coisas consistentes, você deve apagar
os arquivos de configuração (.cfg), NVRAM (.nv) e o cartão de
memória.
\begin{quote}

O valor predefinido é \textbf{NULO} (sem reprodução).
\end{quote}
\end{quote}
\phantomsection\label{commandline/commandline-all:mame-commandline-exitafterplayback}
\textbf{-exit\_after\_playback}
\begin{quote}

Diz ao MAME para encerrar a emulação depois que terminar a
reprodução (playback).
\end{quote}
\phantomsection\label{commandline/commandline-all:mame-commandline-record}
\textbf{-record} / \textbf{-rec} \textless{}\emph{filename}\textgreater{}
\begin{quote}

Faz a gravação de todos comandos feitos pelo usuários durante uma
seção e define o nome do arquivo onde será registrado todos esses
comandos durante uma seção.
Esse recurso não funciona de forma confiável com todos os sistemas.
\begin{quote}

O valor predefinido é \textbf{NULO} (sem gravação).
\end{quote}
\end{quote}
\phantomsection\label{commandline/commandline-all:mame-commandline-recordtimecode}
\textbf{-record\_timecode}
\begin{quote}

Diz ao MAME para criar um arquivo de \emph{timecode}. Ele contém uma linha
com os tempos decorridos a cada pressão da tecla de atalho
(\emph{O valor predefinido é F12}). Esta opção funciona apenas quando o modo de
gravação está ativado (opção \textbf{-record}). O arquivo é salvo na
pasta \emph{inp}. É predefinido que nenhum arquivo de timecode seja
gravado.
\end{quote}
\phantomsection\label{commandline/commandline-all:mame-commandline-mngwrite}
\textbf{-mngwrite} \textless{}\emph{filename}\textgreater{}.mng
\begin{quote}

Escreve cada quadro de vídeo em um arquivo \textless{}\emph{filename}\textgreater{} no formato
MNG, produzindo uma animação da sessão.
Note que \textbf{-mngwrite} só grava quadros de vídeo. Ele não grava
nenhum dado de áudio, para tanto use \textbf{-wavwrite} em conjunto com o
comando e remonte o áudio e vídeo posteriormente usando outras
ferramentas.
\begin{quote}

O valor predefinido é \textbf{NULO} (sem gravação).
\end{quote}
\end{quote}
\phantomsection\label{commandline/commandline-all:mame-commandline-aviwrite}
\textbf{-aviwrite} \textless{}\emph{filename}\textgreater{}.avi
\begin{quote}

Grava todos os dados de áudio e vídeo em um arquivo,
\textless{}\emph{filename}\textgreater{}.avi é o nome do arquivo de vídeo. O arquivo é gravado
em formato AVI puro (raw), note que o arquivo final ficará bem
grande. Caso o seu HDD não seja rápido o suficiente haverá
travamentos e lentidão na emulação.
\begin{quote}

O valor predefinido é \textbf{NULO} (sem gravação).
\end{quote}
\end{quote}
\phantomsection\label{commandline/commandline-all:mame-commandline-wavwrite}
\textbf{-wavwrite} \textless{}\emph{filename}\textgreater{}.wav
\begin{quote}

Grava todos os dados de áudio da seção em formato WAV em um arquivo
\textless{}\emph{filename}\textgreater{}.wav .
\begin{quote}

O valor predefinido é \textbf{NULO} (sem gravação).
\end{quote}
\end{quote}
\phantomsection\label{commandline/commandline-all:mame-commandline-snapname}
\textbf{-snapname} \textless{}\emph{name}\textgreater{}
\begin{quote}

Descreve como MAME deve nomear arquivos de instantâneos de tela.
\textless{}\emph{name}\textgreater{} será o guia que o MAME usará para nomear o arquivo.

São disponibilizadas três substituições simples: o caractere \sphinxcode{/}
representa o separador de caminho em qualquer plataforma de destino
(até mesmo o Windows); a string \sphinxcode{\%g} representa o nome do driver
do sistema atual; e a string \sphinxcode{\%i} representa um índice de
incremento. Caso o \sphinxcode{\%i} seja omitido, cada instantâneo tirado
substituirá o anterior; caso contrário, o MAME encontrará o próximo
valor vazio para \sphinxcode{\%i} e o usará como um nome de arquivo.

O valor predefinido é \sphinxcode{\%g/\%i}, que cria uma pasta separada para
cada sistema e nomeia os instantâneos dentro ele, começando com
\textbf{0000} e incrementando a partir daí.

Em adição ao que foi dito acima, para os drivers que usam mídias
diferentes, como cartões ou disquetes, você também pode usar o
indicador \sphinxcode{\%d\_{[}media{]}}. Substitua \sphinxcode{{[}media{]}} pelo comutador de
mídia que você deseja usar.

Alguns exemplos: se você usar \sphinxcode{mame robby -snapname foo/\%g\%i} os
instantâneos serão salvos em \textbf{snapsfoorobby0000.png},
\textbf{snapsfoorobby0001.png} e assim por diante. Caso você use
\sphinxcode{mame nes -cart robby -snapname \%g/\%d\_cart} os instantâneos serão
salvos como \textbf{snaps\textbackslash{}nes\textbackslash{}robby.png}, caso você use
\sphinxcode{mame c64 -flop1 robby -snapname \%g/\%d\_flop1/\%i} estes serão
salvos como \textbf{snaps\textbackslash{}c64\textbackslash{}robby\textbackslash{}0000.png}.
\end{quote}
\phantomsection\label{commandline/commandline-all:mame-commandline-snapsize}
\textbf{-snapsize} \textless{}\emph{width\textgreater{}x\textless{}height}\textgreater{}
\begin{quote}

Define um tamanho fixo para os instantâneos e vídeos.
É predefinido que o MAME criará instantâneos, assim como os vídeos,
na resolução original do sistema em pixels brutos. Caso você use
esta opção, o MAME criará instantâneos e vídeos no tamanho que você
determinou, com filtro bilinear (filtro de embaçamento de pixels)
aplicado no resultado final. Observe que ao definir este tamanho a
tela não gira automaticamente caso o sistema seja orientado
verticalmente.
\begin{quote}

O valor predefinido é \textbf{auto}.
\end{quote}
\end{quote}
\phantomsection\label{commandline/commandline-all:mame-commandline-snapview}
\textbf{-snapview} \textless{}\emph{viewname}\textgreater{}
\begin{quote}

Define a exibição a ser usada ao renderizar instantâneos e vídeos.
É predefinido que ambos usem uma exibição especial `interna', que
renderize uma captura instantânea separada por tela ou renderize
os vídeos somente da primeira tela. Ao usar essa opção, você
pode sobrepor esse comportamento predefinido de exibição e
selecionar apenas uma exibição que será aplicada a todos os
instantâneos e vídeos. Observe que o nome de visualização.
Observe que \textless{}\emph{viewname}\textgreater{} não precisa ser uma combinação perfeita,
ao invés disso, ele selecionará a primeira exibição cujo nome
corresponda a todos os caracteres definidos por \textless{}\emph{viewname}\textgreater{}.

Por exemplo, \textbf{-snapview native} corresponderá visualização
``Nativa em (15:14)'' ainda que não seja uma combinação ideal.
O \textless{}\emph{viwename}\textgreater{} também pode ser ``auto'' onde será escolhida a primeira
exibição de todas as telas presentes.
\begin{quote}

O valor predefinido é \textbf{internal}.
\end{quote}
\end{quote}
\phantomsection\label{commandline/commandline-all:mame-commandline-nosnapbilinear}
\textbf{-{[}no{]}snapbilinear}
\begin{quote}

Especifique se o instantâneo ou vídeo deve ter filtragem bilinear
aplicada, o filtro bilinear aplica um leve efeito de embaçamento ou
suavização à tela, amenizando um pouco o serrilhado nos contornos
gráficos e suavizando a tela do sistema. Desligar essa opção pode
fazer a diferença melhorando a performance durante a gravação do
vídeo.
\begin{quote}

O valor predefinido é \textbf{Ligado} (\textbf{-snapbilinear}).
\end{quote}
\end{quote}
\phantomsection\label{commandline/commandline-all:mame-commandline-statename}
\textbf{-statename} \textless{}\emph{name}\textgreater{}
\begin{quote}

Descreve como o MAME deve armazenar os arquivos de estado salvos
relativo ao caminho do state\_directory. \textless{}\emph{name}\textgreater{} é uma string que
fornece um modelo a ser usado usado para gerar um nome de arquivo.

São disponibilizadas duas substituições simples: o caractere \sphinxcode{/}
representa o separador de caminho em qualquer plataforma de destino
(até mesmo no Windows); a string \sphinxcode{\%g} representa o nome do driver
do sistema atual.

O valor predefinido é \sphinxcode{\%g}, que cria uma pasta separada para cada
sistema.

Em adição ao que foi dito acima, para os drivers que usem mídias
diferentes, como cartões ou disquetes, você também pode usar o
indicador \sphinxcode{\%d\_{[}media{]}}. Substitua \sphinxcode{{[}media{]}} pelo comutador de
mídia que você deseja usar.

Alguns exemplos: se você usar \sphinxcode{mame robby -statename foo/\%g\%i} os
instantâneos serão salvos em \textbf{sta\textbackslash{}foo\textbackslash{}robby\textbackslash{}}. Caso você use
\sphinxcode{mame nes -cart robby -statename \%g/\%d\_cart} os instantâneos serão
salvos em \textbf{sta\textbackslash{}nes\textbackslash{}robby}. Caso você use
\sphinxcode{mame c64 -flop1 robby -statename \%g/\%d\_flop1/\%i} estes serão
salvos como \textbf{sta\textbackslash{}c64\textbackslash{}robby\textbackslash{}0000.png}.
\end{quote}
\phantomsection\label{commandline/commandline-all:mame-commandline-noburnin}
\textbf{-{[}no{]}burnin}
\begin{quote}

Rastreia o brilho da tela durante a reprodução e no final da
emulação, gera um PNG que pode ser usado para simular um efeito
burn-in \footnote[4]{\sphinxAtStartFootnote%
Quando uma imagem ficava estática em uma tela de tubo CRT
durante muito tempo, a fina película de fósforo que fica por de
trás da tela de vidro sofria uma leve \textbf{queima} nas regiões de
maior intensidade ficando uma marca no lugar. Uma vez marcada,
essa mancha ficava sobre a imagem como se fosse uma sombra e nem
sempre era necessário que a tela estivesse ligada para que a
mancha pudesse ser visualizada na tela. (Nota do tradutor)
} na tela. O PNG é criado de tal maneira que as
áreas menos usadas da tela ficam totalmente brancas (pois as áreas a
serem marcadas são escuras, todo o resto da tela deverá ficar um
pouco mais iluminada).

A intenção é que este PNG possa ser carregado através de um arquivo
de ilustração usando um valor alpha pequeno como valores entre \emph{0.1}
e \emph{0.2} que se misturam bem com o resto da tela.
Os arquivos PNG gerados são gravados no diretório snap dentro do
\emph{systemname/burnin-\textless{}nome.da.tela\textgreater{}.png}.
\begin{quote}

O valor predefinido é \textbf{Desligado} (\textbf{-noburnin}).
\end{quote}
\end{quote}


\subsection{Principais opções de performance}
\label{commandline/commandline-all:principais-opcoes-de-performance}\phantomsection\label{commandline/commandline-all:mame-commandline-noautoframeskip}
\textbf{-{[}no{]}autoframeskip} / \textbf{-{[}no{]}afs}
\begin{quote}

Determina automaticamente quantos quadros pular no sistema que você
estiver rodando, realizando ajustes constantes na tentativa de
manter o sistema rodando a toda velocidade. Habilitando essa opção
ela sobrescreve o valor predefinido por \textbf{-frameskip}.
\begin{quote}

O valor predefinido é \textbf{Desligado} (\textbf{-noautoframeskip}).
\end{quote}
\end{quote}
\phantomsection\label{commandline/commandline-all:mame-commandline-frameskip}
\textbf{-frameskip} / \textbf{-fs} \textless{}\emph{level}\textgreater{}
\begin{quote}

Determina o valor de pulo de quadros (frameskip). Ela elimina
cerca de 12 quadros enquanto estiver sendo executado. Por exemplo,
se você definir \emph{-frameskip 2} então MAME irá exibir 10 de cada 12
quadros. Ao pular estes quadros, pode ser que você consiga rodar
o sistema emulado em velocidade máxima sem que sobrecarregue o
seu computador ainda que ele não tenha todo este poder de
processamento.
\begin{quote}

O valor predefinido é não pular nenhum quadro
(\textbf{-frameskip 0}).
\end{quote}
\end{quote}
\phantomsection\label{commandline/commandline-all:mame-commandline-secondstorun}
\textbf{-seconds\_to\_run} / \textbf{-str} \textless{}\emph{seconds}\textgreater{}
\begin{quote}

Este comando pode ser usado para realizar um teste de velocidade de
forma automatizada. O comando diz ao MAME para para interromper a
emulação depois de alguns segundos. Ao combinar com outras opções
fixas de linha de comando você pode definir um ambiente para
realizar testes de performance. Em adição, ao sair, a opção \textbf{-str}
faz com que seja gravado um instantâneo da tela chamado \emph{final.png}
no diretório de instantâneos.
\end{quote}
\phantomsection\label{commandline/commandline-all:mame-commandline-nothrottle}
\textbf{-{[}no{]}throttle}
\begin{quote}

Ativa ou não a função de controle de velocidade do emulador \footnote[5]{\sphinxAtStartFootnote%
O termo \emph{throttle} no Inglês significa \emph{parar/interromper a
respiração através da esganadura da garganta}. O termo então
significa manter o controle do fluxo da velocidade. Em Inglês
este termo também é usado para descrever o acelerador de um
veículo, onde o \emph{acelerador} faz o controle da velocidade do
mesmo. (Nota do tradutor)
}.
Quando este controle está ligado, o MAME tenta manter o sistema
rodando em sua velocidade original, ao ser deligado o MAME roda o
sistema na velocidade mais rápida possível.
Note que a velocidade mais rápida geralmente não é limitada pela sua
placa de vídeo, especialmente em sistemas mais antigos.
\begin{quote}

O valor predefinido é \textbf{Ligado} (\textbf{-throttle}).
\end{quote}
\end{quote}
\phantomsection\label{commandline/commandline-all:mame-commandline-nosleep}
\textbf{-{[}no{]}sleep}
\begin{quote}

Permite que o MAME devolva tempo de CPU ao sistema quando
estiver rodando com \textbf{-throttle}. Isso permite que outros programas
tenham mais tempo de CPU, assumindo que a emulação não esteja
consumindo 100\% dos recursos do processador. Essa opção pode causar
uma certa intermitência na performance caso outros programas que
também demandem de processamento estejam rodando junto com o MAME.
\begin{quote}

O valor predefinido é \textbf{Ligado} (\textbf{-sleep}).
\end{quote}
\end{quote}
\phantomsection\label{commandline/commandline-all:mame-commandline-speed}
\textbf{-speed} \textless{}\emph{factor}\textgreater{}
\begin{quote}

Muda a maneira que o MAME controla o fluxo de dados do sistema
emulado de uma maneira que rode em múltiplos da velocidade original.
Um \textless{}\emph{fator}\textgreater{} de \textbf{1.0} significa rodar o sistema em velocidade
normal. Um fator de \textbf{0.5} significa rodar o sistema na metade da
velocidade normal.
Já um \textless{}\emph{fator}\textgreater{} de \textbf{2.0} significa rodar o sistema 2x acima da
velocidade normal. Repare que ao mudar este valor, a velocidade e a
tonalidade do áudio que está sendo executado irá mudar também
proporcionalmente. A resolução interna da fração são dois ponto
decimais, então o valor \textbf{1.002} é o mesmo que \textbf{1.0}.
\begin{quote}

O valor predefinido é \textbf{1.0}.
\end{quote}
\end{quote}
\phantomsection\label{commandline/commandline-all:mame-commandline-norefreshspeed}
\textbf{-{[}no{]}refreshspeed} / \textbf{-{[}no{]}rs}
\begin{quote}

Permite ao MAME ajustar a velocidade do sistema dinamicamente de
maneira que não exceda o valor mais baixo da taxa de atualização da
tela de qualquer monitor no seu sistema. Assim, se você tem um
monitor com \textbf{60 Hz} e roda um sistema configurado para \textbf{60.6 Hz}
o MAME irá reduzir a velocidade dinamicamente para \textbf{99\%} visando
prevenir cortes no som ou outro problemas indesejáveis enquanto
estiver rodando com uma velocidade de taxa de atualização de tela
mais baixa.
\begin{quote}

O valor predefinido é \textbf{Desligado} (\textbf{-norefreshspeed}).
\end{quote}
\end{quote}
\phantomsection\label{commandline/commandline-all:mame-commandline-numprocessors}
\textbf{-numprocessors} \textless{}\emph{auto\textbar{}value}\textgreater{} / \textbf{-np} \textless{}\emph{auto\textbar{}value}\textgreater{}
\begin{quote}

Define a quantidade de núcleos do processador a serem usados.
A opção \textbf{auto} usará a quantidade de núcleos informada pelo seu
sistema ou pela variável de ambiente \textbf{OSDPROCESSORS}. Para evitar
abusos esse valor é limitado internamente a quantidade de núcleos
informados pelo seu sistema.
\begin{quote}

O valor predefinido é \textbf{auto}.
\end{quote}
\end{quote}
\phantomsection\label{commandline/commandline-all:mame-commandline-bench}
\textbf{-bench} \emph{{[}n{]}}
\begin{quote}

Define a quantidade de segundos de emulação em \emph{{[}n{]}} usado para
teste de performance, o comando é um atalho com comando abaixo:
\begin{quote}

\textbf{-str {[}n{]} -video none -sound none -nothrottle}
\end{quote}
\end{quote}


\subsection{Principais opções de rotação}
\label{commandline/commandline-all:principais-opcoes-de-rotacao}\phantomsection\label{commandline/commandline-all:mame-commandline-norotate}
\textbf{-{[}no{]}rotate}
\begin{quote}

Gira a tela para corresponder ao seu estado normal do sistema
(horizontal / vertical). Isso garante que os sistemas vertical e
horizontalmente orientados sejam exibidos corretamente sem que haja
a necessidade de girar fisicamente a sua tela.
\begin{quote}

O valor predefinido é \textbf{Ligado} (\textbf{-rotate}).
\end{quote}
\end{quote}
\phantomsection\label{commandline/commandline-all:mame-commandline-noror}\phantomsection\label{commandline/commandline-all:mame-commandline-norol}
\textbf{-{[}no{]}ror}
\textbf{-{[}no{]}rol}
\begin{quote}

Rotacione a tela do sistema para a direita (sentido horário) ou para
a esquerda (sentido anti-horário) em relação ao seu estado normal
(caso o \textbf{-rotate} seja definido) ou seu estado nativo
(caso \textbf{-norotate} for definido).
\begin{quote}

O valor predefinido para ambas as opções é \textbf{Desligado}
(\textbf{-noror} \textbf{-norol}).
\end{quote}
\end{quote}
\phantomsection\label{commandline/commandline-all:mame-commandline-noautoror}\phantomsection\label{commandline/commandline-all:mame-commandline-noautorol}
\textbf{-{[}no{]}autoror}
\textbf{-{[}no{]}autorol}
\begin{quote}

Essas opções são projetadas para uso com telas giratórias que giram
apenas em uma única direção. Caso a tela gire somente no sentido
horário, use o comando \textbf{-autorol} para garantir que o sistema
encha a tela horizontalmente ou verticalmente em uma das direções
que você pode manipular. Caso a sua tela gire somente no sentido
anti-horário, use \textbf{-autoror}.
\end{quote}
\phantomsection\label{commandline/commandline-all:mame-commandline-noflipx}\phantomsection\label{commandline/commandline-all:mame-commandline-noflipy}
\textbf{-{[}no{]}flipx}
\textbf{-{[}no{]}flipy}
\begin{quote}

Espelhe a tela do sistema horizontalmente (\textbf{-flipx}) ou
verticalmente (\textbf{-flipy}). As inversões são aplicadas depois que as
opções de rotação \textbf{-rotate} e rolagem \textbf{-ror/-rol} forem
aplicadas.
\begin{quote}

O valor predefinido para ambas as opções é \textbf{Desligado}
(\textbf{-noflipx} \textbf{-noflipy}).
\end{quote}
\end{quote}


\subsection{Principais opções de vídeo}
\label{commandline/commandline-all:principais-opcoes-de-video}\phantomsection\label{commandline/commandline-all:mame-commandline-video}
\textbf{-video} \textless{}\emph{bgfx\textbar{}gdi\textbar{}d3d\textbar{}opengl\textbar{}soft\textbar{}accel\textbar{}none}\textgreater{}
\begin{quote}

Define qual tipo de saída de vídeo usar. As opções aqui descritas
dependem do sistema operacional utilizado e se a versão do MAME é
uma versão SDL ou não.
\end{quote}
\begin{itemize}
\item {} 
\textbf{Geralmente Disponível:}
\begin{quote}

\textbf{bgfx} determina o novo renderizador acelerado por hardware.

\textbf{opengl} faz a renderização do vídeo usando a aceleração OpenGL.

\textbf{none} não exibe janelas e nem mostra nada na tela.

Essa última é usada principalmente para realizar testes de
performance do processador sem fazer uso da placa de vídeo.
\end{quote}

\item {} 
\textbf{No Windows:}
\begin{quote}

\textbf{gdi} diz ao MAME para renderizar o vídeo usando funções gráficas mais
antigas do Windows. Esta é a opção mais lenta porém a mais compatível
com as versões mais antigas do Windows.

\textbf{d3d} diz MAME para renderizar a tela com o Direct3D.
Isso produz uma saída de melhor qualidade que o gdi e permite opções
adicionais de renderização da tela. É recomendável que você tenha uma
placa de vídeo mediana (2002+) ou uma placa de vídeo Intel embutida
modelo \emph{HD3000} ou superior.
\end{quote}

\item {} 
\textbf{Em outras plataformas (incluindo o SDL no Windows):}
\begin{quote}

\textbf{accel} diz ao MAME para, se possível, processar o vídeo usando a
aceleração 2D do SDL.

\textbf{soft} faz com que a tela seja renderizada através de software.
Isso não é tão rápido ou tão bom quanto o OpenGL, mas favorece uma
melhor compatibilidade em qualquer plataforma.
\end{quote}

\item {} 
\textbf{Predefinições:}
\begin{quote}

O valor predefinido no Windows é \textbf{d3d}.

Para Mac OS X é \textbf{opengl} pois é quase certo que o Mac OS X tenha
uma pilha OpenGL compatível.
\begin{quote}

O valor predefinido para todos os outros sistemas é \textbf{soft}.
\end{quote}
\end{quote}

\end{itemize}
\phantomsection\label{commandline/commandline-all:mame-commandline-numscreens}
\textbf{-numscreens} \textless{}\emph{count}\textgreater{}
\begin{quote}

Diz ao MAME quantas telas devem ser criadas. Para a maioria dos
sistemas só exite uma, porém alguns sistemas originalmente usavam
mais de uma (\emph{como as máquinas Darius e máquinas Arcade
PlayChoice-10 por exemplo}). Cada tela (até 4), possem as suas
próprias configurações, taxa de proporção de tela, resolução e
exibição, que podem ser definidas usando as opções abaixo.
\begin{quote}

O valor predefinido é \textbf{1}.
\end{quote}
\end{quote}
\phantomsection\label{commandline/commandline-all:mame-commandline-window}
\textbf{-{[}no{]}window} / \textbf{-{[}no{]}w}
\begin{quote}

Faz o MAME exibir a tela em uma janela ou em uma tela inteira.
\begin{quote}

O valor predefinido é \textbf{Desligado} (\textbf{-nowindow}).
\end{quote}
\end{quote}
\phantomsection\label{commandline/commandline-all:mame-commandline-maximize}
\textbf{-{[}no{]}maximize} / \textbf{-{[}no{]}max}
\begin{quote}

Controla o tamanho inicial da janela no modo de janelado. Caso seja
ativado, ao iniciar o MAME a janela será configurada para o tamanho
máximo suportado. Caso esteja desativado, a janela será exibida no
menor tamanho suportado. Esta opção só tem efeito quando a opção
\textbf{-window} for usada.
\begin{quote}

O valor predefinido é \textbf{Ligado} (\textbf{-maximize}).
\end{quote}
\end{quote}
\phantomsection\label{commandline/commandline-all:mame-commandline-keepaspect}
\textbf{-{[}no{]}keepaspect} / \textbf{-{[}no{]}ka}
\begin{quote}

Faz com que a proporção de tela seja mantida. Quando essa opção está
ativa, a taxa de proporção adequada da tela do sistema é aplicada
(geralmente 4:3 ou 3:4), mantendo a proporção original do sistema.
Ao usar essa opção no modo janelado, ao redimensionar a janela ela
tentara manter as proporções originais a menos que você mantenha
pressionada a tecla \textbf{CONTROL} para que você consiga dimensionar a
janela livremente.
Desativando a opção, a proporção de tela pode ser alterada
livremente no modo janelado. Em tela cheia, isso significa que a
imagem vai preencher toda a tela (até mesmo em sistemas verticais)
de maneira desproporcional.
\begin{quote}

O valor predefinido é \textbf{Ligado} (\textbf{-keepaspect}).
\end{quote}

A equipe do MAME, veementemente sugere que você deixe o valor
predefinido inalterado. Esticando a tela do sistema além da
proporção original vai causar distorções na aparência do sistema
que vai além da capacidade de reparo dos filtros ou HLSL.
\end{quote}
\phantomsection\label{commandline/commandline-all:mame-commandline-waitvsync}
\textbf{-{[}no{]}waitvsync}
\begin{quote}

Aguarda acabar o período de atualização da tela do monitor do seu
computador antes de começar a desenhar na tela. Caso esta opção
esteja desligada, o MAME só irá desenhar na tela com tempo
posterior ou até mesmo durante um ciclo de atualização de tela. Isso
pode causar um ``screen tearing'' \footnote[6]{\sphinxAtStartFootnote%
Faz com que a metade da parte de cima da tela saia de
sincronismo com a parte de baixo, surgindo um efeito ou
um ``\emph{defeito}'' onde cada metade se desloca para lados opostos
horizontalmente. (Nota do tradutor)
}.
O efeito ``tearing'' não é perceptível em todos os sistemas, porém
algumas pessoas acham o efeito desagradável, algumas mais do que as
outras.
Entretanto, ao ativar esta opção, saiba que você desperdiçará
preciosos ciclos de CPU enquanto o mesmo espera o tempo certo para
desenhar na tela, fazendo com que a performance no geral seja
prejudicada.
Só é necessário ligar esta opção caso você jogue em modo janelado.

Em modo de tela cheia, só será necessário caso a opção
\textbf{-triplebuffer} não remova o efeito tearing, então você deve usar
as duas opções juntas \textbf{-notriplebuffer -waitvsync}. Essa opção não
funciona com a opção \textbf{-video gdi}.
\begin{quote}

O valor predefinido é \textbf{Desligado} (\textbf{-nowaitvsync}).
\end{quote}

Essa opção funcionará com o MAME SDL dependendo exclusivamente do
seu sistema operacional e dos drivers da sua placa de vídeo que no
geral não funcionam em modo janelado, portanto você obterá maior
chances de sucesso ao usar o modo de tela inteira com a opção
\textbf{-video opengl}.
\end{quote}
\phantomsection\label{commandline/commandline-all:mame-commandline-syncrefresh}
\textbf{-{[}no{]}syncrefresh}
\begin{quote}

Ativa o controle de velocidade da taxa de atualização do seu
monitor. Isso significa que a taxa de atualização usada pelo sistema
é ignorada, porém, o código responsável pelo som tentará manter o
sincronismo com a taxa de atualização usada pelo sistema, assim
haverá problemas com o som. Essa opção foi pensada naqueles que
modificaram as configurações da sua placa de vídeo, combinando uma
opção a mais com as de atualização de tela.
Essa opção não funciona com a opção \textbf{-video gdi}.
\begin{quote}

O valor predefinido é \textbf{Desligado} (\textbf{-nosyncrefresh}).
\end{quote}
\end{quote}
\phantomsection\label{commandline/commandline-all:mame-commandline-prescale}
\textbf{-prescale} \textless{}\emph{amount}\textgreater{}
\begin{quote}

Controla o tamanho das imagens na tela enquanto são repassadas para
o sistema gráfico de redimensionamento. No ajuste mínimo de \textbf{1}, a
tela é renderizada no seu tamanho original antes de ser
dimensionada. Com valores maiores a tela é expandida pelo fator
definido em \textless{}\emph{amount}\textgreater{} antes de ser dimensionado. Isso gera imagens
menos borradas com a opção \textbf{-video d3d} ao custo da perda de
alguma performance.
\begin{quote}

O valor predefinido é \textbf{1}.
\end{quote}

Funciona com todos os modos de vídeo no Windows (bgfx, d3d, etc) e
nas outras plataformas \textbf{APENAS} aquelas que forem compatíveis com
o OpenGL.
\end{quote}
\phantomsection\label{commandline/commandline-all:mame-commandline-filter}
\textbf{-{[}no{]}filter} / \textbf{-{[}no{]}d3dfilter} / \textbf{-{[}no{]}flt}
\begin{quote}

O filtro bilinear, aplica um leve efeito de embaçamento ou
suavização à tela, amenizando um pouco o serrilhado nos contornos
gráficos e suavizando a tela do sistema.
Quando desabilitado você terá uma imagem pura e com aparência mais
serrilhada e também ocasiona artefatos na tela em caso de
dimensionamento. Caso não goste da aparência filtrada e amaciada da
imagem, tente incrementar o valor da opção \textbf{-prescale} ao invés de
desabilitar todos os filtros.
\begin{quote}

O valor predefinido é \textbf{Ligado} (\textbf{-filter}).
\end{quote}

Funciona com todos os modos de vídeo (bgfx, d3d, etc) no Windows e
nas outras plataformas \textbf{APENAS} aquelas com o OpenGL.
\end{quote}
\phantomsection\label{commandline/commandline-all:mame-commandline-unevenstretch}
\textbf{-{[}no{]}unevenstretch}
\begin{quote}

Permite fatores não integrais permitindo a flexibilização no momento
do dimensionamento e o esticamento da janela.
\begin{quote}

O valor predefinido é \textbf{Ligado} (\textbf{-unevenstretch}).
\end{quote}
\end{quote}


\subsection{Principais opções de tela inteira}
\label{commandline/commandline-all:principais-opcoes-de-tela-inteira}\phantomsection\label{commandline/commandline-all:mame-commandline-switchres}
\textbf{-{[}no{]}switchres}
\begin{quote}

Ativa a alteração, comutação ou troca da resolução. Esta opção é
necessária para as opções \textbf{-resolution} evitando a troca das
resoluções enquanto estiver no modo de tela inteira. Em placas de
vídeo modernas, há poucas razões para alternar as resoluções a menos
que você esteja tentando alcançar as resoluções ``exatas'' dos pixels
dos sistemas originais, o que exige ajustes significativos.
Esta opção também é útil em monitores de LCD, uma vez que eles rodam
com uma resolução fixa e as comutações da resolução algumas vezes
são exageradas. Essa opção não funciona com a opção \textbf{-video gdi}.
\begin{quote}

O valor predefinido é \textbf{Desligado} (\textbf{-noswitchres}).
\end{quote}
\end{quote}


\subsection{Principais opções de janela individual}
\label{commandline/commandline-all:principais-opcoes-de-janela-individual}\phantomsection\label{commandline/commandline-all:mame-commandline-screen}
NOTA: \textbf{A partir de agora a opção de várias telas simultâneas podem não
funcionar corretamente em algumas máquinas Mac.}

\begin{DUlineblock}{0em}
\item[] \textbf{-screen} \textless{}\emph{display}\textgreater{}
\item[] \textbf{-screen0} \textless{}\emph{display}\textgreater{}
\item[] \textbf{-screen1} \textless{}\emph{display}\textgreater{}
\item[] \textbf{-screen2} \textless{}\emph{display}\textgreater{}
\item[] \textbf{-screen3} \textless{}\emph{display}\textgreater{}
\end{DUlineblock}
\begin{quote}

Define qual o monitor físico em seu sistema você deseja que cada
janela use por padrão. Para usar várias janelas, você deve ter
aumentado o valor da opção \textbf{-numscreens}.
O nome de cada exibição em seu sistema pode ser determinado
executando o MAME com a opção \textbf{-verbose}.
Os nomes de exibição geralmente estão no formato: \emph{\textbackslash{}\textbackslash{}.DISPLAYn},
onde \textbf{n} é um número do monitor conectado.

O valor predefinido para essas opções é \textbf{auto}.
O que significa que a primeira janela é colocada na primeira
exibição, a segunda janela na segunda exibição e assim por diante.

Os parâmetros \textbf{-screen0}, \textbf{-screen1}, \textbf{-screen2}, \textbf{-screen3}
aplicam-se as janelas definidas. O parâmetro \textbf{screen} se aplica
a todas as janelas.
As opções definidas da janela substituem os valores da opções de
todas as janelas.
\end{quote}
\phantomsection\label{commandline/commandline-all:mame-commandline-aspect}
\begin{DUlineblock}{0em}
\item[] \textbf{-aspect} \textless{}\emph{width:height}\textgreater{} / \textbf{-screen\_aspect} \textless{}\emph{num:den}\textgreater{}
\item[] \textbf{-aspect0} \textless{}\emph{width:height}\textgreater{}
\item[] \textbf{-aspect1} \textless{}\emph{width:height}\textgreater{}
\item[] \textbf{-aspect2} \textless{}\emph{width:height}\textgreater{}
\item[] \textbf{-aspect3} \textless{}\emph{width:height}\textgreater{}
\end{DUlineblock}
\begin{quote}

Define a proporção física do monitor para cada janela. Para usar
várias janelas, você deve ter aumentado o valor da opção
\textbf{-numscreens}.
A proporção física pode ser determinada medindo a largura e a altura
da imagem da tela visível e definindo-as separadas por dois pontos.
\begin{quote}

O valor predefinido para essas opções é \textbf{auto}.
\end{quote}

Significa que o MAME assume que a proporção de tela é proporcional
ao número de pixels no modo de vídeo da área de trabalho para cada
monitor.

O parâmetro \textbf{-aspect0}, \textbf{-aspect1}, \textbf{-aspect2} e \textbf{-aspect3}
se aplica a todas as janelas definidas. O parâmetro \textbf{-aspect} se
aplica a todas as janelas.
As opções definidas da janela substituem os valores da opções de
todas as janelas.
\end{quote}
\phantomsection\label{commandline/commandline-all:mame-commandline-resolution}
\begin{DUlineblock}{0em}
\item[] \textbf{-resolution} \textless{}\emph{widthxheight{[}@refresh{]}}\textgreater{} / \textbf{-r} \textless{}\emph{widthxheight{[}@refresh{]}}\textgreater{}
\item[] \textbf{-resolution0} \textless{}\emph{widthxheight{[}@refresh{]}}\textgreater{} / \textbf{-r0} \textless{}\emph{widthxheight{[}@refresh{]}}\textgreater{}
\item[] \textbf{-resolution1} \textless{}\emph{widthxheight{[}@refresh{]}}\textgreater{} / \textbf{-r1} \textless{}\emph{widthxheight{[}@refresh{]}}\textgreater{}
\item[] \textbf{-resolution2} \textless{}\emph{widthxheight{[}@refresh{]}}\textgreater{} / \textbf{-r2} \textless{}\emph{widthxheight{[}@refresh{]}}\textgreater{}
\item[] \textbf{-resolution3} \textless{}\emph{widthxheight{[}@refresh{]}}\textgreater{} / \textbf{-r3} \textless{}\emph{widthxheight{[}@refresh{]}}\textgreater{}
\end{DUlineblock}
\begin{quote}

Define a resolução exata a ser exibida. No modo de tela cheia o MAME
tentará usar a resolução solicitada. A largura e a altura são
obrigatórias, a taxa de atualização é opcional.

Caso seja omitido ou configurado para \textbf{0}, o MAME determinará o
modo automaticamente. Por exemplo, a opção \textbf{-resolution 640x480}
forçará a resolução de 640x480 porém o MAME escolherá a taxa de
atualização por conta própria.

Da mesma forma que \textbf{-resolution 0x0@60} obrigará que a taxa de
atualização seja de 60 Hz, mas permite que o MAME escolha a
resolução. O comando também funciona com ``\emph{auto}'' e é equivalente a
\emph{0x0@0}.

No modo janelado essa resolução é usada para determinar o tamanho
máximo para a janela. Essa opção também requer que seja usada a
opção \textbf{-switchres} para ativar a comutação de resolução junto com
\textbf{-video d3d}.
\begin{quote}

O valor predefinido para essas opções é \textbf{auto}.
\end{quote}

O parâmetro \textbf{-resolution0}, \textbf{-resolution1}, \textbf{-resolution2} e
\textbf{-resolution3} se aplica a todas as janelas definidas.
O parâmetro \textbf{-resolution} se aplica a todas as janelas.
As opções específicas da janela substituem os valores da opções de
todas as janelas.
\end{quote}
\phantomsection\label{commandline/commandline-all:mame-commandline-view}
\begin{DUlineblock}{0em}
\item[] \textbf{-view} \textless{}\emph{viewname}\textgreater{}
\item[] \textbf{-view0} \textless{}\emph{viewname}\textgreater{}
\item[] \textbf{-view1} \textless{}\emph{viewname}\textgreater{}
\item[] \textbf{-view2} \textless{}\emph{viewname}\textgreater{}
\item[] \textbf{-view3} \textless{}\emph{viewname}\textgreater{}
\end{DUlineblock}
\begin{quote}

Define a configuração da visualização inicial de cada janela.
Note que o nome de visualização \textless{}\emph{viewname}\textgreater{} não precisa ser uma
combinação exata, em vez disso, será selecionado a primeira exibição
cujo nome corresponde a todos os caracteres especificados por
\textless{}\emph{viewname}\textgreater{}.
Por exemplo, \textbf{-view native} corresponderá à visualização
``Native (15:14)'', mesmo que não seja uma correspondência perfeita.
O valor funciona com a opção \textbf{auto} também e solicita que o MAME
execute uma seleção predefinida.
\begin{quote}

O valor predefinido para essas opções é \textbf{auto}.
\end{quote}

Os parâmetros \textbf{-view0}, \textbf{-view1}, \textbf{-view2} e \textbf{-view3} se
aplicam a todas as janelas especificadas. O parâmetro \textbf{-view} se
aplica a todas as janelas.
As opções definidas para a janela substituem os valores da opções de
todas as janelas.
\end{quote}


\subsection{Principais opções para as ilustrações}
\label{commandline/commandline-all:principais-opcoes-para-as-ilustracoes}\phantomsection\label{commandline/commandline-all:mame-commandline-noartworkcrop}
\textbf{-{[}no{]}artwork\_crop} / \textbf{-{[}no{]}artcrop}
\begin{quote}

Ativar o recorte de arte somente na área da tela do sistema. Isso
funciona melhor com a opção \textbf{-video gdi} ou \textbf{-video d3d}
e significa que os sistemas orientados verticalmente em tela cheia
podem exibir as suas ilustrações nos lados esquerdo e direito da
tela. Essa opção também pode ser configurada pela opção de vídeo
acessada através das opções da interface do usuário.
\begin{quote}

O valor predefinido é \textbf{Desligado} (\textbf{-noartwork\_crop}).
\end{quote}
\end{quote}
\phantomsection\label{commandline/commandline-all:mame-commandline-nousebackdrops}
\textbf{-{[}no{]}use\_backdrops} / \textbf{-{[}no{]}backdrop}
\begin{quote}

Ativa ou desativa a exibição dos cenários ou pano de fundo.
\begin{quote}

O valor predefinido é \textbf{Ligado} (\textbf{-use\_backdrops}).
\end{quote}
\end{quote}
\phantomsection\label{commandline/commandline-all:mame-commandline-nouseoverlays}
\textbf{-{[}no{]}use\_overlays} / \textbf{-{[}no{]}overlay}
\begin{quote}

Ativa ou desativa a exibição de sobreposições.
\begin{quote}

O valor predefinido é \textbf{Ligado} (\textbf{-use\_overlays}).
\end{quote}
\end{quote}
\phantomsection\label{commandline/commandline-all:mame-commandline-nousebezels}
\textbf{-{[}no{]}use\_bezels} / \textbf{-{[}no{]}bezels}
\begin{quote}

Ativa ou desativa a exibição de molduras.
\begin{quote}

O valor predefinido é \textbf{Ligado} (\textbf{-use\_bezels}).
\end{quote}
\end{quote}
\phantomsection\label{commandline/commandline-all:mame-commandline-nousecpanels}
\textbf{-{[}no{]}use\_cpanels} / \textbf{-{[}no{]}cpanels}
\begin{quote}

Ativa ou desativa a exibição dos painéis de controle.
\begin{quote}

O valor predefinido é \textbf{Ligado} (\textbf{-use\_cpanels}).
\end{quote}
\end{quote}
\phantomsection\label{commandline/commandline-all:mame-commandline-nousemarquees}
\textbf{-{[}no{]}use\_marquees} / \textbf{-{[}no{]}marquees}
\begin{quote}

Ativa ou desativa a exibição de marquises ou molduras que sustentem
a arte do jogo na parte de cima da máquina.
\begin{quote}

O valor predefinido é \textbf{Ligado} (\textbf{-use\_marquees}).
\end{quote}
\end{quote}
\phantomsection\label{commandline/commandline-all:mame-commandline-fallbackartwork}
\textbf{-fallback\_artwork}
\begin{quote}

Define uma ilustração alternativa caso nenhuma ilustração interna ou
externa de layout seja definida.
\end{quote}
\phantomsection\label{commandline/commandline-all:mame-commandline-overrideartwork}
\textbf{-override\_artwork}
\begin{quote}

Define uma ilustração para sobrepor a ilustração interna ou externa
de layout.
\end{quote}


\subsection{Principais opções de tela}
\label{commandline/commandline-all:principais-opcoes-de-tela}\phantomsection\label{commandline/commandline-all:mame-commandline-brightness}
\textbf{-brightness} \textless{}\emph{value}\textgreater{}
\begin{quote}

Controla o valor de brilho ou nível de preto da tela.
Essa opção não afeta a arte ou outras partes da tela. Usando a
interface interna do MAME, você pode configurar o brilho para cada
tela do sistema e para todos os sistemas individualmente.
Ao selecionar valores menores (não menor que \textbf{0.1}) produzirá uma
tela mais escura, enquanto valores maiores até \textbf{2.0} produzirão
uma tela mais clara.
\begin{quote}

O valor predefinido é \textbf{1.0}.
\end{quote}
\end{quote}
\phantomsection\label{commandline/commandline-all:mame-commandline-contrast}
\textbf{-contrast} \textless{}\emph{value}\textgreater{}
\begin{quote}

Controla o contraste da tela ou os nível de branco da tela.
Essa opção não afeta a arte ou outras partes da tela. Usando a
interface interna do MAME, você pode configurar o brilho para cada
tela do sistema e para todos os sistemas individualmente.
Essa opção define o valor inicial de todas as telas visíveis de
todos os sistemas.
Selecionando valores (não menor que \textbf{0.1}) produzirá uma tela mais
apagada, enquanto valores maiores até \textbf{2.0} produzirão uma tela
mais saturada.
\begin{quote}

O valor predefinido é \textbf{1.0}.
\end{quote}
\end{quote}
\phantomsection\label{commandline/commandline-all:mame-commandline-gamma}
\textbf{-gamma} \textless{}\emph{value}\textgreater{}
\begin{quote}

Controle de gamma, ajusta a escala de luminância da tela. Essa opção
não afeta a arte ou outras partes da tela. Usando a interface
interna do MAME, você pode configurar o gamma para cada tela do
sistema e para todos os sistemas individualmente. Essa opção define
o valor inicial de todas as telas visíveis de todos os sistemas.
Essa configuração oferece um ajuste de luminância linear de preto
para o branco. Ao selecionar valores menores (até \textbf{0.1})
trará a luminância mais para o preto, enquanto valores maiores
(até \textbf{3.0}) empurrarão essa luminância para o branco.
\begin{quote}

O valor predefinido é \textbf{1.0}.
\end{quote}
\end{quote}
\phantomsection\label{commandline/commandline-all:mame-commandline-pausebrightness}
\textbf{-pause\_brightness} \textless{}\emph{value}\textgreater{}
\begin{quote}

Faz o controle do nível de brilho durante a pausa.
\begin{quote}

O valor predefinido é \textbf{0.65}.
\end{quote}
\end{quote}
\phantomsection\label{commandline/commandline-all:mame-commandline-effect}
\textbf{-effect} \textless{}\emph{filename}\textgreater{}
\begin{quote}

Define um único arquivo PNG que será usado como sobreposição na tela
de qualquer sistema. Presume-se que o aquivo PNG esteja em um dos
diretórios raiz do artpath. Ambas as combinações horizontais e
verticais dentro do arquivo PNG é repetido para cobrir toda a tela
(mas nenhuma parte da arte externa).
Ela é renderizada na resolução nativa do sistema. Para os modos de
vídeo \textbf{-video gdi} e \textbf{-video d3d} significa que um pixel dentro
do PNG será mapeado para um pixel da sua tela. Os valores RGB de
cada pixel dentro do PNG são multiplicados com os valores de RGB da
tela de destino.
\begin{quote}

O valor predefinido é \textbf{none} ou nenhum efeito.
\end{quote}
\end{quote}


\subsection{Principais opções para vetores}
\label{commandline/commandline-all:principais-opcoes-para-vetores}\phantomsection\label{commandline/commandline-all:mame-commandline-beamwidthmin}
\textbf{-beam\_width\_min} \textless{}\emph{width}\textgreater{}
\begin{quote}

Define a espessura mínima do feixe do vetor.
\end{quote}
\phantomsection\label{commandline/commandline-all:mame-commandline-beamwidthmax}
\textbf{-beam\_width\_max} \textless{}\emph{width}\textgreater{}
\begin{quote}

Define a espessura máxima do feixe do vetor.
\end{quote}
\phantomsection\label{commandline/commandline-all:mame-commandline-beamintensityweight}
\textbf{-beam\_intensity\_weight} \textless{}\emph{weight}\textgreater{}
\begin{quote}

Define a intensidade do feixe do vetor.
\end{quote}
\phantomsection\label{commandline/commandline-all:mame-commandline-flicker}
\textbf{-flicker} \textless{}\emph{value}\textgreater{}
\begin{quote}

Simula um vetor de efeito de ``tremulação'' ou oscilação da tela
semelhante aos monitores desregulados usados nos jogos vetoriais.
Essa opção espera um valor flutuante (float) no intervalo
entre \textbf{0.00} e \textbf{100.00} (\textbf{0} = nenhum e \textbf{100} = máximo).
\begin{quote}

O valor predefinido é \textbf{0}.
\end{quote}
\end{quote}


\subsection{Principais opções para a depuração de vídeo OpenGL}
\label{commandline/commandline-all:principais-opcoes-para-a-depuracao-de-video-opengl}
Essas são as opções compatíveis com \textbf{-video opengl}.
Caso você note artefatos renderizados na tela, poderá ser solicitado
pelos desenvolvedores que você tente alterá-los, porém normalmente esses
os valores devem ser mantidos em seus valores originais para que se
obtenha a melhor performance possível.
\phantomsection\label{commandline/commandline-all:mame-commandline-glforcepow2texture}
\textbf{-{[}no{]}gl\_forcepow2texture}
\begin{quote}

Sempre utilize a potência de 2 para o tamanhos das texturas.
\begin{quote}

O valor predefinido é \textbf{Desligado}
(\textbf{-nogl\_forcepow2texture}).
\end{quote}
\end{quote}
\phantomsection\label{commandline/commandline-all:mame-commandline-glnotexturerect}
\textbf{-{[}no{]}gl\_notexturerect}
\begin{quote}

Não use o \emph{OpenGL GL\_ARB\_texture\_rectangle}
\begin{quote}

O valor predefinido é \textbf{Ligado} (\textbf{-gl\_notexturerect}).
\end{quote}
\end{quote}
\phantomsection\label{commandline/commandline-all:mame-commandline-glvbo}
\textbf{-{[}no{]}gl\_vbo}
\begin{quote}

Ative o \emph{OpenGL VBO} (Vertex Buffer Objects) caso esteja disponível.
\begin{quote}

O valor predefinido é \textbf{Ligado} (\textbf{-gl\_vbo}).
\end{quote}
\end{quote}
\phantomsection\label{commandline/commandline-all:mame-commandline-glpbo}
\textbf{-{[}no{]}gl\_pbo}
\begin{quote}

Ativar o \emph{OpenGL PBO} (Pixel Buffer Objects) caso esteja disponível.
\begin{quote}

O valor predefinido é \textbf{Ligado} (\textbf{-gl\_pbo}).
\end{quote}
\end{quote}


\subsection{Principais opções de vídeo OpenGL GLSL}
\label{commandline/commandline-all:principais-opcoes-de-video-opengl-glsl}\phantomsection\label{commandline/commandline-all:mame-commandline-glglsl}
\textbf{-{[}no{]}gl\_glsl}
\begin{quote}

Ativar o \emph{OpenGL GLSL} caso esteja disponível.
\begin{quote}

O valor predefinido é \textbf{Desligado} (\textbf{-nogl\_glsl}).
\end{quote}
\end{quote}
\phantomsection\label{commandline/commandline-all:mame-commandline-glglslfilter}
\textbf{-gl\_glsl\_filter}
\begin{quote}

Habilite a filtragem \emph{OpenGL GLSL} em vez da filtragem FF
\emph{0-simples, 1-bilinear, 2-bicúbica}
\begin{quote}

O valor predefinido é \textbf{1} (\textbf{-gl\_glsl\_filter 1}).
\end{quote}
\end{quote}
\phantomsection\label{commandline/commandline-all:mame-commandline-glslshadermame}
\begin{DUlineblock}{0em}
\item[] \textbf{-glsl\_shader\_mame0}
\item[] \textbf{-glsl\_shader\_mame1}
\item[] ...
\item[] \textbf{-glsl\_shader\_mame9}
\end{DUlineblock}
\begin{quote}

O shader personalizado do OpenGL GLSL configura o bitmap do MAME no
slot fornecido entre (\emph{0-9}). É possível aplicar um para a cada slot.

A ser feito: Descrever mais detalhes sobre a utilização em algum
momento no futuro. Veja:
\url{http://forums.bannister.org/ubbthreads.php?ubb=showflat\&Number=100988\#Post100988}\textgreater{} para maiores informações.
\end{quote}
\phantomsection\label{commandline/commandline-all:mame-commandline-glslshaderscreen}
\begin{DUlineblock}{0em}
\item[] \textbf{-glsl\_shader\_screen0}
\item[] \textbf{-glsl\_shader\_screen1}
\item[] ...
\item[] \textbf{-glsl\_shader\_screen9}
\end{DUlineblock}
\begin{quote}

O shader personalizado de tela do OpenGL GLSL configura o bitmap do
MAME no slot fornecido entre (0-9).

A ser feito: Descrever mais detalhes sobre a utilização em algum
momento no futuro. Veja:

\url{http://forums.bannister.org/ubbthreads.php?ubb=showflat\&Number=100988\#Post100988} para maiores informações.
\end{quote}
\phantomsection\label{commandline/commandline-all:mame-commandline-glglslvidattr}
\textbf{-gl\_glsl\_vid\_attr}
\begin{quote}

Ative o manuseio do GLSL em OpenGL de brilho e contraste.
Melhor desempenho do sistema RGB.
\begin{quote}

O valor predefinido é \textbf{Ligado} (\textbf{-gl\_glsl\_vid\_attr}).
\end{quote}
\end{quote}


\subsection{Principais opções de áudio}
\label{commandline/commandline-all:principais-opcoes-de-audio}\phantomsection\label{commandline/commandline-all:mame-commandline-samplerate}
\textbf{-samplerate} \textless{}\emph{value}\textgreater{} / \textbf{-sr} \textless{}\emph{value}\textgreater{}
\begin{quote}

Define a taxa de amostragem do áudio. Valores menores como 11025 por
exemplo, reduzem a qualidade da áudio porém a performance da
emulação melhora.
Valores maiores que 48000, aumentam a qualidade do áudio ao custo da
perda de performance da emulação.
\begin{quote}

O valor predefinido é \textbf{48000} (\textbf{-samplerate 48000}).
\end{quote}
\end{quote}
\phantomsection\label{commandline/commandline-all:mame-commandline-nosamples}
\textbf{-{[}no{]}samples}
\begin{quote}

Usar amostras caso estejam disponíveis.
\begin{quote}

O valor predefinido é \textbf{Ligado} (\textbf{-samples}).
\end{quote}
\end{quote}
\phantomsection\label{commandline/commandline-all:mame-commandline-volume}
\textbf{-volume} / \textbf{-vol} \textless{}\emph{value}\textgreater{}
\begin{quote}

Define o volume inicial. Pode ser alterado posteriormente usando
a interface do usuário.
O valor do volume está definido em decibéis (dB): Por exemplo,
``\textbf{-volume -12}'' começará com uma atenuação de -12 dB no som.
\begin{quote}

O valor predefinido é \textbf{0} (\textbf{-volume 0}).
\end{quote}
\end{quote}
\phantomsection\label{commandline/commandline-all:mame-commandline-sound}
\textbf{-sound} \textless{}\emph{dsound\textbar{}sdl\textbar{}coreaudio\textbar{}xaudio\textbar{}portaudio\textbar{}none}\textgreater{}
\begin{quote}

Define qual o tipo de saída de áudio usar. \textbf{none} desativa o áudio
completamente.
\begin{quote}

O valor predefinido é \textbf{dsound} no Windows, no Mac é
\textbf{coreaudio} nas outras plataformas é \textbf{sdl}.
\end{quote}

No Windows e no Linux a opção \textbf{portaudio} provavelmente dará uma
menor latência possível, enquanto no Mac a opção \textbf{coreaudio}
oferecerá os melhores resultados.
\end{quote}
\phantomsection\label{commandline/commandline-all:mame-commandline-audiolatency}
\textbf{-audio\_latency} \textless{}\emph{value}\textgreater{}
\begin{quote}

Controla a quantidade de latência (atraso) incorporada no streaming
de áudio. É predefinido que o MAME tente manter a memória intermédia
(buffer) do áudio do DirectSound cheia entre 1/5 e 2/5.
Em alguns sistemas, isso poderá ficar muito próximo do limite, o que
ocasiona em algumas vezes, um som ruim. O parâmetro de latência
controla o limite inferior.
\begin{quote}

O valor predefinido é \textbf{1} (significando inferior=1/5 e
superior=2/5). Para manter a memória intermédia sempre cheia entre
2/5 e 3/5, defina o valor para \textbf{2} (\textbf{-audio\_latency 2}).
Caso você exagere nesse valor, como \textbf{4} por exemplo, você um
notará um atraso significativo no som.
\end{quote}
\end{quote}
\clearpage

\subsection{Principais opções de entrada}
\label{commandline/commandline-all:principais-opcoes-de-entrada}\phantomsection\label{commandline/commandline-all:mame-commandline-nocoinlockout}
\textbf{-{[}no{]}coin\_lockout} / \textbf{-{[}no{]}coinlock}
\begin{quote}

Permite a simulação do recurso ``bloqueio de ficha'' implementado em
vários PCBs de jogos de arcade. Cabia ao operador saber se as saídas
de bloqueio da moeda estavam realmente conectadas aos mecanismos das
moedas. Se esse recurso estiver ativado, as tentativas de inserir
uma moeda enquanto o bloqueio estiver ativo falharão e exibirão uma
mensagem na tela (no modo de depuração). Caso esta função esteja
desativada, o sinal de bloqueio da moeda será ignorado.
\begin{quote}

O valor predefinido é \textbf{Ligado} (\textbf{-coin\_lockout}).
\end{quote}
\end{quote}
\phantomsection\label{commandline/commandline-all:mame-commandline-ctrlr}
\textbf{-ctrlr} \textless{}\emph{controller}\textgreater{}
\begin{quote}

Ativa o suporte para controladores especiais. Os arquivos de
configuração são carregados do \emph{ctrlrpath}. Eles estão no mesmo
formato dos arquivos .cfg, mas somente os dados de configuração de
controle são lidos do arquivo.
\begin{quote}

O valor predefinido é \textbf{NULO} (nenhum arquivo de controle)
\end{quote}
\end{quote}
\phantomsection\label{commandline/commandline-all:mame-commandline-nomouse}
\textbf{-{[}no{]}mouse}
\begin{quote}

Controla se o MAME faz uso ou não dos controladores do mouse.
Se estiver ligado o mouse ficará reservado para uso exclusivo do
MAME até que você saia ou pause a emulação.
\begin{quote}

O valor predefinido é \textbf{Desligado} (\textbf{-nomouse}).
\end{quote}
\end{quote}
\phantomsection\label{commandline/commandline-all:mame-commandline-nojoystick}
\textbf{-{[}no{]}joystick} / \textbf{-{[}no{]}joy}
\begin{quote}

Controla se o MAME usa ou não os controles do joystick/gamepad.
Se estiver ligado o MAME perguntará ao DirectInput sobre quais
controles estão conectados atualmente.
\begin{quote}

O valor predefinido é \textbf{Desligado} (\textbf{-nojoystick}).
\end{quote}
\end{quote}
\phantomsection\label{commandline/commandline-all:mame-commandline-nolightgun}
\textbf{-{[}no{]}lightgun} / \textbf{-{[}no{]}gun}
\begin{quote}

Controla se o MAME usa ou não os controles da pistola de luz
(lightgun). Observe que a maioria das pistolas de luz são mapeadas
para o mouse, assim, ao se usar ambas as opções \textbf{-lightgun} e
\textbf{-mouse} juntos, isso pode poderá trazer resultados inesperados.
\begin{quote}

O valor predefinido é \textbf{Desligado} (\textbf{-nolightgun}).
\end{quote}
\end{quote}
\phantomsection\label{commandline/commandline-all:mame-commandline-nomultikeyboard}
\textbf{-{[}no{]}multikeyboard} / \textbf{-{[}no{]}multikey}
\begin{quote}

Determina se o MAME diferencia entre os vários teclados disponíveis.
Alguns sistemas podem reportar mais de um teclado; por padrão, os
dados de todos esses teclados são combinados para que pareçam um só.
Ativando essa opção permitirá que o MAME retorne quais teclas foram
pressionadas em diferentes teclados de maneira independente.
\begin{quote}

O valor predefinido é \textbf{Desligado} (\textbf{-nomultikeyboard}).
\end{quote}
\end{quote}
\phantomsection\label{commandline/commandline-all:mame-commandline-nomultimouse}
\textbf{-{[}no{]}multimouse}
\begin{quote}

Determina se o MAME diferencia entre os vários mouses disponíveis.
Alguns sistemas podem reportar mais de um dispositivo de mouse;
por padrão, os dados de todos esses mouses são combinados para que
pareçam um só. Ativando esta opção fará com que o MAME relate o
movimento e o pressionar de botões do mouse em diferentes mouses de
maneira independente.
\begin{quote}

O valor predefinido é \textbf{Desligado} (\textbf{-nomultimouse}).
\end{quote}
\end{quote}
\phantomsection\label{commandline/commandline-all:mame-commandline-nosteadykey}
\textbf{-{[}no{]}steadykey} / \textbf{-{[}no{]}steady}
\begin{quote}

Alguns sistemas exigem que dois ou mais botões sejam pressionados
exatamente ao mesmo tempo para realizar movimentos ou comandos
especiais. Devido a limitação do hardware do teclado, pode ser
difícil ou até mesmo impossível de realizar usando um teclado comum.
Essa opção seleciona diferentes modos de manuseio o que torna mais
fácil registrar o pressionamento simultâneo das teclas, porém tem a
desvantagem de deixar a sua capacidade de resposta mais lenta.
\begin{quote}

O valor predefinido é \textbf{Desligado} (\textbf{-nosteadykey}).
\end{quote}
\end{quote}
\phantomsection\label{commandline/commandline-all:mame-commandline-uiactive}
\textbf{-{[}no{]}ui\_active}
\begin{quote}

Habilita a opção para que a interface do usuário se sobreponha a do
teclado emulado caso esteja presente.
\begin{quote}

O valor predefinido é \textbf{Desligado} (\textbf{-noui\_active}).
\end{quote}
\end{quote}
\phantomsection\label{commandline/commandline-all:mame-commandline-nooffscreenreload}
\textbf{-{[}no{]}offscreen\_reload} / \textbf{-{[}no{]}reload}
\begin{quote}

Controla se o MAME trata o segundo botão da pistola de luz
(lightgun) como um sinal para recarregar a arma. Neste caso, o MAME
reportará a posição da arma como (\textbf{0,MAX}) com o gatilho
pressionado, o que é o equivalente a uma recarga da arma com ela
apontada para fora da tela. Isso só é necessário para jogos que
precisam que o usuário atire para fora da tela para recarregar a
arma e se também a sua arma não tiver essa funcionalidade.
\begin{quote}

O valor predefinido é \textbf{Desligado} (\textbf{-nooffscreen\_reload}).
\end{quote}
\end{quote}
\phantomsection\label{commandline/commandline-all:mame-commandline-joystickmap}
\textbf{-joystick\_map} \textless{}\emph{map}\textgreater{} / \textbf{-joymap} \textless{}\emph{map}\textgreater{}
\begin{quote}

Controla como mapear os valores analógicos do controle (joystick)
para o controle (joystick) digital. O MAME aceita qualquer dado
analógico de todos os controles (joystick). Para controles
analógicos de verdade, os valores precisam ser mapeados para valores
de controles digitais com 4 direções ou 8 direções. Para fazer isso
o MAME divide o alcance do valor analógico numa grade de 9x9.
Então usa a posição do eixo (para eixos X e Y apenas), mapeia para
essa grade e procura compatibilizar a tradução para um mapa de
controle conhecido. Este parâmetro permite especificar o mapa.

O valor predefinido é \textbf{auto} o que significa que um mapa diagonal
de 4 ou 8 direções, ou um mapa diagonal 4 direções é selecionado
automaticamente com base na configuração da porta de entrada do
sistema atual.

Estes mapas são definidos como uma sequência de números e
caracteres. Sabendo que a grade é de 9x9, há um total de 81
caracteres necessários para definir um mapa completo.
Abaixo está um exemplo de um mapa para um controle (joystick) com
8 direções:
\begin{quote}

\noindent\begin{tabular}{|p{0.475\linewidth}|p{0.475\linewidth}|}
\hline

\begin{DUlineblock}{0em}
\item[] 777888999
\item[] 777888999
\item[] 777888999
\item[] 444555666
\item[] 444555666
\item[] 444555666
\item[] 111222333
\item[] 111222333
\item[] 111222333
\end{DUlineblock}
&
\begin{DUlineblock}{0em}
\item[] Note que os dígitos numéricos correspondem às chaves
\item[] em um teclado numérico. Então o `7' mapeia para cima + esquerda, o `4' mapeia
\item[] para a esquerda, o `5' mapeia para o neutro, etc. Em adição aos valores
\item[] numéricos, você pode especificar o caractere `s',
\item[] que significa `pegajoso' . Neste caso, o valor do
\item[] mapa é o mesmo que foi da última vez que um valor não pegajoso
\item[] foi lido.
\end{DUlineblock}
\\
\hline\end{tabular}

\end{quote}

Para definir o mapa para este parâmetro, você pode usar uma cadeia
de dessas linhas separadas por um `.' (que indica o fim de uma
linha), dessa maneira:
\begin{quote}

\noindent\begin{tabulary}{\linewidth}{|L|}
\hline

777888999.777888999.777888999.444555666.444555666.444555666.111222333.111222333.111222333
\\
\hline\end{tabulary}

\end{quote}

No entanto, isso pode ser reduzido usando vários atalhos compatíveis
com o parâmetro \textless{}map\textgreater{}. Caso as informações sobre uma linha estejam
ausentes, presume-se que os dados ausentes nas colunas 5-9 são
simétricos da esquerda/direita com os dados da coluna 0-4; qualquer
dados ausentes das colunas 0-4, assume-se então que estas serão
cópias dos dados anteriores. A mesma lógica se aplica a linhas
ausentes, exceto que a simetria cima/baixo seja assumida.

Usando essas abreviações o mapa com 81 caracteres pode ser
simplesmente definido por essas 11 cadeias de caracteres:
7778...4445

Olhando para a primeira linha, 7778 são apenas 4 caracteres longos.
A 5º entrada não pode usar valores simétricos então assume-se que
seja igual ao valor anterior, `8'. O 6º caractere é esquerda/direita
em simetria com o 4º caractere, resultando em `8'. O 7º caractere é
esquerda/direita em simétrica com o 3º caractere, resultando em `9'
(que é `7' invertido com esquerda/direita). Eventualmente isso
resulta numa cadeia de 777888999 na linha.

A segunda e a terceira linhas estão ausentes, portanto, elas são
consideradas idênticas à primeira linha. A quarta linha decodifica
de forma semelhante à primeira linha, produzindo 444555666.
A quinta linha está faltando, então é assumido como sendo o mesmo
que o quarto.

As três linhas restantes também estão faltando, então elas são
consideradas os espelhos cima/baixo das três primeiras linhas, dando
três linhas finais de 111222333.
\end{quote}
\phantomsection\label{commandline/commandline-all:mame-commandline-joystickdeadzone}
\textbf{-joystick\_deadzone} \textless{}\emph{value}\textgreater{} / \textbf{-joy\_deadzone} \textless{}\emph{value}\textgreater{} / \textbf{-jdz} \textless{}\emph{value}\textgreater{}
\begin{quote}

Caso você jogue com um joystick analógico ele poderá estar um pouco
fora de contro. O \textbf{-joystick\_deadzone} informa uma folga ao longo
de um eixo que você deve mover antes que o eixo comece a mudar.
Essa opção espera um valor flutuante (float) no intervalo entre
\textbf{0.0} e \textbf{1.0}. Onde \textbf{0} é o centro do joystick e \textbf{1} o
limite externo.
\begin{quote}

O valor predefinido é \textbf{0.3} (\textbf{-joystick\_deadzone 0.3}).
\end{quote}
\end{quote}
\phantomsection\label{commandline/commandline-all:mame-commandline-joysticksaturation}
\textbf{-joystick\_saturation} \textless{}\emph{value}\textgreater{} / \textbf{joy\_saturation} \textless{}\emph{value}\textgreater{} / \textbf{-jsat} \textless{}\emph{value}\textgreater{}
\begin{quote}

Caso você jogue com um joystick analógico as extremidades podem
estar um pouco fora e podem não corresponder nas direções + /.
O \textbf{-joystick\_saturation} define se uma folga no movimento do eixo
será aceita até que se atinja o alcance máximo. Essa opção espera um
valor flutuante (float) no intervalo entre \textbf{0.0} até \textbf{1.0} onde
\textbf{0} é o centro do joystick e \textbf{1} é o limite externo.
\begin{quote}

O valor predefinido é \textbf{0.85} (\textbf{-joystick\_saturation 0.85}).
\end{quote}
\end{quote}
\phantomsection\label{commandline/commandline-all:mame-commandline-natural}
\textbf{-natural}
\begin{quote}

Permite que o usuário defina se deve ou não usar um teclado natural.
Isso permite que você inicie seu sistema em um modo `nativo'
dependendo da sua região, permitindo compatibilidade para teclados
fora do padrão ``QWERTY''.

O valor predefinido é \textbf{Desligado} (\textbf{-nonatural}).

No modo de ``teclado emulado'' (predefinido) o MAME traduz o
pressionamento/liberação de teclas/botões do host para
pressionamentos emulados de tecla. Quando você pressiona/solta uma
tecla/botão mapeado para uma tecla emulada, o MAME pressiona/libera
a tecla emulada.

No modo ``teclado natural'', o MAME tenta traduzir os caracteres para
as teclas digitadas. O sistema operacional traduz pressionamentos
de tecla a caracteres (da mesma forma quando você digita em um
editor de texto) e o MAME tenta traduzir esses caracteres para
pressionamentos de tecla emulados.
\end{quote}

\textbf{Existem várias limitações inevitáveis no modo ``teclado natural'':}
\begin{itemize}
\item {} 
O driver do sistema emulado ou do dispositivo de teclado precisam
ser compatíveis e haver suporte para eles.

\item {} 
O teclado selecionado \textbf{deve} corresponder ao layout do teclado
selecionado no sistema operacional emulado!

\item {} 
As teclas que não produzam caracteres não podem ser traduzidas.

\item {} 
Segurar uma tecla até que o caractere se repitam fará com que a
tecla emulada seja pressionada repetidamente em vez de ser mantida
pressionada.

\item {} 
As sequências de chaves inativas na melhor das hipóteses, são
complicadas de se usar.

\item {} 
Não funcionará se a edição do IME estiver envolvida como
Chinês/Japonês/Coreano por exemplo)

\end{itemize}
\phantomsection\label{commandline/commandline-all:mame-commandline-joystickcontradictory}
\textbf{-joystick\_contradictory}
\begin{quote}

Aceita a entrada de comandos contraditórios e simultâneos no
controle digital como \textbf{Esquerda e Direita} ou \textbf{Cima e Baixo} ao
mesmo tempo.
\begin{quote}

O valor predefinido é \textbf{Desligado}
(\textbf{-nojoystick\_contradictory})
\end{quote}
\end{quote}
\phantomsection\label{commandline/commandline-all:mame-commandline-coinimpulse}
\textbf{-coin\_impulse} \emph{{[}n{]}}
\begin{quote}

Define o tempo de impulso da moeda com base em \emph{n} (\textbf{n\textless{}0}
desabilita, \textbf{n==0} obedeça o driver, \textbf{0\textless{}n} defina o tempo em
\emph{n}).
\begin{quote}

O valor predefinido é \textbf{0} (\textbf{-coin\_impulse 0}).
\end{quote}
\end{quote}


\subsection{Principais opções de entrada automaticamente ativas}
\label{commandline/commandline-all:principais-opcoes-de-entrada-automaticamente-ativas}\phantomsection\label{commandline/commandline-all:mame-commandline-paddledevice}
\textbf{-paddle\_device}
\begin{quote}

ativa (\emph{none\textbar{}keyboard\textbar{}mouse\textbar{}lightgun\textbar{}joystick})
caso haja um controle de pá ou remo presente.
\end{quote}
\phantomsection\label{commandline/commandline-all:mame-commandline-adstickdevice}
\textbf{-adstick\_device}
\begin{quote}

ativa (\emph{none\textbar{}keyboard\textbar{}mouse\textbar{}lightgun\textbar{}joystick})
caso haja um controle analógico presente.
\end{quote}
\phantomsection\label{commandline/commandline-all:mame-commandline-pedaldevice}
\textbf{-pedal\_device}
\begin{quote}

ativa (\emph{none\textbar{}keyboard\textbar{}mouse\textbar{}lightgun\textbar{}joystick})
caso haja um controle de pedal presente.
\end{quote}
\phantomsection\label{commandline/commandline-all:mame-commandline-dialdevice}
\textbf{-dial\_device}
\begin{quote}

ativa (\emph{none\textbar{}keyboard\textbar{}mouse\textbar{}lightgun\textbar{}joystick})
caso haja um controle de um discador presente.
\end{quote}
\phantomsection\label{commandline/commandline-all:mame-commandline-trackballdevice}
\textbf{-trackball\_device}
\begin{quote}

ativa (\emph{none\textbar{}keyboard\textbar{}mouse\textbar{}lightgun\textbar{}joystick})
caso haja um controle de trackball presente.
\end{quote}
\phantomsection\label{commandline/commandline-all:mame-commandline-lightgundevice}
\textbf{-lightgun\_device}
\begin{quote}

ativa (\emph{none\textbar{}keyboard\textbar{}mouse\textbar{}lightgun\textbar{}joystick})
caso haja um controle de pistola de luz presente.
\end{quote}
\phantomsection\label{commandline/commandline-all:mame-commandline-positionaldevice}
\textbf{-positional\_device}
\begin{quote}

ativa (\emph{none\textbar{}keyboard\textbar{}mouse\textbar{}lightgun\textbar{}joystick})
caso haja um controle de posição presente.
\end{quote}
\phantomsection\label{commandline/commandline-all:mame-commandline-mousedevice}
\textbf{-mouse\_device}
\begin{quote}

ativa (\emph{none\textbar{}keyboard\textbar{}mouse\textbar{}lightgun\textbar{}joystick})
caso haja um controle de mouse presente.

Cada uma dessas opções de controle são habilitadas automaticamente
para o mouse, controle (joystick) ou pistola de luz (lightgun)
dependendo de uma classe em particular de controle analógico para um
sistema em particular. Por exemplo, se você definir a opção
\textbf{-paddle mouse} então qualquer jogo que tenha um remo ou pá como
controle será automaticamente configurada para ser usada pelo mouse
da mesma maneira como se você tivesse especificado explicitamente
com a opção \textbf{-mouse}.

Observe que estes controles sobrescrevem as opções \textbf{-{[}no{]}mouse}
\textbf{-{[}no{]}joystick}, etc.
\end{quote}


\subsection{Opções de depuração}
\label{commandline/commandline-all:opcoes-de-depuracao}\phantomsection\label{commandline/commandline-all:mame-commandline-verbose}
\textbf{-{[}no{]}verbose} / \textbf{-{[}no{]}v}
\begin{quote}

Este é o \textbf{modo loquaz} \footnote[7]{\sphinxAtStartFootnote%
Tagarela, que verbaliza muito, falador. (Nota do tradutor)
}, exibe todas as informações de
diagnósticos disponíveis.
Essas informações são úteis para apurar qualquer tipo de problemas
com a sua configuração ou qualquer outra que possa aparecer.
IMPORTANTE: favor rodar com \textbf{mame -verbose} e incluir a
saída junto caso você queira entrar em contato conosco para relatar
um erro.
\begin{quote}

O valor predefinido é \textbf{Desligado} (\textbf{-noverbose}).
\end{quote}
\end{quote}
\phantomsection\label{commandline/commandline-all:mame-commandline-oslog}
\textbf{-{[}no{]}oslog}
\begin{quote}

Escreve uma saída de dados no arquivo error.log para o depurador do
sistema.
\begin{quote}

O valor predefinido é \textbf{Desligado} (\textbf{-nooslog}).
\end{quote}
\end{quote}
\phantomsection\label{commandline/commandline-all:mame-commandline-log}
\textbf{-{[}no{]}log}
\begin{quote}

Cria um arquivo chamado error.log que contém todos os registros de
mensagens internas gerada pelo cerne do MAME e drivers de sistema.
Isso pode ser usado ao mesmo tempo que \textbf{-oslog} para escrever os
dados de saída de ambos ao mesmo tempo.
\begin{quote}

O valor predefinido é \textbf{Desligado} (\textbf{-nolog}).
\end{quote}
\end{quote}
\phantomsection\label{commandline/commandline-all:mame-commandline-debug}
\textbf{-{[}no{]}debug}
\begin{quote}

Habilita o depurador embutido no MAME. É predefinido que o depurador
entre em ação ao pressionar a tela til (\textbf{\textasciitilde{}}) \footnote[8]{\sphinxAtStartFootnote%
Até que o teclado \textbf{ABNT-2} seja mapeado pela equipe do MAMEDev,
essa tecla fica do lado esquerdo da tecla 1, logo abaixo da
tecla ESQ. (Nota do tradutor)
} durante a
emulação.
Ele também entra em ação imediatamente ao iniciar a emulação.
\begin{quote}

O valor predefinido é \textbf{Desligado} (\textbf{-nodebug}).
\end{quote}
\end{quote}
\phantomsection\label{commandline/commandline-all:mame-commandline-debugscript}
\textbf{-debugscript} \textless{}\emph{filename}\textgreater{}
\begin{quote}

Define um arquivo que vai conter a lista de comandos de depuração a
serem executados no momento da inicialização.
\begin{quote}

O valor predefinido é \textbf{NULO} (nenhum comando).
\end{quote}
\end{quote}
\phantomsection\label{commandline/commandline-all:mame-commandline-updateinpause}
\textbf{-{[}no{]}update\_in\_pause}
\begin{quote}

Habilita a atualização do bitmap inicial da tela enquanto o sistema
estiver pausado. Isso significa que a opção de retorno
\textbf{VIDEO\_UPDATE} sempre será chamada durante a pausa, o que pode ser
útil durante a depuração.

O valor predefinido é \textbf{Desligado} (\textbf{-noupdate\_in\_pause}).
\end{quote}
\phantomsection\label{commandline/commandline-all:mame-commandline-watchdog}
\textbf{-watchdog} \textless{}\emph{duration}\textgreater{} / \textbf{-wdog} \textless{}\emph{duration}\textgreater{}
\begin{quote}

Habilita o temporizador watchdog interno que vai automaticamente
matar o processo do MAME caso o tempo de duração definido em
\textless{}\emph{duration}\textgreater{} passe caso não haja nenhuma atualização de quadro.
Tenha em mente que alguns sistemas ficam parados por algum tempo
durante o carregamento da tela, então \textless{}\emph{duration}\textgreater{} deve ser grande
o suficiente para levar esse tempo extra em consideração.
Geralmente, um valor entre \textbf{10} e \textbf{30} segundos devem ser
suficientes.
\begin{quote}

Nenhum watchdog vem habilitado.
\end{quote}
\end{quote}
\phantomsection\label{commandline/commandline-all:mame-commandline-debuggerfont}
\textbf{-debugger\_font} \textless{}\emph{fontname}\textgreater{} / \textbf{-dfont} \textless{}\emph{fontname}\textgreater{}
\begin{quote}

Define o nome da fonte a ser usada nas janelas do depurador.

A fonte predefinida da janela é \textbf{Lucida Console}.
A fonte predefinida do Mac (\textbf{Cocoa}) é o padrão de fonte de
tamanho fixo do sistema (geralmente a fonte \textbf{Monaco}).
A fonte padrão do Qt é \textbf{Courier New}.
\end{quote}
\phantomsection\label{commandline/commandline-all:mame-commandline-debuggerfontsize}
\textbf{-debugger\_font\_size} \textless{}\emph{points}\textgreater{} / \textbf{-dfontsize} \textless{}\emph{points}\textgreater{}
\begin{quote}

Define o tamanho da fonte a ser usada nas janelas do depurador
em pontos.

O tamanho padrão da janela é de \textbf{9} pontos.
O tamanho padrão do Qt é de \textbf{11} pontos.
O tamanho padrão do Mac (\textbf{Cocoa}) é o tamanho padrão do sistema.
\end{quote}


\subsection{Principais opções de comunicação de rede}
\label{commandline/commandline-all:principais-opcoes-de-comunicacao-de-rede}\phantomsection\label{commandline/commandline-all:mame-commandline-commlocalhost}
\textbf{-comm\_localhost} \textless{}\emph{string}\textgreater{}
\begin{quote}

Definição para o endereço local. Este pode ser um endereço
tradicional xxx.xxx.xxx.xxx ou uma cadeia contendo um nome de host
resolvível.
\begin{quote}

O valor predefinido é \textbf{0.0.0.0}
\end{quote}
\end{quote}
\phantomsection\label{commandline/commandline-all:mame-commandline-commlocalport}
\textbf{-comm\_localport} \textless{}\emph{string}\textgreater{}
\begin{quote}

Definição da porta local. Esta pode ser qualquer porta de
comunicação tradicional como um valor inteiro non-signed com 16-bit
(\textbf{0-65535}).
\begin{quote}

O valor predefinido é \textbf{15122}.
\end{quote}
\end{quote}
\phantomsection\label{commandline/commandline-all:mame-commandline-commremotehost}
\textbf{-comm\_remotehost} \textless{}\emph{string}\textgreater{}
\begin{quote}

Definição do endereço remoto. Este pode ser um endereço tradicional
xxx.xxx.xxx.xxx ou uma cadeia contendo um nome de host resolvível.

O valor predefinido é \textbf{0.0.0.0}
\end{quote}
\phantomsection\label{commandline/commandline-all:mame-commandline-commremoteport}
\textbf{-comm\_remoteport} \textless{}\emph{string}\textgreater{}
\begin{quote}

Definição da porta remota. Esta pode ser qualquer porta de
comunicação tradicional como um valor inteiro non-signed com 16-bit
(\textbf{0-65535}).
\begin{quote}

O valor predefinido é \textbf{15122}.
\end{quote}
\end{quote}
\phantomsection\label{commandline/commandline-all:mame-commandline-commframesync}
\textbf{-{[}no{]}comm\_framesync}
\begin{quote}

Sincroniza os frames entre a rede de comunicação.
\begin{quote}

O valor predefinido é \textbf{Desligado} (\textbf{-nocomm\_framesync}).
\end{quote}
\end{quote}


\subsection{Principais opções diversas}
\label{commandline/commandline-all:principais-opcoes-diversas}\phantomsection\label{commandline/commandline-all:mame-commandline-drc}\begin{description}
\item[{\textbf{-{[}no{]}drc}}] \leavevmode
Ativa o núcleo o DRC (recompilador dinâmico) da CPU visando uma
velocidade máxima de emulação, caso esteja disponível.
\begin{quote}

O valor predefinido é \textbf{Ligado} (\textbf{-drc}).
\end{quote}

\end{description}
\phantomsection\label{commandline/commandline-all:mame-commandline-drcusec}
\textbf{-drc\_use\_c}
\begin{quote}

Force o uso de DRC usando infra-estrutura em código C.

O valor predefinido é \textbf{Desligado} (\textbf{-nodrc\_use\_c}).
\end{quote}
\phantomsection\label{commandline/commandline-all:mame-commandline-drcloguml}
\textbf{-drc\_log\_uml}
\begin{quote}

Escreva um registro descompilado DRC UML em um arquivo de registro
(log).
\begin{quote}

O valor predefinido é (\textbf{-nodrc\_log\_uml}).
\end{quote}
\end{quote}
\phantomsection\label{commandline/commandline-all:mame-commandline-drclognative}
\textbf{-drc\_log\_native}
\begin{quote}

escreva o DRC nativo num registro descompilado em assembler.
\begin{quote}

O valor predefinido é \textbf{Desligado} (\textbf{-nodrc\_log\_native}).
\end{quote}
\end{quote}
\phantomsection\label{commandline/commandline-all:mame-commandline-bios}
\textbf{-bios} \textless{}\emph{biosname}\textgreater{}
\begin{quote}

Determina qual BIOS usar no sistema a ser emulado em sistemas
que fazem uso de uma BIOS. A saída \textbf{-listxml} listará todos os
nomes das BIOS disponíveis para o sistema.
\begin{quote}

Não há valor predefinido (O MAME usará a primeira BIOS nativa
do sistema que for encontrada, caso uma esteja disponível).
\end{quote}
\end{quote}
\phantomsection\label{commandline/commandline-all:mame-commandline-cheat}
\textbf{-{[}no{]}cheat} / \textbf{-{[}no{]}c}
\begin{quote}

Ativa o cardápio de trapaças, exibindo uma lista de trapaças que
ficam armazenadas em um arquivo externo chamado \textbf{cheat.7z}.
Essa opção também habilita as opções de turbo dos botões.
\begin{quote}

O valor predefinido é \textbf{Desligado} (\textbf{-nocheat}).
\end{quote}
\end{quote}
\phantomsection\label{commandline/commandline-all:mame-commandline-skipgameinfo}
\textbf{-{[}no{]}skip\_gameinfo}
\begin{quote}

Força o MAME a não exibir a tela de informações do sistema ou jogo.
\begin{quote}

O valor predefinido é \textbf{Desligado} (\textbf{-noskip\_gameinfo}).
\end{quote}
\end{quote}
\phantomsection\label{commandline/commandline-all:mame-commandline-uifont}
\textbf{-uifont} \textless{}\emph{fontname}\textgreater{}
\begin{quote}

Define o nome da fonte ou um nome do arquivo de fonte a ser usada na
interface do usuário. Caso esta fonte não possa ser encontrada ou
não puder ser carregada, o MAME usará a sua própria fonte embutida.
Em algumas plataformas o \textless{}\emph{fontname}\textgreater{} (nome da fonte) pode ser um
nome da fonte do sistema em vez de um arquivo fonte com extensão
BDF.
\begin{quote}

O valor predefinido é \textbf{default} (O MAME usará a fonte nativa).
\end{quote}
\end{quote}
\phantomsection\label{commandline/commandline-all:mame-commandline-ui}
\textbf{-ui} \textless{}\emph{type}\textgreater{}
\begin{quote}

Define o tipo de interface do usuário a ser usada, as opções ficam
entre \emph{simple} ou \emph{cabinet}.
\begin{quote}

O valor predefinido é \textbf{Cabinet} (\textbf{-ui cabinet}).
\end{quote}
\end{quote}
\phantomsection\label{commandline/commandline-all:mame-commandline-ramsize}
\textbf{-ramsize} {[}\emph{n}{]}
\begin{quote}

Permite que você altere o tamanho padrão da RAM (caso exista suporte
para tanto no driver).
\end{quote}
\phantomsection\label{commandline/commandline-all:mame-commandline-confirmquit}
\textbf{-confirm\_quit}
\begin{quote}

Exibir um aviso na tela ``\emph{Confirmar Sair}'' antes de sair, exigindo
que o usuário confirme a ação antes de sair do MAME.
\begin{quote}

O valor predefinido é \textbf{Desligado} (\textbf{-noconfirm\_quit}).
\end{quote}
\end{quote}
\phantomsection\label{commandline/commandline-all:mame-commandline-uimouse}
\textbf{-ui\_mouse}
\begin{quote}

Exibe o ponteiro do mouse na interface do usuário do MAME.
\begin{quote}

O valor predefinido é \textbf{sem mouse} (\textbf{-noui\_mouse}).
\end{quote}
\end{quote}
\phantomsection\label{commandline/commandline-all:mame-commandline-language}
\textbf{-language} \textless{}\emph{language}\textgreater{}
\begin{quote}

Especifique um idioma para ser usado na interface do usuário, os
arquivos de tradução para cada idioma estão no caminho definido em
\textbf{languagepath}.
\end{quote}
\phantomsection\label{commandline/commandline-all:mame-commandline-nvramsave}
\textbf{-{[}no{]}nvram\_save}
\begin{quote}

Salva o conteúdo da NVRAM ao sair da emulação. Caso essa opção seja
desligada, o conteúdo que foi gravado anteriormente não será apagado
e qualquer alteração atual não será gravada.
\begin{quote}

O valor predefinido é \textbf{Ligado} (\textbf{-nvram\_save})
\end{quote}
\end{quote}
\phantomsection\label{commandline/commandline-all:mame-commandline-autobootcommand}
\textbf{-autoboot\_command} ``\textless{}\emph{command}\textgreater{}''
\begin{quote}

Cadeia de comandos que serão executados após a inicialização da
máquina (entre aspas '' ''). Para emitir uma cotação para a
emulação, use ``'''' no comando. Usando \textbf{\textbackslash{}n} irá criar uma nova
linha, emitindo o que foi digitado antes como comando.

Exemplo: \sphinxcode{-autoboot\_command "load """\$""",8,1\textbackslash{}\textbackslash{}n}
\end{quote}
\phantomsection\label{commandline/commandline-all:mame-commandline-autobootdelay}
\textbf{-autoboot\_delay} {[}\emph{n}{]}
\begin{quote}

Tempo de atraso (em segundos) para o \textbf{-autoboot\_command}.
\end{quote}
\phantomsection\label{commandline/commandline-all:mame-commandline-autobootscript}
\textbf{-autoboot\_script} / \textbf{-script} {[}\emph{filename.lua}{]}
\begin{quote}

Carrega e executa um scrit após a inicialização da máquina.
\end{quote}
\phantomsection\label{commandline/commandline-all:mame-commandline-console}
\textbf{-console}
\begin{quote}

Habilita emulador do Console Lua.
\begin{quote}

O valor predefinido é \textbf{Desligado} (\textbf{-noconsole})
\end{quote}
\end{quote}
\phantomsection\label{commandline/commandline-all:mame-commandline-plugins}\begin{description}
\item[{\textbf{-plugins}}] \leavevmode
Habilita o uso de plug-ins Lua
\begin{quote}

O valor predefinido é \textbf{Ligado} (\textbf{-plugins}).
\end{quote}

\end{description}
\phantomsection\label{commandline/commandline-all:mame-commandline-plugin}
\textbf{-plugin} {[}\emph{plugin shortname}{]}
\begin{quote}

Permite o uso de uma lista de plug-ins Lua separados por vírgula.
\end{quote}
\phantomsection\label{commandline/commandline-all:mame-commandline-noplugin}
\textbf{-noplugin} {[}\emph{plugin shortname}{]}
\begin{quote}

Permite desabilitar uma lista de plug-ins Lua separados por vírgula.
\end{quote}


\subsection{Opções do servidor HTTP}
\label{commandline/commandline-all:opcoes-do-servidor-http}\phantomsection\label{commandline/commandline-all:mame-commandline-http}\begin{description}
\item[{\textbf{-http}}] \leavevmode
Habilita o servidor de HTTP.
\begin{quote}

O valor predefinido é \textbf{Desligado} (\textbf{-nohttp}).
\end{quote}

\end{description}
\phantomsection\label{commandline/commandline-all:mame-commandline-httpport}
\textbf{-http\_port} {[}\emph{port}{]}
\begin{quote}

Define uma porta para o servidor HTTP.
\begin{quote}

O valor predefinido é \textbf{8080}.
\end{quote}
\end{quote}
\phantomsection\label{commandline/commandline-all:mame-commandline-httproot}
\textbf{-http\_root} {[}\emph{rootfolder}{]}
\begin{quote}

Define a pasta raíz para os documentos do servidor HTTP.
\begin{quote}

O valor predefinido é \textbf{web}.
\end{quote}
\end{quote}
\clearpage

\section{Opções de linha de comando específicos para a versão Windows}
\label{commandline/windowsconfig:opcoes-de-linha-de-comando-especificos-para-a-versao-windows}\label{commandline/windowsconfig::doc}
Nesta seção descrevemos todas as opções de configuração disponível
apenas para a versão nativa do MAME no Windows (não SDL).


\subsection{Opções de performance}
\label{commandline/windowsconfig:opcoes-de-performance}\phantomsection\label{commandline/windowsconfig:mame-wcommandline-priority}
\textbf{-priority} \textless{}\emph{priority}\textgreater{}
\begin{quote}

Define a prioridade de tarefas (\emph{threads}) usadas pelo MAME. Não há
nenhuma tarefa predefinida para que não haja interferência com
outros aplicativos.
Um valor válido fica entre \textbf{-15} até \textbf{1} onde \textbf{1} é a maior
prioridade.
\begin{quote}

O valor predefinido é \textbf{0} (\textbf{prioridade NORMAL}).
\end{quote}
\end{quote}
\phantomsection\label{commandline/windowsconfig:mame-wcommandline-profile}
\textbf{-profile} {[}\emph{n}{]}
\begin{quote}

Faz o uso de um perfil determinado por {[}\emph{n}{]}.
\end{quote}


\subsection{Opções de tela inteira}
\label{commandline/windowsconfig:opcoes-de-tela-inteira}\phantomsection\label{commandline/windowsconfig:mame-wcommandline-triplebuffer}
\textbf{-{[}no{]}triplebuffer} / \textbf{-{[}no{]}tb}
\begin{quote}

Ativa ou não o ``triple buffering'', buffering é o nome da técnica ou
função que faz o armazenamento prévio de dados em uma memória
preliminar. Normalmente o MAME escreve ``a seco'' diretamente na tela
sem fazer firulas com a memória preliminar. Porém com essa opção
ativa o MAME cria e escreve seus ciclos intermediários e em ordem
usando três memórias preliminares. Essa é uma maneira de tentar
manter o fluxo contínuo de dados, evitando interrupções. Dentre as
três, apenas a primeira é exibida, a segunda fica na espera sendo
acumulada e a terceira fica sendo escrita constantemente.
A opção \textbf{-triplebuffer} sobrescreve a opção \textbf{-waitvsync} caso a
memória preliminar seja criada com sucesso.

Essa opção não funciona com \textbf{-video gdi}.
\begin{quote}

O valor predefinido é \textbf{Desligado} (\textbf{-notriplebuffer}).
\end{quote}
\end{quote}
\phantomsection\label{commandline/windowsconfig:mame-wcommandline-fullscreenbrightness}
\textbf{-full\_screen\_brightness} \textless{}\emph{value}\textgreater{} / \textbf{-fsb} \textless{}\emph{value}\textgreater{}
\begin{quote}

Controla o brilho ou nível de preto da tela.
Selecionando valores menores (até \textbf{0.1}) produzirá uma tela mais
escura, enquanto valores maiores (até \textbf{2.0}) produzirão uma tela
mais clara.

Note que nem todas as placa de vídeo são compatíveis com essa opção.
Essa opção também não funciona com \textbf{-video gdi}.
\begin{quote}

O valor predefinido é \textbf{1.0}.
\end{quote}
\end{quote}
\phantomsection\label{commandline/windowsconfig:mame-wcommandline-fullscreencontrast}
\textbf{-full\_screen\_contrast} \textless{}\emph{value}\textgreater{} / \textbf{-fsc} \textless{}\emph{value}\textgreater{}
\begin{quote}

Controla o contraste ou nível de branco da tela.
Selecionando valores menores (até \textbf{0.1}) produzirá uma tela mais
apagada, enquanto valores maiores (até \textbf{2.0}) produzirão uma tela
mais saturada.

Note que nem todas as placa de vídeo são compatíveis com essa opção.
Essa opção também não funciona com \textbf{-video gdi}.
\begin{quote}

O valor predefinido é \textbf{1.0}.
\end{quote}
\end{quote}
\phantomsection\label{commandline/windowsconfig:mame-wcommandline-fullscreengamma}
\textbf{-full\_screen\_gamma} \textless{}\emph{value}\textgreater{} / \textbf{-fsg} \textless{}\emph{value}\textgreater{}
\begin{quote}

Ajuste de gama da tela, faz o ajuste da escala de luminância da
tela ajustando o contraste entre o claro e escuro.
Essa opção não afeta a arte ou outras partes da tela.

Note que nem todas as placa de vídeo são compatíveis com essa opção.
Essa opção não funciona com \textbf{-video gdi}.
\begin{quote}

O valor predefinido é \textbf{1.0}.
\end{quote}
\end{quote}


\subsection{Opções para a entrada de controle}
\label{commandline/windowsconfig:opcoes-para-a-entrada-de-controle}\phantomsection\label{commandline/windowsconfig:mame-wcommandline-duallightgun}
\textbf{-{[}no{]}dual\_lightgun} / \textbf{-{[}no{]}dual}
\begin{quote}

Controla se o MAME tenta ou não rastrear duas pistolas de luz
conectadas ao mesmo tempo. Essa opção requer \textbf{-lightgun}. Essa
opção é um quebra galho para ser compatível com certos tipos antigos
de pistolas de luz. Se você possuí múltiplas pistolas de luz
conectadas, basta apenas usar a opção \textbf{-mouse} e configurar cada
pistola individualmente.
\begin{quote}

O valor predefinido é \textbf{Desligado} (\textbf{-nodual\_lightgun}).
\end{quote}
\end{quote}


\section{Opções de linha de comando específicos para a versão SDL}
\label{commandline/sdlconfig:opcoes-de-linha-de-comando-especificos-para-a-versao-sdl}\label{commandline/sdlconfig::doc}
Nesta seção descreveremos as opções de configuração voltadas
especificamente para qualquer versão compatível com o SDL (incluindo o
Windows caso o MAME tenha sido compilado com o SDL ao invés da sua forma
nativa).


\subsection{Opções de performance}
\label{commandline/sdlconfig:opcoes-de-performance}\phantomsection\label{commandline/sdlconfig:mame-scommandline-sdlvideofps}
\textbf{-sdlvideofps}
\begin{quote}

Ativa a saída de dados para benchmark no subsistema de vídeo SDL
incluindo o driver de vídeo do seu sistema, o servidor X (caso seja
aplicável) e stack Opengl em modo \textbf{-video opengl}.
\end{quote}


\subsection{Opções de vídeo}
\label{commandline/sdlconfig:opcoes-de-video}\phantomsection\label{commandline/sdlconfig:mame-scommandline-centerh}
\textbf{-{[}no{]}centerh}
\begin{quote}

Centraliza o eixo horizontal da tela.
\begin{quote}

O valor predefinido é \textbf{Ligado} (\textbf{-centerh}).
\end{quote}
\end{quote}
\phantomsection\label{commandline/sdlconfig:mame-scommandline-centerv}
\textbf{-{[}no{]}centerv}
\begin{quote}

Centraliza o eixo vertical da tela.
\begin{quote}

O valor predefinido é \textbf{Ligado} (\textbf{-centerv}).
\end{quote}
\end{quote}


\subsection{Opções de software específicos para vídeo}
\label{commandline/sdlconfig:opcoes-de-software-especificos-para-video}\phantomsection\label{commandline/sdlconfig:mame-scommandline-scalemode}
\textbf{-scalemode}
\begin{quote}

Modos de escala da família de espaços de cor: \textbf{none}, \textbf{async},
\textbf{yv12}, \textbf{yuy2}, \textbf{yv12x2}, \textbf{yuy2x2} (apenas com \textbf{-video
soft}).
\begin{quote}

O valor predefinido é \textbf{none} (nenhum).
\end{quote}
\end{quote}


\subsection{Mapeamento do teclado SDL}
\label{commandline/sdlconfig:mapeamento-do-teclado-sdl}\phantomsection\label{commandline/sdlconfig:mame-scommandline-keymap}
\textbf{-keymap}
\begin{quote}

Ativa o mapa de teclado.
\begin{quote}

O valor predefinido é \textbf{Desligado} (\textbf{-nokeymap}).
\end{quote}
\end{quote}
\phantomsection\label{commandline/sdlconfig:mame-scommandline-keymapfile}
\textbf{-keymap\_file} \textless{}\emph{file}\textgreater{}
\begin{quote}

Nome do aquivo de mapa de teclado.

O valor predefinido é \textbf{keymap.dat}.
\end{quote}


\subsection{Mapeamento de joystick SDL}
\label{commandline/sdlconfig:mapeamento-de-joystick-sdl}\phantomsection\label{commandline/sdlconfig:mame-scommandline-joyidx}
\begin{DUlineblock}{0em}
\item[] \textbf{-joy\_idx1} \textless{}\emph{name}\textgreater{}
\item[] \textbf{-joy\_idx2} \textless{}\emph{name}\textgreater{}
\item[] ...
\item[] \textbf{-joy\_idx8} \textless{}\emph{name}\textgreater{}
\end{DUlineblock}
\begin{quote}

Nome do controle joystick mapeado para um determinado slot de
joystick.
\begin{quote}

O valor predefinido é \textbf{auto}.
\end{quote}
\end{quote}
\phantomsection\label{commandline/sdlconfig:mame-scommandline-sixaxis}
\textbf{-sixaxis}
\begin{quote}

Usar um tratamento especial para lidar com os controles SixAxis do
PS3.
\begin{quote}

O valor predefinido é \textbf{Desligado} (\textbf{-nosixaxis})
\end{quote}
\end{quote}


\subsection{Opções de baixo nível para drivers SDL}
\label{commandline/sdlconfig:opcoes-de-baixo-nivel-para-drivers-sdl}\phantomsection\label{commandline/sdlconfig:mame-scommandline-videodriver}
\textbf{-videodriver} \textless{}\emph{driver}\textgreater{}
\begin{quote}

Define o driver de vídeo SDL a ser usado (\textbf{x11}, \textbf{directfb} ou
\textbf{auto}).
\begin{quote}

O valor predefinido é \textbf{auto}
\end{quote}
\end{quote}
\phantomsection\label{commandline/sdlconfig:mame-scommandline-audiodriver}
\textbf{-audiodriver} \textless{}\emph{driver}\textgreater{}
\begin{quote}

Define o driver de áudio SDL a ser usado (\textbf{alsa}, \textbf{arts} ou
\textbf{auto}).
\begin{quote}

O valor predefinido é \textbf{auto}
\end{quote}
\end{quote}
\phantomsection\label{commandline/sdlconfig:mame-scommandline-gllib}
\textbf{-gl\_lib} \textless{}\emph{driver}\textgreater{}
\begin{quote}

Define o \textbf{libGL.so} alternativo a ser usado.
\begin{quote}

O valor predefinido para o sistema é \textbf{auto}
\end{quote}
\end{quote}


\section{Indice das opções de linha de comando}
\label{commandline/commandline-index::doc}\label{commandline/commandline-index:indice-das-opcoes-de-linha-de-comando}\label{commandline/commandline-index:index-commandline}
Este é o índice completo de todas as opções de linha de comandos para o
MAME, muito conveniente para a localização rápida de um determinado
comando.


\subsection{Opções universais de linha de comando}
\label{commandline/commandline-index:opcoes-universais-de-linha-de-comando}
Esssa seção contém as opções de configuração aplicáveis à \emph{todas} versões do do MAME (seja SDL ou Windows).


\subsubsection{Comandos principais}
\label{commandline/commandline-index:comandos-principais}
\begin{DUlineblock}{0em}
\item[] {\hyperref[commandline/commandline\string-all:mame\string-commandline\string-help]{\sphinxcrossref{\DUrole{std,std-ref}{help}}}}
\item[] {\hyperref[commandline/commandline\string-all:mame\string-commandline\string-validate]{\sphinxcrossref{\DUrole{std,std-ref}{validate}}}}
\end{DUlineblock}


\subsubsection{Comandos de configuração}
\label{commandline/commandline-index:comandos-de-configuracao}
\begin{DUlineblock}{0em}
\item[] {\hyperref[commandline/commandline\string-all:mame\string-commandline\string-createconfig]{\sphinxcrossref{\DUrole{std,std-ref}{createconfig}}}}
\item[] {\hyperref[commandline/commandline\string-all:mame\string-commandline\string-showconfig]{\sphinxcrossref{\DUrole{std,std-ref}{showconfig}}}}
\item[] {\hyperref[commandline/commandline\string-all:mame\string-commandline\string-showusage]{\sphinxcrossref{\DUrole{std,std-ref}{showusage}}}}
\end{DUlineblock}


\subsubsection{Comandos frontend}
\label{commandline/commandline-index:comandos-frontend}
\begin{DUlineblock}{0em}
\item[] {\hyperref[commandline/commandline\string-all:mame\string-commandline\string-listxml]{\sphinxcrossref{\DUrole{std,std-ref}{listxml}}}}
\item[] {\hyperref[commandline/commandline\string-all:mame\string-commandline\string-listfull]{\sphinxcrossref{\DUrole{std,std-ref}{listfull}}}}
\item[] {\hyperref[commandline/commandline\string-all:mame\string-commandline\string-listsource]{\sphinxcrossref{\DUrole{std,std-ref}{listsource}}}}
\item[] {\hyperref[commandline/commandline\string-all:mame\string-commandline\string-listclones]{\sphinxcrossref{\DUrole{std,std-ref}{listclones}}}}
\item[] {\hyperref[commandline/commandline\string-all:mame\string-commandline\string-listbrothers]{\sphinxcrossref{\DUrole{std,std-ref}{listbrothers}}}}
\item[] {\hyperref[commandline/commandline\string-all:mame\string-commandline\string-listcrc]{\sphinxcrossref{\DUrole{std,std-ref}{listcrc}}}}
\item[] {\hyperref[commandline/commandline\string-all:mame\string-commandline\string-listroms]{\sphinxcrossref{\DUrole{std,std-ref}{listroms}}}}
\item[] {\hyperref[commandline/commandline\string-all:mame\string-commandline\string-listsamples]{\sphinxcrossref{\DUrole{std,std-ref}{listsamples}}}}
\item[] {\hyperref[commandline/commandline\string-all:mame\string-commandline\string-verifyroms]{\sphinxcrossref{\DUrole{std,std-ref}{verifyroms}}}}
\item[] {\hyperref[commandline/commandline\string-all:mame\string-commandline\string-verifysamples]{\sphinxcrossref{\DUrole{std,std-ref}{verifysamples}}}}
\item[] {\hyperref[commandline/commandline\string-all:mame\string-commandline\string-romident]{\sphinxcrossref{\DUrole{std,std-ref}{romident}}}}
\item[] {\hyperref[commandline/commandline\string-all:mame\string-commandline\string-listdevices]{\sphinxcrossref{\DUrole{std,std-ref}{listdevices}}}}
\item[] {\hyperref[commandline/commandline\string-all:mame\string-commandline\string-listslots]{\sphinxcrossref{\DUrole{std,std-ref}{listslots}}}}
\item[] {\hyperref[commandline/commandline\string-all:mame\string-commandline\string-listmedia]{\sphinxcrossref{\DUrole{std,std-ref}{listmedia}}}}
\item[] {\hyperref[commandline/commandline\string-all:mame\string-commandline\string-listsoftware]{\sphinxcrossref{\DUrole{std,std-ref}{listsoftware}}}}
\item[] {\hyperref[commandline/commandline\string-all:mame\string-commandline\string-verifysoftware]{\sphinxcrossref{\DUrole{std,std-ref}{verifysoftware}}}}
\item[] {\hyperref[commandline/commandline\string-all:mame\string-commandline\string-getsoftlist]{\sphinxcrossref{\DUrole{std,std-ref}{getsoftlist}}}}
\item[] {\hyperref[commandline/commandline\string-all:mame\string-commandline\string-verifysoftlist]{\sphinxcrossref{\DUrole{std,std-ref}{verifysoftlist}}}}
\end{DUlineblock}


\subsubsection{Opções relacionadas ao que é exibido na tela}
\label{commandline/commandline-index:opcoes-relacionadas-ao-que-e-exibido-na-tela}
\begin{DUlineblock}{0em}
\item[] {\hyperref[commandline/commandline\string-all:mame\string-commandline\string-uimodekey]{\sphinxcrossref{\DUrole{std,std-ref}{uimodekey}}}}
\item[] {\hyperref[commandline/commandline\string-all:mame\string-commandline\string-uifontprovider]{\sphinxcrossref{\DUrole{std,std-ref}{uifontprovider}}}}
\item[] {\hyperref[commandline/commandline\string-all:mame\string-commandline\string-keyboardprovider]{\sphinxcrossref{\DUrole{std,std-ref}{keyboardprovider}}}}
\item[] {\hyperref[commandline/commandline\string-all:mame\string-commandline\string-mouseprovider]{\sphinxcrossref{\DUrole{std,std-ref}{mouseprovider}}}}
\item[] {\hyperref[commandline/commandline\string-all:mame\string-commandline\string-lightgunprovider]{\sphinxcrossref{\DUrole{std,std-ref}{lightgunprovider}}}}
\item[] {\hyperref[commandline/commandline\string-all:mame\string-commandline\string-joystickprovider]{\sphinxcrossref{\DUrole{std,std-ref}{joystickprovider}}}}
\end{DUlineblock}


\subsubsection{Opções CLI relacionados ao que é exibido na tela}
\label{commandline/commandline-index:opcoes-cli-relacionados-ao-que-e-exibido-na-tela}
\begin{DUlineblock}{0em}
\item[] {\hyperref[commandline/commandline\string-all:mame\string-commandline\string-listmidi]{\sphinxcrossref{\DUrole{std,std-ref}{listmidi}}}}
\item[] {\hyperref[commandline/commandline\string-all:mame\string-commandline\string-listnetwork]{\sphinxcrossref{\DUrole{std,std-ref}{listnetwork}}}}
\end{DUlineblock}


\subsubsection{Opções de saída relacionados ao que é exibido na tela}
\label{commandline/commandline-index:opcoes-de-saida-relacionados-ao-que-e-exibido-na-tela}
\begin{DUlineblock}{0em}
\item[] {\hyperref[commandline/commandline\string-all:mame\string-commandline\string-output]{\sphinxcrossref{\DUrole{std,std-ref}{output}}}}
\end{DUlineblock}


\subsubsection{Opções de configuração}
\label{commandline/commandline-index:opcoes-de-configuracao}
\begin{DUlineblock}{0em}
\item[] {\hyperref[commandline/commandline\string-all:mame\string-commandline\string-noreadconfig]{\sphinxcrossref{\DUrole{std,std-ref}{noreadconfig}}}}
\end{DUlineblock}


\subsubsection{Principais opções de pesquisa de caminho}
\label{commandline/commandline-index:principais-opcoes-de-pesquisa-de-caminho}
\begin{DUlineblock}{0em}
\item[] {\hyperref[commandline/commandline\string-all:mame\string-commandline\string-homepath]{\sphinxcrossref{\DUrole{std,std-ref}{homepath}}}}
\item[] {\hyperref[commandline/commandline\string-all:mame\string-commandline\string-rompath]{\sphinxcrossref{\DUrole{std,std-ref}{rompath}}}}
\item[] {\hyperref[commandline/commandline\string-all:mame\string-commandline\string-hashpath]{\sphinxcrossref{\DUrole{std,std-ref}{hashpath}}}}
\item[] {\hyperref[commandline/commandline\string-all:mame\string-commandline\string-samplepath]{\sphinxcrossref{\DUrole{std,std-ref}{samplepath}}}}
\item[] {\hyperref[commandline/commandline\string-all:mame\string-commandline\string-artpath]{\sphinxcrossref{\DUrole{std,std-ref}{artpath}}}}
\item[] {\hyperref[commandline/commandline\string-all:mame\string-commandline\string-ctrlrpath]{\sphinxcrossref{\DUrole{std,std-ref}{ctrlrpath}}}}
\item[] {\hyperref[commandline/commandline\string-all:mame\string-commandline\string-inipath]{\sphinxcrossref{\DUrole{std,std-ref}{inipath}}}}
\item[] {\hyperref[commandline/commandline\string-all:mame\string-commandline\string-fontpath]{\sphinxcrossref{\DUrole{std,std-ref}{fontpath}}}}
\item[] {\hyperref[commandline/commandline\string-all:mame\string-commandline\string-cheatpath]{\sphinxcrossref{\DUrole{std,std-ref}{cheatpath}}}}
\item[] {\hyperref[commandline/commandline\string-all:mame\string-commandline\string-crosshairpath]{\sphinxcrossref{\DUrole{std,std-ref}{crosshairpath}}}}
\item[] {\hyperref[commandline/commandline\string-all:mame\string-commandline\string-pluginspath]{\sphinxcrossref{\DUrole{std,std-ref}{pluginspath}}}}
\item[] {\hyperref[commandline/commandline\string-all:mame\string-commandline\string-languagepath]{\sphinxcrossref{\DUrole{std,std-ref}{languagepath}}}}
\item[] {\hyperref[commandline/commandline\string-all:mame\string-commandline\string-swpath]{\sphinxcrossref{\DUrole{std,std-ref}{swpath}}}}
\end{DUlineblock}


\subsubsection{Principais opções para o destino de diretório}
\label{commandline/commandline-index:principais-opcoes-para-o-destino-de-diretorio}
\begin{DUlineblock}{0em}
\item[] {\hyperref[commandline/commandline\string-all:mame\string-commandline\string-cfgdirectory]{\sphinxcrossref{\DUrole{std,std-ref}{cfg\_directory}}}}
\item[] {\hyperref[commandline/commandline\string-all:mame\string-commandline\string-nvramdirectory]{\sphinxcrossref{\DUrole{std,std-ref}{nvram\_directory}}}}
\item[] {\hyperref[commandline/commandline\string-all:mame\string-commandline\string-inputdirectory]{\sphinxcrossref{\DUrole{std,std-ref}{input\_directory}}}}
\item[] {\hyperref[commandline/commandline\string-all:mame\string-commandline\string-statedirectory]{\sphinxcrossref{\DUrole{std,std-ref}{state\_directory}}}}
\item[] {\hyperref[commandline/commandline\string-all:mame\string-commandline\string-snapshotdirectory]{\sphinxcrossref{\DUrole{std,std-ref}{snapshot\_directory}}}}
\item[] {\hyperref[commandline/commandline\string-all:mame\string-commandline\string-diffdirectory]{\sphinxcrossref{\DUrole{std,std-ref}{diff\_directory}}}}
\item[] {\hyperref[commandline/commandline\string-all:mame\string-commandline\string-commentdirectory]{\sphinxcrossref{\DUrole{std,std-ref}{comment\_directory}}}}
\end{DUlineblock}


\subsubsection{Principais opções de estado e reprodução}
\label{commandline/commandline-index:principais-opcoes-de-estado-e-reproducao}
\begin{DUlineblock}{0em}
\item[] {\hyperref[commandline/commandline\string-all:mame\string-commandline\string-norewind]{\sphinxcrossref{\DUrole{std,std-ref}{{[}no{]}rewind / rewind}}}}
\item[] {\hyperref[commandline/commandline\string-all:mame\string-commandline\string-rewindcapacity]{\sphinxcrossref{\DUrole{std,std-ref}{rewind\_capacity}}}}
\item[] {\hyperref[commandline/commandline\string-all:mame\string-commandline\string-state]{\sphinxcrossref{\DUrole{std,std-ref}{state}}}}
\item[] {\hyperref[commandline/commandline\string-all:mame\string-commandline\string-noautosave]{\sphinxcrossref{\DUrole{std,std-ref}{{[}no{]}autosave}}}}
\item[] {\hyperref[commandline/commandline\string-all:mame\string-commandline\string-playback]{\sphinxcrossref{\DUrole{std,std-ref}{playback}}}}
\item[] {\hyperref[commandline/commandline\string-all:mame\string-commandline\string-exitafterplayback]{\sphinxcrossref{\DUrole{std,std-ref}{exit\_after\_playback}}}}
\item[] {\hyperref[commandline/commandline\string-all:mame\string-commandline\string-record]{\sphinxcrossref{\DUrole{std,std-ref}{record}}}}
\item[] {\hyperref[commandline/commandline\string-all:mame\string-commandline\string-recordtimecode]{\sphinxcrossref{\DUrole{std,std-ref}{record\_timecode}}}}
\item[] {\hyperref[commandline/commandline\string-all:mame\string-commandline\string-mngwrite]{\sphinxcrossref{\DUrole{std,std-ref}{mngwrite}}}}
\item[] {\hyperref[commandline/commandline\string-all:mame\string-commandline\string-aviwrite]{\sphinxcrossref{\DUrole{std,std-ref}{aviwrite}}}}
\item[] {\hyperref[commandline/commandline\string-all:mame\string-commandline\string-wavwrite]{\sphinxcrossref{\DUrole{std,std-ref}{wavwrite}}}}
\item[] {\hyperref[commandline/commandline\string-all:mame\string-commandline\string-snapname]{\sphinxcrossref{\DUrole{std,std-ref}{snapname}}}}
\item[] {\hyperref[commandline/commandline\string-all:mame\string-commandline\string-snapsize]{\sphinxcrossref{\DUrole{std,std-ref}{snapsize}}}}
\item[] {\hyperref[commandline/commandline\string-all:mame\string-commandline\string-snapview]{\sphinxcrossref{\DUrole{std,std-ref}{snapview}}}}
\item[] {\hyperref[commandline/commandline\string-all:mame\string-commandline\string-nosnapbilinear]{\sphinxcrossref{\DUrole{std,std-ref}{{[}no{]}snapbilinear}}}}
\item[] {\hyperref[commandline/commandline\string-all:mame\string-commandline\string-statename]{\sphinxcrossref{\DUrole{std,std-ref}{statename}}}}
\item[] {\hyperref[commandline/commandline\string-all:mame\string-commandline\string-noburnin]{\sphinxcrossref{\DUrole{std,std-ref}{{[}no{]}burnin}}}}
\end{DUlineblock}


\subsubsection{Principais opções de performance}
\label{commandline/commandline-index:principais-opcoes-de-performance}
\begin{DUlineblock}{0em}
\item[] {\hyperref[commandline/commandline\string-all:mame\string-commandline\string-noautoframeskip]{\sphinxcrossref{\DUrole{std,std-ref}{{[}no{]}autoframeskip}}}}
\item[] {\hyperref[commandline/commandline\string-all:mame\string-commandline\string-frameskip]{\sphinxcrossref{\DUrole{std,std-ref}{frameskip}}}}
\item[] {\hyperref[commandline/commandline\string-all:mame\string-commandline\string-secondstorun]{\sphinxcrossref{\DUrole{std,std-ref}{seconds\_to\_run}}}}
\item[] {\hyperref[commandline/commandline\string-all:mame\string-commandline\string-nothrottle]{\sphinxcrossref{\DUrole{std,std-ref}{{[}no{]}throttle}}}}
\item[] {\hyperref[commandline/commandline\string-all:mame\string-commandline\string-nosleep]{\sphinxcrossref{\DUrole{std,std-ref}{{[}no{]}sleep}}}}
\item[] {\hyperref[commandline/commandline\string-all:mame\string-commandline\string-speed]{\sphinxcrossref{\DUrole{std,std-ref}{speed}}}}
\item[] {\hyperref[commandline/commandline\string-all:mame\string-commandline\string-norefreshspeed]{\sphinxcrossref{\DUrole{std,std-ref}{{[}no{]}refreshspeed}}}}
\item[] {\hyperref[commandline/commandline\string-all:mame\string-commandline\string-numprocessors]{\sphinxcrossref{\DUrole{std,std-ref}{numprocessors}}}}
\item[] {\hyperref[commandline/commandline\string-all:mame\string-commandline\string-bench]{\sphinxcrossref{\DUrole{std,std-ref}{bench}}}}
\end{DUlineblock}


\subsubsection{Principais opções de rotação}
\label{commandline/commandline-index:principais-opcoes-de-rotacao}
\begin{DUlineblock}{0em}
\item[] {\hyperref[commandline/commandline\string-all:mame\string-commandline\string-norotate]{\sphinxcrossref{\DUrole{std,std-ref}{{[}no{]}rotate}}}}
\item[] {\hyperref[commandline/commandline\string-all:mame\string-commandline\string-noror]{\sphinxcrossref{\DUrole{std,std-ref}{{[}no{]}ror}}}}
\item[] {\hyperref[commandline/commandline\string-all:mame\string-commandline\string-norol]{\sphinxcrossref{\DUrole{std,std-ref}{{[}no{]}rol}}}}
\item[] {\hyperref[commandline/commandline\string-all:mame\string-commandline\string-noautoror]{\sphinxcrossref{\DUrole{std,std-ref}{{[}no{]}autoror}}}}
\item[] {\hyperref[commandline/commandline\string-all:mame\string-commandline\string-noautorol]{\sphinxcrossref{\DUrole{std,std-ref}{{[}no{]}autorol}}}}
\item[] {\hyperref[commandline/commandline\string-all:mame\string-commandline\string-noflipx]{\sphinxcrossref{\DUrole{std,std-ref}{{[}no{]}flipx}}}}
\item[] {\hyperref[commandline/commandline\string-all:mame\string-commandline\string-noflipy]{\sphinxcrossref{\DUrole{std,std-ref}{{[}no{]}flipy}}}}
\end{DUlineblock}


\subsubsection{Principais opções de vídeo}
\label{commandline/commandline-index:principais-opcoes-de-video}
\begin{DUlineblock}{0em}
\item[] {\hyperref[commandline/commandline\string-all:mame\string-commandline\string-video]{\sphinxcrossref{\DUrole{std,std-ref}{video}}}}
\item[] {\hyperref[commandline/commandline\string-all:mame\string-commandline\string-numscreens]{\sphinxcrossref{\DUrole{std,std-ref}{numscreens}}}}
\item[] {\hyperref[commandline/commandline\string-all:mame\string-commandline\string-window]{\sphinxcrossref{\DUrole{std,std-ref}{{[}no{]}window}}}}
\item[] {\hyperref[commandline/commandline\string-all:mame\string-commandline\string-maximize]{\sphinxcrossref{\DUrole{std,std-ref}{{[}no{]}maximize}}}}
\item[] {\hyperref[commandline/commandline\string-all:mame\string-commandline\string-keepaspect]{\sphinxcrossref{\DUrole{std,std-ref}{{[}no{]}keepaspect}}}}
\item[] {\hyperref[commandline/commandline\string-all:mame\string-commandline\string-waitvsync]{\sphinxcrossref{\DUrole{std,std-ref}{{[}no{]}waitvsync}}}}
\item[] {\hyperref[commandline/commandline\string-all:mame\string-commandline\string-syncrefresh]{\sphinxcrossref{\DUrole{std,std-ref}{{[}no{]}syncrefresh}}}}
\item[] {\hyperref[commandline/commandline\string-all:mame\string-commandline\string-prescale]{\sphinxcrossref{\DUrole{std,std-ref}{prescale}}}}
\item[] {\hyperref[commandline/commandline\string-all:mame\string-commandline\string-filter]{\sphinxcrossref{\DUrole{std,std-ref}{{[}no{]}filter}}}}
\item[] {\hyperref[commandline/commandline\string-all:mame\string-commandline\string-unevenstretch]{\sphinxcrossref{\DUrole{std,std-ref}{{[}no{]}unevenstretch}}}}
\end{DUlineblock}


\subsubsection{Opções de tela inteira}
\label{commandline/commandline-index:opcoes-de-tela-inteira}
\begin{DUlineblock}{0em}
\item[] {\hyperref[commandline/commandline\string-all:mame\string-commandline\string-switchres]{\sphinxcrossref{\DUrole{std,std-ref}{{[}no{]}switchres}}}}
\end{DUlineblock}


\subsubsection{Opções de janelas individuais de vídeo}
\label{commandline/commandline-index:opcoes-de-janelas-individuais-de-video}
\begin{DUlineblock}{0em}
\item[] {\hyperref[commandline/commandline\string-all:mame\string-commandline\string-screen]{\sphinxcrossref{\DUrole{std,std-ref}{screen}}}}
\item[] {\hyperref[commandline/commandline\string-all:mame\string-commandline\string-aspect]{\sphinxcrossref{\DUrole{std,std-ref}{aspect}}}}
\item[] {\hyperref[commandline/commandline\string-all:mame\string-commandline\string-resolution]{\sphinxcrossref{\DUrole{std,std-ref}{resolution}}}}
\item[] {\hyperref[commandline/commandline\string-all:mame\string-commandline\string-view]{\sphinxcrossref{\DUrole{std,std-ref}{view}}}}
\end{DUlineblock}


\subsubsection{Opções de ilustração (Artwork)}
\label{commandline/commandline-index:opcoes-de-ilustracao-artwork}
\begin{DUlineblock}{0em}
\item[] {\hyperref[commandline/commandline\string-all:mame\string-commandline\string-noartworkcrop]{\sphinxcrossref{\DUrole{std,std-ref}{{[}no{]}artwork\_crop}}}}
\item[] {\hyperref[commandline/commandline\string-all:mame\string-commandline\string-nousebackdrops]{\sphinxcrossref{\DUrole{std,std-ref}{{[}no{]}use\_backdrops}}}}
\item[] {\hyperref[commandline/commandline\string-all:mame\string-commandline\string-nouseoverlays]{\sphinxcrossref{\DUrole{std,std-ref}{{[}no{]}use\_overlays}}}}
\item[] {\hyperref[commandline/commandline\string-all:mame\string-commandline\string-nousebezels]{\sphinxcrossref{\DUrole{std,std-ref}{{[}no{]}use\_bezels}}}}
\item[] {\hyperref[commandline/commandline\string-all:mame\string-commandline\string-nousecpanels]{\sphinxcrossref{\DUrole{std,std-ref}{{[}no{]}use\_cpanels}}}}
\item[] {\hyperref[commandline/commandline\string-all:mame\string-commandline\string-nousemarquees]{\sphinxcrossref{\DUrole{std,std-ref}{{[}no{]}use\_marquees}}}}
\item[] {\hyperref[commandline/commandline\string-all:mame\string-commandline\string-fallbackartwork]{\sphinxcrossref{\DUrole{std,std-ref}{fallback\_artwork}}}}
\item[] {\hyperref[commandline/commandline\string-all:mame\string-commandline\string-overrideartwork]{\sphinxcrossref{\DUrole{std,std-ref}{override\_artwork}}}}
\end{DUlineblock}


\subsubsection{Opções de tela}
\label{commandline/commandline-index:opcoes-de-tela}
\begin{DUlineblock}{0em}
\item[] {\hyperref[commandline/commandline\string-all:mame\string-commandline\string-brightness]{\sphinxcrossref{\DUrole{std,std-ref}{brightness}}}}
\item[] {\hyperref[commandline/commandline\string-all:mame\string-commandline\string-contrast]{\sphinxcrossref{\DUrole{std,std-ref}{contrast}}}}
\item[] {\hyperref[commandline/commandline\string-all:mame\string-commandline\string-gamma]{\sphinxcrossref{\DUrole{std,std-ref}{gamma}}}}
\item[] {\hyperref[commandline/commandline\string-all:mame\string-commandline\string-pausebrightness]{\sphinxcrossref{\DUrole{std,std-ref}{pause\_brightness}}}}
\item[] {\hyperref[commandline/commandline\string-all:mame\string-commandline\string-effect]{\sphinxcrossref{\DUrole{std,std-ref}{effect}}}}
\end{DUlineblock}


\subsubsection{Opções de vetores}
\label{commandline/commandline-index:opcoes-de-vetores}
\begin{DUlineblock}{0em}
\item[] {\hyperref[commandline/commandline\string-all:mame\string-commandline\string-beamwidthmin]{\sphinxcrossref{\DUrole{std,std-ref}{beam\_width\_min}}}}
\item[] {\hyperref[commandline/commandline\string-all:mame\string-commandline\string-beamwidthmax]{\sphinxcrossref{\DUrole{std,std-ref}{beam\_width\_max}}}}
\item[] {\hyperref[commandline/commandline\string-all:mame\string-commandline\string-beamintensityweight]{\sphinxcrossref{\DUrole{std,std-ref}{beam\_intensity\_weight}}}}
\item[] {\hyperref[commandline/commandline\string-all:mame\string-commandline\string-flicker]{\sphinxcrossref{\DUrole{std,std-ref}{flicker}}}}
\end{DUlineblock}


\subsubsection{Opções de depuração de vídeo OpenGL}
\label{commandline/commandline-index:opcoes-de-depuracao-de-video-opengl}
\begin{DUlineblock}{0em}
\item[] {\hyperref[commandline/commandline\string-all:mame\string-commandline\string-glforcepow2texture]{\sphinxcrossref{\DUrole{std,std-ref}{{[}no{]}gl\_forcepow2texture}}}}
\item[] {\hyperref[commandline/commandline\string-all:mame\string-commandline\string-glnotexturerect]{\sphinxcrossref{\DUrole{std,std-ref}{{[}no{]}gl\_notexturerect}}}}
\item[] {\hyperref[commandline/commandline\string-all:mame\string-commandline\string-glvbo]{\sphinxcrossref{\DUrole{std,std-ref}{{[}no{]}gl\_vbo}}}}
\item[] {\hyperref[commandline/commandline\string-all:mame\string-commandline\string-glpbo]{\sphinxcrossref{\DUrole{std,std-ref}{{[}no{]}gl\_pbo}}}}
\end{DUlineblock}


\subsubsection{Opções de vídeo OpenGL GLSL}
\label{commandline/commandline-index:opcoes-de-video-opengl-glsl}
\begin{DUlineblock}{0em}
\item[] {\hyperref[commandline/commandline\string-all:mame\string-commandline\string-glglsl]{\sphinxcrossref{\DUrole{std,std-ref}{gl\_glsl}}}}
\item[] {\hyperref[commandline/commandline\string-all:mame\string-commandline\string-glglslfilter]{\sphinxcrossref{\DUrole{std,std-ref}{gl\_glsl\_filter}}}}
\item[] {\hyperref[commandline/commandline\string-all:mame\string-commandline\string-glslshadermame]{\sphinxcrossref{\DUrole{std,std-ref}{glsl\_shader\_mame{[}0-9{]}}}}}
\item[] {\hyperref[commandline/commandline\string-all:mame\string-commandline\string-glslshaderscreen]{\sphinxcrossref{\DUrole{std,std-ref}{glsl\_shader\_screen{[}0-9{]}}}}}
\item[] {\hyperref[commandline/commandline\string-all:mame\string-commandline\string-glglslvidattr]{\sphinxcrossref{\DUrole{std,std-ref}{gl\_glsl\_vid\_attr}}}}
\end{DUlineblock}


\subsubsection{Opções de áudio}
\label{commandline/commandline-index:opcoes-de-audio}
\begin{DUlineblock}{0em}
\item[] {\hyperref[commandline/commandline\string-all:mame\string-commandline\string-samplerate]{\sphinxcrossref{\DUrole{std,std-ref}{samplerate}}}}
\item[] {\hyperref[commandline/commandline\string-all:mame\string-commandline\string-nosamples]{\sphinxcrossref{\DUrole{std,std-ref}{{[}no{]}samples}}}}
\item[] {\hyperref[commandline/commandline\string-all:mame\string-commandline\string-volume]{\sphinxcrossref{\DUrole{std,std-ref}{volume}}}}
\item[] {\hyperref[commandline/commandline\string-all:mame\string-commandline\string-sound]{\sphinxcrossref{\DUrole{std,std-ref}{sound}}}}
\item[] {\hyperref[commandline/commandline\string-all:mame\string-commandline\string-audiolatency]{\sphinxcrossref{\DUrole{std,std-ref}{audio\_latency}}}}
\end{DUlineblock}


\subsubsection{Opções de entrada}
\label{commandline/commandline-index:opcoes-de-entrada}
\begin{DUlineblock}{0em}
\item[] {\hyperref[commandline/commandline\string-all:mame\string-commandline\string-nocoinlockout]{\sphinxcrossref{\DUrole{std,std-ref}{{[}no{]}coin\_lockout}}}}
\item[] {\hyperref[commandline/commandline\string-all:mame\string-commandline\string-ctrlr]{\sphinxcrossref{\DUrole{std,std-ref}{ctrlr}}}}
\item[] {\hyperref[commandline/commandline\string-all:mame\string-commandline\string-nomouse]{\sphinxcrossref{\DUrole{std,std-ref}{{[}no{]}mouse}}}}
\item[] {\hyperref[commandline/commandline\string-all:mame\string-commandline\string-nojoystick]{\sphinxcrossref{\DUrole{std,std-ref}{{[}no{]}joystick}}}}
\item[] {\hyperref[commandline/commandline\string-all:mame\string-commandline\string-nolightgun]{\sphinxcrossref{\DUrole{std,std-ref}{{[}no{]}lightgun}}}}
\item[] {\hyperref[commandline/commandline\string-all:mame\string-commandline\string-nomultikeyboard]{\sphinxcrossref{\DUrole{std,std-ref}{{[}no{]}multikeyboard}}}}
\item[] {\hyperref[commandline/commandline\string-all:mame\string-commandline\string-nomultimouse]{\sphinxcrossref{\DUrole{std,std-ref}{{[}no{]}multimouse}}}}
\item[] {\hyperref[commandline/commandline\string-all:mame\string-commandline\string-nosteadykey]{\sphinxcrossref{\DUrole{std,std-ref}{{[}no{]}steadykey}}}}
\item[] {\hyperref[commandline/commandline\string-all:mame\string-commandline\string-uiactive]{\sphinxcrossref{\DUrole{std,std-ref}{{[}no{]}ui\_active}}}}
\item[] {\hyperref[commandline/commandline\string-all:mame\string-commandline\string-nooffscreenreload]{\sphinxcrossref{\DUrole{std,std-ref}{{[}no{]}offscreen\_reload}}}}
\item[] {\hyperref[commandline/commandline\string-all:mame\string-commandline\string-joystickmap]{\sphinxcrossref{\DUrole{std,std-ref}{joystick\_map}}}}
\item[] {\hyperref[commandline/commandline\string-all:mame\string-commandline\string-joystickdeadzone]{\sphinxcrossref{\DUrole{std,std-ref}{joystick\_deadzone}}}}
\item[] {\hyperref[commandline/commandline\string-all:mame\string-commandline\string-joysticksaturation]{\sphinxcrossref{\DUrole{std,std-ref}{joystick\_saturation}}}}
\item[] {\hyperref[commandline/commandline\string-all:mame\string-commandline\string-natural]{\sphinxcrossref{\DUrole{std,std-ref}{natural}}}}
\item[] {\hyperref[commandline/commandline\string-all:mame\string-commandline\string-joystickcontradictory]{\sphinxcrossref{\DUrole{std,std-ref}{joystick\_contradictory}}}}
\item[] {\hyperref[commandline/commandline\string-all:mame\string-commandline\string-coinimpulse]{\sphinxcrossref{\DUrole{std,std-ref}{coin\_impulse}}}}
\end{DUlineblock}


\subsubsection{Opções de entrada automaticamente habilitadas}
\label{commandline/commandline-index:opcoes-de-entrada-automaticamente-habilitadas}
\begin{DUlineblock}{0em}
\item[] {\hyperref[commandline/commandline\string-all:mame\string-commandline\string-paddledevice]{\sphinxcrossref{\DUrole{std,std-ref}{paddle\_device}}}}
\item[] {\hyperref[commandline/commandline\string-all:mame\string-commandline\string-adstickdevice]{\sphinxcrossref{\DUrole{std,std-ref}{adstick\_device}}}}
\item[] {\hyperref[commandline/commandline\string-all:mame\string-commandline\string-pedaldevice]{\sphinxcrossref{\DUrole{std,std-ref}{pedal\_device}}}}
\item[] {\hyperref[commandline/commandline\string-all:mame\string-commandline\string-dialdevice]{\sphinxcrossref{\DUrole{std,std-ref}{dial\_device}}}}
\item[] {\hyperref[commandline/commandline\string-all:mame\string-commandline\string-trackballdevice]{\sphinxcrossref{\DUrole{std,std-ref}{trackball\_device}}}}
\item[] {\hyperref[commandline/commandline\string-all:mame\string-commandline\string-lightgundevice]{\sphinxcrossref{\DUrole{std,std-ref}{lightgun\_device}}}}
\item[] {\hyperref[commandline/commandline\string-all:mame\string-commandline\string-positionaldevice]{\sphinxcrossref{\DUrole{std,std-ref}{positional\_device}}}}
\item[] {\hyperref[commandline/commandline\string-all:mame\string-commandline\string-mousedevice]{\sphinxcrossref{\DUrole{std,std-ref}{mouse\_device}}}}
\end{DUlineblock}


\subsubsection{Opções de depuração}
\label{commandline/commandline-index:opcoes-de-depuracao}
\begin{DUlineblock}{0em}
\item[] {\hyperref[commandline/commandline\string-all:mame\string-commandline\string-verbose]{\sphinxcrossref{\DUrole{std,std-ref}{{[}no{]}verbose}}}}
\item[] {\hyperref[commandline/commandline\string-all:mame\string-commandline\string-oslog]{\sphinxcrossref{\DUrole{std,std-ref}{{[}no{]}oslog}}}}
\item[] {\hyperref[commandline/commandline\string-all:mame\string-commandline\string-log]{\sphinxcrossref{\DUrole{std,std-ref}{{[}no{]}log}}}}
\item[] {\hyperref[commandline/commandline\string-all:mame\string-commandline\string-debug]{\sphinxcrossref{\DUrole{std,std-ref}{{[}no{]}debug}}}}
\item[] {\hyperref[commandline/commandline\string-all:mame\string-commandline\string-debugscript]{\sphinxcrossref{\DUrole{std,std-ref}{debugscript}}}}
\item[] {\hyperref[commandline/commandline\string-all:mame\string-commandline\string-updateinpause]{\sphinxcrossref{\DUrole{std,std-ref}{{[}no{]}update\_in\_pause}}}}
\item[] {\hyperref[commandline/commandline\string-all:mame\string-commandline\string-watchdog]{\sphinxcrossref{\DUrole{std,std-ref}{watchdog}}}}
\item[] {\hyperref[commandline/commandline\string-all:mame\string-commandline\string-debuggerfont]{\sphinxcrossref{\DUrole{std,std-ref}{debugger\_font}}}}
\item[] {\hyperref[commandline/commandline\string-all:mame\string-commandline\string-debuggerfontsize]{\sphinxcrossref{\DUrole{std,std-ref}{debugger\_font\_size}}}}
\end{DUlineblock}


\subsubsection{Opções de comunicação}
\label{commandline/commandline-index:opcoes-de-comunicacao}
\begin{DUlineblock}{0em}
\item[] {\hyperref[commandline/commandline\string-all:mame\string-commandline\string-commlocalhost]{\sphinxcrossref{\DUrole{std,std-ref}{comm\_localhost}}}}
\item[] {\hyperref[commandline/commandline\string-all:mame\string-commandline\string-commlocalport]{\sphinxcrossref{\DUrole{std,std-ref}{comm\_localport}}}}
\item[] {\hyperref[commandline/commandline\string-all:mame\string-commandline\string-commremotehost]{\sphinxcrossref{\DUrole{std,std-ref}{comm\_remotehost}}}}
\item[] {\hyperref[commandline/commandline\string-all:mame\string-commandline\string-commremoteport]{\sphinxcrossref{\DUrole{std,std-ref}{comm\_remoteport}}}}
\item[] {\hyperref[commandline/commandline\string-all:mame\string-commandline\string-commframesync]{\sphinxcrossref{\DUrole{std,std-ref}{{[}no{]}comm\_framesync}}}}
\end{DUlineblock}


\subsubsection{Opções diversas}
\label{commandline/commandline-index:opcoes-diversas}
\begin{DUlineblock}{0em}
\item[] {\hyperref[commandline/commandline\string-all:mame\string-commandline\string-drc]{\sphinxcrossref{\DUrole{std,std-ref}{{[}no{]}drc}}}}
\item[] {\hyperref[commandline/commandline\string-all:mame\string-commandline\string-drcusec]{\sphinxcrossref{\DUrole{std,std-ref}{drc\_use\_c}}}}
\item[] {\hyperref[commandline/commandline\string-all:mame\string-commandline\string-drcloguml]{\sphinxcrossref{\DUrole{std,std-ref}{drc\_log\_uml}}}}
\item[] {\hyperref[commandline/commandline\string-all:mame\string-commandline\string-drclognative]{\sphinxcrossref{\DUrole{std,std-ref}{drc\_log\_native}}}}
\item[] {\hyperref[commandline/commandline\string-all:mame\string-commandline\string-bios]{\sphinxcrossref{\DUrole{std,std-ref}{bios}}}}
\item[] {\hyperref[commandline/commandline\string-all:mame\string-commandline\string-cheat]{\sphinxcrossref{\DUrole{std,std-ref}{{[}no{]}cheat}}}}
\item[] {\hyperref[commandline/commandline\string-all:mame\string-commandline\string-skipgameinfo]{\sphinxcrossref{\DUrole{std,std-ref}{{[}no{]}skip\_gameinfo}}}}
\item[] {\hyperref[commandline/commandline\string-all:mame\string-commandline\string-uifont]{\sphinxcrossref{\DUrole{std,std-ref}{uifont}}}}
\item[] {\hyperref[commandline/commandline\string-all:mame\string-commandline\string-ui]{\sphinxcrossref{\DUrole{std,std-ref}{ui}}}}
\item[] {\hyperref[commandline/commandline\string-all:mame\string-commandline\string-ramsize]{\sphinxcrossref{\DUrole{std,std-ref}{ramsize}}}}
\item[] {\hyperref[commandline/commandline\string-all:mame\string-commandline\string-confirmquit]{\sphinxcrossref{\DUrole{std,std-ref}{confirm\_quit}}}}
\item[] {\hyperref[commandline/commandline\string-all:mame\string-commandline\string-uimouse]{\sphinxcrossref{\DUrole{std,std-ref}{ui\_mouse}}}}
\item[] {\hyperref[commandline/commandline\string-all:mame\string-commandline\string-language]{\sphinxcrossref{\DUrole{std,std-ref}{language}}}}
\item[] {\hyperref[commandline/commandline\string-all:mame\string-commandline\string-nvramsave]{\sphinxcrossref{\DUrole{std,std-ref}{{[}no{]}nvram\_save}}}}
\end{DUlineblock}


\subsubsection{Opções de script}
\label{commandline/commandline-index:opcoes-de-script}
\begin{DUlineblock}{0em}
\item[] {\hyperref[commandline/commandline\string-all:mame\string-commandline\string-autobootcommand]{\sphinxcrossref{\DUrole{std,std-ref}{autoboot\_command}}}}
\item[] {\hyperref[commandline/commandline\string-all:mame\string-commandline\string-autobootdelay]{\sphinxcrossref{\DUrole{std,std-ref}{autoboot\_delay}}}}
\item[] {\hyperref[commandline/commandline\string-all:mame\string-commandline\string-autobootscript]{\sphinxcrossref{\DUrole{std,std-ref}{autoboot\_script}}}}
\item[] {\hyperref[commandline/commandline\string-all:mame\string-commandline\string-console]{\sphinxcrossref{\DUrole{std,std-ref}{{[}no{]}console}}}}
\item[] {\hyperref[commandline/commandline\string-all:mame\string-commandline\string-plugins]{\sphinxcrossref{\DUrole{std,std-ref}{{[}no{]}plugins}}}}
\item[] {\hyperref[commandline/commandline\string-all:mame\string-commandline\string-plugin]{\sphinxcrossref{\DUrole{std,std-ref}{plugin}}}}
\item[] {\hyperref[commandline/commandline\string-all:mame\string-commandline\string-noplugin]{\sphinxcrossref{\DUrole{std,std-ref}{noplugin}}}}
\end{DUlineblock}


\subsection{Opções do servidor HTTP}
\label{commandline/commandline-index:opcoes-do-servidor-http}
\begin{DUlineblock}{0em}
\item[] {\hyperref[commandline/commandline\string-all:mame\string-commandline\string-http]{\sphinxcrossref{\DUrole{std,std-ref}{http}}}}
\item[] {\hyperref[commandline/commandline\string-all:mame\string-commandline\string-httpport]{\sphinxcrossref{\DUrole{std,std-ref}{http\_port}}}}
\item[] {\hyperref[commandline/commandline\string-all:mame\string-commandline\string-httproot]{\sphinxcrossref{\DUrole{std,std-ref}{http\_root}}}}
\end{DUlineblock}


\subsection{Opções de linha de comando específicas para o Windows}
\label{commandline/commandline-index:opcoes-de-linha-de-comando-especificas-para-o-windows}

\subsubsection{Opções de performance para o Windows}
\label{commandline/commandline-index:opcoes-de-performance-para-o-windows}
\begin{DUlineblock}{0em}
\item[] {\hyperref[commandline/windowsconfig:mame\string-wcommandline\string-priority]{\sphinxcrossref{\DUrole{std,std-ref}{priority}}}}
\item[] {\hyperref[commandline/windowsconfig:mame\string-wcommandline\string-profile]{\sphinxcrossref{\DUrole{std,std-ref}{profile}}}}
\end{DUlineblock}


\subsubsection{Opções de tela inteira para o Windows}
\label{commandline/commandline-index:opcoes-de-tela-inteira-para-o-windows}
\begin{DUlineblock}{0em}
\item[] {\hyperref[commandline/windowsconfig:mame\string-wcommandline\string-triplebuffer]{\sphinxcrossref{\DUrole{std,std-ref}{{[}no{]}triplebuffer}}}}
\item[] {\hyperref[commandline/windowsconfig:mame\string-wcommandline\string-fullscreenbrightness]{\sphinxcrossref{\DUrole{std,std-ref}{full\_screen\_brightness}}}}
\item[] {\hyperref[commandline/windowsconfig:mame\string-wcommandline\string-fullscreencontrast]{\sphinxcrossref{\DUrole{std,std-ref}{full\_screen\_contrast}}}}
\item[] {\hyperref[commandline/windowsconfig:mame\string-wcommandline\string-fullscreengamma]{\sphinxcrossref{\DUrole{std,std-ref}{full\_screen\_gamma}}}}
\end{DUlineblock}


\subsubsection{Opções de controle de entrada para o Windows}
\label{commandline/commandline-index:opcoes-de-controle-de-entrada-para-o-windows}
\begin{DUlineblock}{0em}
\item[] {\hyperref[commandline/windowsconfig:mame\string-wcommandline\string-duallightgun]{\sphinxcrossref{\DUrole{std,std-ref}{{[}no{]}dual\_lightgun}}}}
\end{DUlineblock}


\subsection{Opções de linha de comando específicas para o SDL}
\label{commandline/commandline-index:opcoes-de-linha-de-comando-especificas-para-o-sdl}
Esta seção contém opções de configuração que são específicas para as
versões SDL compatíveis (incluindo versões Windows que foram compiladas
com SDL ao invés da sua versão nativa).


\subsubsection{Opções relacionadas a performance SDL}
\label{commandline/commandline-index:opcoes-relacionadas-a-performance-sdl}
\begin{DUlineblock}{0em}
\item[] {\hyperref[commandline/sdlconfig:mame\string-scommandline\string-sdlvideofps]{\sphinxcrossref{\DUrole{std,std-ref}{sdlvideofps}}}}
\end{DUlineblock}


\subsubsection{Opções de vídeo SDL}
\label{commandline/commandline-index:opcoes-de-video-sdl}
\begin{DUlineblock}{0em}
\item[] {\hyperref[commandline/sdlconfig:mame\string-scommandline\string-centerh]{\sphinxcrossref{\DUrole{std,std-ref}{{[}no{]}centerh}}}}
\item[] {\hyperref[commandline/sdlconfig:mame\string-scommandline\string-centerv]{\sphinxcrossref{\DUrole{std,std-ref}{{[}no{]}centerv}}}}
\end{DUlineblock}


\subsubsection{Opções específicas de software para vídeo SDL}
\label{commandline/commandline-index:opcoes-especificas-de-software-para-video-sdl}
\begin{DUlineblock}{0em}
\item[] {\hyperref[commandline/sdlconfig:mame\string-scommandline\string-scalemode]{\sphinxcrossref{\DUrole{std,std-ref}{scalemode}}}}
\end{DUlineblock}


\subsubsection{Opções de mapeamento de teclado SDL}
\label{commandline/commandline-index:opcoes-de-mapeamento-de-teclado-sdl}
\begin{DUlineblock}{0em}
\item[] {\hyperref[commandline/sdlconfig:mame\string-scommandline\string-keymap]{\sphinxcrossref{\DUrole{std,std-ref}{keymap}}}}
\item[] {\hyperref[commandline/sdlconfig:mame\string-scommandline\string-keymapfile]{\sphinxcrossref{\DUrole{std,std-ref}{keymap\_file}}}}
\end{DUlineblock}


\subsubsection{Opções de mapeamento de controle joystick SDL}
\label{commandline/commandline-index:opcoes-de-mapeamento-de-controle-joystick-sdl}
\begin{DUlineblock}{0em}
\item[] {\hyperref[commandline/sdlconfig:mame\string-scommandline\string-joyidx]{\sphinxcrossref{\DUrole{std,std-ref}{joyidx}}}}
\item[] {\hyperref[commandline/sdlconfig:mame\string-scommandline\string-sixaxis]{\sphinxcrossref{\DUrole{std,std-ref}{sixaxis}}}}
\end{DUlineblock}


\subsubsection{Opções de baixo nível para drivers SDL}
\label{commandline/commandline-index:opcoes-de-baixo-nivel-para-drivers-sdl}
\begin{DUlineblock}{0em}
\item[] {\hyperref[commandline/sdlconfig:mame\string-scommandline\string-videodriver]{\sphinxcrossref{\DUrole{std,std-ref}{videodriver}}}}
\item[] {\hyperref[commandline/sdlconfig:mame\string-scommandline\string-audiodriver]{\sphinxcrossref{\DUrole{std,std-ref}{audiodriver}}}}
\item[] {\hyperref[commandline/sdlconfig:mame\string-scommandline\string-gllib]{\sphinxcrossref{\DUrole{std,std-ref}{gl\_lib}}}}
\end{DUlineblock}


\chapter{CONFIGURAÇÕES AVANÇADAS}
\label{advanced/index:configuracoes-avancadas}\label{advanced/index::doc}

\section{Múltiplos arquivos de configuração}
\label{advanced/multiconfig::doc}\label{advanced/multiconfig:multiplos-arquivos-de-configuracao}
O MAME tem um poderoso sistema de configuração que permite ajustar as
opções de cada jogo, sistema ou até mesmo um tipo de monitor em
específico de maneira individual, porém requer um cuidado especial na
organização das configurações.


\subsection{A ordem de leitura dos arquivos}
\label{advanced/multiconfig:advanced-multi-cfg}\label{advanced/multiconfig:a-ordem-de-leitura-dos-arquivos}\begin{enumerate}
\item {} 
Inicialmente a linha de comando é interpretada primeiro, depois
as configurações que \emph{terão prioridade sobre qualquer outra
opção que esteja no arquivo .INI}.

\item {} 
O \textbf{MAME.INI} (ou qualquer outra plataforma que use um .INI
como o \textbf{MESS.INI} por exemplo) são interpretadas duas vezes.
Na primeira passada pode alterar várias configurações de
caminho, a segunda passagem é feita para ver se há um arquivo de
configuração válido nesse novo local (caso haja, altera as
configurações com as informações desse arquivo).
E o \textbf{DEBUG.INI} se estiver no modo de depuração.
Este é um arquivo de configuração avançado que a maioria das
pessoas não precisam sequer se preocupar com ele.

\end{enumerate}
\begin{enumerate}
\setcounter{enumi}{3}
\item {} 
Quando for apropriado, os arquivos INI específicos voltado para
um sistema como \textbf{NEOGEO\_NOSLOT.INI} ou \textbf{CPS2.INI} por
exemplo.
O jogo \textbf{Street Fighter Alpha} é um jogo do sistema CPS2, então
o arquivo \textbf{CPS2.INI} será lido duas vezes.

\item {} 
Arquivo INI de orientação do monitor (seja \textbf{HORIZONT.INI} ou
\textbf{VERTICAL.INI}).
O jogo Pac-Man por exemplo, usa uma configuração de monitor
vertical, então ele leria o arquivo \textbf{VERTICAL.INI}.
Já o jogo \textbf{Street Fighter Alpha} é um jogo com tela
horizontal, então leria o arquivo \textbf{HORIZONT.INI}.

\item {} 
Arquivos INI voltado para diferentes sistemas (\textbf{ARCADE.INI},
\textbf{CONSOLE.INI}, \textbf{COMPUTER.INI}, ou \textbf{OTHERSYS.INI}).
Tanto o jogo \textbf{Pac-Man} quanto o jogo \textbf{Street Fighter Alpha}
são jogos de arcade, então o arquivo a ser lido seria o
\textbf{ARCADE.INI}.
Já no caso de um console como o \textbf{Atari 2600}, o arquivo a ser
lido seria o \textbf{CONSOLE.INI}.

\item {} 
Os arquivos INI voltados para diferentes tipos de tela
(\textbf{VECTOR.INI} para jogos vetoriais, \textbf{RASTER.INI} para jogos
rasterizados, \textbf{LCD.INI} para jogos em LCD).
Ambos os jogos \textbf{Pac-Man} e \textbf{Street Fighter Alpha} são jogos
rasterizados, então o arquivo a ser lido seria o \textbf{RASTER.INI}.
Tempest é um jogo que usa uma tela com vetores, então o arquivo
a ser lido seria o \textbf{VECTOR.INI}.

\item {} 
Os Arquivos INI do Códico Fonte.
Este também é um arquivo de configuração avançado, a maioria das
pessoas não precisam usá-lo para nada.
O MAME tentará ler os arquivos \textbf{SOURCE/SOURCEFILE.INI} e
\textbf{SOURCEFILE.INI} onde o sourcefile é o nome real do arquivo de
código-fonte.
O \textbf{mame -listsource \textless{}game\textgreater{}} mostrará o arquivo de código-fonte
para um determinado nome.
Por exemplo, o jogo \textbf{Sailor Moon} da Banpresto, \textbf{Dodonpachi}
da Atlus e \textbf{Dangun Feveron} da Nihon System compartilham uma
grande quantidade de hardware compatíveis entre si e são
agrupados em um único arquivo \textbf{CAVE.C}, o que significa que
todos eles interpretarão o arquivo \textbf{source/cave.ini}.

\item {} 
O arquivo INI Pai.
Se rodar o jogo \textbf{Pac-Man} por exemplo, que é um clone do jogo
\textbf{Puck-Man}, seria então o arquivo \textbf{PUCKMAN.INI}.

\item {} 
O arquivo INI de Driver.
Usando o nosso exemplo anterior do Pac-Man, isso seria um
arquivo chamado \textbf{PACMAN.INI}.

\end{enumerate}


\subsection{Exemplos da sequência de leitura dos arquivos}
\label{advanced/multiconfig:exemplos-da-sequencia-de-leitura-dos-arquivos}\begin{enumerate}
\item {} 
O jogo \textbf{Alcon}, que é um clone Americano do jogo
\textbf{Slap Fight}.  (\textbf{mame alcon})
Linha de comando, \textbf{MAME.INI}, \textbf{VERTICAL.INI},
\textbf{ARCADE.INI}, \textbf{RASTER.INI}, \textbf{SLAPFGHT.INI}, e por último
\textbf{ALCON.INI}
(\emph{lembre-se que os parâmetros na linha de comando tem
preferência sobre todos os outros arquivos!})

\item {} 
O jogo \textbf{Super Street Fighter 2 Turbo} (\textbf{mame ssf2t})
Linha de comando, \textbf{MAME.INI}, \textbf{HORIZONT.INI},
\textbf{ARCADE.INI}, \textbf{RASTER.INI}, \textbf{CPS2.INI}, e por último
\textbf{SSF2T.INI}
(\emph{lembre-se que os parâmetros na linha de comando tem
preferência sobre todos os outros arquivos!})

\end{enumerate}


\subsection{Truques para tornar a vida mais fácil}
\label{advanced/multiconfig:truques-para-tornar-a-vida-mais-facil}
Alguns usuários podem ter um monitor montado na parede ou um monitor
rotativo, e podem querer realmente jogar jogos verticais com a tela
girada na vertical. A maneira mais fácil de fazer isso é colocar as suas
configurações de rotação no arquivo \textbf{VERTICAL.INI}, onde afetaria
apenas os jogos verticais.

{[}a fazer: mais exemplos práticos{]}


\section{Como o MAME lida com o caminho dos arquivos}
\label{advanced/paths:como-o-mame-lida-com-o-caminho-dos-arquivos}\label{advanced/paths::doc}
O MAME obedece uma sequência lógica quando verifica os arquivos dos
usuários como ROMs e os arquivos de trapaça.


\subsection{A sequência de leitura dos caminhos}
\label{advanced/paths:a-sequencia-de-leitura-dos-caminhos}
Vamos usar o exemplo de um arquivo de trapaça do Sega Genesis/Megadrive
para o jogo After Burner 2 (aburner2 na lista de jogos para Megadrive),
o caminho predefinido para o ``cheatpath'' é cheat. É assim que o MAME vai
fazer a pesquisa por um arquivo de trapaça:
\begin{enumerate}
\item {} 
cheat/megadriv/aburner2.xml

\item {} 
cheat/megadriv.zip -\textgreater{} aburner2.xml
Repare que ele pesquisa por um arquivo \emph{.ZIP} primeiro, depois um
arquivo \emph{.7Z}.

\item {} 
cheat/megadriv.zip -\textgreater{} \textless{}arbitrary path\textgreater{}/aburner2.xml
Ele procurará (caso haja) pelo primeiro arquivo aburner2.xml que
ele puder encontrar dentro daquele arquivo zip, independente de onde
esteja.

\item {} 
cheat.zip -\textgreater{} megadriv/aburner2.xml
Agora está procurando especificamente por uma combinação de arquivo
ou pasta, porém agora, dentro do arquivo cheat.zip.

\item {} 
cheat.zip -\textgreater{} \textless{}qualquer caminho\textgreater{}/megadriv/aburner2.xml
Como antes, menos dentro do primeiro (se houver) o arquivo
aburner2.xml dentro da pasta megadriv que esteja dentro de um
arquivo zip.

\item {} 
cheat/megadriv.7z -\textgreater{} aburner2.xml
Agora começa a procurar dentro de arquivos \emph{7ZIP}.

\item {} 
cheat/megadriv.7z -\textgreater{} \textless{}qualquer caminho\textgreater{}/aburner2.xml

\item {} 
cheat.7z -\textgreater{} megadriv/aburner2.xml

\item {} 
cheat.7z -\textgreater{} \textless{}qualquer caminho\textgreater{}/megadriv/aburner2.xml
Similar ao zip, soque agora com arquivos \emph{7ZIP}.

\end{enumerate}

{[}a fazer: A leitura do conjunto de arquivos ROM é um pouco mais
complicado, adicionar CRC. Documentar isso no próximo dia ou dois.


\section{Desabilitando o interruptor de câmbio}
\label{advanced/shiftertoggle::doc}\label{advanced/shiftertoggle:desabilitando-o-interruptor-de-cambio}
Este é um recurso avançado para lidar com o interruptor de certas
máquinas arcade antigas como a \emph{Spy Junter} e \emph{Outrun} que usava uma
chave de via dupla que funcionava como um câmbio de marchas. Por
predefinição esse câmbio é tratado como um interruptor. Um toque na
configuração do controle para que o câmbio alterne entre marcha alta e
baixa, com outro toque ele volta para a posição anterior. Essa pode não
ser a melhor opção se você tiver em mãos um câmbio que trabalhe como os
câmbios antigos usados nas máquinas originais.
O câmbio estará engrenado quando a chave for ligada, funcionando ao
contrário quando estiver desligada.

Observe que este recurso \emph{não} ajudará o controle dos usuários e tão
pouco será de qualquer ajuda para jogos que tenham um câmbio com
mais de dois estados (como jogos modernos com mais de uma marcha por
exemplo).

Este recurso não é exibido através da tela do usuário pois é uma
customização extrema feita apenas pessoas que têm essa necessidade em
específico e o conhecimento para fazer o uso dela da forma correta.


\subsection{Alternando o interruptor de câmbio}
\label{advanced/shiftertoggle:alternando-o-interruptor-de-cambio}
O jogo Spy Hunter (do conjunto \emph{spyhunt}) será usado como exemplo para
explicar as alterações necessárias. Dentro da pasta CFG há um arquivo
\emph{.CFG} que você precisará editar como \emph{spyhunt.cfg} por exemplo.

No MAME, comece rodando o jogo que estamos usando de exemplo, o
\emph{spyhunt}, assim:

\textbf{mame spyhunt}.

Configure os controles da maneira que faria em outros jogos, incluindo a
configuração do câmbio. Saia do MAME e abra o arquivo .cfg do jogo no
editor de texto de sua preferência.

Dentro do arquivo \emph{spyhunt.cfg}, você deverá encontrar as seguintes
linhas de texto de configuração para a entrada. O código de entrada
exibido no meio da configuração pode variar dependendo da posição do
controle que você tiver configurado:

\begin{Verbatim}[commandchars=\\\{\}]
\PYG{o}{\PYGZlt{}}\PYG{n}{port} \PYG{n}{tag}\PYG{o}{=}\PYG{l+s+s2}{\PYGZdq{}}\PYG{l+s+s2}{:ssio:IP0}\PYG{l+s+s2}{\PYGZdq{}} \PYG{n+nb}{type}\PYG{o}{=}\PYG{l+s+s2}{\PYGZdq{}}\PYG{l+s+s2}{P1\PYGZus{}BUTTON2}\PYG{l+s+s2}{\PYGZdq{}} \PYG{n}{mask}\PYG{o}{=}\PYG{l+s+s2}{\PYGZdq{}}\PYG{l+s+s2}{16}\PYG{l+s+s2}{\PYGZdq{}} \PYG{n}{defvalue}\PYG{o}{=}\PYG{l+s+s2}{\PYGZdq{}}\PYG{l+s+s2}{16}\PYG{l+s+s2}{\PYGZdq{}}\PYG{o}{\PYGZgt{}}
    \PYG{o}{\PYGZlt{}}\PYG{n}{newseq} \PYG{n+nb}{type}\PYG{o}{=}\PYG{l+s+s2}{\PYGZdq{}}\PYG{l+s+s2}{standard}\PYG{l+s+s2}{\PYGZdq{}}\PYG{o}{\PYGZgt{}}
        \PYG{n}{JOYCODE\PYGZus{}1\PYGZus{}RYAXIS\PYGZus{}NEG\PYGZus{}SWITCH} \PYG{n}{OR} \PYG{n}{JOYCODE\PYGZus{}1\PYGZus{}RYAXIS\PYGZus{}POS\PYGZus{}SWITCH}
    \PYG{o}{\PYGZlt{}}\PYG{o}{/}\PYG{n}{newseq}\PYG{o}{\PYGZgt{}}
\PYG{o}{\PYGZlt{}}\PYG{o}{/}\PYG{n}{port}\PYG{o}{\PYGZgt{}}
\end{Verbatim}

Você precisa editar a linha da porta que definirá a entrada. Para o jogo
Spy Hunter será \emph{P1\_BUTTON2}. Adicione \sphinxcode{toggle="no"} no final da tag,
como mostra o exemplo abaixo:

\begin{Verbatim}[commandchars=\\\{\}]
\PYG{o}{\PYGZlt{}}\PYG{n}{port} \PYG{n}{tag}\PYG{o}{=}\PYG{l+s+s2}{\PYGZdq{}}\PYG{l+s+s2}{:ssio:IP0}\PYG{l+s+s2}{\PYGZdq{}} \PYG{n+nb}{type}\PYG{o}{=}\PYG{l+s+s2}{\PYGZdq{}}\PYG{l+s+s2}{P1\PYGZus{}BUTTON2}\PYG{l+s+s2}{\PYGZdq{}} \PYG{n}{mask}\PYG{o}{=}\PYG{l+s+s2}{\PYGZdq{}}\PYG{l+s+s2}{16}\PYG{l+s+s2}{\PYGZdq{}} \PYG{n}{defvalue}\PYG{o}{=}\PYG{l+s+s2}{\PYGZdq{}}\PYG{l+s+s2}{16}\PYG{l+s+s2}{\PYGZdq{}} \PYG{n}{toggle}\PYG{o}{=}\PYG{l+s+s2}{\PYGZdq{}}\PYG{l+s+s2}{no}\PYG{l+s+s2}{\PYGZdq{}}\PYG{o}{\PYGZgt{}}
    \PYG{o}{\PYGZlt{}}\PYG{n}{newseq} \PYG{n+nb}{type}\PYG{o}{=}\PYG{l+s+s2}{\PYGZdq{}}\PYG{l+s+s2}{standard}\PYG{l+s+s2}{\PYGZdq{}}\PYG{o}{\PYGZgt{}}
        \PYG{n}{JOYCODE\PYGZus{}1\PYGZus{}RYAXIS\PYGZus{}NEG\PYGZus{}SWITCH} \PYG{n}{OR} \PYG{n}{JOYCODE\PYGZus{}1\PYGZus{}RYAXIS\PYGZus{}POS\PYGZus{}SWITCH}
    \PYG{o}{\PYGZlt{}}\PYG{o}{/}\PYG{n}{newseq}\PYG{o}{\PYGZgt{}}
\PYG{o}{\PYGZlt{}}\PYG{o}{/}\PYG{n}{port}\PYG{o}{\PYGZgt{}}
\end{Verbatim}

Salve e saia.
Para desabilitar, simplesmente remova a opção \textbf{toggle=''no''} de cada
arquivo de .CFG que desejar.


\section{Efeitos BGFX para (quase) todo mundo}
\label{advanced/bgfx::doc}\label{advanced/bgfx:efeitos-bgfx-para-quase-todo-mundo}
Por predefinição, o MAME gera um sinal de vídeo puro, assim como seria
também no hardware original do arcade até o sinal chegar aos circuitos
que levam o sinal ao monitor CRT do arcade, com pequenas modificações na
saída (em geral, esticar a imagem do jogo de volta à proporção que se
teria num monitor CRT, geralmente na proporção 4:3), no geral isso
funciona bem, mas perde-se um pouco do fator nostalgia. Os monitores de
arcade, ainda que em perfeitas condições, nunca foram ideais pois devido
a sua natureza o monitor CRT distorciam a imagem original de maneira
a distorcerem significativamente a sua aparência final na tela.

Os monitores CRT dos arcades são uma experiência única na maneira que a
imagem é formada e apresentada na tela, imagem essa que os monitores de
LCD e até mesmo monitores CRT não possuem.

É aí então que entra em cena os novos processamentos BGFX com HLSL.

O filtro HLSL simula a maioria dos efeitos de vídeo que um monitor CRT
de arcade teria, fazendo com que o resultado visual seja muito mais
realista. Porém, os filtros HLSL exigem um esforço extra dos recursos do
seu computador e em especial do monitor que você estiver usando.
Além disso, havia centenas de milhares de tipos monitores diferentes nos
fliperamas. Cada um foi ajustado e mantido de forma diferente, o que
significa que, não tem como escolher e definir entre todos eles, apenas
um como referência. Diretrizes básicas serão fornecidas aqui para
ajudá-lo, mas você também poderá pedir mais opiniões em qualquer um dos
fóruns conhecidos sobre o MAME espalhados pela internet.


\subsection{Resolução e relação de aspecto da tela}
\label{advanced/bgfx:resolucao-e-relacao-de-aspecto-da-tela}
A resolução é um assunto muito importante para as configurações do HLSL.
Você desejará que o MAME esteja usando a resolução nativa do seu monitor
para evitar distorções e atrasos adicionais criados pelo seu monitor ao
tentar preencher a imagem na tela.

Enquanto a maioria das máquinas de arcade usava um monitor com proporção
de tela no formato 4:3 (ou 3:4 se o monitor estivesse orientado
verticalmente como é no caso do Pac Man), a essa altura do campeonato é
difícil encontrar nos dias de hoje um monitor ou TV que tenha uma
proporção de tela no formato 4:3. A boa notícia é que esse espaço extra
que sobra nas laterais não é desperdiçado. Muitos gabinetes de arcade na
época utilizavam uma moldura com ilustrações ao redor da tela, caso você
tenha esses arquivos o MAME também irá exibir essas ilustrações na tela.
Para se obter um melhor resultado, ative o visualizador de ilustrações e
selecione o modo recortado ou cropped em Inglês.

Alguns monitores de LCD mais antigos usavam uma resolução nativa de
1280x1024 onde tinham uma proporção de tela no formato 5:4.
Neste exemplo, não há muito espaço extra suficiente para exibir a
ilustração e você vai notar um leve esticamento vertical, porém os
resultados ainda serão bons o suficiente, como se fossem um monitor com
formato 4:3.


\subsection{Introdução ao BGFX}
\label{advanced/bgfx:introducao-ao-bgfx}
Antes de começar, você precisará seguir as instruções de configuração
inicial do MAME encontrada em outra parte deste manual.
As distribuições oficiais do MAME à partir da versão 0.172 já incluem o
BGFX, então você não precisa baixar nenhum outro arquivo adicional.

Abra o seu MAME.INI no seu editor de texto preferido como o bloco de
notas por exemplo e verifique se as seguintes opções estão definidas
corretamente:
\begin{itemize}
\item {} 
\textbf{video bgfx}

\end{itemize}

Agora tire um momento para ler as definições de configuração na seção
abaixo para aprender como melhor configurar as opções do BGFX.

Como descrito em {\hyperref[advanced/multiconfig:advanced\string-multi\string-cfg]{\sphinxcrossref{\DUrole{std,std-ref}{A ordem de leitura dos arquivos}}}}, o MAME segue uma sequência
na hora de processar os arquivos INI. As configurações BGFX podem ser
editadas diretamente no arquivo MAME.INI, porém para tirar melhor
proveito do poder dos arquivos de configuração do MAME, talvez seja
melhor copiar as opções do BGFX do MAME.INI para um outro arquivo de
configuração e fazer as modificações lá.

Particularmente, você vai querer que as configurações
\textbf{bgfx\_screen\_chains} sejam específicas es customizáveis para cada jogo
individualmente ao invés de uma única configuração para todos os jogos.

Salve o arquivo .INI e já estamos pronto para começar.


\subsection{Alterando as configurações}
\label{advanced/bgfx:alterando-as-configuracoes}
\begin{DUlineblock}{0em}
\item[] \textbf{bgfx\_path}
\item[] 
\item[]
\begin{DUlineblock}{\DUlineblockindent}
\item[] É aqui que seus arquivos de sombreamento BGFX (BGFX shader) são armazenados. Por definição inicial, o nome desta pasta será BGFX localizado onde o seu MAME estiver instalado.
\item[] 
\end{DUlineblock}
\item[] \textbf{bgfx\_backend}
\item[] 
\item[]
\begin{DUlineblock}{\DUlineblockindent}
\item[] Seleciona um tipo de infraestrutura de renderização que o BGFX possa usar. As escolhas possíveis são \textbf{d3d9}, \textbf{d3d11}, \textbf{opengl}, and \textbf{metal}. O valor predefinido é \textbf{auto} que permite que o MAME escolha a melhor opção para você.
\item[] 
\item[] \textbf{d3d9} -- Renderizador do Direct3D 9.0 (Requer o Windows XP ou mais recente)
\item[] \textbf{d3d11} -- Renderizador do Direct3D 11.0 (Requer Windows Vista com o D3D11 atualizado ou o  Windows 7 ou mais recente)
\item[] \textbf{opengl} -- Renderizador OpenGL (Requer Drivers OpenGL, pode funcionar melhor em algumas placas de vídeo mais antigas ou mal projetadas, compatível com Linux/Mac OS X)
\item[] \textbf{metal} -- Metal Apple Graphics API (Requer Mac OS X 10.11 El Capitan ou mais recente)
\item[] 
\end{DUlineblock}
\item[] \textbf{bgfx\_debug}
\item[] 
\item[]
\begin{DUlineblock}{\DUlineblockindent}
\item[] Ativa funcionalidades de depuração. A maioria dos usuários não precisará usar isso.
\item[] 
\end{DUlineblock}
\item[] \textbf{bgfx\_screen\_chains}
\item[] 
\item[]
\begin{DUlineblock}{\DUlineblockindent}
\item[] Determina como manipular a renderização BGFX tela a tela. As opções disponíveis são \textbf{hlsl}, \textbf{unfiltered}, and \textbf{default}.
\item[] 
\item[] \textbf{default} -- saída de filtro bilinear predefinido
\item[] \textbf{unfiltered} -- saída sem filtro mais próxima do original
\item[] \textbf{hlsl} -- saída com simulação de tela HLSL usando sombreadores
\item[] 
\item[] Nós fazemos um distinção entre dispositivos de tela emuladas (na qual a chamamos de \textbf{screen} ou \textbf{tela}) e tela física (na qual a chamaremos de \textbf{window} ou \textbf{janela}, configurável através da opção \textbf{-numscreens}). Nós usamos dois pontos (:) para separar janelas e vírgulas (,) para separar as telas. As vírgulas sempre saem do lado de fora da cadeia (veja o exemplo do \textbf{House Mannequin})
\item[] 
\item[] Em uma combinação de só uma janela, no caso de jogos com uma única tela, como o Pac Man em um monitor de PC físico, você pode definir a opção como:
\item[] 
\item[]
\begin{DUlineblock}{\DUlineblockindent}
\item[] \textbf{bgfx\_screen\_chains hlsl}
\item[] 
\end{DUlineblock}
\item[] As coisas se complicam um pouco mais quando chegarmos a várias janelas e várias telas.
\item[] 
\item[] Para usar uma só janela, num jogo com múltiplas telas, como é o caso do jogo Darius usando só um monitor físico de PC, defina as opções para cada uma dessas telas individualmente, assim:
\item[] 
\item[]
\begin{DUlineblock}{\DUlineblockindent}
\item[] \textbf{bgfx\_screen\_chains hlsl,hlsl,hlsl}
\item[] 
\end{DUlineblock}
\item[] Isso também funciona com jogos que usam uma só tela e você está espelhando a saída dela para vários outros monitores físicos. Por exemplo, você pode configurar o jogo Pac Man para ter uma saída não filtrada para ser usada em uma transmissão de vídeo enquanto a saída para segunda tela é configurada para exibir uma tela com os efeitos como HLSL.
\item[] 
\item[] Em um jogo com várias telas em várias janelas, como o jogo Darius em três monitores físicos, defina as opções como mostra abaixo (individual para cada janela):
\item[] 
\item[]
\begin{DUlineblock}{\DUlineblockindent}
\item[] \textbf{bgfx\_screen\_chains hlsl:hlsl:hlsl}
\item[] 
\end{DUlineblock}
\item[] Outro exemplo seria o jogo Taisen Hot Gimmick que usa dois monitores CRT, um para cada jogador que mostra a mão de cada jogador individualmente. Se estiver usando duas janelas (com duas telas físicas):
\item[] 
\item[]
\begin{DUlineblock}{\DUlineblockindent}
\item[] \textbf{bgfx\_screen\_chains hlsl:hlsl}
\item[] 
\end{DUlineblock}
\item[] Outro caso especial, a Nichibutsu tinha uma máquina coquetel especial de Mahjongg que usa uma tela CRT no meio da máquina, junto com outras duas telas LCD individuais para cada jogador que mostrava a mão que cada um tinha. Nós gostaríamos que os LCDs não fossem tão filtrados como eram, enquanto o CRT seria melhorado através do uso do HLSL. Como queremos dar a cada jogador sua própria tela cheia (dois monitores físicos) junto com o LCD, nós fazemos assim:
\item[] 
\item[]
\begin{DUlineblock}{\DUlineblockindent}
\item[] \textbf{-numscreens 2 -view0 ``Player 1'' -view1 ``Player 2'' -video bgfx -bgfx\_screen\_chains hlsl,unfiltered,unfiltered:hlsl,unfiltered,unfiltered}
\item[] 
\end{DUlineblock}
\item[] Isso configura a visualização de cada tela respectivamente, mantendo o efeito de tela CRT com HLSL para cada janela física enquanto fica sem os filtros nas telas LCD.
\item[] 
\item[] Se estiver usando apenas uma janela (uma tela), tendo em mente que o jogo ainda tem três telas, nós faríamos:
\item[] 
\item[]
\begin{DUlineblock}{\DUlineblockindent}
\item[] \textbf{bgfx\_screen\_chains hlsl,unfiltered,unfiltered}
\item[] 
\item[] 
\end{DUlineblock}
\item[] Observe que as vírgulas estão nas bordas externas e qualquer dois-pontos estão no meio.
\item[] 
\end{DUlineblock}
\item[] \textbf{bgfx\_shadow\_mask}
\item[] 
\item[]
\begin{DUlineblock}{\DUlineblockindent}
\item[] Especifica o arquivo PNG para ser usado como efeito de máscara de sombra. Por definição inicial o nome do arquivo é \textbf{slot-mask.png}.
\item[] 
\item[] 
\end{DUlineblock}
\end{DUlineblock}


\subsection{Customizando as configurações de BGFX HLSL dentro do MAME}
\label{advanced/bgfx:customizando-as-configuracoes-de-bgfx-hlsl-dentro-do-mame}
\textbf{Aviso:} \emph{As configurações BGFX HLSL não são gravados ou lidas de
qualquer arquivo de configuração. É esperado que isso mude no futuro.}

Comece rodando o MAME com o jogo de sua preferência (\textbf{mame pacman} por
exemplo)

Use a tecla til (\textbf{\textasciitilde{}}) \footnote[1]{\sphinxAtStartFootnote%
Até que o teclado \textbf{ABNT-2} seja mapeado pela equipe do MAMEDev,
essa tecla fica do lado esquerdo da tecla 1, logo abaixo da
tecla ESQ. (Nota do tradutor)
} para chamar a tela de opções que vai
aparecer na parte de baixo da tela. Use as teclas cima e baixo para
navegar dentre as várias opções, enquanto as teclas esquerda e direita
irão permitir que você altere o valor dessas opções. Os resultados
aparecerão em tempo real conforme elas forem sendo alteradas.

Observe que as configurações são individuais para cada tela.


\section{Os efeitos HLSL para Windows}
\label{advanced/hlsl::doc}\label{advanced/hlsl:os-efeitos-hlsl-para-windows}
Por predefinição, o MAME gera um sinal de vídeo puro, assim como seria
também no hardware original do arcade até o sinal chegar aos circuitos
que levam ao monitor CRT do arcade, com pequenas modificações na saída
(em geral, esticar a imagem do jogo de volta à proporção que se teria
num monitor CRT, geralmente na proporção 4:3), no geral isso funciona
bem, mas perde-se um pouco do fator nostalgia. Os monitores de arcade,
ainda que em perfeitas condições, nunca foram ideais pois devido a sua
natureza o monitor CRT distorciam a imagem original de maneira a
distorcerem significativamente a sua aparência final na tela.

Os monitores CRT dos arcades são uma experiência única na maneira que a
imagem é formada e apresentada na tela, imagem essa que os monitores de
LCD e até mesmo monitores CRT não possuem.

É aí então que que o HLSL entra em cena.

O filtro HLSL simula a maioria dos efeitos de vídeo que um monitor CRT
de arcade teria, fazendo com que o resultado visual seja muito mais
realista. Porém, os filtros HLSL exigem um esforço extra dos recursos do
seu computador e em especial do monitor que você estiver usando.
Além disso, havia centenas de milhares de tipos monitores diferentes nos
fliperamas. Cada um foi ajustado e mantido de forma diferente, o que
significa que, não tem como escolher e definir entre todos eles, apenas
um como referência. Diretrizes básicas serão fornecidas aqui para
ajudá-lo, mas você também poderá pedir mais opiniões em qualquer um dos
fóruns conhecidos sobre o MAME espalhados pela internet.


\subsection{Resolução e relação de aspecto da tela}
\label{advanced/hlsl:resolucao-e-relacao-de-aspecto-da-tela}
A resolução é um assunto muito importante para as configurações do HLSL.
Você desejará que o MAME esteja usando a resolução nativa do seu monitor
para evitar distorções e atrasos adicionais criados pelo seu monitor ao
tentar preencher a imagem na tela.

Enquanto a maioria das máquinas de arcade usava um monitor com proporção
de tela no formato 4:3 (ou 3:4 se o monitor estivesse orientado
verticalmente como é no caso do Pac Man), a essa altura do campeonato é
difícil encontrar nos dias de hoje um monitor ou TV que tenha uma
proporção de tela no formato 4:3. A boa notícia é que esse espaço extra
que sobra nas laterais não é desperdiçado. Muitos gabinetes de arcade na
época utilizavam uma moldura com ilustrações ao redor da tela, caso você
tenha esses arquivos o MAME também irá exibir essas ilustrações na tela.
Para se obter um melhor resultado, ative o visualizador de ilustrações e
selecione o modo recortado ou cropped em Inglês.

Alguns monitores de LCD mais antigos usavam uma resolução nativa de
1280x1024 onde tinham uma proporção de tela no formato 5:4.
Neste exemplo, não há muito espaço extra suficiente para exibir a
ilustração e você vai notar um leve esticamento vertical, porém os
resultados ainda serão bons o suficiente, como se fossem um monitor com
formato 4:3.


\subsection{Introdução ao HLSL}
\label{advanced/hlsl:introducao-ao-hlsl}
Antes de começar, você precisará seguir as instruções de configuração
inicial do MAME encontrada em outra parte deste manual.
As distribuições oficiais do MAME já incluem o HLSL, então você não
precisa baixar nenhum outro arquivo adicional.

Abra o seu MAME.INI no seu editor de texto preferido como o bloco de
notas por exemplo e verifique se as seguintes opções estão definidas
corretamente:
\begin{itemize}
\item {} 
\textbf{video d3d}

\item {} 
\textbf{filter 0}

\end{itemize}

O primeiro é necessário porque HLSL requer suporte do Direct3D. O último
desliga filtro extras que possam interferir com a saída HLSL.

Por último, uma edição a mais para ativar o HLSL:
\begin{itemize}
\item {} 
\textbf{hlsl\_enable 1}

\end{itemize}

Salve o arquivo .INI e você está pronto para começar.

Várias predefinições foram incluídas na pasta INI junto com o MAME,
permitindo um bom ponto de partida para as configurações iniciais de
tela para os consoles Nintendo Game Boy, Nintendo Game Boy Advance,
Rasterizado e Vetores.


\subsection{Customizando as configurações HLSL dentro MAME}
\label{advanced/hlsl:customizando-as-configuracoes-hlsl-dentro-mame}
Por vários motivos complicados de explicar, as configurações HLSL não
são mais salvas quando você sai do MAME. Isso significa que apesar das
configurações exigirem um pouco mais de trabalho de sua parte, os
resultados sempre sairão conforme esperado.

Comece rodando o MAME com o jogo de sua preferência como por exemplo
\textbf{mame pacman}.

Use a tecla til (\textbf{\textasciitilde{}}) \footnote[1]{\sphinxAtStartFootnote%
Até que o teclado \textbf{ABNT-2} seja mapeado pela equipe do MAMEDev,
essa tecla fica do lado esquerdo da tecla 1, logo abaixo da
tecla ESQ. (Nota do tradutor)
} para chamar a tela de opções que vai
aparecer na parte de baixo da tela. Use as teclas cima e baixo para
navegar dentre as várias opções, enquanto as teclas esquerda e direita
irão permitir que você altere o valor dessas opções. Os resultados
aparecerão em tempo real conforme elas forem sendo alteradas.

Depois de encontrar as configurações desejadas, anote os números em um
bloco de notas e saia do MAME.


\subsection{Alterando as configurações}
\label{advanced/hlsl:alterando-as-configuracoes}
Como descrito em {\hyperref[advanced/multiconfig:advanced\string-multi\string-cfg]{\sphinxcrossref{\DUrole{std,std-ref}{A ordem de leitura dos arquivos}}}}, o MAME segue uma sequência
na hora de processar os arquivos INI. As configurações HLSL podem ser
editadas diretamente no arquivo MAME.INI, porém para tirar melhor
proveito do poder dos arquivos de configuração do MAME, talvez seja
melhor copiar as opções do HLSL do MAME.INI para um outro arquivo de
configuração e fazer as modificações lá.

Por exemplo, uma vez que você encontrou configurações de HLSL que acha
que são apropriadas para os jogos de Neo-Geo, você pode colocar essas
configurações num arquivo neogeo.ini para que todos os jogos de Neo-Geo
usem essas configurações sem que você tenha que adicioná-las manualmente
uma a uma em diferentes arquivos INI como o nome do jogo.


\subsection{Alterando as configurações}
\label{advanced/hlsl:id2}
\begin{DUlineblock}{0em}
\item[] \textbf{hlslpath}
\item[] 
\item[]
\begin{DUlineblock}{\DUlineblockindent}
\item[] É aqui que seus arquivos de sombreamento HLSL são armazenados. Por definição inicial, o nome desta pasta será HLSL localizado onde o seu MAME estiver instalado.
\item[] 
\end{DUlineblock}
\item[] \textbf{hlsl\_snap\_width}
\item[] \textbf{hlsl\_snap\_height}
\item[] 
\item[]
\begin{DUlineblock}{\DUlineblockindent}
\item[] Define a resolução de saída na qual as capturas de tela HLSL Alt+F12 terão.
\item[] 
\end{DUlineblock}
\item[] \textbf{shadow\_mask\_alpha} (\emph{Quantidade da Máscara de Sombra})
\item[] 
\item[]
\begin{DUlineblock}{\DUlineblockindent}
\item[] Isso define o quão forte o efeito da máscara de sombra será. O intervalo aceitável vai de 0 a 1, onde 0 não mostra nenhum efeito de sombra da máscara, 1 a mascara será completamente opaca e 0.5 será 50\% transparente.
\item[] 
\end{DUlineblock}
\item[] \textbf{shadow\_mask\_tile\_mode} (\emph{Máscara de Sombra em Modo Ladrilhado})
\item[] 
\item[]
\begin{DUlineblock}{\DUlineblockindent}
\item[] Isso define se a máscara de sombra deve ser lado a lado com base na resolução de tela do seu monitor ou com base na resolução de origem do sistema emulado. Os valores válidos são entre 0 para modo te tela \emph{Screen} e 1 para modo de origem \emph{Source}.
\item[] 
\end{DUlineblock}
\item[] \textbf{shadow\_mask\_texture}
\item[] \textbf{shadow\_mask\_x\_count} (\emph{Quantidade X de Pixels Máscara de Sombra})
\item[] \textbf{shadow\_mask\_y\_count} (\emph{Quantidade Y de Pixels Máscara de Sombra})
\item[] \textbf{shadow\_mask\_usize} (\emph{Tamanho U da Máscara de Sombra})
\item[] \textbf{shadow\_mask\_vsize} (\emph{Tamanho V da Máscara de Sombra})
\item[] \textbf{shadow\_mask\_x\_count} (\emph{Deslocamento U da Máscara de Sombra})
\item[] \textbf{shadow\_mask\_y\_count} (\emph{Deslocamento V da Máscara de Sombra})
\item[] 
\item[]
\begin{DUlineblock}{\DUlineblockindent}
\item[] Essas configurações devem estar em harmonia entre si. As regras \textbf{shadow\_mask\_texture} em particular, definem as regras de como você deve configurar as outras opções.
\item[] 
\item[] \textbf{shadow\_mask\_texture} configura a textura do efeito de máscara de sombra. O MAME vem com três máscaras de sombra: \emph{aperture-grille.png}, \emph{shadow-mask.png}, e \emph{slot-mask.png}
\item[] 
\item[] \textbf{shadow\_mask\_usize} e \textbf{shadow\_mask\_vsize} definem o tamanho a ser usado pela textura do efeito  shadow\_mask\_texture em valores de porcentagem, começando pelo canto superior esquerdo. Isso significa que, para uma textura com o tamanho real com pixels de 24x24 e um tamanho de u/v com 0.5,0.5, serão usados 12x12 pixels no canto superior esquerdo. Lembre-se de definir um tamanho de u/v que possibilite organizar a textura lado a lado sem lacunas ou falhas. 0.5,0.5 é bom para qualquer uma das textura de máscara de sombra que estão inclusas no MAME.
\item[] 
\item[] \textbf{shadow\_mask\_x\_count} e \textbf{shadow\_mask\_y\_count} define quantos pixels de tela devem ser usados para exibir o tamanho u/v da textura. se você usar o exemplo acima e configurar a quantidade x/y em proporção de 12,12 pixels de textura, ela será exibida numa proporção de 1:1 na tela. Caso você defina a quantidade x/y em proporção de 24,24 a textura será exibida de maneira que ficará duas vezes maior.
\item[] 
\end{DUlineblock}
\item[] amostra de configuração para \textbf{shadow\_mask.png}:
\item[] 
\item[]
\begin{DUlineblock}{\DUlineblockindent}
\item[] shadow\_mask\_texture shadow-mask.png
\item[] shadow\_mask\_x\_count 12
\item[] shadow\_mask\_y\_count 6 or 12
\item[] shadow\_mask\_usize 0.5
\item[] shadow\_mask\_vsize 0.5
\item[] 
\end{DUlineblock}
\item[] amostra de configuração para \textbf{slot-mask.png}:
\item[] 
\item[]
\begin{DUlineblock}{\DUlineblockindent}
\item[] shadow\_mask\_texture slot-mask.png
\item[] shadow\_mask\_x\_count 12
\item[] shadow\_mask\_y\_count 8 or 16
\item[] shadow\_mask\_usize 0.5
\item[] shadow\_mask\_vsize 0.5
\item[] 
\end{DUlineblock}
\item[] amostra de configuração para \textbf{aperture-grille}:
\item[] 
\item[]
\begin{DUlineblock}{\DUlineblockindent}
\item[] shadow\_mask\_texture aperture-grille.png
\item[] shadow\_mask\_x\_count 12
\item[] shadow\_mask\_y\_count 12 or any
\item[] shadow\_mask\_usize 0.5
\item[] shadow\_mask\_vsize 0.5
\item[] 
\item[] \textbf{shadow\_mask\_uoffset} e \textbf{shadow\_mask\_voffset} podem ser usados para customizar o alcance do alinhamento final da máscara de sombreamento a nível de subpixel. O intervalo aceitável vai de -1.00 até 1.00, onde 0.5 move a máscara de sombreamento em 50\% com relação ao tamanho u/v da textura.
\item[] 
\end{DUlineblock}
\item[] \textbf{distortion} (\emph{Quantidade de Distorção Quádrica})
\item[] 
\item[]
\begin{DUlineblock}{\DUlineblockindent}
\item[] Essa opção determina a intensidade da distorção quádrica da imagem na tela.
\item[] 
\end{DUlineblock}
\item[] \textbf{cubic\_distortion} (\emph{Quantidade de Distorção Cúbica})
\item[] 
\item[]
\begin{DUlineblock}{\DUlineblockindent}
\item[] Essa configuração determina a intensidade da distorção cúbica da imagem na tela.
\item[] 
\item[] Os fatores de distorção em ambos podem ser negativos para que um seja compensado pelo outro, como distorção 0.5 e \emph{cubic\_distortion} -0.5 por exemplo
\item[] 
\end{DUlineblock}
\item[] \textbf{distort\_corner} (\emph{Quantidade de Distorção dos Cantos da Tela})
\item[] 
\item[]
\begin{DUlineblock}{\DUlineblockindent}
\item[] Essa configuração determina a intensidade de distorção dos cantos da tela, o que não afeta a distorção da imagem na tela em si.
\item[] 
\end{DUlineblock}
\item[] \textbf{round\_corner} (\emph{Quantidade de Arredondamento dos Cantos})
\item[] 
\item[]
\begin{DUlineblock}{\DUlineblockindent}
\item[] Os cantos da tela podem ser arredondados com o uso dessa configuração.
\item[] 
\end{DUlineblock}
\item[] \textbf{smooth\_border} (\emph{Quantidade de Suavização das Bordas})
\item[] 
\item[]
\begin{DUlineblock}{\DUlineblockindent}
\item[] Configura a suavização/desfoque dos cantos na região das bordas da tela.
\item[] 
\end{DUlineblock}
\item[] \textbf{reflection} (\emph{Quantidade de Reflexo})
\item[] 
\item[]
\begin{DUlineblock}{\DUlineblockindent}
\item[] Se configurado acima de 0, cria um efeito mancha esbranquiçada na tela. Por definição inicial, este é colocado no canto superior direito da tela. Editando o arquivo \emph{POST.FX} na seção GetSpotAddend, você poderá alterar a posição do local. Os valores válidos ficam entre 0.00 até 1.00.
\item[] 
\end{DUlineblock}
\item[] \textbf{vignetting} (\emph{Quantidade do Efeito Vinheta})
\item[] 
\item[]
\begin{DUlineblock}{\DUlineblockindent}
\item[] Se configurado acima de 0, incrementa a vinheta nos cantos da tela com pseudo efeito 3D. Os valores válidos ficam entre 0.00 até 1.00.
\item[] 
\end{DUlineblock}
\item[] \textbf{scanline\_alpha} (\emph{Quantidade de Linhas de Escaneamento})
\item[] 
\item[]
\begin{DUlineblock}{\DUlineblockindent}
\item[] Determina o quão forte será o efeito de scanlines na tela. O intervalo aceitável fica entre 0 e 1, onde 0 não mostra nenhum efeito, 1 seria uma linha de escaneamento totalmente preta e 0.5 será 50\% transparente. Observe que nos monitores arcade as linhas de escaneamento não completamente pretas na tela.
\item[] 
\end{DUlineblock}
\item[] \textbf{scanline\_size} (\emph{Escala Total das Linhas de Escaneamento})
\item[] 
\item[]
\begin{DUlineblock}{\DUlineblockindent}
\item[] Determina o espaçamento das linhas de escaneamento da tela. Se configurado como 1, mostra uma consistente alternância de espaço entre as linhas da tela e as linhas de escaneamento.
\item[] 
\end{DUlineblock}
\item[] \textbf{scanline\_height} (\emph{Escala Individual das Linhas de Escaneamento})
\item[] 
\item[]
\begin{DUlineblock}{\DUlineblockindent}
\item[] Determina o tamanho total de cada linha de escaneamento. Configurando com um valor menor que 1, fazem elas ficarem mais finas, maiores que 1 ficam mais grossas.
\item[] 
\end{DUlineblock}
\item[] \textbf{scanline\_variation} (\emph{Variação das Linhas de Escaneamento})
\item[] 
\item[]
\begin{DUlineblock}{\DUlineblockindent}
\item[] Determina o o tamanho de cada linhas de escaneamento, dependendo do seu brilho. Linhas de escaneamento mais claras ficarão mais finas em comparação com as mais escuras. Os valores válidos ficam entre 0 e 2.0, onde o valor predefinido é 1.0. Em 0.0 as linhas de escaneamento ficam com o mesmo tamanho, independente do seu brilho.
\item[] 
\end{DUlineblock}
\item[] \textbf{scanline\_bright\_scale} (\emph{Escala de Brilho das Linhas de Escaneamento})
\item[] 
\item[]
\begin{DUlineblock}{\DUlineblockindent}
\item[] Determina o quão brilhante a linha de escalonamento será. Valores maior que 1 faz com que elas fiquem mais clara, valores menores as deixam mais obscuras. Configurando-a para 0, fazem com que as linhas de escaneamento desapareçam por completo.
\item[] 
\end{DUlineblock}
\item[] \textbf{scanline\_bright\_offset} (\emph{Deslocamento de Brilho das Linhas de Escaneamento})
\item[] 
\item[]
\begin{DUlineblock}{\DUlineblockindent}
\item[] Define o efeito de brilho/saturação das linhas de escaneamento, suavizando e alisando a parte de cima e de baixo de cada linha de escaneamento.
\item[] 
\end{DUlineblock}
\item[] \textbf{scanline\_jitter} (\emph{Quantidade de tremulação das Linhas de Escaneamento})
\item[] 
\item[]
\begin{DUlineblock}{\DUlineblockindent}
\item[] Determina intensidade de oscilação ou tremulação das linhas de escaneamento na tela do monitor. Alerta: Valores muitos altos podem irritar seus olhos.
\item[] 
\end{DUlineblock}
\item[] \textbf{hum\_bar\_alpha} (\emph{Quantidade de Interferência Vertical})
\item[] 
\item[]
\begin{DUlineblock}{\DUlineblockindent}
\item[] Determina a intensidade deste efeito de interferência.
\item[] 
\end{DUlineblock}
\item[] \textbf{defocus} (\emph{Desfoco})
\item[] 
\item[]
\begin{DUlineblock}{\DUlineblockindent}
\item[] Determina o desfoco da tela borrando os pixels individualmente como as bordas de um monitor com manutenção precária. Especifique com valores X,Y (\textbf{defocus 1,1} por exemplo)
\item[] 
\end{DUlineblock}
\item[] \textbf{converge\_x} (\emph{Convergência Linear X, RGB})
\item[] \textbf{converge\_y} (\emph{Convergência Linear Y, RGB})
\item[] \textbf{radial\_converge\_x} (\emph{Convergência Radial X, RGB})
\item[] \textbf{radial\_converge\_y} (\emph{Convergência Radial Y, RGB})
\item[] 
\item[]
\begin{DUlineblock}{\DUlineblockindent}
\item[] Ajusta os valores de convergência dos canais vermelho, verde e azul para uma determinada direção. Muitos monitores mal cuidados tem uma péssima convergência que causa um efeito fantasma devido ao vazamento de cores que ficam fora do eixo de um sprite, essa opção simula esse efeito.
\item[] 
\end{DUlineblock}
\item[] \textbf{red\_ratio} (\emph{Proporção de Vermelho do RGB})
\item[] \textbf{grn\_ratio} (\emph{Proporção de Verde do RGB})
\item[] \textbf{blu\_ratio} (\emph{Proporção do Azul do RGB})
\item[] 
\item[]
\begin{DUlineblock}{\DUlineblockindent}
\item[] Define a matriz 3x3 que será multiplicado com os sinais RGB para simular a proporção de interferência no canal de cor. Por exemplo, o sinal verde com (0.100, 1.000, 0.250) é 10\% mais fraco que o sinal vermelho e 25\% mais forte no sinal azul.
\item[] 
\end{DUlineblock}
\item[] \textbf{offset} (\emph{Deslocamento do Sinal})
\item[] 
\item[]
\begin{DUlineblock}{\DUlineblockindent}
\item[] Fortalece ou enfraquece a intensidade do sinal de uma determinada cor. Por exemplo, o sinal vermelho com um valor de 0.5 com um desvio/deslocamento de 0.2 será intensificado para 0.7.
\item[] 
\end{DUlineblock}
\item[] \textbf{scale} (\emph{Signal Scale})
\item[] 
\item[]
\begin{DUlineblock}{\DUlineblockindent}
\item[] Aplica uma escala ao valor da cor do sinal atual. Por exemplo, o sinal vermelho com um valor de 0.5 em uma escala de 1.1, resultará num sinal de vermelho com 0.55
\item[] 
\end{DUlineblock}
\item[] \textbf{power} (\emph{Expoente do Sinal, RGB})
\item[] 
\item[]
\begin{DUlineblock}{\DUlineblockindent}
\item[] Valor expoente do valor da cor do sinal atual, também conhecido como gama. O gama é o valor relativo entre o claro e o escuro de uma imagem. Por exemplo, o sinal vermelho com um valor de 0.5 com o valor power no vermelho de 2, resultará num sinal de vermelho com 0.25.
\item[] 
\item[] Em jogos com vetores, essa configuração também pode ser usada para ajudar a grossura dessas linhas.
\item[] 
\end{DUlineblock}
\item[] \textbf{floor} (\emph{Piso do Sinal RGB})
\item[] 
\item[]
\begin{DUlineblock}{\DUlineblockindent}
\item[] Define o valor mínimo absoluto para um sinal de cor. Por exemplo, o sinal vermelho com um valor de 0.0 (ausência total do sinal vermelho) com o piso do sinal vermelho de 0.2, resultará num sinal vermelho com 0.2.
\item[] 
\item[] Normalmente usado em conjunto com a ilustração ativada para fazer a tela ter um brilho da trama mais fraca.
\item[] 
\end{DUlineblock}
\item[] \textbf{phosphor\_life} (\emph{Tempo de Vida do Fósforo RGB})
\item[] 
\item[]
\begin{DUlineblock}{\DUlineblockindent}
\item[] Define por quanto tempo a cor do sinal continua ativa na tela, também conhecido como efeito fantasma de tela. O valor 0 não produz nenhum efeito fantasma, enquanto o valor 1 deixa um rastro para trás que só volta a ser alterado por sinal de cor de maior valor.
\item[] 
\item[] Isso também afeta bastante os jogos vetoriais.
\item[] 
\end{DUlineblock}
\item[] \textbf{saturation} (\emph{Saturação de Cor})
\item[] 
\item[]
\begin{DUlineblock}{\DUlineblockindent}
\item[] Define os níveis de saturação de cor.
\item[] 
\end{DUlineblock}
\item[] \textbf{bloom\_blend\_mode} (\emph{Tipo de Combinação do Efeito Bloom})
\item[] 
\item[]
\begin{DUlineblock}{\DUlineblockindent}
\item[] Define o tipo de efeito efeito Bloom. Os valores válidos ficam entre 0 para um tipo mais \emph{Claro} e 1 para um tipo mais \emph{Escuro}, essa última só é útil com monitores do tipo STN LCD.
\item[] 
\end{DUlineblock}
\item[] \textbf{bloom\_scale} (\emph{Escala do Efeito Bloom})
\item[] 
\item[]
\begin{DUlineblock}{\DUlineblockindent}
\item[] Determina a intensidade do efeito bloom. Os monitores CRT dos arcades tem uma tendência a ter esse efeito bloom naturalmente, onde as cores mais claras se misturam com os pixels que ficam ao redor. Esse efeito faz um uso intensivo das placas de vídeo e pode ser totalmente desligada ao definir essa opção com 0 para economizar os recursos de processamento da GPU.
\item[] 
\end{DUlineblock}
\item[] \textbf{bloom\_overdrive} (\emph{Saturação do Efeito, RGB})
\item[] 
\item[]
\begin{DUlineblock}{\DUlineblockindent}
\item[] Define o valor, separados por vírgula, o quão claro um sinal de cor RGB pode chegar, onde a saturação ficará voltada para o branco. Isso só é útil em jogos com tramas coloridas, LCD colorido ou jogos vetorizados coloridos.
\item[] 
\end{DUlineblock}
\item[] \textbf{bloom\_lvl0\_weight} (\emph{Escala do Nível do Bloom 0})
\item[] \textbf{bloom\_lvl1\_weight} (\emph{Escala do Nível do Bloom 1})
\item[]
\begin{DUlineblock}{\DUlineblockindent}
\item[] .  .  .  .
\end{DUlineblock}
\item[] \textbf{bloom\_lvl7\_weight} (\emph{Escala do Nível do Bloom 7})
\item[] \textbf{bloom\_lvl8\_weight} (\emph{Escala do Nível do Bloom 8})
\item[] 
\item[]
\begin{DUlineblock}{\DUlineblockindent}
\item[] Define a quantidade do efeito bloom. Os valores válidos ficam entre 0.00 até 1.00. Se for usado da maneira correta em conjunto com o phosphor\_life o efeito de brilho/fantasma enquanto os objetos se movem na tela poderá ser atingido.
\item[] 
\end{DUlineblock}
\item[] \textbf{hlsl\_write}
\item[] 
\item[]
\begin{DUlineblock}{\DUlineblockindent}
\item[] Ativa a gravação de vídeo com os efeitos HLSL em contêiner AVI em formado RAW se for definido como \emph{1}. Por ser um formato sem compressão (RAW), o arquivo ocupa muito espaço de forma muito rápida, é recomendável usar um HD com uma velocidade rápida de escrita. O valor predefinido é desligado ou \emph{0}.
\item[] 
\end{DUlineblock}
\end{DUlineblock}

\begin{DUlineblock}{0em}
\item[] Padrões sugeridos para os jogos rasterizados:
\item[] 
\end{DUlineblock}

\noindent\begin{tabular}{|p{0.317\linewidth}|p{0.317\linewidth}|p{0.317\linewidth}|}
\hline

\begin{DUlineblock}{0em}
\item[] bloom\_lvl0\_weight    1.00
\item[] bloom\_lvl1\_weight    0.64
\item[] bloom\_lvl2\_weight    0.32
\item[] bloom\_lvl3\_weight    0.16
\item[] bloom\_lvl4\_weight    0.08
\item[] bloom\_lvl5\_weight    0.06
\item[] bloom\_lvl6\_weight    0.04
\item[] bloom\_lvl7\_weight    0.02
\item[] bloom\_lvl8\_weight    0.01
\end{DUlineblock}
&
\begin{DUlineblock}{0em}
\item[] Peso 0 do Nível Bloom
\item[] Peso 1 do Nível Bloom
\item[] Peso 2 do Nível Bloom
\item[] Peso 3 do Nível Bloom
\item[] Peso 4 do Nível Bloom
\item[] Peso 1 do Nível Bloom
\item[] Peso 1 do Nível Bloom
\item[] Peso 1 do Nível Bloom
\item[] Peso 1 do Nível Bloom
\end{DUlineblock}
&
\begin{DUlineblock}{0em}
\item[] Tamanho Máximo.
\item[] 1/4 menor que o nível 0
\item[] 1/4 menor que o nível 1
\item[] 1/4 menor que o nível 2
\item[] 1/4 menor que o nível 3
\item[] 1/4 menor que o nível 4
\item[] 1/4 menor que o nível 5
\item[] 1/4 menor que o nível 6
\item[] 1/4 menor que o nível 7
\end{DUlineblock}
\\
\hline\end{tabular}



\subsection{Jogos vetorizados}
\label{advanced/hlsl:jogos-vetorizados}
Os efeitos HLSL também podem ser usados com jogos vetorizados. Devido a
uma grande variedade de opções para a configuração individual de jogos
vetoriais, é altamente recomendável que você os adicione em arquivos INI
individuais jogo a jogo (tempest.ini por exemplo).

As máscaras de sombreamento só estão disponíveis em jogos vetoriais e
não devem ser usados em jogos vetoriais monocromáticos. Além disso, os
jogos de vetoriais não usavam linhas de varredura, de modo que também
devem ser desativados.

Abra o seu arquivo INI no seu editor de texto preferido (o Bloco de
notas por exemplo) e verifique se as seguintes opções estão configuradas
corretamente:
\begin{itemize}
\item {} 
\textbf{video d3d}

\item {} 
\textbf{filter 0}

\item {} 
\textbf{hlsl\_enable 1}

\end{itemize}

Nas Opções Principais de Vetores:
\begin{itemize}
\item {} 
\textbf{beam\_width\_min 1.0} (\emph{Feixe Com o Máximo de})

\item {} 
\textbf{beam\_width\_max 1.0} (\emph{Feixe Com o Mínimo de})

\item {} 
\textbf{beam\_intensity\_weight 0.0} (\emph{Altura da Intensidade do Feixe})

\item {} 
\textbf{flicker 0.0} (\emph{Vector Flicker})

\end{itemize}

Na Seção das Opções de Pós Processamento de Vetores:
\begin{itemize}
\item {} 
\textbf{vector\_beam\_smooth 0.0} (\emph{Quantidade de Suavização do Feixe do
Vetor})

\item {} 
\textbf{vector\_length\_scale 0.5} (\emph{Atenuação Máxima do Vetor})

\item {} 
\textbf{vector\_length\_ratio 0.5} (\emph{Extensão Mínima de Atenuação do Vetor})

\end{itemize}

Padrões sugeridos para jogos vetoriais:
\begin{itemize}
\item {} 
\textbf{bloom\_scale} o valor dever ser maior em jogos vetoriais do que os
jogos rasterizados. Para obter um melhor efeito, tente valores entre
0.4 e 1.0.

\item {} 
\textbf{bloom\_overdrive} só deve ser usado em com jogos vetoriais
coloridos.

\item {} 
\textbf{bloom\_lvl\_weights} deve ser configurado como mostrado abaixo:

\end{itemize}

\noindent\begin{tabular}{|p{0.317\linewidth}|p{0.317\linewidth}|p{0.317\linewidth}|}
\hline

\begin{DUlineblock}{0em}
\item[] bloom\_lvl0\_weight    1.00
\item[] bloom\_lvl1\_weight    0.48
\item[] bloom\_lvl2\_weight    0.32
\item[] bloom\_lvl3\_weight    0.24
\item[] bloom\_lvl4\_weight    0.16
\item[] bloom\_lvl5\_weight    0.24
\item[] bloom\_lvl6\_weight    0.32
\item[] bloom\_lvl7\_weight    0.48
\item[] bloom\_lvl8\_weight    0.64
\end{DUlineblock}
&
\begin{DUlineblock}{0em}
\item[] Peso 0 do Nível Bloom
\item[] Peso 1 do Nível Bloom
\item[] Peso 2 do Nível Bloom
\item[] Peso 3 do Nível Bloom
\item[] Peso 4 do Nível Bloom
\item[] Peso 1 do Nível Bloom
\item[] Peso 1 do Nível Bloom
\item[] Peso 1 do Nível Bloom
\item[] Peso 1 do Nível Bloom
\end{DUlineblock}
&
\begin{DUlineblock}{0em}
\item[] Tamanho Máximo.
\item[] 1/4 menor que o nível 0
\item[] 1/4 menor que o nível 1
\item[] 1/4 menor que o nível 2
\item[] 1/4 menor que o nível 3
\item[] 1/4 menor que o nível 4
\item[] 1/4 menor que o nível 5
\item[] 1/4 menor que o nível 6
\item[] 1/4 menor que o nível 7
\end{DUlineblock}
\\
\hline\end{tabular}



\section{Efeitos GLSL para *nix, OS X e Windows}
\label{advanced/glsl::doc}\label{advanced/glsl:efeitos-glsl-para-nix-os-x-e-windows}
Por predefinição, o MAME gera um sinal de vídeo puro, assim como seria
também no hardware original do arcade até o sinal chegar aos circuitos
que levam o sinal ao monitor CRT do arcade, com pequenas modificações na
saída (em geral, esticar a imagem do jogo de volta à proporção que se
teria num monitor CRT, geralmente na proporção 4:3), no geral isso
funciona bem, mas perde-se um pouco do fator nostalgia. Os monitores de
arcade, ainda que em perfeitas condições, nunca foram ideais pois devido
a sua natureza o monitor CRT distorciam a imagem original de maneira
a distorcerem significativamente a sua aparência final na tela.

Os monitores CRT dos arcades são uma experiência única na maneira que a
imagem é formada e apresentada na tela, imagem essa que os monitores de
LCD e até mesmo monitores CRT não possuem.

É aí então que que o GLSL entra em cena.

O filtro GLSL simula a maioria dos efeitos de vídeo que um monitor CRT
de arcade teria, fazendo com que o resultado visual seja muito mais
realista. Porém, os filtros GLSL exigem um esforço extra dos recursos do
seu computador e em especial do monitor que você estiver usando.
Além disso, havia centenas de milhares de tipos monitores diferentes nos
fliperamas. Cada um foi ajustado e mantido de forma diferente, o que
significa que, não tem como escolher e definir entre todos eles, apenas
um como referência. Diretrizes básicas serão fornecidas aqui para
ajudá-lo, mas você também poderá pedir mais opiniões em qualquer um dos
fóruns conhecidos sobre o MAME espalhados pela internet.


\subsection{Resolução e relação de aspecto da tela}
\label{advanced/glsl:resolucao-e-relacao-de-aspecto-da-tela}
A resolução é um assunto muito importante para as configurações do GLSL.
Você desejará que o MAME esteja usando a resolução nativa do seu monitor
para evitar distorções e atrasos adicionais criados pelo seu monitor ao
tentar preencher a imagem na tela.

Enquanto a maioria das máquinas de arcade usava um monitor com proporção
de tela no formato 4:3 (ou 3:4 se o monitor estivesse orientado
verticalmente como é no caso do Pac Man), a essa altura do campeonato é
difícil encontrar nos dias de hoje um monitor ou TV que tenha uma
proporção de tela no formato 4:3. A boa notícia é que esse espaço extra
que sobra nas laterais não é desperdiçado. Muitos gabinetes de arcade na
época utilizavam uma moldura com ilustrações ao redor da tela, caso você
tenha esses arquivos o MAME também irá exibir essas ilustrações na tela.
Para se obter um melhor resultado, ative o visualizador de ilustrações e
selecione o modo recortado ou cropped em Inglês.

Alguns monitores de LCD mais antigos usavam uma resolução nativa de
1280x1024 onde tinham uma proporção de tela no formato 5:4.
Neste exemplo, não há muito espaço extra suficiente para exibir a
ilustração e você vai notar um leve esticamento vertical, porém os
resultados ainda serão bons o suficiente, como se fossem um monitor com
formato 4:3.


\subsection{Introdução ao GLSL}
\label{advanced/glsl:introducao-ao-glsl}
Antes de começar, você precisará seguir as instruções de configuração
inicial do MAME encontrada em outra parte deste manual.
As distribuições oficiais do MAME já são compatíveis com o GLSL, mas
\textbf{NÃO} incluem os arquivos de sombreamento GLSL. Você precisará obter
esses arquivos de sombreamento através de um outro fornecedor qualquer
pela internet.

Abra o seu MAME.INI no seu editor de texto preferido como o bloco de
notas por exemplo e verifique se as seguintes opções estão definidas
corretamente:
\begin{itemize}
\item {} 
\textbf{video opengl}

\item {} 
\textbf{filter 0}

\end{itemize}

O primeiro é necessário pois o GLSL requer suporte ao OpenGL. Já o
último desliga os filtros extras que possam interferir com a saída GLSL.

Por último, resta uma edição a mais para ativar o GLSL:
\begin{itemize}
\item {} 
\textbf{gl\_glsl 1}

\end{itemize}

Salve o arquivo .INI e já estamos pronto para começar.


\subsection{Customizando as configurações GLSL de dentro do MAME}
\label{advanced/glsl:customizando-as-configuracoes-glsl-de-dentro-do-mame}
Por vários motivos complicados de explicar, as configurações GLSL não
são mais salvas quando você sai do MAME. Isso significa que apesar das
configurações exigirem um pouco mais de trabalho de sua parte, os
resultados sempre sairão conforme esperado.

Comece rodando o MAME com o jogo de sua preferência como por exemplo
\textbf{mame pacman}.

Use a tecla til (\textbf{\textasciitilde{}}) \footnote[1]{\sphinxAtStartFootnote%
Até que o teclado \textbf{ABNT-2} seja mapeado pela equipe do MAMEDev,
essa tecla fica do lado esquerdo da tecla 1, logo abaixo da
tecla ESQ. (Nota do tradutor)
} para chamar a tela de opções que vai
aparecer na parte de baixo da tela. Use as teclas cima e baixo para
navegar dentre as várias opções, enquanto as teclas esquerda e direita
irão permitir que você altere o valor dessas opções. Os resultados
aparecerão em tempo real conforme elas forem sendo alteradas.

Depois de encontrar as configurações desejadas, anote os números em um
bloco de notas e saia do MAME.


\subsection{Alterando as configurações}
\label{advanced/glsl:alterando-as-configuracoes}
Como descrito em {\hyperref[advanced/multiconfig:advanced\string-multi\string-cfg]{\sphinxcrossref{\DUrole{std,std-ref}{A ordem de leitura dos arquivos}}}}, o MAME segue uma sequência
na hora de processar os arquivos INI. As configurações GLSL podem ser
editadas diretamente no arquivo MAME.INI, porém para tirar melhor
proveito do poder dos arquivos de configuração do MAME, talvez seja
melhor copiar as opções do GLSL do MAME.INI para um outro arquivo de
configuração e fazer as modificações lá.

Por exemplo, uma vez que você encontrou configurações de GLSL que acha
que são apropriadas para os jogos de Neo-Geo, você pode colocar essas
configurações num arquivo neogeo.ini para que todos os jogos de Neo-Geo
usem essas configurações sem que você tenha que adicioná-las manualmente
uma a uma em diferentes arquivos INI como o nome do jogo.


\subsection{Alterando as configurações}
\label{advanced/glsl:id2}
\begin{DUlineblock}{0em}
\item[] \textbf{gl\_glsl}
\item[] 
\item[]
\begin{DUlineblock}{\DUlineblockindent}
\item[] Caso o valor seja \textbf{1} ativa o GLSL, desativa se for definido como \textbf{0}. O valor predefinido é \textbf{0}.
\item[] 
\end{DUlineblock}
\item[] \textbf{gl\_glsl\_filter}
\item[] 
\item[]
\begin{DUlineblock}{\DUlineblockindent}
\item[] Ativa o filtro na saída do GLSL. Reduz o serrilhado no contorno da imagem,  essa opção deixa a imagem um pouco suavizada.
\item[] 
\end{DUlineblock}
\item[] \textbf{glsl\_shader\_mame0}
\item[]
\begin{DUlineblock}{\DUlineblockindent}
\item[] ...
\end{DUlineblock}
\item[] \textbf{glsl\_shader\_mame9}
\item[] 
\item[]
\begin{DUlineblock}{\DUlineblockindent}
\item[] Especifica quais dos sombreadores usar, na ordem entre \textbf{0} a \textbf{9}. Se informe com o autor do seu pacote de sombreadores para saber em que ordem rodar primeiro para obter o melhor efeito.
\item[] 
\end{DUlineblock}
\item[] \textbf{glsl\_shader\_screen0}
\item[]
\begin{DUlineblock}{\DUlineblockindent}
\item[] ...
\end{DUlineblock}
\item[] \textbf{glsl\_shader\_screen9}
\item[] 
\item[]
\begin{DUlineblock}{\DUlineblockindent}
\item[] Determina em qual tela aplicar os efeitos.
\item[] 
\end{DUlineblock}
\end{DUlineblock}


\section{Controladores estáticos de IDs}
\label{advanced/devicemap:controladores-estaticos-de-ids}\label{advanced/devicemap::doc}
Já é predefinido que os IDs de mapeamento entre dispositivos e
controladores não são estáticos. Por exemplo, o controlador de um
controle joystick pode ser atribuído inicialmente para ``Joy 1'', mas
depois de uma reinicialização, ele pode ser redefinido como ``Joy 3''.

O MAME enumera os dispositivos conectados e os atribui IDs de
controlador com base na ordem de enumeração. Os fatores que podem causar
a alteração dessas IDs são conectar ou desconectar os dispositivos USB,
alterar as portas ou hubs, assim como até mesmo a reinicialização do
sistema.

É um pouco complicado garantir que as IDs do controlador sejam sempre as
mesmas, é para isso que usamos a configuração ``mapdevice''.
Essa configuração permite especificar um ID de dispositivo para um ID de
controlador, garantindo ao MAME que o dispositivo especificado sempre seja
mapeado para o mesmo ID de controlador.
\clearpage

\subsection{O uso do mapdevice}
\label{advanced/devicemap:o-uso-do-mapdevice}
O ``mapdevice'' é um elemento de configuração definido através de um
marcador salvo em um arquivo no formato xml. Ele necessita de dois
atributos, o ``device'' e o ``controller''.
Nota: Essa configuração só entra em vigor quando for adicionada ao
arquivo de configuração \textbf{ctrl}.

O atributo ``device'' define o ID do dispositivo a ser mapeado. Também
pode ser uma subcategoria de caracteres deste ID. No MAME use o modo
loquaz através da opção \emph{-verbose} para que você possa ver os
dispositivos disponíveis durante a inicialização (mais detalhes logo
abaixo).

No MAME o atributo ``controller'' define o ID do controlador que é
composto pelo índice do controlador e por uma classe de controladores
como ``JOYCODE'', ``GUNCODE'', ``MOUSECODE'', etc.


\subsection{Exemplo de configuração}
\label{advanced/devicemap:exemplo-de-configuracao}
\begin{DUlineblock}{0em}
\item[] \textless{}mameconfig version=''10''\textgreater{}
\item[]
\begin{DUlineblock}{\DUlineblockindent}
\item[] \textless{}system name=''default''\textgreater{}
\item[]
\begin{DUlineblock}{\DUlineblockindent}
\item[] \textless{}input\textgreater{}
\item[]
\begin{DUlineblock}{\DUlineblockindent}
\item[] \textbf{\textless{}mapdevice device=''VID\_D209\&amp;PID\_1601'' controller=''GUNCODE\_1'' /\textgreater{}}
\item[] \textbf{\textless{}mapdevice device=''VID\_D209\&amp;PID\_1602'' controller=''GUNCODE\_2'' /\textgreater{}}
\item[] \textbf{\textless{}mapdevice device=''XInput Player 1'' controller=''JOYCODE\_1'' /\textgreater{}}
\item[] \textbf{\textless{}mapdevice device=''XInput Player 2'' controller=''JOYCODE\_2'' /\textgreater{}}
\item[] 
\item[] \textless{}port type=''P1\_JOYSTICK\_UP''\textgreater{}
\item[]
\begin{DUlineblock}{\DUlineblockindent}
\item[] \textless{}newseq type=''standard''\textgreater{}
\item[]
\begin{DUlineblock}{\DUlineblockindent}
\item[] JOYCODE\_1\_YAXIS\_UP\_SWITCH OR KEYCODE\_8PAD
\end{DUlineblock}
\item[] \textless{}/newseq\textgreater{}
\end{DUlineblock}
\item[] \textless{}/port\textgreater{}
\item[] ...
\item[] 
\end{DUlineblock}
\end{DUlineblock}
\end{DUlineblock}
\end{DUlineblock}

Acima especificamos quatro mapeamentos de dispositivos, GUNCODE 1/2 e
JOYCODE 1/2:
\begin{itemize}
\item {} 
As duas primeiras entradas ``mapdevice'' definem o controle de pistola
de luz do ``jogador 1 e 2'' (player 1 e 2) para Gun 1 e Gun 2
respectivamente.

\item {} 
Nós usamos uma cadeia de caracteres com os nomes brutos dos
dispositivos para que combinem com cada dispositivo de forma
individual. Observe que, como este é um arquivo em formato XML,
precisamos usar o caractere de escape `\&amp;' para representar `\&'.

\item {} 
As duas últimas entradas mapdevices definem o jogador 1 e 2 para
Joy 1 e Joy 2 respectivamente. Neste caso, estes são dispositivos
XInput.

\end{itemize}
\clearpage

\subsection{Listando os dispositivos disponíveis}
\label{advanced/devicemap:listando-os-dispositivos-disponiveis}
Você deve estar se perguntando, como foi que nós obtivemos os IDs dos
dispositivos usados no exemplo acima?
Fácil!

Rode o MAME com o parâmetro -v para ativar o modo loquaz (verbose).
Assim será exibido no terminal uma lista de dispositivos disponíveis
correspondentes ao ID do dispositivo com a etiqueta ``device id''.

Aqui um exemplo:

\begin{DUlineblock}{0em}
\item[] Input: Adding Gun \#0:
\item[] Input: Adding Gun \#1:
\item[] Input: Adding Gun \#2: HID-compliant mouse (\textbf{device id: \textbackslash{}?HID\#VID\_045E\&PID\_0053\#7\&18297dcb\&0\&0000\#\{378de44c-56ef-11d1-bc8c-00a0c91405dd\}})
\item[] Input: Adding Gun \#3: HID-compliant mouse (\textbf{device id: \textbackslash{}?HID\#IrDeviceV2\&Col08\#2\&2818a073\&0\&0007\#\{378de44c-56ef-11d1-bc8c-00a0c91405dd\}})
\item[] Input: Adding Gun \#4: HID-compliant mouse (\textbf{device id: \textbackslash{}?HID\#VID\_D209\&PID\_1602\&MI\_02\#8\&389ab7f3\&0\&0000\#\{378de44c-56ef-11d1-bc8c-00a0c91405dd\}})
\item[] Input: Adding Gun \#5: HID-compliant mouse (\textbf{device id: \textbackslash{}?HID\#VID\_D209\&PID\_1601\&MI\_02\#9\&375eebb1\&0\&0000\#\{378de44c-56ef-11d1-bc8c-00a0c91405dd\}})
\item[] Input: Adding Gun \#6: HID-compliant mouse (\textbf{device id: \textbackslash{}?HID\#VID\_1241\&PID\_1111\#8\&198f3adc\&0\&0000\#\{378de44c-56ef-11d1-bc8c-00a0c91405dd\}})
\item[] Skipping DirectInput for XInput compatible joystick Controller (XBOX 360 For Windows).
\item[] Input: Adding Joy \#0: ATRAK Device \#1 (\textbf{device id: ATRAK Device \#1})
\item[] Skipping DirectInput for XInput compatible joystick Controller (XBOX 360 For Windows).
\item[] Input: Adding Joy \#1: ATRAK Device \#2 (\textbf{device id: ATRAK Device \#2})
\item[] Input: Adding Joy \#2: XInput Player 1 (\textbf{device id: XInput Player 1})
\item[] Input: Adding Joy \#3: XInput Player 2 (\textbf{device id: XInput Player 2})
\item[] 
\end{DUlineblock}

Além disso, quando os dispositivos são definidos usando o \emph{mapdevice},
você os verá também no modo loquaz:

\begin{DUlineblock}{0em}
\item[] Input: Remapped Gun \#0: HID-compliant mouse (device id: \textbackslash{}?HID\#VID\_D209\&PID\_1601\&MI\_02\#9\&375eebb1\&0\&0000\#\{378de44c-56ef-11d1-bc8c-00a0c91405dd\})
\item[] Input: Remapped Gun \#1: HID-compliant mouse (device id: \textbackslash{}?HID\#VID\_D209\&PID\_1602\&MI\_02\#8\&389ab7f3\&0\&0000\#\{378de44c-56ef-11d1-bc8c-00a0c91405dd\})
\item[] Input: Remapped Joy \#0: XInput Player 1 (device id: XInput Player 1)
\item[] Input: Remapped Joy \#1: XInput Player 2 (device id: XInput Player 2)
\item[] 
\end{DUlineblock}


\chapter{O DEPURADOR DO MAME}
\label{debugger/index::doc}\label{debugger/index:o-depurador-do-mame}
Essa seção descreve as funcionalidades do depurador embutido no MAME.


\section{Comandos gerais do depurador}
\label{debugger/general:debugger-general-list}\label{debugger/general::doc}\label{debugger/general:comandos-gerais-do-depurador}
Na interface de depuração do MAME você pode digitar \textbf{help \textless{}command\textgreater{}}
para uma melhor descrição de cada comando.

\begin{DUlineblock}{0em}
\item[] {\hyperref[debugger/general:debugger\string-command\string-do]{\sphinxcrossref{\DUrole{std,std-ref}{do}}}} -- avalia a expressão dada
\item[] {\hyperref[debugger/general:debugger\string-command\string-symlist]{\sphinxcrossref{\DUrole{std,std-ref}{symlist}}}} -- lista os símbolos registrados
\item[] {\hyperref[debugger/general:debugger\string-command\string-softreset]{\sphinxcrossref{\DUrole{std,std-ref}{softreset}}}} -- executa um soft reset
\item[] {\hyperref[debugger/general:debugger\string-command\string-hardreset]{\sphinxcrossref{\DUrole{std,std-ref}{hardreset}}}} -- executa um hard reset
\item[] {\hyperref[debugger/general:debugger\string-command\string-print]{\sphinxcrossref{\DUrole{std,std-ref}{print}}}} -- imprime um ou mais \textless{}\emph{item}\textgreater{}s para o console
\item[] {\hyperref[debugger/general:debugger\string-command\string-printf]{\sphinxcrossref{\DUrole{std,std-ref}{printf}}}} -- imprime um ou mais \textless{}\emph{item}\textgreater{}s para o console usando \textless{}\emph{format}\textgreater{}
\item[] {\hyperref[debugger/general:debugger\string-command\string-logerror]{\sphinxcrossref{\DUrole{std,std-ref}{logerror}}}} -- exibe um ou mais \textless{}\emph{item}\textgreater{}s para o arquivo error.log
\item[] {\hyperref[debugger/general:debugger\string-command\string-tracelog]{\sphinxcrossref{\DUrole{std,std-ref}{tracelog}}}} -- exibe um ou mais \textless{}\emph{item}\textgreater{}s para o arquivo de rastreio usando \textless{}\emph{format}\textgreater{}
\item[] {\hyperref[debugger/general:debugger\string-command\string-tracesym]{\sphinxcrossref{\DUrole{std,std-ref}{tracesym}}}} -- exibe um ou mais \textless{}\emph{item}\textgreater{}s para o arquivo de rastreio
\item[] history -- produz um breve histórico de \emph{opcodes} visitados (\textbf{a ser concluído: ainda não há ajuda para este comando})
\item[] {\hyperref[debugger/general:debugger\string-command\string-trackpc]{\sphinxcrossref{\DUrole{std,std-ref}{trackpc}}}} -- rastreia visualmente os opcodes visitados {[}booleano para ligar e desligar, para o dado CPU, limpa{]}
\item[] {\hyperref[debugger/general:debugger\string-command\string-trackmem]{\sphinxcrossref{\DUrole{std,std-ref}{trackmem}}}} -- grava qual PC grava em cada endereço de memória {[}booleano para ligar e desligar, limpar{]}
\item[] {\hyperref[debugger/general:debugger\string-command\string-pcatmem]{\sphinxcrossref{\DUrole{std,std-ref}{pcatmem}}}} -- consulta qual PC escreveu para um determinado endereço de memória para o CPU atual
\item[] {\hyperref[debugger/general:debugger\string-command\string-rewind]{\sphinxcrossref{\DUrole{std,std-ref}{rewind}}}} -- volta no tempo carregando o estado de retrocesso mais recente
\item[] {\hyperref[debugger/general:debugger\string-command\string-statesave]{\sphinxcrossref{\DUrole{std,std-ref}{statesave}}}} -- salva um arquivo de estado para o driver atual
\item[] {\hyperref[debugger/general:debugger\string-command\string-stateload]{\sphinxcrossref{\DUrole{std,std-ref}{stateload}}}} -- carrega um arquivo de estado para o driver atual
\item[] {\hyperref[debugger/general:debugger\string-command\string-snap]{\sphinxcrossref{\DUrole{std,std-ref}{snap}}}} -- salva um instantâneo da tela.
\item[] {\hyperref[debugger/general:debugger\string-command\string-source]{\sphinxcrossref{\DUrole{std,std-ref}{source}}}} -- lê os comandos do \textless{}\emph{filename}\textgreater{} e os executa um por um
\item[] {\hyperref[debugger/general:debugger\string-command\string-quit]{\sphinxcrossref{\DUrole{std,std-ref}{quit}}}} -- sai do MAME e do depurador
\end{DUlineblock}
\begin{quote}
\phantomsection\label{debugger/general:debugger-command-do}\end{quote}


\subsection{do}
\label{debugger/general:do}\label{debugger/general:debugger-command-do}
\begin{DUlineblock}{0em}
\item[]
\begin{DUlineblock}{\DUlineblockindent}
\item[] \textbf{do \textless{}expression\textgreater{}}
\item[] 
\end{DUlineblock}
\item[] O comando do avalia a expressão \textless{}\emph{expression}\textgreater{} dada. Isso é normalmente usado para definir ou modificar variáveis.
\item[] 
\item[] Exemplo:
\item[] 
\item[]
\begin{DUlineblock}{\DUlineblockindent}
\item[] do pc = 0
\item[] 
\end{DUlineblock}
\item[] Define o registro `pc' para 0.
\item[] 
\item[] Voltar para {\hyperref[debugger/general:debugger\string-general\string-list]{\sphinxcrossref{\DUrole{std,std-ref}{Comandos gerais do depurador}}}}
\end{DUlineblock}
\begin{quote}
\phantomsection\label{debugger/general:debugger-command-symlist}\end{quote}


\subsection{symlist}
\label{debugger/general:symlist}\label{debugger/general:debugger-command-symlist}
\begin{DUlineblock}{0em}
\item[]
\begin{DUlineblock}{\DUlineblockindent}
\item[] \textbf{symlist} {[}\textless{}\emph{cpu}\textgreater{}{]}
\item[] 
\end{DUlineblock}
\item[] Lista os símbolos registrados. Caso \textless{}\emph{cpu}\textgreater{} não seja definido, os símbolos na tabela de símbolos globais serão exibidos; caso contrário, os símbolos para o CPU em específico serão exibidas. Os símbolos estão listados em ordem alfabética, os símbolos que forem de apenas leitura serão marcados com um asterisco.
\item[] 
\item[] Exemplo:
\item[] 
\item[]
\begin{DUlineblock}{\DUlineblockindent}
\item[] \sphinxcode{symlist}
\item[] 
\end{DUlineblock}
\item[] Exibe a tabela de símbolos globais.
\item[] Exibe os símbolos específicos para a CPU \sphinxcode{\#2}.
\item[] 
\item[] Voltar para {\hyperref[debugger/general:debugger\string-general\string-list]{\sphinxcrossref{\DUrole{std,std-ref}{Comandos gerais do depurador}}}}
\end{DUlineblock}
\begin{quote}
\phantomsection\label{debugger/general:debugger-command-softreset}\end{quote}


\subsection{softreset}
\label{debugger/general:softreset}\label{debugger/general:debugger-command-softreset}
\begin{DUlineblock}{0em}
\item[]
\begin{DUlineblock}{\DUlineblockindent}
\item[] \textbf{softreset}
\item[] 
\end{DUlineblock}
\item[] Executa um soft reset.
\item[] Exemplo:
\item[] 
\item[] \sphinxcode{softreset}
\item[] 
\item[] 
\item[] Voltar para {\hyperref[debugger/general:debugger\string-general\string-list]{\sphinxcrossref{\DUrole{std,std-ref}{Comandos gerais do depurador}}}}
\end{DUlineblock}
\begin{quote}
\phantomsection\label{debugger/general:debugger-command-hardreset}\end{quote}


\subsection{hardreset}
\label{debugger/general:debugger-command-hardreset}\label{debugger/general:hardreset}
\begin{DUlineblock}{0em}
\item[]
\begin{DUlineblock}{\DUlineblockindent}
\item[] \textbf{hardreset}
\item[] 
\end{DUlineblock}
\item[] Executa um hard reset.
\item[] Exemplo:
\item[] 
\item[] \sphinxcode{hardreset}
\item[] 
\item[] 
\item[] Voltar para {\hyperref[debugger/general:debugger\string-general\string-list]{\sphinxcrossref{\DUrole{std,std-ref}{Comandos gerais do depurador}}}}
\end{DUlineblock}
\begin{quote}
\phantomsection\label{debugger/general:debugger-command-print}\end{quote}


\subsection{print}
\label{debugger/general:print}\label{debugger/general:debugger-command-print}
\begin{DUlineblock}{0em}
\item[] O comando print imprime os resultados de uma ou mais expressões no console do depurador usando valores hexadecimais.
\item[] 
\item[] Exemplo:
\item[] 
\item[]
\begin{DUlineblock}{\DUlineblockindent}
\item[] \sphinxcode{print pc}
\item[] 
\end{DUlineblock}
\item[] Imprime o valor de \textbf{pc} no console como um número hexadecimal.
\item[] Imprime \textbf{a}, \textbf{b} e o valor de \textbf{a+b} no console como números hexadecimais.
\item[] 
\item[] Voltar para {\hyperref[debugger/general:debugger\string-general\string-list]{\sphinxcrossref{\DUrole{std,std-ref}{Comandos gerais do depurador}}}}
\end{DUlineblock}
\begin{quote}
\phantomsection\label{debugger/general:debugger-command-printf}\end{quote}


\subsection{printf}
\label{debugger/general:debugger-command-printf}\label{debugger/general:printf}
\begin{DUlineblock}{0em}
\item[]
\begin{DUlineblock}{\DUlineblockindent}
\item[] \textbf{printf} \textless{}\emph{format}\textgreater{}{[},\textless{}\emph{item}\textgreater{}{[},...{]}{]}
\item[] 
\end{DUlineblock}
\item[] O comando ``\emph{printf}'' executa um printf no estilo C para o console do depurador. Apenas um conjunto muito limitado de opções de formatação está disponível:
\item[] 
\item[]
\begin{DUlineblock}{\DUlineblockindent}
\item[] \sphinxcode{\%{[}0{]}{[}\textless{}n\textgreater{}{]}d} -- imprime \textless{}\emph{item}\textgreater{} como um valor decimal com contagem de dígitos opcional e preenchimento zero
\item[] \sphinxcode{\%{[}0{]}{[}\textless{}n\textgreater{}{]}x} -- imprime \textless{}\emph{item}\textgreater{} como um valor hexadecimal com contagem de dígitos opcional e preenchimento zero
\item[] 
\end{DUlineblock}
\item[] Todas as opções restantes de formatação são ignoradas. Use \textbf{\%\%} junto para gerar um caractere \textbf{\%}. Várias linhas podem ser impressas incorporando um \textbf{\textbackslash{}n} no texto.
\item[] 
\item[] Exemplos:
\item[] 
\item[]
\begin{DUlineblock}{\DUlineblockindent}
\item[] \sphinxcode{printf "PC=\%04X",pc}
\item[] 
\end{DUlineblock}
\item[] Imprime \textbf{PC=\textless{}pcval\textgreater{}} onde \textless{}\emph{pcval}\textgreater{} é exibido em hexadecimal com \textbf{4} dígitos e com zero preenchimento.
\item[] 
\item[]
\begin{DUlineblock}{\DUlineblockindent}
\item[] \sphinxcode{printf "A=\%d, B=\%d\textbackslash{}\textbackslash{}nC=\%d",a,b,a+b}
\item[] 
\end{DUlineblock}
\item[] Imprime \textbf{A=\textless{}aval\textgreater{}}, \textbf{B=\textless{}bval\textgreater{}} em uma linha e \textbf{C=\textless{}a+bval\textgreater{}} na segunda linha.
\item[] 
\item[] Voltar para {\hyperref[debugger/general:debugger\string-general\string-list]{\sphinxcrossref{\DUrole{std,std-ref}{Comandos gerais do depurador}}}}
\end{DUlineblock}
\begin{quote}
\phantomsection\label{debugger/general:debugger-command-logerror}\end{quote}


\subsection{logerror}
\label{debugger/general:logerror}\label{debugger/general:debugger-command-logerror}
\begin{DUlineblock}{0em}
\item[]
\begin{DUlineblock}{\DUlineblockindent}
\item[] \textbf{logerror} \textless{}\emph{format}\textgreater{}{[},\textless{}\emph{item}\textgreater{}{[},...{]}{]}
\item[] 
\end{DUlineblock}
\item[] O comando ``\emph{logerror}'' executa um printf no estilo C no registro de erro. Apenas um conjunto muito limitado de opções de formatação está disponível:
\item[] 
\item[]
\begin{DUlineblock}{\DUlineblockindent}
\item[] \sphinxcode{\%{[}0{]}{[}\textless{}n\textgreater{}{]}d} -- registra \textless{}\emph{item}\textgreater{} como um valor decimal com contagem de dígitos opcional e preenchimento zero
\item[] \sphinxcode{\%{[}0{]}{[}\textless{}n\textgreater{}{]}x} -- registra \textless{}\emph{item}\textgreater{} como um valor hexadecimal com contagem de dígitos opcional e preenchimento zero
\item[] 
\end{DUlineblock}
\item[] Todas as opções restantes de formatação são ignoradas. Use \textbf{\%\%} junto para gerar um caractere \textbf{\%}. Várias linhas podem ser impressas incorporando um \textbf{\textbackslash{}n} no texto.
\item[] 
\item[] Exemplos:
\item[] 
\item[]
\begin{DUlineblock}{\DUlineblockindent}
\item[] \sphinxcode{logerror "PC=\%04X",pc}
\item[] 
\end{DUlineblock}
\item[] Registra \textbf{PC=\textless{}pcval\textgreater{}} onde \textless{}\emph{pcval}\textgreater{} é exibido em hexadecimal com \textbf{4} dígitos e com preenchimento zero.
\item[] 
\item[]
\begin{DUlineblock}{\DUlineblockindent}
\item[] \sphinxcode{logerror "A=\%d, B=\%d\textbackslash{}\textbackslash{}nC=\%d",a,b,a+b}
\item[] 
\end{DUlineblock}
\item[] Registra \textbf{A=\textless{}aval\textgreater{}}, \textbf{B=\textless{}bval\textgreater{}} em uma linha, e \textbf{C=\textless{}a+bval\textgreater{}} na segunda linha.
\item[] 
\item[] Voltar para {\hyperref[debugger/general:debugger\string-general\string-list]{\sphinxcrossref{\DUrole{std,std-ref}{Comandos gerais do depurador}}}}
\end{DUlineblock}
\begin{quote}
\phantomsection\label{debugger/general:debugger-command-tracelog}\end{quote}


\subsection{tracelog}
\label{debugger/general:debugger-command-tracelog}\label{debugger/general:tracelog}
\begin{DUlineblock}{0em}
\item[]
\begin{DUlineblock}{\DUlineblockindent}
\item[] \textbf{tracelog} \textless{}\emph{format}\textgreater{}{[},\textless{}\emph{item}\textgreater{}{[},...{]}{]}
\item[] 
\end{DUlineblock}
\item[] O comando ``\emph{tracelog}'' executa um printf no estilo C e roteia a saída para o arquivo de rastreio atualmente aberto (consulte o comando `trace' para mais detalhes). Caso nenhum arquivo esteja aberto no momento, o tracelog não fará nada. Apenas um conjunto muito limitado de opções de formatação está disponível. Veja {\hyperref[debugger/general:debugger\string-command\string-printf]{\sphinxcrossref{\DUrole{std,std-ref}{printf}}}} para mais detalhes.
\item[] 
\item[] Exemplos:
\item[] 
\item[]
\begin{DUlineblock}{\DUlineblockindent}
\item[] \sphinxcode{tracelog "PC=\%04X",pc}
\item[] 
\end{DUlineblock}
\item[] Registra \textbf{PC=\textless{}pcval\textgreater{}} onde \textless{}\emph{pcval}\textgreater{} é exibido em hexadecimal com \textbf{4} digitos com preenchimento zero.
\item[] 
\item[]
\begin{DUlineblock}{\DUlineblockindent}
\item[] \sphinxcode{printf "A=\%d, B=\%d\textbackslash{}\textbackslash{}nC=\%d",a,b,a+b}
\item[] 
\end{DUlineblock}
\item[] Registra \textbf{A=\textless{}aval\textgreater{}}, \textbf{B=\textless{}bval\textgreater{}} em uma linha, e \textbf{C=\textless{}a+bval\textgreater{}} na segunda.
\item[] 
\item[] Voltar para {\hyperref[debugger/general:debugger\string-general\string-list]{\sphinxcrossref{\DUrole{std,std-ref}{Comandos gerais do depurador}}}}
\end{DUlineblock}
\begin{quote}
\phantomsection\label{debugger/general:debugger-command-tracesym}\end{quote}


\subsection{tracesym}
\label{debugger/general:tracesym}\label{debugger/general:debugger-command-tracesym}
\begin{DUlineblock}{0em}
\item[]
\begin{DUlineblock}{\DUlineblockindent}
\item[] \textbf{tracesym} \textless{}\emph{item}\textgreater{}{[},...{]}
\item[] 
\end{DUlineblock}
\item[] O comando ``\emph{tracesym}'' imprime os símbolos especificados e roteia a saída para o arquivo de rastreio aberto no momento (consulte o comando `trace' para obter detalhes). Caso nenhum arquivo esteja aberto no momento, o tracesym não faz nada.
\item[] 
\item[] Exemplo:
\item[] 
\item[]
\begin{DUlineblock}{\DUlineblockindent}
\item[] \sphinxcode{tracelog pc}
\item[] 
\end{DUlineblock}
\item[] Registra \textbf{PC=\textless{}pcval\textgreater{}} onde \textless{}\emph{pcval}\textgreater{} é exibido em seu formato predefinido.
\item[] 
\item[] Voltar para {\hyperref[debugger/general:debugger\string-general\string-list]{\sphinxcrossref{\DUrole{std,std-ref}{Comandos gerais do depurador}}}}
\end{DUlineblock}
\begin{quote}
\phantomsection\label{debugger/general:debugger-command-trackpc}\end{quote}


\subsection{trackpc}
\label{debugger/general:trackpc}\label{debugger/general:debugger-command-trackpc}
\begin{DUlineblock}{0em}
\item[]
\begin{DUlineblock}{\DUlineblockindent}
\item[] \textbf{trackpc} {[}\textless{}\emph{bool}\textgreater{},\textless{}\emph{cpu}\textgreater{},\textless{}\emph{bool}\textgreater{}{]}
\item[] 
\end{DUlineblock}
\item[] O comando ``\emph{trackpc}'' exibe quais contadores do programa já foram visitados em todas as janelas do desmontador. O primeiro argumento booleano ativa e desativa o processo. O segundo argumento é um seletor de CPU; caso nenhuma CPU seja especificada a CPU atual é selecionada automaticamente. O terceiro argumento é um booleano denotando se os dados existentes devem ser limpos ou não.
\item[] 
\item[] Exemplos:
\item[] 
\item[]
\begin{DUlineblock}{\DUlineblockindent}
\item[] \sphinxcode{trackpc 1}
\item[] 
\end{DUlineblock}
\item[] Comece a rastrear o PC atual da CPU.
\item[] 
\item[]
\begin{DUlineblock}{\DUlineblockindent}
\item[] \sphinxcode{trackpc 1, 0, 1}
\item[] 
\end{DUlineblock}
\item[] Continue rastreando o PC na CPU 0, mas limpe as informações de faixa existentes.
\item[] 
\item[] Voltar para {\hyperref[debugger/general:debugger\string-general\string-list]{\sphinxcrossref{\DUrole{std,std-ref}{Comandos gerais do depurador}}}}
\end{DUlineblock}
\begin{quote}
\phantomsection\label{debugger/general:debugger-command-trackmem}\end{quote}


\subsection{trackmem}
\label{debugger/general:debugger-command-trackmem}\label{debugger/general:trackmem}
\begin{DUlineblock}{0em}
\item[]
\begin{DUlineblock}{\DUlineblockindent}
\item[] \textbf{trackmem} {[}\textless{}\emph{bool}\textgreater{},\textless{}\emph{cpu}\textgreater{},\textless{}\emph{bool}\textgreater{}{]}
\item[] 
\end{DUlineblock}
\item[] O comando ``\emph{trackmem}'' registra o PC a cada vez que um endereço de memória é gravado. O primeiro argumento booleano ativa e desativa o processo. O segundo argumento é um seletor de CPU; caso nenhuma CPU seja especificada, a CPU atual é selecionada automaticamente. O terceiro argumento é um booleano denotando se os dados existentes devem ser limpos ou não. Favor consultar o comando {\hyperref[debugger/general:debugger\string-command\string-pcatmem]{\sphinxcrossref{\DUrole{std,std-ref}{pcatmem}}}} para obter informações sobre como recuperar esses dados. Além disso, clicar com o botão direito em uma janela de memória exibirá o PC registrado para o endereço fornecido.
\item[] 
\item[] Exemplos:
\item[] 
\item[]
\begin{DUlineblock}{\DUlineblockindent}
\item[] \sphinxcode{trackmem}
\item[] 
\end{DUlineblock}
\item[] Comece a rastrear o PC atual da CPU.
\item[] 
\item[]
\begin{DUlineblock}{\DUlineblockindent}
\item[] \sphinxcode{trackmem 1, 0, 1}
\item[] 
\end{DUlineblock}
\item[] Continue rastreando as gravações de memória na CPU 0, mas limpe as informações de faixa existentes.
\item[] 
\item[] Voltar para {\hyperref[debugger/general:debugger\string-general\string-list]{\sphinxcrossref{\DUrole{std,std-ref}{Comandos gerais do depurador}}}}
\end{DUlineblock}
\begin{quote}
\phantomsection\label{debugger/general:debugger-command-pcatmem}\end{quote}


\subsection{pcatmem}
\label{debugger/general:pcatmem}\label{debugger/general:debugger-command-pcatmem}
\begin{DUlineblock}{0em}
\item[]
\begin{DUlineblock}{\DUlineblockindent}
\item[] \textbf{pcatmem(p/d/i)} \textless{}\emph{address}\textgreater{}{[},\textless{}\emph{cpu}\textgreater{}{]}
\item[] 
\end{DUlineblock}
\item[] \textbf{pcatmemp \textless{}address\textgreater{}{[},\textless{}cpu\textgreater{}{]}} -- consulta qual PC escreveu para um dado endereço de memória do programa para o CPU atual
\item[] \textbf{pcatmemd \textless{}address\textgreater{}{[},\textless{}cpu\textgreater{}{]}} -- consulta qual PC escreveu para um endereço de dados na memória para a CPU atual
\item[] \textbf{pcatmemi \textless{}address\textgreater{}{[},\textless{}cpu\textgreater{}{]}} -- consulta qual PC escreveu para um endereço de I/O para a CPU atual (você também pode consultar esta informação clicando com o botão direito em uma janela de memória)
\item[] 
\item[] O comando ``\emph{pcatmem}'' retorna qual PC gravou em um determinado endereço de memória para a CPU atual. O primeiro argumento é o endereço solicitado. O segundo argumento é um seletor de CPU; caso nenhuma CPU seja especificada, a CPU atual é selecionada automaticamente. Clicar com o botão direito em uma janela de memória também exibirá o PC registrado para o endereço fornecido.
\item[] 
\item[] Exemplo:
\item[] 
\item[]
\begin{DUlineblock}{\DUlineblockindent}
\item[] \sphinxcode{pcatmem 400000}
\item[] 
\end{DUlineblock}
\item[] Imprimir qual PC escreveu a localização de memória da CPU \textbf{0x400000}.
\item[] 
\item[] Voltar para {\hyperref[debugger/general:debugger\string-general\string-list]{\sphinxcrossref{\DUrole{std,std-ref}{Comandos gerais do depurador}}}}
\end{DUlineblock}
\begin{quote}
\phantomsection\label{debugger/general:debugger-command-rewind}\end{quote}


\subsection{rewind}
\label{debugger/general:debugger-command-rewind}\label{debugger/general:rewind}
\begin{DUlineblock}{0em}
\item[]
\begin{DUlineblock}{\DUlineblockindent}
\item[] \textbf{rewind{[}rw{]}}
\item[] 
\end{DUlineblock}
\item[] O comando de retrocesso ``\emph{rewind}'' carrega o estado mais recente baseado em RAM. Os estados de retrocesso, quando ativados, são salvos quando o comando ``step'', ``over'' ou ``out'' é executado, armazenando o estado da máquina a partir do momento antes de realmente avançar. Consecutivamente, o carregamento de estados de retrocesso pode funcionar como uma execução reversa. Dependendo de quais passos foram dados anteriormente, o comportamento pode ser similar ao ``reverse stepi'' do GDB ou ``reverse next''. Toda a saída para este comando está atualmente ecoada na janela da máquina em execução. A memória anterior e as estatísticas de rastreamento do PC serão limpas, a execução reversa atual não ocorre.
\item[] 
\item[] Voltar para {\hyperref[debugger/general:debugger\string-general\string-list]{\sphinxcrossref{\DUrole{std,std-ref}{Comandos gerais do depurador}}}}
\end{DUlineblock}
\begin{quote}
\phantomsection\label{debugger/general:debugger-command-statesave}\end{quote}


\subsection{statesave}
\label{debugger/general:debugger-command-statesave}\label{debugger/general:statesave}
\begin{DUlineblock}{0em}
\item[]
\begin{DUlineblock}{\DUlineblockindent}
\item[] \textbf{statesave{[}ss{]}} \textless{}\emph{filename}\textgreater{}
\item[] 
\end{DUlineblock}
\item[] O comando ``\emph{statesave}'' cria um estado de salvaguarda neste exato momento no tempo. O arquivo de estado fornecido é gravado no diretório de estado padrão (sta) e recebe .sta adicionado a ele, sem necessidade de extensão de arquivo. Toda a saída para este comando está atualmente ecoada na janela da máquina em execução.
\item[] 
\item[] Exemplo:
\item[] 
\item[]
\begin{DUlineblock}{\DUlineblockindent}
\item[] \sphinxcode{statesave foo}
\item[] 
\end{DUlineblock}
\item[] Grava o arquivo `foo.sta' no diretório de salvamento de estado padrão.
\item[] 
\item[] Voltar para {\hyperref[debugger/general:debugger\string-general\string-list]{\sphinxcrossref{\DUrole{std,std-ref}{Comandos gerais do depurador}}}}
\end{DUlineblock}
\begin{quote}
\phantomsection\label{debugger/general:debugger-command-stateload}\end{quote}


\subsection{stateload}
\label{debugger/general:stateload}\label{debugger/general:debugger-command-stateload}
\begin{DUlineblock}{0em}
\item[]
\begin{DUlineblock}{\DUlineblockindent}
\item[] \textbf{stateload{[}sl{]}} \textless{}\emph{filename}\textgreater{}
\item[] 
\end{DUlineblock}
\item[] O comando ``\emph{stateload}'' recupera um estado de salvamento do disco. O arquivo de estado fornecido é lido a partir do diretório de estado padrão (sta) e recebe .sta adicionado a ele, sem necessidade de extensão de arquivo. Toda a saída para este comando está atualmente ecoada na janela da máquina em execução. A memória anterior e as estatísticas de rastreamento do PC serão apagadas.
\item[] 
\item[] Exemplo:
\item[] 
\item[]
\begin{DUlineblock}{\DUlineblockindent}
\item[] \sphinxcode{stateload foo}
\item[] 
\end{DUlineblock}
\item[] Carrega o arquivo `foo.sta' do diretório padrão de salvamento de estado.
\item[] 
\item[] Voltar para {\hyperref[debugger/general:debugger\string-general\string-list]{\sphinxcrossref{\DUrole{std,std-ref}{Comandos gerais do depurador}}}}
\end{DUlineblock}
\begin{quote}
\phantomsection\label{debugger/general:debugger-command-snap}\end{quote}


\subsection{snap}
\label{debugger/general:snap}\label{debugger/general:debugger-command-snap}
\begin{DUlineblock}{0em}
\item[]
\begin{DUlineblock}{\DUlineblockindent}
\item[] \textbf{snap} {[}{[}\textless{}\emph{filename}\textgreater{}{]}, \textless{}\emph{scrnum}\textgreater{}{]}
\item[] 
\end{DUlineblock}
\item[] O comando snap tira um instantâneo da exibição de vídeo atual e a salva no diretório snapshot. Caso o \textless{}\emph{filename}\textgreater{} seja definido explicitamente, uma única captura de tela \emph{\textless{}scrnum\textgreater{}} é salva sob o nome do arquivo solicitado. Caso \textless{}\emph{filename}\textgreater{} seja omitido, todas as telas são salvas usando as mesmas regras predefinidas que a tecla ``salvar instantâneo'' no MAME.
\item[] 
\item[] Exemplos:
\item[] 
\item[]
\begin{DUlineblock}{\DUlineblockindent}
\item[] \sphinxcode{snap}
\item[] 
\end{DUlineblock}
\item[] Obtém um instantâneo da tela de vídeo atual e salva no próximo nome de arquivo não conflitante no diretório \textbf{snapshot}.
\item[] 
\item[]
\begin{DUlineblock}{\DUlineblockindent}
\item[] \sphinxcode{snap shinobi}
\item[] 
\end{DUlineblock}
\item[] Obtém um instantâneo da tela de vídeo atual e a salva como `shinobi.png' no diretório \textbf{snapshot}.
\item[] 
\item[] Voltar para {\hyperref[debugger/general:debugger\string-general\string-list]{\sphinxcrossref{\DUrole{std,std-ref}{Comandos gerais do depurador}}}}
\end{DUlineblock}
\begin{quote}
\phantomsection\label{debugger/general:debugger-command-source}\end{quote}


\subsection{source}
\label{debugger/general:source}\label{debugger/general:debugger-command-source}
\begin{DUlineblock}{0em}
\item[]
\begin{DUlineblock}{\DUlineblockindent}
\item[] \textbf{source \textless{}filename\textgreater{}}
\item[] 
\end{DUlineblock}
\item[] O comando source lê um conjunto de comandos do depurador de um arquivo e os executa um por um, semelhante a um arquivo em lotes.
\item[] 
\item[] Exemplo:
\item[] 
\item[]
\begin{DUlineblock}{\DUlineblockindent}
\item[] \sphinxcode{source break\_and\_trace.cmd}
\item[] 
\end{DUlineblock}
\item[] Lê nos comandos do depurador a partir do \textbf{break\_and\_trace.cmd} e os executa.
\item[] 
\item[] Voltar para {\hyperref[debugger/general:debugger\string-general\string-list]{\sphinxcrossref{\DUrole{std,std-ref}{Comandos gerais do depurador}}}}
\end{DUlineblock}
\begin{quote}
\phantomsection\label{debugger/general:debugger-command-quit}\end{quote}


\subsection{quit}
\label{debugger/general:quit}\label{debugger/general:debugger-command-quit}
\begin{DUlineblock}{0em}
\item[]
\begin{DUlineblock}{\DUlineblockindent}
\item[] \textbf{quit}
\item[] 
\end{DUlineblock}
\item[] O comando quit sai do MAME imediatamente.
\item[] 
\item[] Voltar para {\hyperref[debugger/general:debugger\string-general\string-list]{\sphinxcrossref{\DUrole{std,std-ref}{Comandos gerais do depurador}}}}
\end{DUlineblock}


\section{Comandos para a depuração de Memória}
\label{debugger/memory:debugger-memory-list}\label{debugger/memory::doc}\label{debugger/memory:comandos-para-a-depuracao-de-memoria}
Na interface de depuração do MAME você pode digitar \textbf{help \textless{}command\textgreater{}}
para uma melhor descrição de cada comando.

\begin{DUlineblock}{0em}
\item[] {\hyperref[debugger/memory:debugger\string-command\string-dasm]{\sphinxcrossref{\DUrole{std,std-ref}{dasm}}}} -- desmonta para um determinado arquivo
\item[] {\hyperref[debugger/memory:debugger\string-command\string-find]{\sphinxcrossref{\DUrole{std,std-ref}{find}}}} -- pesquisa a memória do programa, dados de memória, ou a memória I/O por dados
\item[] {\hyperref[debugger/memory:debugger\string-command\string-dump]{\sphinxcrossref{\DUrole{std,std-ref}{dump}}}} -- extrai a memória do programa, dados da memória, ou os dados I/O como texto
\item[] {\hyperref[debugger/memory:debugger\string-command\string-save]{\sphinxcrossref{\DUrole{std,std-ref}{save}}}} -- salva o binário do programa, dados, ou a memória I/O para um determinado aquivo
\item[] {\hyperref[debugger/memory:debugger\string-command\string-load]{\sphinxcrossref{\DUrole{std,std-ref}{load}}}} -- carrega um programa binário na memória, memória de dados, ou a memória I/O de um determinado arquivo
\item[] {\hyperref[debugger/memory:debugger\string-command\string-map]{\sphinxcrossref{\DUrole{std,std-ref}{map}}}} -- mapeia um programa lógico, dados, ou um endereço de I/O para um endereço físico e um banco
\end{DUlineblock}
\begin{quote}
\phantomsection\label{debugger/memory:debugger-command-dasm}\end{quote}


\subsection{dasm}
\label{debugger/memory:debugger-command-dasm}\label{debugger/memory:dasm}
\begin{DUlineblock}{0em}
\item[]
\begin{DUlineblock}{\DUlineblockindent}
\item[] \textbf{dasm} \textless{}\emph{filename}\textgreater{},\textless{}\emph{address}\textgreater{},\textless{}\emph{length}\textgreater{}{[},\textless{}\emph{opcodes}\textgreater{}{[},\textless{}\emph{cpu}\textgreater{}{]}{]}
\item[] 
\end{DUlineblock}
\item[] O comando ``\emph{dasm}'' desmonta a memória de um programa para um arquivo definido no parâmetro \textless{}\emph{filename}\textgreater{}. O \textless{}\emph{address}\textgreater{} indica o endereço do início do desmonte e \textless{}\emph{length}\textgreater{} indica quanta memória deve ser desmontada. O intervalo \emph{\textless{}address\textgreater{}} através de \emph{\textless{}address\textgreater{}+\textless{}length\textgreater{}-1} será inclusive salvo para um arquivo. É predefinido que os dados \textbf{raw opcode} sejam enviados para a saída com cada linha. O parâmetro \textless{}\emph{opcodes}\textgreater{} opcional pode ser usado para ativar (\textbf{1}) ou desativar (\textbf{0}) essa função. Finalmente, você pode desmontar o código de outra CPU, ao definir o parâmetro \textless{}\emph{cpu}\textgreater{}.
\item[] 
\item[] Exemplos:
\item[] 
\item[]
\begin{DUlineblock}{\DUlineblockindent}
\item[] \sphinxcode{dasm venture.asm,0,10000}
\item[] 
\end{DUlineblock}
\item[] Desmonta os intervalos de endereços \textbf{0-ffff} no CPU atual, incluindo dados de \textbf{raw opcode} para o arquivo `venture.asm'.
\item[] 
\item[]
\begin{DUlineblock}{\DUlineblockindent}
\item[] \sphinxcode{dasm harddriv.asm,3000,1000,0,2}
\item[] 
\end{DUlineblock}
\item[] Desmonta os intervalos de endereços \textbf{3000-3fff} da CPU \textbf{\#2} sem nenhum dado de \textbf{raw opcode} para o arquivo `harddriv.asm'.
\item[] 
\item[] Back to {\hyperref[debugger/memory:debugger\string-memory\string-list]{\sphinxcrossref{\DUrole{std,std-ref}{Comandos para a depuração de Memória}}}}
\end{DUlineblock}
\begin{quote}
\phantomsection\label{debugger/memory:debugger-command-find}\end{quote}


\subsection{find}
\label{debugger/memory:find}\label{debugger/memory:debugger-command-find}
\begin{DUlineblock}{0em}
\item[]
\begin{DUlineblock}{\DUlineblockindent}
\item[] \textbf{f{[}ind{]}{[}\{d\textbar{}i\}{]}} \textless{}\emph{address}\textgreater{},\textless{}\emph{length}\textgreater{}{[},\textless{}\emph{data}\textgreater{}{[},...{]}{]}
\item[] 
\end{DUlineblock}
\item[] Os comandos \textbf{find}/\textbf{findd}/\textbf{findi} pesquisam na memória por uma sequência específica de dados. o `find' procurará o espaço do programa na memória, enquanto `findd' procurará o espaço dos dados na memória e `findi' procurará pelo espaço I/O na memória. O \emph{\textless{}address\textgreater{}} indica o endereço que será iniciado a pesquisa, e \textless{}\emph{length}\textgreater{} indica quanta memória será pesquisada. \emph{\textless{}data\textgreater{}} pode ser tanto uma string citada, um valor numérico, uma expressão ou um caractere coringa `?'. As strings por padrão implicam em uma pesquisa de tamanho de byte. Os dados não string são pesquisados por padrão no tamanho da palavra nativa da CPU. Para substituir o tamanho da pesquisa por sequências sem string, você pode prefixar o valor com \textbf{b}. para forçar pesquisa de tamanho de byte, \textbf{w} para pesquisa por tamanho da palavra, \textbf{d} para o tamanho dword e \textbf{q} para o tamanho qword. As substituições são memorizadas, então se você quiser procurar por uma série de palavras, basta prefixar o primeiro valor com um \emph{w}. Observe também que você pode misturar os tamanhos para executar as pesquisas mais complexas. Todo o intervalo \emph{\textless{}address\textgreater{}} através de \emph{\textless{}address\textgreater{}+\textless{}length\textgreater{}-1} será inclusive pesquisada na sequência e todas as ocorrências serão exibidas.
\item[] 
\item[] Exemplos:
\item[] 
\item[]
\begin{DUlineblock}{\DUlineblockindent}
\item[] \sphinxcode{find 0,10000,"HIGH SCORE",0}
\item[] 
\end{DUlineblock}
\item[] Procura no intervalo de endereços \textbf{0-ffff} na CPU atual pela string ``\textbf{HIGH SCORE}'' seguido por um \textbf{0} byte.
\item[] 
\item[]
\begin{DUlineblock}{\DUlineblockindent}
\item[] \sphinxcode{findd 3000,1000,w.abcd,4567}
\item[] 
\end{DUlineblock}
\item[] Pesquisa o intervalo de endereços da memória de dados \textbf{3000-3fff} por um valor com o tamanho word \textbf{abcd} seguido pelo valor word com tamanho \textbf{4567}.
\item[] 
\item[]
\begin{DUlineblock}{\DUlineblockindent}
\item[] \sphinxcode{find 0,8000,"AAR",d.0,"BEN",w.0}
\item[] 
\end{DUlineblock}
\item[] Procura no intervalo de endereços \textbf{0000-7fff} pela string ``\textbf{AAR}'' seguindo por um dwrod com tamanho \textbf{0} seguido pela string ``\textbf{BEN}'' e seguido por uma word com tamanho \textbf{0}.
\item[] 
\item[] Back to {\hyperref[debugger/memory:debugger\string-memory\string-list]{\sphinxcrossref{\DUrole{std,std-ref}{Comandos para a depuração de Memória}}}}
\end{DUlineblock}
\begin{quote}
\phantomsection\label{debugger/memory:debugger-command-dump}\end{quote}


\subsection{dump}
\label{debugger/memory:debugger-command-dump}\label{debugger/memory:dump}
\begin{DUlineblock}{0em}
\item[]
\begin{DUlineblock}{\DUlineblockindent}
\item[] \textbf{dump{[}\{d\textbar{}i\}{]}} \textless{}\emph{filename}\textgreater{},\textless{}\emph{address}\textgreater{},\textless{}\emph{length}\textgreater{}{[},\textless{}\emph{size}\textgreater{}{[},\textless{}\emph{ascii}\textgreater{}{[},\textless{}\emph{cpu}\textgreater{}{]}{]}{]}
\item[] 
\end{DUlineblock}
\item[] Os comandos \textbf{dump}/\textbf{dumpd}/\textbf{dumpi} extraem a memória para um arquivo texto especificado com o parâmetro \textless{}\emph{filename}\textgreater{}.
\item[] `dump' despejará o espaço de memória do programa, enquanto `dumpd' despejará o espaço de memória dos dados e `dumpi' despejará o espeço de memória do I/O.
\item[] \textless{}\emph{address}\textgreater{} Indica o endereço inicial do despejo, e \textless{}\emph{length}\textgreater{} indica o quanto será despejado. O intervalo \textless{}\emph{address}\textgreater{} através de \textless{}\emph{address}\textgreater{}+\textless{}\emph{length}\textgreater{}-1 será inclusive salvo em um arquivo.
\item[] É predefinido que os dados serão emitidos em formato de byte, a menos que o espaço de endereço subjacente seja apenas \emph{word/dword/qword-only}. Você pode sobrescrever isso definindo o parâmetro \textless{}\emph{size}\textgreater{}, que pode ser usado para agrupar os dados em pedaços de 1, 2, 4 e 8 bytes.
\item[] O parâmetro \textless{}\emph{ascii}\textgreater{} opcional pode ser usado para ativar (1) ou desativar (0) a saída de caracteres ASCII à direita de cada linha; por padrão, isso está ativado.
\item[] Finalmente, você pode despejar a memória de outro CPU ao definir o parâmetro \textless{}\emph{cpu}\textgreater{}.
\item[] 
\item[] 
\item[] Exemplos:
\item[] 
\item[]
\begin{DUlineblock}{\DUlineblockindent}
\item[] \sphinxcode{dump venture.dmp,0,10000}
\item[] 
\end{DUlineblock}
\item[] Despeja o intervalo de endereços \textbf{0-ffff} em pedaços de \textbf{1 byte} na CPU atual, incluindo dados ASCII no arquivo `venture.dmp'.
\item[] 
\item[]
\begin{DUlineblock}{\DUlineblockindent}
\item[] \sphinxcode{dumpd harddriv.dmp,3000,1000,4,0,3}
\item[] 
\end{DUlineblock}
\item[] Despeja o intervalo de endereços \textbf{3000-3fff} da CPU \textbf{\#3} em pedaços de \textbf{4 bytes}, sem nenhum dado ASCII no arquivo `harddriv.dmp'.
\item[] 
\item[] Back to {\hyperref[debugger/memory:debugger\string-memory\string-list]{\sphinxcrossref{\DUrole{std,std-ref}{Comandos para a depuração de Memória}}}}
\end{DUlineblock}
\begin{quote}
\phantomsection\label{debugger/memory:debugger-command-save}\end{quote}


\subsection{save}
\label{debugger/memory:save}\label{debugger/memory:debugger-command-save}
\begin{DUlineblock}{0em}
\item[]
\begin{DUlineblock}{\DUlineblockindent}
\item[] \textbf{save{[}\{d\textbar{}i\}{]}} \textless{}\emph{filename}\textgreater{},\textless{}\emph{address}\textgreater{},\textless{}\emph{length}\textgreater{}{[},\textless{}\emph{cpu}\textgreater{}{]}
\item[] 
\end{DUlineblock}
\item[] O comando \textbf{save}/\textbf{saved}/\textbf{savei} gravam memória pura (raw) no arquivo de binário especificado com o parâmetro \textless{}\emph{filename}\textgreater{}.
\item[] `save' salvará o espaço de memória do programa, enquanto `saved' salvará o espaço de dados da memória e `savei' salvará o espaço de memória I/O.
\item[] \textless{}\emph{address}\textgreater{} Indica o endereço inicial que será salvo, e \textless{}\emph{length}\textgreater{} indica o quanto dessa memória será salva. O intervalo \textless{}\emph{address}\textgreater{} através de \textless{}\emph{address}\textgreater{}+\textless{}\emph{length}\textgreater{}-1 será inclusive salvo para um arquivo.
\item[] Você também pode salvar a memória de outro CPU ao definir o parâmetro \textless{}\emph{cpu}\textgreater{}.
\item[] 
\item[] 
\item[] Exemplos:
\item[] 
\item[]
\begin{DUlineblock}{\DUlineblockindent}
\item[] \sphinxcode{save venture.bin,0,10000}
\item[] 
\end{DUlineblock}
\item[] Salva o intervalo de endereços \textbf{0-ffff} na CPU atual para o arquivo `venture.bin'.
\item[] 
\item[]
\begin{DUlineblock}{\DUlineblockindent}
\item[] \sphinxcode{saved harddriv.bin,3000,1000,3}
\item[] 
\end{DUlineblock}
\item[] Salva o intervalo de dados da memória \textbf{3000-3fff} da CPU \textbf{\#3} para o arquivo binário `harddriv.bin'.
\item[] 
\item[] Back to {\hyperref[debugger/memory:debugger\string-memory\string-list]{\sphinxcrossref{\DUrole{std,std-ref}{Comandos para a depuração de Memória}}}}
\end{DUlineblock}
\begin{quote}
\phantomsection\label{debugger/memory:debugger-command-load}\end{quote}


\subsection{load}
\label{debugger/memory:load}\label{debugger/memory:debugger-command-load}
\begin{DUlineblock}{0em}
\item[]
\begin{DUlineblock}{\DUlineblockindent}
\item[] \textbf{load{[}\{d\textbar{}i\}{]}} \textless{}\emph{filename}\textgreater{},\textless{}\emph{address}\textgreater{}{[},\textless{}\emph{length}\textgreater{},\textless{}\emph{cpu}\textgreater{}{]}
\item[] 
\end{DUlineblock}
\item[] Os comandos \textbf{load}/\textbf{loadd}/\textbf{loadi} carregam dados puros vindos de um arquivo binário ao ser especificado com o parâmetro \textless{}\emph{filename}\textgreater{}.
\item[] `load' carregará o programa no espaço de memória enquanto `loadd' carregará os dados no espaço de memória e `loadi' carregará o I/O no espaço de memória.
\item[] \textless{}\emph{address}\textgreater{} indica o endereço do início do salvamento, e \textless{}\emph{length}\textgreater{} indica o quanto dessa memória será lida. O intervalo \textless{}\emph{address}\textgreater{} através de \textless{}\emph{address}\textgreater{}+\textless{}\emph{length}\textgreater{}-1 será inclusive lido de um arquivo.
\item[] Se você definir \textless{}\emph{length}\textgreater{} = \emph{0} ou um comprimento maior que o comprimento total do arquivo, ele carregará todo o conteúdo do arquivo e nada mais.
\item[] Você também pode carregar memória de outra CPU definindo o parâmetro \textless{}\emph{cpu}\textgreater{}.
\item[] 
\item[] NOTA: A escrita só será possível caso seja possível sobrescrever na janela da memória.
\item[] 
\item[] 
\item[] Exemplos:
\item[] 
\item[]
\begin{DUlineblock}{\DUlineblockindent}
\item[] \sphinxcode{load venture.bin,0,10000}
\item[] 
\end{DUlineblock}
\item[] Carrega o intervalo de endereços \textbf{0-ffff} na CPU atual vindo do arquivo binário `venture.bin'.
\item[] 
\item[]
\begin{DUlineblock}{\DUlineblockindent}
\item[] \sphinxcode{loadd harddriv.bin,3000,1000,3}
\item[] 
\end{DUlineblock}
\item[] Carrega dados de memória do intervalo de endereços \textbf{3000-3fff} da CPU \textbf{\#3} vindo do arquivo binário `harddriv.bin'.
\item[] 
\item[] Back to {\hyperref[debugger/memory:debugger\string-memory\string-list]{\sphinxcrossref{\DUrole{std,std-ref}{Comandos para a depuração de Memória}}}}
\end{DUlineblock}
\begin{quote}
\phantomsection\label{debugger/memory:debugger-command-map}\end{quote}


\subsection{map}
\label{debugger/memory:map}\label{debugger/memory:debugger-command-map}
\begin{DUlineblock}{0em}
\item[]
\begin{DUlineblock}{\DUlineblockindent}
\item[] \textbf{map{[}\{d\textbar{}i\}{]}} \textless{}\emph{address}\textgreater{}
\item[] 
\end{DUlineblock}
\item[] O comando \textbf{map}/\textbf{mapd}/\textbf{mapi} faz o mapeamento lógico de endereço na memória para o endereço físico correto, além de definir o banco.
\item[] `map' mapeará o espaço do programa na memória enquanto `mapd' mapeará o espaço dos dados na memória e `mapi' mapeará o espaço I/O na memória.
\item[] 
\item[] Exemplo:
\item[] 
\item[]
\begin{DUlineblock}{\DUlineblockindent}
\item[] \sphinxcode{map 152d0}
\item[] 
\end{DUlineblock}
\item[] Fornece o endereço físico e o banco para o endereço lógico \textbf{152d0} na memória do programa
\item[] 
\item[] Back to {\hyperref[debugger/memory:debugger\string-memory\string-list]{\sphinxcrossref{\DUrole{std,std-ref}{Comandos para a depuração de Memória}}}}
\end{DUlineblock}


\section{Comandos de execução do depurador}
\label{debugger/execution:comandos-de-execucao-do-depurador}\label{debugger/execution::doc}\label{debugger/execution:debugger-execution-list}
Na interface de depuração do MAME você pode digitar \textbf{help \textless{}command\textgreater{}}
para uma melhor descrição de cada comando.

\begin{DUlineblock}{0em}
\item[] {\hyperref[debugger/execution:debugger\string-command\string-step]{\sphinxcrossref{\DUrole{std,std-ref}{step}}}} -- passo único para instruções \textless{}\emph{count}\textgreater{} (F11)
\item[] {\hyperref[debugger/execution:debugger\string-command\string-over]{\sphinxcrossref{\DUrole{std,std-ref}{over}}}} -- passo único durante instruções \textless{}\emph{count}\textgreater{} (F10)
\item[] {\hyperref[debugger/execution:debugger\string-command\string-out]{\sphinxcrossref{\DUrole{std,std-ref}{out}}}} -- passo único até o manipulador atual de subrotina/execução seja terminado (Shift-F11)
\item[] {\hyperref[debugger/execution:debugger\string-command\string-go]{\sphinxcrossref{\DUrole{std,std-ref}{go}}}} -- resume a execução, define breakpoint temporário no endereço \textless{}\emph{address}\textgreater{} (F5)
\item[] {\hyperref[debugger/execution:debugger\string-command\string-gint]{\sphinxcrossref{\DUrole{std,std-ref}{gint}}}} -- resume a execução, define breakpoint temporário se \textless{}\emph{irqline}\textgreater{} for tomada (F7)
\item[] {\hyperref[debugger/execution:debugger\string-command\string-gtime]{\sphinxcrossref{\DUrole{std,std-ref}{gtime}}}} -- resume a execução até que o atraso determinado termine
\item[] {\hyperref[debugger/execution:debugger\string-command\string-gvblank]{\sphinxcrossref{\DUrole{std,std-ref}{gvblank}}}} -- resume a execução, define breakpoint temporário até o próximo VBLANK (F8)
\item[] {\hyperref[debugger/execution:debugger\string-command\string-next]{\sphinxcrossref{\DUrole{std,std-ref}{next}}}} -- executa até que o próximo CPU alterne (F6)
\item[] {\hyperref[debugger/execution:debugger\string-command\string-focus]{\sphinxcrossref{\DUrole{std,std-ref}{focus}}}} -- foca o depurados apenas na \textless{}\emph{cpu}\textgreater{}
\item[] {\hyperref[debugger/execution:debugger\string-command\string-ignore]{\sphinxcrossref{\DUrole{std,std-ref}{ignore}}}} -- para a depuração na \textless{}\emph{cpu}\textgreater{}
\item[] {\hyperref[debugger/execution:debugger\string-command\string-observe]{\sphinxcrossref{\DUrole{std,std-ref}{observe}}}} -- continua a depuração na \textless{}\emph{cpu}\textgreater{}
\item[] {\hyperref[debugger/execution:debugger\string-command\string-trace]{\sphinxcrossref{\DUrole{std,std-ref}{trace}}}} -- rastreia o dado CPU para um arquivo (defaults to active CPU)
\item[] {\hyperref[debugger/execution:debugger\string-command\string-traceover]{\sphinxcrossref{\DUrole{std,std-ref}{traceover}}}} -- rastreia o dado CPU para um arquivo, mas pule as subrotinas (defaults to active CPU)
\item[] {\hyperref[debugger/execution:debugger\string-command\string-traceflush]{\sphinxcrossref{\DUrole{std,std-ref}{traceflush}}}} -- elimine todos os arquivo open trace.
\end{DUlineblock}
\begin{quote}
\phantomsection\label{debugger/execution:debugger-command-step}\end{quote}


\subsection{step}
\label{debugger/execution:debugger-command-step}\label{debugger/execution:step}
\begin{DUlineblock}{0em}
\item[]
\begin{DUlineblock}{\DUlineblockindent}
\item[] \textbf{s{[}tep{]}} {[}\textless{}\emph{count}\textgreater{}=1{]}
\item[] 
\end{DUlineblock}
\item[] O comando ``\emph{step}'' avança uma ou mais instruções na CPU que estiverem sendo executadas. É predefinido que o comando execute apenas uma instrução a cada vez que for chamado. Também é possível dar um passo em diferentes instruções ao incluir o parâmetro opcional \textless{}\emph{count}\textgreater{}.
\item[] 
\item[] Exemplos:
\item[] 
\item[]
\begin{DUlineblock}{\DUlineblockindent}
\item[] \sphinxcode{s}
\item[] 
\end{DUlineblock}
\item[] Avança apenas uma instrução da CPU.
\item[] 
\item[]
\begin{DUlineblock}{\DUlineblockindent}
\item[] \sphinxcode{step 4}
\item[] 
\end{DUlineblock}
\item[] Avança quatro instruções da CPU.
\item[] 
\item[] Voltar para {\hyperref[debugger/execution:debugger\string-execution\string-list]{\sphinxcrossref{\DUrole{std,std-ref}{Comandos de execução do depurador}}}}
\end{DUlineblock}
\begin{quote}
\phantomsection\label{debugger/execution:debugger-command-over}\end{quote}


\subsection{over}
\label{debugger/execution:debugger-command-over}\label{debugger/execution:over}
\begin{DUlineblock}{0em}
\item[]
\begin{DUlineblock}{\DUlineblockindent}
\item[] \textbf{o{[}ver{]}} {[}\textless{}\emph{count}\textgreater{}=1{]}
\item[] 
\end{DUlineblock}
\item[] O comando ``\emph{over}'' avança um passo simples sobre uma ou mais instruções que estiverem sendo executadas na CPU, passando por cima de chamadas de sub-rotina e traps do manipulador de exceção, contando-os como uma única instrução. Observe que, ao passar por cima de uma chamada de sub-rotina o código pode ser executado em outras CPUs antes da conclusão da chamada. É predefinido que o comando execute apenas uma instrução a cada vez que for chamado. Também é possível dar um passo em diferentes instruções ao incluir o parâmetro opcional \textless{}\emph{count}\textgreater{}.
\item[] 
\item[] Observe que a funcionalidade step over pode não estar implementada em todos os tipos de CPU. Caso não esteja, então o comando `over' se comportará exatamente como o comando `step'.
\item[] 
\item[] Exemplos:
\item[] 
\item[]
\begin{DUlineblock}{\DUlineblockindent}
\item[] \sphinxcode{o}
\item[] 
\end{DUlineblock}
\item[] Avança e passa por cima de apenas uma instrução da CPU.
\item[] 
\item[]
\begin{DUlineblock}{\DUlineblockindent}
\item[] \sphinxcode{over 4}
\item[] 
\end{DUlineblock}
\item[] Avança e passa por cima sobre quatro instruções da CPU atual.
\item[] 
\item[] Voltar para {\hyperref[debugger/execution:debugger\string-execution\string-list]{\sphinxcrossref{\DUrole{std,std-ref}{Comandos de execução do depurador}}}}
\end{DUlineblock}
\begin{quote}
\phantomsection\label{debugger/execution:debugger-command-out}\end{quote}


\subsection{out}
\label{debugger/execution:debugger-command-out}\label{debugger/execution:out}
\begin{DUlineblock}{0em}
\item[]
\begin{DUlineblock}{\DUlineblockindent}
\item[] \textbf{out}
\item[] 
\end{DUlineblock}
\item[] O comando ``\emph{out}'' avança passos simples até encontrar um retorno da sub-rotina ou retorno da instrução em exceção. Observe que, como ele detecta o retorno das condições de exceção, caso você tente sair de uma sub-rotina e ocorrer uma interrupção/exceção antes de atingir o final, você poderá parar prematuramente no final do manipulador de exceções.
\item[] 
\item[] Observe que a funcionalidade de saída não pode estar implementada em todos os tipos de CPU. Caso não esteja, então o comando `out' se comportará exatamente como o comando `step'.
\item[] 
\item[] Exemplos:
\item[] 
\item[]
\begin{DUlineblock}{\DUlineblockindent}
\item[] \sphinxcode{out}
\item[] 
\end{DUlineblock}
\item[] Avance até que a sub-rotina atual ou o manipulador de exceções retorne.
\item[] 
\item[] Voltar para {\hyperref[debugger/execution:debugger\string-execution\string-list]{\sphinxcrossref{\DUrole{std,std-ref}{Comandos de execução do depurador}}}}
\end{DUlineblock}
\begin{quote}
\phantomsection\label{debugger/execution:debugger-command-go}\end{quote}


\subsection{go}
\label{debugger/execution:go}\label{debugger/execution:debugger-command-go}
\begin{DUlineblock}{0em}
\item[]
\begin{DUlineblock}{\DUlineblockindent}
\item[] \textbf{g{[}o{]}} {[}\textless{}\emph{address}\textgreater{}{]}
\item[] 
\end{DUlineblock}
\item[] O comando ``\emph{go}'' retoma a execução do código atual. O controle não será retornado ao depurador até que um breakpoint ou um watchpoint seja atingido, ou até que você interrompa manualmente usando a chave designada. O comando go usa um parâmetro opcional \emph{\textless{}address\textgreater{}} que é um breakpoint incondicional que é definido antes de ser executado e removido automaticamente quando for pressionado.
\item[] 
\item[] Exemplos:
\item[] 
\item[]
\begin{DUlineblock}{\DUlineblockindent}
\item[] \sphinxcode{g}
\item[] 
\end{DUlineblock}
\item[] Retomar a execução até o próximo \textbf{break/watchpoint} ou até uma parada manual.
\item[] 
\item[]
\begin{DUlineblock}{\DUlineblockindent}
\item[] \sphinxcode{g 1234}
\item[] 
\end{DUlineblock}
\item[] Retomar a execução parando no endereço \textbf{1234} a não ser que algo nos pare primeiro.
\item[] 
\item[] Voltar para {\hyperref[debugger/execution:debugger\string-execution\string-list]{\sphinxcrossref{\DUrole{std,std-ref}{Comandos de execução do depurador}}}}
\end{DUlineblock}
\begin{quote}
\phantomsection\label{debugger/execution:debugger-command-gvblank}\end{quote}


\subsection{gvblank}
\label{debugger/execution:gvblank}\label{debugger/execution:debugger-command-gvblank}
\begin{DUlineblock}{0em}
\item[]
\begin{DUlineblock}{\DUlineblockindent}
\item[] \textbf{gv{[}blank{]}}
\item[] 
\end{DUlineblock}
\item[] O comando ``\emph{gvblank}'' retoma a execução do código atual. O controle não será retornado ao depurador até que um breakpoint ou watchpoint seja atingido ou até que o próximo \textbf{VBLANK} ocorra no emulador.
\item[] 
\item[] Exemplos:
\item[] 
\item[]
\begin{DUlineblock}{\DUlineblockindent}
\item[] \sphinxcode{gv}
\item[] 
\end{DUlineblock}
\item[] Retomar a execução até o próximo \textbf{break/watchpoint} ou até o próximo \textbf{VBLANK}.
\item[] 
\item[] Voltar para {\hyperref[debugger/execution:debugger\string-execution\string-list]{\sphinxcrossref{\DUrole{std,std-ref}{Comandos de execução do depurador}}}}
\end{DUlineblock}
\begin{quote}
\phantomsection\label{debugger/execution:debugger-command-gint}\end{quote}


\subsection{gint}
\label{debugger/execution:debugger-command-gint}\label{debugger/execution:gint}
\begin{DUlineblock}{0em}
\item[]
\begin{DUlineblock}{\DUlineblockindent}
\item[] \textbf{gi{[}nt{]}} {[}\textless{}\emph{irqline}\textgreater{}{]}
\item[] 
\end{DUlineblock}
\item[] O comando ``\emph{gint}'' retoma a execução do código atual. O controle não será retornado ao depurador até que um breakpoint ou watchpoint seja atingido ou até que um IRQ seja declarado e reconhecido na CPU atual. Você pode definir um \textless{}\emph{irqline}\textgreater{} caso deseje interromper a execução apenas em uma determinada linha de IRQ que estiver sendo declarada e confirmada. Caso o \textless{}\emph{irqline}\textgreater{} seja omitido, então qualquer linha IRQ irá parar a execução.
\item[] 
\item[] Exemplos:
\item[] 
\item[]
\begin{DUlineblock}{\DUlineblockindent}
\item[] \sphinxcode{gi}
\item[] 
\end{DUlineblock}
\item[] Retomar a execução até o próximo \textbf{break/watchpoint} ou até que qualquer IRQ seja declarado e reconhecido na CPU atual.
\item[] 
\item[]
\begin{DUlineblock}{\DUlineblockindent}
\item[] \sphinxcode{gint 4}
\item[] 
\end{DUlineblock}
\item[] Retomar a execução até a próxima \textbf{break/watchpoint} ou até que a linha IRQ seja declarada e confirmada na CPU atual.
\item[] 
\item[] Voltar para {\hyperref[debugger/execution:debugger\string-execution\string-list]{\sphinxcrossref{\DUrole{std,std-ref}{Comandos de execução do depurador}}}}
\end{DUlineblock}
\begin{quote}
\phantomsection\label{debugger/execution:debugger-command-gtime}\end{quote}


\subsection{gtime}
\label{debugger/execution:debugger-command-gtime}\label{debugger/execution:gtime}
\begin{DUlineblock}{0em}
\item[]
\begin{DUlineblock}{\DUlineblockindent}
\item[] \textbf{gt{[}ime{]}} \textless{}\emph{milliseconds}\textgreater{}
\item[] 
\end{DUlineblock}
\item[] O comando ``\emph{gtime}'' retoma a execução do código atual. O controle não será retornado ao depurador até que um atraso especificado tenha decorrido. O atraso é em milissegundos.
\item[] 
\item[] Examplo:
\item[] 
\item[]
\begin{DUlineblock}{\DUlineblockindent}
\item[] \sphinxcode{gtime \#10000}
\item[] 
\end{DUlineblock}
\item[] Retomar a execução por dez segundos
\item[] 
\item[] Voltar para {\hyperref[debugger/execution:debugger\string-execution\string-list]{\sphinxcrossref{\DUrole{std,std-ref}{Comandos de execução do depurador}}}}
\end{DUlineblock}
\begin{quote}
\phantomsection\label{debugger/execution:debugger-command-next}\end{quote}


\subsection{next}
\label{debugger/execution:debugger-command-next}\label{debugger/execution:next}
\begin{DUlineblock}{0em}
\item[]
\begin{DUlineblock}{\DUlineblockindent}
\item[] \textbf{n{[}ext{]}}
\item[] 
\end{DUlineblock}
\item[] O comando ``\emph{next}'' retoma a execução e continua a execução até a próxima vez que uma CPU diferente for planejada. Note que se você usou `ignore' para ignorar certas CPUs, você não irá parar até que uma CPU não-`ignore' seja agendada.
\item[] 
\item[] Voltar para {\hyperref[debugger/execution:debugger\string-execution\string-list]{\sphinxcrossref{\DUrole{std,std-ref}{Comandos de execução do depurador}}}}
\end{DUlineblock}
\begin{quote}
\phantomsection\label{debugger/execution:debugger-command-focus}\end{quote}


\subsection{focus}
\label{debugger/execution:debugger-command-focus}\label{debugger/execution:focus}
\begin{DUlineblock}{0em}
\item[]
\begin{DUlineblock}{\DUlineblockindent}
\item[] \textbf{focus} \textless{}\emph{cpu}\textgreater{}
\item[] 
\end{DUlineblock}
\item[] O comando ``\emph{focus}'' Define o foco do depurador exclusivamente para o dado \textless{}\emph{cpu}\textgreater{}. Isso é equivalente a especificar `ignore' em todas as outras CPUs.
\item[] 
\item[] Example:
\item[] 
\item[]
\begin{DUlineblock}{\DUlineblockindent}
\item[] \sphinxcode{focus 1}
\item[] 
\end{DUlineblock}
\item[] Concentre-se exclusivamente CPU \textbf{\#1} enquanto ignora todas as outras CPUs ao usar o depurador.
\item[] 
\item[] Voltar para {\hyperref[debugger/execution:debugger\string-execution\string-list]{\sphinxcrossref{\DUrole{std,std-ref}{Comandos de execução do depurador}}}}
\end{DUlineblock}
\begin{quote}
\phantomsection\label{debugger/execution:debugger-command-ignore}\end{quote}


\subsection{ignore}
\label{debugger/execution:ignore}\label{debugger/execution:debugger-command-ignore}
\begin{DUlineblock}{0em}
\item[]
\begin{DUlineblock}{\DUlineblockindent}
\item[] \textbf{ignore} {[}\textless{}\emph{cpu}\textgreater{}{[},\textless{}\emph{cpu}\textgreater{}{[},...{]}{]}{]}
\item[] 
\end{DUlineblock}
\item[] Ignora a \textless{}\emph{cpu}\textgreater{} definida ao usar o depurador. Isso significa que você nunca verá a execução nessa CPU e tão pouco poderá definir breakpoints nela. Para desfazer essa mudança, use o comando `observe'. Você pode definir diferentes \emph{\textless{}cpu\textgreater{}s} em um único comando. Note também que você não tem permissão para ignorar todas as CPUs; pelo menos um deve estar ativo em todos os momentos.
\item[] 
\item[] Exemplos:
\item[] 
\item[]
\begin{DUlineblock}{\DUlineblockindent}
\item[] \sphinxcode{ignore 1}
\item[] 
\end{DUlineblock}
\item[] Ignore o CPU \textbf{\#1} ao usar o depurador.
\item[] 
\item[]
\begin{DUlineblock}{\DUlineblockindent}
\item[] \sphinxcode{ignore 2,3,4}
\item[] 
\end{DUlineblock}
\item[] Ignora a CPU \textbf{\#2}, \textbf{\#3} e \textbf{\#4} ao usar o depurador.
\item[] 
\item[]
\begin{DUlineblock}{\DUlineblockindent}
\item[] \sphinxcode{ignore}
\item[] 
\end{DUlineblock}
\item[] Liste todas as CPUs atualmente ignoradas.
\item[] 
\item[] Voltar para {\hyperref[debugger/execution:debugger\string-execution\string-list]{\sphinxcrossref{\DUrole{std,std-ref}{Comandos de execução do depurador}}}}
\end{DUlineblock}
\begin{quote}
\phantomsection\label{debugger/execution:debugger-command-observe}\end{quote}


\subsection{observe}
\label{debugger/execution:observe}\label{debugger/execution:debugger-command-observe}
\begin{DUlineblock}{0em}
\item[]
\begin{DUlineblock}{\DUlineblockindent}
\item[] \textbf{observe} {[}\textless{}\emph{cpu}\textgreater{}{[},\textless{}\emph{cpu}\textgreater{}{[},...{]}{]}{]}
\item[] 
\end{DUlineblock}
\item[] Reativa a interação com a \textless{}\emph{cpu}\textgreater{} definida no depurador. Este comando desfaz os efeitos do comando `ignore'. Você pode especificar diferentes \textless{}\emph{cpu}\textgreater{}s em um único comando.
\item[] 
\item[] Exemplos:
\item[] 
\item[]
\begin{DUlineblock}{\DUlineblockindent}
\item[] \sphinxcode{observe 1}
\item[] 
\end{DUlineblock}
\item[] Pare de ignorar a CPU \textbf{\#1} ao usar o depurador.
\item[] 
\item[]
\begin{DUlineblock}{\DUlineblockindent}
\item[] \sphinxcode{observe 2,3,4}
\item[] 
\end{DUlineblock}
\item[] Pare de ignorar a CPU \textbf{\#2}, \textbf{\#3} e \textbf{\#4} quando usar o depurador.
\item[] 
\item[]
\begin{DUlineblock}{\DUlineblockindent}
\item[] \sphinxcode{observe}
\item[] 
\end{DUlineblock}
\item[] Liste todas as CPUs sendo observadas atualmente.
\item[] 
\item[] Voltar para {\hyperref[debugger/execution:debugger\string-execution\string-list]{\sphinxcrossref{\DUrole{std,std-ref}{Comandos de execução do depurador}}}}
\end{DUlineblock}
\begin{quote}
\phantomsection\label{debugger/execution:debugger-command-trace}\end{quote}


\subsection{trace}
\label{debugger/execution:trace}\label{debugger/execution:debugger-command-trace}
\begin{DUlineblock}{0em}
\item[]
\begin{DUlineblock}{\DUlineblockindent}
\item[] \textbf{trace} \{\textless{}\emph{filename}\textgreater{} \textbar{} \emph{OFF}\}{[},\textless{}\emph{cpu}\textgreater{}{[},{[}\emph{noloop} \textbar{} \emph{logerror}{]}{[},\textless{}\emph{action}\textgreater{}{]}{]}{]}
\item[] 
\end{DUlineblock}
\item[] Inicia ou para o rastreio da execução da \textless{}\emph{cpu}\textgreater{} definida. Caso a \textless{}\emph{cpu}\textgreater{} seja omitida a CPU que estiver ativa no momento será definida.
\item[] 
\item[] Ao habilitar o rastreamento, defina um nome do arquivo \textless{}\emph{filename}\textgreater{} no parâmetro. Para desabilitar o rastreamento, substitua a palavra-chave `off' no \textless{}\emph{filename}\textgreater{}.
\item[] 
\item[] \textless{}\emph{detectloops}\textgreater{} deve ser \textbf{true} ou \textbf{false}.
\item[] 
\item[] Caso o `noloop' seja omitido, o rastreamento terá loops detectados e será condensado em uma única linha. Caso o `noloop' seja definido, o rastreio irá conter cada opcode conforme for sendo executado.
\item[] 
\item[] Caso o `logerror' seja definido, a saída do logerror irá aumentar o rastreamento. Se você deseja obter informações adicionais sobre cada vestígio de log, você pode acrescentar o parâmetro \textless{}\emph{action}\textgreater{} que é um comando que é executado antes que cada traço que for registrado. Geralmente, isso é usado para incluir um comando `tracelog'. Observe que você pode precisar incorporar a ação entre chaves \textbf{\{ \}} para evitar que as vírgulas e os pontos-e-vírgulas sejam interpretados como se aplicassem ao próprio comando trace.
\item[] 
\item[] 
\item[] Exemplos:
\item[] 
\item[]
\begin{DUlineblock}{\DUlineblockindent}
\item[] \sphinxcode{trace joust.tr}
\item[] 
\end{DUlineblock}
\item[] Iniciar o rastreamento da CPU atualmente ativa, registrando a saída para `joust.tr'.
\item[] 
\item[]
\begin{DUlineblock}{\DUlineblockindent}
\item[] \sphinxcode{trace dribling.tr,0}
\item[] 
\end{DUlineblock}
\item[] Comece a rastrear a execução da CPU \textbf{\#0}, registrando a saída para `dribling.tr'.
\item[] 
\item[]
\begin{DUlineblock}{\DUlineblockindent}
\item[] \sphinxcode{trace starswep.tr,0,noloop}
\item[] 
\end{DUlineblock}
\item[] Comece a rastrear a execução da CPU \textbf{\#0}, registrando a saída em `starswep.tr', com a detecção de loop desativada.
\item[] 
\item[]
\begin{DUlineblock}{\DUlineblockindent}
\item[] trace starswep.tr,0,logerror
\item[] 
\end{DUlineblock}
\item[] Comece a rastrear a execução da CPU \textbf{\#0}, registrando a saída (junto com a saída logerror) para `starswep.tr'.
\item[] 
\item[]
\begin{DUlineblock}{\DUlineblockindent}
\item[] \sphinxcode{trace starswep.tr,0,logerror\textbar{}noloop}
\item[] 
\end{DUlineblock}
\item[] Comece a rastrear a execução da CPU \textbf{\#0}, registrando a saída (junto com a saída logerror) para `starswep.tr' com a detecção de loop desativada.
\item[] 
\item[]
\begin{DUlineblock}{\DUlineblockindent}
\item[] \sphinxcode{trace \textgreater{}\textgreater{}pigskin.tr}
\item[] 
\end{DUlineblock}
\item[] Comece a rastrear a CPU atualmente ativa, anexando a saída de log para `pigskin.tr'.
\item[] 
\item[]
\begin{DUlineblock}{\DUlineblockindent}
\item[] \sphinxcode{trace off,0}
\item[] 
\end{DUlineblock}
\item[] Desativar o rastreio na CPU \textbf{\#0}.
\item[] 
\item[]
\begin{DUlineblock}{\DUlineblockindent}
\item[] \sphinxcode{trace asteroid.tr,0,,\{tracelog "A=\%02X ",a\}}
\item[] 
\item[] 
\item[] \sphinxcode{trace dribling.tr,0}
\item[] 
\end{DUlineblock}
\item[] Comece a rastrear a execução da CPU \textbf{\#0}, registrando a saída para `dribling.tr'. Antes de cada linha, a saída \textbf{A=\textless{}aval\textgreater{}} para o tracelog.
\item[] 
\item[] Voltar para {\hyperref[debugger/execution:debugger\string-execution\string-list]{\sphinxcrossref{\DUrole{std,std-ref}{Comandos de execução do depurador}}}}
\end{DUlineblock}
\begin{quote}
\phantomsection\label{debugger/execution:debugger-command-traceover}\end{quote}


\subsection{traceover}
\label{debugger/execution:traceover}\label{debugger/execution:debugger-command-traceover}
\begin{DUlineblock}{0em}
\item[]
\begin{DUlineblock}{\DUlineblockindent}
\item[] \textbf{traceover} \{\textless{}\emph{filename}\textgreater{} \textbar{} \emph{OFF}\}{[},\textless{}\emph{cpu}\textgreater{}{[},\textless{}\emph{detectloops}\textgreater{}{[},\textless{}\emph{action}\textgreater{}{]}{]}{]}
\item[] 
\end{DUlineblock}
\item[] Inicia ou para o rastreio na execução da \textless{}\emph{cpu}\textgreater{} especificada.
\item[] 
\item[] Quando o rastreamento atinge uma sub-rotina ou chamada, a sub-rotina será ignorada pelo rastreamento. O mesmo algoritmo é usado como é usado no comando \emph{step over}. Isso significa que o rastreio não funcionará corretamente quando as chamadas forem recursivas ou o endereço de retorno não estiver seguindo imediatamente a instrução de chamada.
\item[] 
\item[] \textless{}\emph{detectloops}\textgreater{} deve ser true ou false. Caso o \textless{}\emph{detectloops}\textgreater{} seja \emph{true} ou \emph{omitido}, o rastreio terá loops detectados e condensados em uma única linha. Caso seja \emph{false}, o rastreio conterá todos os opcode à medida que forem executados.
\item[] Se o \textless{}\emph{cpu}\textgreater{} for omitido, a CPU atualmente ativa será a especificada.
\item[] Ao habilitar o rastreamento, especifique o nome do arquivo \textless{}\emph{filename}\textgreater{} no parâmetro.
\item[] Para desabilitar o rastreamento, substitui a palavra-chave `off' para \textless{}\emph{filename}\textgreater{}.
\item[] Se você deseja obter informações adicionais sobre cada vestígio de log, você pode acrescentar o parâmetro \textless{}\emph{action}\textgreater{} que é um comando que é executado antes de cada rastreio que for registrado. Geralmente, isso é usado para incluir um comando `tracelog'. Observe que você pode precisar incorporar a ação entre chaves \textbf{\{ \}} para evitar que as vírgulas e os pontos-e-vírgulas sejam interpretados como se aplicassem ao próprio comando trace.
\item[] 
\item[] 
\item[] Exemplos:
\item[] 
\item[]
\begin{DUlineblock}{\DUlineblockindent}
\item[] \sphinxcode{traceover joust.tr}
\item[] 
\end{DUlineblock}
\item[] Iniciar o rastreamento da CPU atualmente ativa, registrando a saída para `joust.tr'.
\item[] 
\item[]
\begin{DUlineblock}{\DUlineblockindent}
\item[] \sphinxcode{traceover dribling.tr,0}
\item[] 
\end{DUlineblock}
\item[] Comece a rastrear a execução da CPU \textbf{\#0}, registrando a saída para `dribling.tr'.
\item[] 
\item[]
\begin{DUlineblock}{\DUlineblockindent}
\item[] \sphinxcode{traceover starswep.tr,0,false}
\item[] 
\end{DUlineblock}
\item[] Comece a rastrear a execução da CPU \textbf{\#0}, registrando a saída para `starswep.tr', com a detecção de loop desativada.
\item[] 
\item[]
\begin{DUlineblock}{\DUlineblockindent}
\item[] \sphinxcode{traceover off,0}
\item[] 
\end{DUlineblock}
\item[] Desativar o rastreio na CPU \textbf{\#0}.
\item[] 
\item[]
\begin{DUlineblock}{\DUlineblockindent}
\item[] \sphinxcode{traceover asteroid.tr,0,true,\{tracelog "A=\%02X ",a\}}
\item[] 
\end{DUlineblock}
\item[] Comece a rastrear a execução da CPU \textbf{\#0}, registrando a saída para `dribling.tr'. Antes de cada linha, a saída \textbf{A=\textless{}aval\textgreater{}} para o tracelog.
\item[] 
\item[] Voltar para {\hyperref[debugger/execution:debugger\string-execution\string-list]{\sphinxcrossref{\DUrole{std,std-ref}{Comandos de execução do depurador}}}}
\end{DUlineblock}
\begin{quote}
\phantomsection\label{debugger/execution:debugger-command-traceflush}\end{quote}


\subsection{traceflush}
\label{debugger/execution:traceflush}\label{debugger/execution:debugger-command-traceflush}
\begin{DUlineblock}{0em}
\item[]
\begin{DUlineblock}{\DUlineblockindent}
\item[] \textbf{traceflush}
\item[] 
\end{DUlineblock}
\item[] Libera todos os arquivos de rastreamento abertos.
\item[] 
\item[] Voltar para {\hyperref[debugger/execution:debugger\string-execution\string-list]{\sphinxcrossref{\DUrole{std,std-ref}{Comandos de execução do depurador}}}}
\end{DUlineblock}


\section{Comandos de breakpoints do depurador}
\label{debugger/breakpoint:debugger-breakpoint-list}\label{debugger/breakpoint::doc}\label{debugger/breakpoint:comandos-de-breakpoints-do-depurador}
Na interface de depuração do MAME você pode digitar \textbf{help \textless{}command\textgreater{}}
para uma melhor descrição de cada comando.

\begin{DUlineblock}{0em}
\item[] {\hyperref[debugger/breakpoint:debugger\string-command\string-bpset]{\sphinxcrossref{\DUrole{std,std-ref}{bpset}}}} -- define o breakpoint no \textless{}\emph{address}\textgreater{}
\item[] {\hyperref[debugger/breakpoint:debugger\string-command\string-bpclear]{\sphinxcrossref{\DUrole{std,std-ref}{bpclear}}}} -- limpa um determinado breakpoint ou todos se nenhum \textless{}\emph{bpnum}\textgreater{} for especificado
\item[] {\hyperref[debugger/breakpoint:debugger\string-command\string-bpdisable]{\sphinxcrossref{\DUrole{std,std-ref}{bpdisable}}}} -- desabilita um determinado breakpoint se nenhum \textless{}\emph{bpnum}\textgreater{} for especificado
\item[] {\hyperref[debugger/breakpoint:debugger\string-command\string-bpenable]{\sphinxcrossref{\DUrole{std,std-ref}{bpenable}}}} -- habilita um determinado breakpoint ou todos se nenhum \textless{}\emph{bpnum}\textgreater{} for especificado
\item[] {\hyperref[debugger/breakpoint:debugger\string-command\string-bplist]{\sphinxcrossref{\DUrole{std,std-ref}{bplist}}}} -- lista todos os breakpoints
\end{DUlineblock}
\begin{quote}
\phantomsection\label{debugger/breakpoint:debugger-command-bpset}\end{quote}


\subsection{bpset}
\label{debugger/breakpoint:debugger-command-bpset}\label{debugger/breakpoint:bpset}
\begin{DUlineblock}{0em}
\item[]
\begin{DUlineblock}{\DUlineblockindent}
\item[] \textbf{bp{[}set{]}} \textless{}\emph{address}\textgreater{}{[},\textless{}\emph{condition}\textgreater{}{[},\textless{}\emph{action}\textgreater{}{]}{]}
\item[] 
\end{DUlineblock}
\item[] Define uma nova execução de breakpoint no \textless{}\emph{address}\textgreater{} especificado.
\item[] O parâmetro de condição opcional \emph{\textless{}condition\textgreater{}} permite que você especifique uma expressão que será avaliada cada vez que um breakpoint for atingido. Caso o resultado da expressão seja verdadeiro (não-zero), o breakpoint irá interromper (halt) a execução; caso contrário, a execução continuará sem nenhuma notificação.
\item[] O parâmetro opcional de ação \emph{\textless{}action\textgreater{}} fornece um comando que é executado sempre que o breakpoint for atingido e a condição \emph{\textless{}condition\textgreater{}} for verdadeira. Observe que você pode precisar incorporar a ação dentro de chaves \textbf{\{ \}} para evitar que vírgulas e ponto e vírgula seja interpretado e aplicado ao comando de bpset em si. Cada breakpoint que for definido é designado a um índice que pode ser usado em outros comandos breakpoint para usar este breakpoint como referência.
\item[] 
\item[] Exemplos:
\item[] 
\item[]
\begin{DUlineblock}{\DUlineblockindent}
\item[] \sphinxcode{bp 1234}
\item[] 
\end{DUlineblock}
\item[] Define um breakpoint que irá interromper uma execução sempre que o PC for igual a \textbf{1234}.
\item[] 
\item[]
\begin{DUlineblock}{\DUlineblockindent}
\item[] \sphinxcode{bp 23456,a0 == 0 \&\& a1 == 0}
\item[] 
\end{DUlineblock}
\item[] Define um breakpoint que irá interromper uma execução sempre que o PC for igual a \textbf{23456} e a expressão \textbf{(a0 == 0 \&\& a1 == 0)} for verdadeira.
\item[] 
\item[]
\begin{DUlineblock}{\DUlineblockindent}
\item[] \sphinxcode{bp 3456,1,\{printf "A0=\%08X\textbackslash{}\textbackslash{}n",a0; g\}}
\item[] 
\end{DUlineblock}
\item[] Define um breakpoint que irá interromper uma execução sempre que o PC for igual a \textbf{3456}. Quando isso acontecer, imprime \textbf{A0=\textless{}a0val\textgreater{}} e continua a execução.
\item[] 
\item[]
\begin{DUlineblock}{\DUlineblockindent}
\item[] \sphinxcode{bp 45678,a0==100,\{a0 = ff; g\}}
\item[] 
\end{DUlineblock}
\item[] Define um breakpoints que irá interromper uma execução sempre que o PC for igual a \textbf{45678} e a expressão \textbf{(a0 == 100)} for verdadeira. Quando isso acontecer, define \textbf{a0} para \textbf{ff} e resume a execução.
\item[] 
\item[]
\begin{DUlineblock}{\DUlineblockindent}
\item[] \sphinxcode{temp0 = 0; bp 567890,++temp0 \textgreater{}= 10}
\item[] 
\end{DUlineblock}
\item[] Define um breakpoints que irá interromper uma execução sempre que o PC for igual a \textbf{567890} e a expressão \textbf{(++temp0 \textgreater{}= 10)} for verdadeira. Isso somente para de fato após o breakpoint ter sido atingido 16 vezes.
\item[] 
\item[] Back to {\hyperref[debugger/breakpoint:debugger\string-breakpoint\string-list]{\sphinxcrossref{\DUrole{std,std-ref}{Comandos de breakpoints do depurador}}}}
\end{DUlineblock}
\begin{quote}
\phantomsection\label{debugger/breakpoint:debugger-command-bpclear}\end{quote}


\subsection{bpclear}
\label{debugger/breakpoint:debugger-command-bpclear}\label{debugger/breakpoint:bpclear}
\begin{DUlineblock}{0em}
\item[]
\begin{DUlineblock}{\DUlineblockindent}
\item[] \textbf{bpclear} {[}\textless{}\emph{bpnum}\textgreater{}{]}
\item[] 
\end{DUlineblock}
\item[] O comando ``\emph{bpclear}'' limpa um breakpoint. Caso um \textless{}\emph{bpnum}\textgreater{} seja definido, apenas o breakpoint requisitado será limpo, caso contrário todos os breakpoints serão limpos.
\item[] 
\item[] Exemplos:
\item[] 
\item[]
\begin{DUlineblock}{\DUlineblockindent}
\item[] \sphinxcode{bpclear 3}
\item[] 
\end{DUlineblock}
\item[] Limpa o indexador \textbf{3} do breakpoints.
\item[] 
\item[]
\begin{DUlineblock}{\DUlineblockindent}
\item[] \sphinxcode{bpclear}
\item[] 
\end{DUlineblock}
\item[] Limpa todos os breakpoints.
\item[] 
\item[] Back to {\hyperref[debugger/breakpoint:debugger\string-breakpoint\string-list]{\sphinxcrossref{\DUrole{std,std-ref}{Comandos de breakpoints do depurador}}}}
\end{DUlineblock}
\begin{quote}
\phantomsection\label{debugger/breakpoint:debugger-command-bpdisable}\end{quote}


\subsection{bpdisable}
\label{debugger/breakpoint:debugger-command-bpdisable}\label{debugger/breakpoint:bpdisable}
\begin{DUlineblock}{0em}
\item[]
\begin{DUlineblock}{\DUlineblockindent}
\item[] \textbf{bpdisable} {[}\textless{}\emph{bpnum}\textgreater{}{]}
\item[] 
\end{DUlineblock}
\item[] O comando ``\emph{bpdisable}'' desabilita um breakpoint. Caso um \textless{}\emph{bpnum}\textgreater{} seja definido, apenas o breakpoint solicitado será desabilitado, caso contrário todos os breakpoints serão desativados. Observe que ao desabilitar um breakpoint ele não será apagado, apenas o marca temporariamente como inativo.
\item[] 
\item[] Exemplos:
\item[] 
\item[]
\begin{DUlineblock}{\DUlineblockindent}
\item[] \sphinxcode{bpdisable 3}
\item[] 
\end{DUlineblock}
\item[] Desabilita o indexador \textbf{3} do breakpoint.
\item[] 
\item[]
\begin{DUlineblock}{\DUlineblockindent}
\item[] \sphinxcode{bpdisable}
\item[] 
\end{DUlineblock}
\item[] Desabilita todos os breakpoints.
\item[] 
\item[] Back to {\hyperref[debugger/breakpoint:debugger\string-breakpoint\string-list]{\sphinxcrossref{\DUrole{std,std-ref}{Comandos de breakpoints do depurador}}}}
\end{DUlineblock}
\begin{quote}
\phantomsection\label{debugger/breakpoint:debugger-command-bpenable}\end{quote}


\subsection{bpenable}
\label{debugger/breakpoint:debugger-command-bpenable}\label{debugger/breakpoint:bpenable}
\begin{DUlineblock}{0em}
\item[]
\begin{DUlineblock}{\DUlineblockindent}
\item[] \textbf{bpenable} {[}\textless{}\emph{bpnum}\textgreater{}{]}
\item[] 
\end{DUlineblock}
\item[] O comando ``\emph{bpenable}'' habilita um breakpoint. Caso um \textless{}\emph{bpnum}\textgreater{} seja definido, apenas o breakpoint solicitado será ativado, caso contrário todos os breakpoints serão desativados.
\item[] 
\item[] Exemplos:
\item[] 
\item[]
\begin{DUlineblock}{\DUlineblockindent}
\item[] \sphinxcode{bpenable 3}
\item[] 
\end{DUlineblock}
\item[] Ativa o indexador 3 do breakpoint.
\item[] 
\item[]
\begin{DUlineblock}{\DUlineblockindent}
\item[] \sphinxcode{bpenable}
\item[] 
\end{DUlineblock}
\item[] Ativa todos os breakpoints.
\item[] 
\item[] Back to {\hyperref[debugger/breakpoint:debugger\string-breakpoint\string-list]{\sphinxcrossref{\DUrole{std,std-ref}{Comandos de breakpoints do depurador}}}}
\end{DUlineblock}
\begin{quote}
\phantomsection\label{debugger/breakpoint:debugger-command-bplist}\end{quote}


\subsection{bplist}
\label{debugger/breakpoint:debugger-command-bplist}\label{debugger/breakpoint:bplist}
\begin{DUlineblock}{0em}
\item[]
\begin{DUlineblock}{\DUlineblockindent}
\item[] \textbf{bplist}
\item[] 
\end{DUlineblock}
\item[] O comando bplist lista todos os breakpoints atuais, junto com seu indexador ou qualquer condições ou ações anexados a eles.
\item[] 
\item[] Back to {\hyperref[debugger/breakpoint:debugger\string-breakpoint\string-list]{\sphinxcrossref{\DUrole{std,std-ref}{Comandos de breakpoints do depurador}}}}
\end{DUlineblock}


\section{Comandos watchpoint do depurador}
\label{debugger/watchpoint:comandos-watchpoint-do-depurador}\label{debugger/watchpoint::doc}\label{debugger/watchpoint:debugger-watchpoints-list}
Na interface de depuração do MAME você pode digitar \textbf{help \textless{}command\textgreater{}}
para uma melhor descrição de cada comando.

\begin{DUlineblock}{0em}
\item[] {\hyperref[debugger/watchpoint:debugger\string-command\string-wpset]{\sphinxcrossref{\DUrole{std,std-ref}{wpset}}}} -- define o espaço de watchpoint para o programa, dados e I/O
\item[] {\hyperref[debugger/watchpoint:debugger\string-command\string-wpclear]{\sphinxcrossref{\DUrole{std,std-ref}{wpclear}}}} -- limpa todos ou nenhum watchpoint caso nenhum \textless{}\emph{wpnum}\textgreater{} seja definido
\item[] {\hyperref[debugger/watchpoint:debugger\string-command\string-wpdisable]{\sphinxcrossref{\DUrole{std,std-ref}{wpdisable}}}} -- desabilita todos ou um determinado watchpoint caso nenhum \textless{}\emph{wpnum}\textgreater{} seja definido
\item[] {\hyperref[debugger/watchpoint:debugger\string-command\string-wpenable]{\sphinxcrossref{\DUrole{std,std-ref}{wpenable}}}} -- habilita todos ou um determinado watchpoint caso nenhum \textless{}\emph{wpnum}\textgreater{} seja definido
\item[] {\hyperref[debugger/watchpoint:debugger\string-command\string-wplist]{\sphinxcrossref{\DUrole{std,std-ref}{wplist}}}} -- lista todos os watchpoints
\end{DUlineblock}
\begin{quote}
\phantomsection\label{debugger/watchpoint:debugger-command-wpset}\end{quote}


\subsection{wpset}
\label{debugger/watchpoint:wpset}\label{debugger/watchpoint:debugger-command-wpset}
\begin{DUlineblock}{0em}
\item[]
\begin{DUlineblock}{\DUlineblockindent}
\item[] \textbf{wp{[}\{d\textbar{}i\}{]}{[}set{]}} \textless{}\emph{address}\textgreater{},\textless{}\emph{length}\textgreater{},\textless{}\emph{type}\textgreater{}{[},\textless{}\emph{condition}\textgreater{}{[},\textless{}\emph{action}\textgreater{}{]}{]}
\item[] 
\end{DUlineblock}
\item[] define um novo ``\emph{watchpoint}'' começando no endereço definido \textless{}\emph{address}\textgreater{} e estendendo para \textless{}\emph{length}\textgreater{}. O intervalo inclusivo do watchpoint é \textless{}\emph{address}\textgreater{} através de \textless{}\emph{address}\textgreater{} + \textless{}\emph{length}\textgreater{} - 1.
\item[] O comando ``\emph{wpset}'' define um watchpoint na memória do programa; o comando ``\emph{wpdset}'' define um watchpoint nos dados da memória; e o comando ``\emph{wpiset}'' define um watchpoint no I/O da memória.
\item[] O parâmetro \textless{}\emph{type}\textgreater{} especifica que tipo de acesso apanhar. Pode ser um dos três valores: `r' para um watchpoint de leitura `w' para um watchpoint de gravação e `rw' para um watchpoint de leitura/escrita.
\item[] 
\item[] O parâmetro de condição opcional \textless{}\emph{condition}\textgreater{} permite que você especifique uma expressão que será avaliada cada vez que o watchpoint for atingido. Se o resultado da expressão for verdadeiro (não-zero), o watchpoint irá interromper (halt) a execução; caso contrário, a execução continuará sem nenhuma notificação.
\item[] O parâmetro opcional de ação \textless{}\emph{action}\textgreater{} fornece um comando que é executado sempre que o watchpoint for atingido e a condição \textless{}\emph{condition}\textgreater{} for verdadeira. Observe que você pode precisar incorporar a ação entre chaves \textbf{\{ \}} para evitar que as vírgulas e os pontos e vírgulas sejam interpretados como se aplicassem ao próprio comando wpset.
\item[] Cada watchpoint que for definido é designado a um índice que pode ser usado em outros comandos watchpoint para usar este watchpoint como referência.
\item[] A fim de ajudar a expressão de condição \textless{}\emph{condition}\textgreater{}, duas variáveis estão disponíveis. Para todos os watchpoints, a variável ``wpaddr'' é definida para o endereço que realmente desencadeou o watchpoint. Para escrever watchpoints, a variável `wpdata' é definida para os dados que estão endo escritos.
\item[] 
\item[] Exemplos:
\item[] 
\item[]
\begin{DUlineblock}{\DUlineblockindent}
\item[] \sphinxcode{wp 1234,6,rw}
\item[] 
\end{DUlineblock}
\item[] Define um watchpoint que interromperá a execução sempre que uma leitura ou escrita acontecer no intervalo de endereço \textbf{1234-1239}, inclusive.
\item[] 
\item[]
\begin{DUlineblock}{\DUlineblockindent}
\item[] \sphinxcode{wp 23456,a,w,wpdata == 1}
\item[] 
\end{DUlineblock}
\item[] Define um watchpoint que interromperá a execução sempre que uma escrita no intervalo do endereço \textbf{23456-2345f} e os dados escritos forem iguais a \textbf{1}.
\item[] 
\item[]
\begin{DUlineblock}{\DUlineblockindent}
\item[] \sphinxcode{wp 3456,20,r,1,\{printf "Read @ \%08X\textbackslash{}\textbackslash{}n",wpaddr; g\}}
\item[] 
\end{DUlineblock}
\item[] Define um watchpoint que interromperá a execução sempre que uma leitura acontecer no intervalo de endereço \textbf{3456-3475}. Quando isso acontecer, imprime \textbf{Read @ \textless{}wpaddr\textgreater{}} e continua a execução.
\item[] 
\item[]
\begin{DUlineblock}{\DUlineblockindent}
\item[] \sphinxcode{temp0 = 0; wp 45678,1,w,wpdata==f0,\{temp0++; g\}}
\item[] 
\end{DUlineblock}
\item[] Define um watchpoint que interromperá a execução sempre que uma escrita acontecer no endereço \textbf{45678} e o valor que estiver sendo escrito for igual a \textbf{f0}. Quando isso acontecer, incrementa a variável \textbf{temp0} e resume a execução.
\item[] 
\item[] Back to {\hyperref[debugger/watchpoint:debugger\string-watchpoints\string-list]{\sphinxcrossref{\DUrole{std,std-ref}{Comandos watchpoint do depurador}}}}
\end{DUlineblock}
\begin{quote}
\phantomsection\label{debugger/watchpoint:debugger-command-wpclear}\end{quote}


\subsection{wpclear}
\label{debugger/watchpoint:debugger-command-wpclear}\label{debugger/watchpoint:wpclear}
\begin{DUlineblock}{0em}
\item[]
\begin{DUlineblock}{\DUlineblockindent}
\item[] \textbf{wpclear} {[}\textless{}\emph{wpnum}\textgreater{}{]}
\item[] 
\end{DUlineblock}
\item[] O comando ``\emph{wpclear}'' limpa o watchpoint. Caso o \textless{}\emph{wpnum}\textgreater{} seja definido, apenas o watchpoint solicitado é limpo, caso contrário, todos os watchpoints serão limpos.
\item[] 
\item[] Exemplos:
\item[] 
\item[]
\begin{DUlineblock}{\DUlineblockindent}
\item[] \sphinxcode{wpclear 3}
\item[] 
\end{DUlineblock}
\item[] Limpa o \textbf{indexador 3} do watchpoint.
\item[] 
\item[]
\begin{DUlineblock}{\DUlineblockindent}
\item[] \sphinxcode{wpclear}
\item[] 
\end{DUlineblock}
\item[] Limpa todos os watchpoints.
\item[] 
\item[] Back to {\hyperref[debugger/watchpoint:debugger\string-watchpoints\string-list]{\sphinxcrossref{\DUrole{std,std-ref}{Comandos watchpoint do depurador}}}}
\end{DUlineblock}
\begin{quote}
\phantomsection\label{debugger/watchpoint:debugger-command-wpdisable}\end{quote}


\subsection{wpdisable}
\label{debugger/watchpoint:wpdisable}\label{debugger/watchpoint:debugger-command-wpdisable}
\begin{DUlineblock}{0em}
\item[]
\begin{DUlineblock}{\DUlineblockindent}
\item[] \textbf{wpdisable} {[}\textless{}\emph{wpnum}\textgreater{}{]}
\item[] 
\end{DUlineblock}
\item[] O comando ``\emph{wpdisable}'' desabilita um watchpoint. Caso o \textless{}\emph{wpnum}\textgreater{} seja definido, apenas o watchpoint solicitado é desativado, caso contrário, todos os watchpoints serão desativados. Note que desabilitar um watchpoint ele não é apagado, o watchpoint fica registrado temporariamente como inativo.
\item[] 
\item[] Exemplos:
\item[] 
\item[]
\begin{DUlineblock}{\DUlineblockindent}
\item[] \sphinxcode{wpdisable 3}
\item[] 
\end{DUlineblock}
\item[] Desabilita o \textbf{indexador 3} do watchpoint.
\item[] 
\item[]
\begin{DUlineblock}{\DUlineblockindent}
\item[] \sphinxcode{wpdisable}
\item[] 
\end{DUlineblock}
\item[] Desabilita todos os watchpoints.
\item[] 
\item[] Back to {\hyperref[debugger/watchpoint:debugger\string-watchpoints\string-list]{\sphinxcrossref{\DUrole{std,std-ref}{Comandos watchpoint do depurador}}}}
\end{DUlineblock}
\begin{quote}
\phantomsection\label{debugger/watchpoint:debugger-command-wpenable}\end{quote}


\subsection{wpenable}
\label{debugger/watchpoint:wpenable}\label{debugger/watchpoint:debugger-command-wpenable}
\begin{DUlineblock}{0em}
\item[]
\begin{DUlineblock}{\DUlineblockindent}
\item[] \textbf{wpenable} {[}\textless{}\emph{wpnum}\textgreater{}{]}
\item[] 
\end{DUlineblock}
\item[] O comando ``\emph{wpenable}'' habilita um watchpoint. Caso o \textless{}\emph{wpnum}\textgreater{} seja definido, apenas o ``\emph{watchpoint}'' solicitado é ativado, caso contrário, todos os watchpoints serão ativados.
\item[] 
\item[] Exemplos:
\item[] 
\item[]
\begin{DUlineblock}{\DUlineblockindent}
\item[] \sphinxcode{wpenable 3}
\item[] 
\end{DUlineblock}
\item[] ativa todos os \textbf{index 3}.
\item[] 
\item[]
\begin{DUlineblock}{\DUlineblockindent}
\item[] wpenable
\item[] 
\end{DUlineblock}
\item[] ativa todos os watchpoints.
\item[] 
\item[] Back to {\hyperref[debugger/watchpoint:debugger\string-watchpoints\string-list]{\sphinxcrossref{\DUrole{std,std-ref}{Comandos watchpoint do depurador}}}}
\end{DUlineblock}
\begin{quote}
\phantomsection\label{debugger/watchpoint:debugger-command-wplist}\end{quote}


\subsection{wplist}
\label{debugger/watchpoint:debugger-command-wplist}\label{debugger/watchpoint:wplist}
\begin{DUlineblock}{0em}
\item[]
\begin{DUlineblock}{\DUlineblockindent}
\item[] \textbf{wplist}
\item[] 
\item[] O comando ``\emph{wplist}'' lista todos os watchpoints atuais, junto com o seu indexador e quaisquer condições anexadas a eles.
\item[] 
\end{DUlineblock}
\item[] Back to {\hyperref[debugger/watchpoint:debugger\string-watchpoints\string-list]{\sphinxcrossref{\DUrole{std,std-ref}{Comandos watchpoint do depurador}}}}
\end{DUlineblock}


\section{Comandos registerpoints do depurador}
\label{debugger/registerpoints:comandos-registerpoints-do-depurador}\label{debugger/registerpoints::doc}\label{debugger/registerpoints:debugger-registerpoints-list}
Na interface de depuração do MAME você pode digitar \textbf{help \textless{}command\textgreater{}}
para uma melhor descrição de cada comando.

\begin{DUlineblock}{0em}
\item[] {\hyperref[debugger/registerpoints:debugger\string-command\string-rpset]{\sphinxcrossref{\DUrole{std,std-ref}{rpset}}}} -- define um registerpoint para disparar com uma condição \textless{}\emph{condition}\textgreater{}
\item[] {\hyperref[debugger/registerpoints:debugger\string-command\string-rpclear]{\sphinxcrossref{\DUrole{std,std-ref}{rpclear}}}} -- limpa todos ou um determinado registerpoint se nenhum \textless{}\emph{rpnum}\textgreater{} for especificado
\item[] {\hyperref[debugger/registerpoints:debugger\string-command\string-rpdisable]{\sphinxcrossref{\DUrole{std,std-ref}{rpdisable}}}} -- desabilita todos ou um determinado registerpoint se nenhum \textless{}\emph{rpnum}\textgreater{} for especificado
\item[] {\hyperref[debugger/registerpoints:debugger\string-command\string-rpenable]{\sphinxcrossref{\DUrole{std,std-ref}{rpenable}}}} -- habilita todo ou um determinado registerpoint se nenhum \textless{}\emph{rpnum}\textgreater{} for especificado
\item[] {\hyperref[debugger/registerpoints:debugger\string-command\string-rplist]{\sphinxcrossref{\DUrole{std,std-ref}{rplist}}}} -- lista todos os registerpoints
\end{DUlineblock}
\begin{quote}
\phantomsection\label{debugger/registerpoints:debugger-command-rpset}\end{quote}


\subsection{rpset}
\label{debugger/registerpoints:debugger-command-rpset}\label{debugger/registerpoints:rpset}
\begin{DUlineblock}{0em}
\item[]
\begin{DUlineblock}{\DUlineblockindent}
\item[] \textbf{rp{[}set{]}} \{\textless{}\emph{condition}\textgreater{}\}{[},\textless{}\emph{action}\textgreater{}{]}{]}
\item[] 
\end{DUlineblock}
\item[] Define um novo registerpoint que será disparado quando a condição \textless{}\emph{condition}\textgreater{} for atingida. A condição deve ser definida entre chaves para evitar que a condição seja avaliada como uma atribuição.
\item[] O parâmetro opcional de ação \textless{}\emph{action}\textgreater{} fornece um comando que é executado sempre que o registerpoint for atingido. Observe que você pode precisar incorporar a ação entre chaves \textbf{\{ \}} para evitar que as vírgulas e os pontos e vírgulas sejam interpretados como se aplicassem ao próprio comando ``\emph{rpset}''.
\item[] Cada registerpoint que for definido é designado a um índice que pode ser usado em outros comandos registerpoint para referenciar este registerpoint.
\item[] 
\item[] Exemplos:
\item[] 
\item[]
\begin{DUlineblock}{\DUlineblockindent}
\item[] \sphinxcode{rp \{PC==0150\}}
\item[] 
\end{DUlineblock}
\item[] Define um registerpoint que interromperá a execução sempre que o registrador PC for igual a \textbf{0x150}.
\item[] 
\item[]
\begin{DUlineblock}{\DUlineblockindent}
\item[] \sphinxcode{temp0=0; rp \{PC==0150\},\{temp0++; g\}}
\item[] 
\end{DUlineblock}
\item[] Define um registerpoint que irá incrementar a variável \textbf{temp0} sempre que o registrador PC for igual a \textbf{0x150}.
\item[] Define um registerpoint que interromperá a execução sempre que a variável \textbf{temp0} for igual a \textbf{5}.
\item[] 
\item[] Back to {\hyperref[debugger/registerpoints:debugger\string-registerpoints\string-list]{\sphinxcrossref{\DUrole{std,std-ref}{Comandos registerpoints do depurador}}}}
\end{DUlineblock}
\begin{quote}
\phantomsection\label{debugger/registerpoints:debugger-command-rpclear}\end{quote}


\subsection{rpclear}
\label{debugger/registerpoints:debugger-command-rpclear}\label{debugger/registerpoints:rpclear}
\begin{DUlineblock}{0em}
\item[]
\begin{DUlineblock}{\DUlineblockindent}
\item[] \textbf{rpclear} {[}\textless{}\emph{rpnum}\textgreater{}{]}
\item[] 
\end{DUlineblock}
\item[] O comando ``\emph{rpclear}'' limpa um registerpoint. Caso o \textless{}\emph{rpnum}\textgreater{} seja definido, apenas o registerpoint solicitado é limpo, caso contrário, todos os registerpoints serão limpos.
\item[] 
\item[] Exemplos:
\item[] 
\item[]
\begin{DUlineblock}{\DUlineblockindent}
\item[] \sphinxcode{rpclear 3}
\item[] 
\end{DUlineblock}
\item[] Limpa o \textbf{indexador 3} do registerpoint.
\item[] 
\item[]
\begin{DUlineblock}{\DUlineblockindent}
\item[] \sphinxcode{rpclear}
\item[] 
\end{DUlineblock}
\item[] Limpa todos os registerpoints.
\item[] 
\item[] Back to {\hyperref[debugger/registerpoints:debugger\string-registerpoints\string-list]{\sphinxcrossref{\DUrole{std,std-ref}{Comandos registerpoints do depurador}}}}
\end{DUlineblock}
\begin{quote}
\phantomsection\label{debugger/registerpoints:debugger-command-rpdisable}\end{quote}


\subsection{rpdisable}
\label{debugger/registerpoints:debugger-command-rpdisable}\label{debugger/registerpoints:rpdisable}
\begin{DUlineblock}{0em}
\item[]
\begin{DUlineblock}{\DUlineblockindent}
\item[] \textbf{rpdisable} {[}\textless{}\emph{rpnum}\textgreater{}{]}
\item[] 
\end{DUlineblock}
\item[] O comando ``\emph{rpdisable}'' desativa um registerpoint. Caso o \textless{}\emph{rpnum}\textgreater{} seja especificado, somente o registerpoint solicitado é desabilitado, caso contrário, todos os registerpoint são desativados. Note que desabilitar um registerpoint ele não é apagado, o registerpoint fica registrado temporariamente como inativo.
\item[] 
\item[] Exemplos:
\item[] 
\item[]
\begin{DUlineblock}{\DUlineblockindent}
\item[] \sphinxcode{rpdisable 3}
\item[] 
\end{DUlineblock}
\item[] Desabilita o \textbf{indexador 3} do registerpoint.
\item[] 
\item[]
\begin{DUlineblock}{\DUlineblockindent}
\item[] \sphinxcode{rpdisable}
\item[] 
\end{DUlineblock}
\item[] Desabilita todo os registerpoints.
\item[] 
\item[] Back to {\hyperref[debugger/registerpoints:debugger\string-registerpoints\string-list]{\sphinxcrossref{\DUrole{std,std-ref}{Comandos registerpoints do depurador}}}}
\end{DUlineblock}
\begin{quote}
\phantomsection\label{debugger/registerpoints:debugger-command-rpenable}\end{quote}


\subsection{rpenable}
\label{debugger/registerpoints:debugger-command-rpenable}\label{debugger/registerpoints:rpenable}
\begin{DUlineblock}{0em}
\item[]
\begin{DUlineblock}{\DUlineblockindent}
\item[] \textbf{rpenable} {[}\textless{}\emph{rpnum}\textgreater{}{]}
\item[] 
\end{DUlineblock}
\item[] O comando ``\emph{rpenable}'' habilita um registerpoint. Caso o \textless{}\emph{rpnum}\textgreater{} seja especificado, somente o registerpoint solicitado é ativado, caso contrário, todos os registerpoint serão habilitados.
\item[] 
\item[] Exemplos:
\item[] 
\item[]
\begin{DUlineblock}{\DUlineblockindent}
\item[] \sphinxcode{rpenable 3}
\item[] 
\end{DUlineblock}
\item[] Habilita o \textbf{indexador 3} do registerpoint.
\item[] 
\item[]
\begin{DUlineblock}{\DUlineblockindent}
\item[] \sphinxcode{rpenable}
\item[] 
\end{DUlineblock}
\item[] Habilita todos os registerpoints.
\item[] 
\item[] Back to {\hyperref[debugger/registerpoints:debugger\string-registerpoints\string-list]{\sphinxcrossref{\DUrole{std,std-ref}{Comandos registerpoints do depurador}}}}
\end{DUlineblock}
\begin{quote}
\phantomsection\label{debugger/registerpoints:debugger-command-rplist}\end{quote}


\subsection{rplist}
\label{debugger/registerpoints:debugger-command-rplist}\label{debugger/registerpoints:rplist}
\begin{DUlineblock}{0em}
\item[]
\begin{DUlineblock}{\DUlineblockindent}
\item[] \textbf{rplist}
\item[] 
\end{DUlineblock}
\item[] O comando ``\emph{rplist}'' lista todos os registerpoints atuais, juntamente com o seu índice e quaisquer ações anexadas à elas.
\item[] 
\item[] Back to {\hyperref[debugger/registerpoints:debugger\string-registerpoints\string-list]{\sphinxcrossref{\DUrole{std,std-ref}{Comandos registerpoints do depurador}}}}
\end{DUlineblock}


\section{Comandos de anotação de código do depurador}
\label{debugger/annotation::doc}\label{debugger/annotation:debugger-annotation-list}\label{debugger/annotation:comandos-de-anotacao-de-codigo-do-depurador}
Na interface de depuração do MAME você pode digitar \textbf{help \textless{}command\textgreater{}}
para uma melhor descrição de cada comando.

\begin{DUlineblock}{0em}
\item[] {\hyperref[debugger/annotation:debugger\string-command\string-comadd]{\sphinxcrossref{\DUrole{std,std-ref}{comadd}}}} -- Inclui um comentário ao código desmontado em um determinado endereço
\item[] {\hyperref[debugger/annotation:debugger\string-command\string-comdelete]{\sphinxcrossref{\DUrole{std,std-ref}{comdelete}}}} -- remove o comentário de um determinado endereço
\item[] {\hyperref[debugger/annotation:debugger\string-command\string-comsave]{\sphinxcrossref{\DUrole{std,std-ref}{comsave}}}} -- salva o comentário atual em um arquivo
\item[] {\hyperref[debugger/annotation:debugger\string-command\string-comlist]{\sphinxcrossref{\DUrole{std,std-ref}{comlist}}}} -- imprime os comentários disponíveis de um arquivo
\item[] {\hyperref[debugger/annotation:debugger\string-command\string-commit]{\sphinxcrossref{\DUrole{std,std-ref}{commit}}}} -- aplica um comadd e depois comsave
\end{DUlineblock}
\begin{quote}
\phantomsection\label{debugger/annotation:debugger-command-comadd}\end{quote}


\subsection{comadd}
\label{debugger/annotation:debugger-command-comadd}\label{debugger/annotation:comadd}
\begin{DUlineblock}{0em}
\item[]
\begin{DUlineblock}{\DUlineblockindent}
\item[] \textbf{comadd{[}//{]}} \textless{}\emph{address}\textgreater{},\textless{}\emph{comment}\textgreater{}
\item[] 
\end{DUlineblock}
\item[] Inclui uma string \textless{}\emph{comment}\textgreater{} ao código desmontado em \textless{}\emph{address}\textgreater{}. O atalho para este comando é só `//'
\item[] 
\item[] Exemplos:
\item[] 
\item[]
\begin{DUlineblock}{\DUlineblockindent}
\item[] \sphinxcode{comadd 0, hello world.}
\item[] 
\end{DUlineblock}
\item[] Inclui o comentário `hello world.' ao código no endereço \textbf{0x0}
\item[] 
\item[]
\begin{DUlineblock}{\DUlineblockindent}
\item[] \sphinxcode{// 10, opcode não documentado!}
\item[] 
\end{DUlineblock}
\item[] Inclui o comentário `opcode não documentado!' ao código no endereço \textbf{0x10}
\end{DUlineblock}
\begin{quote}
\phantomsection\label{debugger/annotation:debugger-command-comdelete}\end{quote}


\subsection{comdelete}
\label{debugger/annotation:debugger-command-comdelete}\label{debugger/annotation:comdelete}
\begin{DUlineblock}{0em}
\item[]
\begin{DUlineblock}{\DUlineblockindent}
\item[] \textbf{comdelete}
\item[] 
\end{DUlineblock}
\item[] Apaga o comentário em um offset determinado da memória. O comentário que é excluído está no banco de memória ativo no momento.
\item[] 
\item[] Exemplos:
\item[] 
\item[]
\begin{DUlineblock}{\DUlineblockindent}
\item[] \sphinxcode{comdelete 10}
\item[] 
\end{DUlineblock}
\item[] Apaga o comentário do código no endereço \textbf{0x10} (usando as configurações do banco atual de memória)
\end{DUlineblock}
\begin{quote}
\phantomsection\label{debugger/annotation:debugger-command-comsave}\end{quote}


\subsection{comsave}
\label{debugger/annotation:debugger-command-comsave}\label{debugger/annotation:comsave}
\begin{DUlineblock}{0em}
\item[]
\begin{DUlineblock}{\DUlineblockindent}
\item[] \textbf{comsave}
\item[] 
\end{DUlineblock}
\item[] Salva as observações de trabalho no arquivo de comentário XML do driver.
\item[] 
\item[] Exemplos:
\item[] 
\item[]
\begin{DUlineblock}{\DUlineblockindent}
\item[] \sphinxcode{comsave}
\item[] 
\end{DUlineblock}
\item[] Salva os comentários no arquivo de comentários do driver
\end{DUlineblock}
\begin{quote}
\phantomsection\label{debugger/annotation:debugger-command-comlist}\end{quote}


\subsection{comlist}
\label{debugger/annotation:comlist}\label{debugger/annotation:debugger-command-comlist}
\begin{DUlineblock}{0em}
\item[]
\begin{DUlineblock}{\DUlineblockindent}
\item[] \textbf{comlist}
\item[] 
\end{DUlineblock}
\item[] Imprime o comentário atual disponível em formato legível para humanos na janela de saída do depurador.
\item[] 
\item[] Exemplos:
\item[] 
\item[]
\begin{DUlineblock}{\DUlineblockindent}
\item[] \sphinxcode{comlist}
\item[] 
\end{DUlineblock}
\item[] Mostra os comentários disponíveis atualmente.
\end{DUlineblock}
\begin{quote}
\phantomsection\label{debugger/annotation:debugger-command-commit}\end{quote}


\subsection{commit}
\label{debugger/annotation:commit}\label{debugger/annotation:debugger-command-commit}
\begin{DUlineblock}{0em}
\item[]
\begin{DUlineblock}{\DUlineblockindent}
\item[] \textbf{commit{[}/*{]}} \textless{}\emph{address}\textgreater{},\textless{}\emph{comment}\textgreater{}
\item[] 
\end{DUlineblock}
\item[] Inclui uma string \textless{}\emph{comment}\textgreater{} ao código desmontado no \textless{}\emph{address}\textgreater{} e salva num arquivo. Basicamente é o mesmo que comadd + comsave em uma única linha.
\item[] O atalho para este comando é \sphinxcode{\textbackslash{}'\textbackslash{}/\textbackslash{}*\textbackslash{}'}
\item[] 
\item[] Exemplos:
\item[] 
\item[]
\begin{DUlineblock}{\DUlineblockindent}
\item[] \sphinxcode{commit 0, hello world.}
\item[] 
\end{DUlineblock}
\item[] Inclui o comentário `hello world.' ao código no endereço \textbf{0x0}
\item[] 
\item[]
\begin{DUlineblock}{\DUlineblockindent}
\item[] \sphinxcode{// 10, undocumented opcode!}
\item[] 
\end{DUlineblock}
\item[] Inclui o comentário `opcode não documentado!' ao código no endereço \textbf{0x10}
\end{DUlineblock}


\section{Comandos para o depurador de trapaça}
\label{debugger/cheats:comandos-para-o-depurador-de-trapaca}\label{debugger/cheats::doc}\label{debugger/cheats:debugger-cheats-list}
Na interface de depuração do MAME você pode digitar \textbf{help \textless{}command\textgreater{}}
para uma melhor descrição de cada comando.

\begin{DUlineblock}{0em}
\item[] {\hyperref[debugger/cheats:debugger\string-command\string-cheatinit]{\sphinxcrossref{\DUrole{std,std-ref}{cheatinit}}}} -- inicializa uma pesquisa de trapaça na região da memória selecionada
\item[] {\hyperref[debugger/cheats:debugger\string-command\string-cheatrange]{\sphinxcrossref{\DUrole{std,std-ref}{cheatrange}}}} -- adiciona uma região da memória a pesquisa de trapaça selecionada
\item[] {\hyperref[debugger/cheats:debugger\string-command\string-cheatnext]{\sphinxcrossref{\DUrole{std,std-ref}{cheatnext}}}} -- continua a pesquisa de uma trapaça comparando-a com o último valor
\item[] {\hyperref[debugger/cheats:debugger\string-command\string-cheatnextf]{\sphinxcrossref{\DUrole{std,std-ref}{cheatnextf}}}} -- continua a pesquisa de uma trapaça comparando-a com o primeiro valor
\item[] {\hyperref[debugger/cheats:debugger\string-command\string-cheatlist]{\sphinxcrossref{\DUrole{std,std-ref}{cheatlist}}}} -- mostra uma lista de pesquisa de trapaças que tiveram correspondências ou salve-as em um nome de arquivo \textless{}filename\textgreater{}
\item[] {\hyperref[debugger/cheats:debugger\string-command\string-cheatundo]{\sphinxcrossref{\DUrole{std,std-ref}{cheatundo}}}} -- desfaz a última pesquisa de trapaça (estado apenas)
\end{DUlineblock}
\begin{quote}
\phantomsection\label{debugger/cheats:debugger-command-cheatinit}\end{quote}


\subsection{cheatinit}
\label{debugger/cheats:cheatinit}\label{debugger/cheats:debugger-command-cheatinit}
\begin{DUlineblock}{0em}
\item[]
\begin{DUlineblock}{\DUlineblockindent}
\item[] \textbf{cheatinit} {[}\textless{}\emph{sign}\textgreater{}\textless{}\emph{width}\textgreater{}\textless{}\emph{swap}\textgreater{},{[}\textless{}\emph{address}\textgreater{},\textless{}\emph{length}\textgreater{}{[},\textless{}\emph{cpu}\textgreater{}{]}{]}{]}
\item[] 
\end{DUlineblock}
\item[] O comando \textbf{cheatinit} inicializa uma pesquisa da trapaça na região da memória selecionada.
\item[] 
\item[] Se nenhum parâmetro for definido para a trapaça a pesquisa é inicializada em toda a memória intercambiável do CPU principal.
\item[] 
\item[] \textbf{\textless{}sign\textgreater{}} pode ser \textbf{s(signed)} ou \textbf{u(unsigned)}
\item[] \textbf{\textless{}width\textgreater{}} pode ser \textbf{b(8 bit)}, \textbf{w(16 bit)}, \textbf{d(32 bit)} ou \textbf{q(64 bit)}
\item[] \textbf{\textless{}swap\textgreater{}} acrescenta \textbf{s} para a pesquisa de troca
\item[] 
\item[] Exemplos:
\item[] 
\item[]
\begin{DUlineblock}{\DUlineblockindent}
\item[] \sphinxcode{cheatinit ub,0x1000,0x10}
\item[] 
\end{DUlineblock}
\item[] Inicializa a pesquisa de trapaça de \textbf{0x1000} para \textbf{0x1010} do primeiro CPU.
\item[] 
\item[]
\begin{DUlineblock}{\DUlineblockindent}
\item[] \sphinxcode{cheatinit sw,0x2000,0x1000,1}
\item[] 
\end{DUlineblock}
\item[] Inicializa a pesquisa de trapaça com uma largura de \textbf{2 bytes} em modo \textbf{signed} de \textbf{0x2000} para \textbf{0x3000} do segundo CPU.
\item[] 
\item[]
\begin{DUlineblock}{\DUlineblockindent}
\item[] \sphinxcode{cheatinit uds,0x0000,0x1000}
\item[] 
\end{DUlineblock}
\item[] Inicializa a pesquisa de trapaça com uma largura de \textbf{4 bytes} trocados de \textbf{0x0000} para \textbf{0x1000}.
\item[] 
\item[] Back to {\hyperref[debugger/cheats:debugger\string-cheats\string-list]{\sphinxcrossref{\DUrole{std,std-ref}{Comandos para o depurador de trapaça}}}}
\end{DUlineblock}
\begin{quote}
\phantomsection\label{debugger/cheats:debugger-command-cheatrange}\end{quote}


\subsection{cheatrange}
\label{debugger/cheats:debugger-command-cheatrange}\label{debugger/cheats:cheatrange}
\begin{DUlineblock}{0em}
\item[]
\begin{DUlineblock}{\DUlineblockindent}
\item[] \textbf{cheatrange} \textless{}\emph{address}\textgreater{},\textless{}\emph{length}\textgreater{}
\item[] 
\end{DUlineblock}
\item[] O comando ``\emph{cheatrange}'' reserva uma regiçao da memóra para que seja feita a pesquisa da trapaça.
\item[] 
\item[] Antes de usar o ``\emph{cheatrange}'' é necessário inicializar a pesquisa da trapaça com ``\emph{cheatinit}''.
\item[] 
\item[] Exemplos:
\item[] 
\item[]
\begin{DUlineblock}{\DUlineblockindent}
\item[] \sphinxcode{cheatrange 0x1000,0x10}
\item[] 
\end{DUlineblock}
\item[] Adiciona os bytes de \textbf{0x1000} para \textbf{0x1010} na pesquisa da trapaça.
\item[] 
\item[] Back to {\hyperref[debugger/cheats:debugger\string-cheats\string-list]{\sphinxcrossref{\DUrole{std,std-ref}{Comandos para o depurador de trapaça}}}}
\end{DUlineblock}
\begin{quote}
\phantomsection\label{debugger/cheats:debugger-command-cheatnext}\end{quote}


\subsection{cheatnext}
\label{debugger/cheats:debugger-command-cheatnext}\label{debugger/cheats:cheatnext}
\begin{DUlineblock}{0em}
\item[]
\begin{DUlineblock}{\DUlineblockindent}
\item[] \textbf{cheatnext} \textless{}\emph{condition}\textgreater{}{[},\textless{}\emph{comparisonvalue}\textgreater{}{]}
\item[] 
\end{DUlineblock}
\item[] O comando ``\emph{cheatnext}'' faz comparações com as últimas pesquisas coincidentes.
\item[] 
\item[] Condição possível \textless{}\emph{condition}\textgreater{}:
\item[] 
\item[]
\begin{DUlineblock}{\DUlineblockindent}
\item[] \textbf{all} (\textbf{todas})
\item[] 
\end{DUlineblock}
\item[] Nenhum valor de comparação \textless{}\emph{comparisonvalue}\textgreater{} é necessário.
\item[] 
\item[] Use para atualizar o último valor sem mudar o os valores já encontrados.
\item[] 
\item[]
\begin{DUlineblock}{\DUlineblockindent}
\item[] \textbf{equal {[}eq{]}}
\item[] 
\end{DUlineblock}
\item[] Sem o valor de comparação \textless{}\emph{comparisonvalue}\textgreater{} pesquise por todos os bytes que são iguais aos da última pesquisa.
\item[] Com o valor de comparação \textless{}\emph{comparisonvalue}\textgreater{} onde todos os bytes sejam iguais com o valor de comparação \textless{}\emph{comparisonvalue}\textgreater{}.
\item[] 
\item[]
\begin{DUlineblock}{\DUlineblockindent}
\item[] \textbf{notequal {[}ne{]}}
\item[] 
\end{DUlineblock}
\item[] Com o valor de comparação \textless{}\emph{comparisonvalue}\textgreater{} pesquise por todos os bytes que não sejam iguais a última pesquisa.
\item[] Com o valor de comparação \textless{}\emph{comparisonvalue}\textgreater{} pesquise por todos os bytes que não são iguais ao valor de comparação \textless{}\emph{comparisonvalue}\textgreater{}.
\item[] 
\item[]
\begin{DUlineblock}{\DUlineblockindent}
\item[] \textbf{decrease {[}de, +{]}}
\item[] 
\end{DUlineblock}
\item[] Sem o valor de comparação \textless{}\emph{comparisonvalue}\textgreater{} pesquise por todos os bytes que tiveram seu valor diminuído desde a última pesquisa.
\item[] Com o valor de comparação \textless{}\emph{comparisonvalue}\textgreater{} pesquise por todos os bytes que tenham diminuído em comparação com o valor de comparação \textless{}\emph{comparisonvalue}\textgreater{} desde a última pesquisa.
\item[] 
\item[]
\begin{DUlineblock}{\DUlineblockindent}
\item[] \textbf{increase {[}in, -{]}}
\item[] 
\end{DUlineblock}
\item[] Sem o valor de comparação \textless{}\emph{comparisonvalue}\textgreater{} pesquise por todos os bytes que tenham aumentando desde a última pesquisa.
\item[] Com o valor de comparação \textless{}\emph{comparisonvalue}\textgreater{} pesquise por todos os bytes que tenham aumentado em comparação com o valor de comparação \textless{}\emph{comparisonvalue}\textgreater{} desde a última pesquisa.
\item[] 
\item[]
\begin{DUlineblock}{\DUlineblockindent}
\item[] \textbf{decreaseorequal {[}deeq{]}}
\item[] 
\end{DUlineblock}
\item[] Nenhum valor de comparação \textless{}\emph{comparisonvalue}\textgreater{} é necessário.
\item[] 
\item[] Pesquise que todos os bytes que tenham diminuído ou tenham o mesmo valor desde a última pesquisa.
\item[] 
\item[]
\begin{DUlineblock}{\DUlineblockindent}
\item[] \textbf{increaseorequal {[}ineq{]}}
\item[] 
\end{DUlineblock}
\item[] Nenhum valor de comparação \textless{}\emph{comparisonvalue}\textgreater{} é necessário.
\item[] 
\item[] Pesquise que todos os bytes que tenham diminuído ou tenham o mesmo valor desde a última pesquisa.
\item[] 
\item[]
\begin{DUlineblock}{\DUlineblockindent}
\item[] \textbf{smallerof {[}lt{]}}
\item[] 
\end{DUlineblock}
\item[] Sem o valor de comparação \textless{}\emph{comparisonvalue}\textgreater{} essa condição é inválida
\item[] Com o valor de comparação \textless{}\emph{comparisonvalue}\textgreater{} pesquise por todos os bytes que são menores que o valor de comparação \textless{}\emph{comparisonvalue}\textgreater{}.
\item[] 
\item[]
\begin{DUlineblock}{\DUlineblockindent}
\item[] \textbf{greaterof {[}gt{]}}
\item[] 
\end{DUlineblock}
\item[] Sem o valor de comparação \textless{}\emph{comparisonvalue}\textgreater{} essa condição é inválida
\item[] Com o valor de comparação \textless{}\emph{comparisonvalue}\textgreater{} pesquise por todos os bytes que são maiores que o valor de comparação \textless{}\emph{comparisonvalue}\textgreater{}.
\item[] 
\item[]
\begin{DUlineblock}{\DUlineblockindent}
\item[] \textbf{changedby {[}ch, \textasciitilde{}{]}}
\item[] 
\end{DUlineblock}
\item[] Sem o valor de comparação \textless{}\emph{comparisonvalue}\textgreater{} essa condição é inválida
\item[] Com o valor de comparação \textless{}\emph{comparisonvalue}\textgreater{} pesquise por todos os bytes que tenham mudado através do valor de comparação \textless{}\emph{comparisonvalue}\textgreater{} desde a última pesquisa
\item[] 
\item[] 
\item[] Exemplos:
\item[] 
\item[]
\begin{DUlineblock}{\DUlineblockindent}
\item[] \sphinxcode{cheatnext increase}
\item[] 
\end{DUlineblock}
\item[] Pesquise por todos os bytes que tenham aumentado desde a última pesquisa
\item[] 
\item[]
\begin{DUlineblock}{\DUlineblockindent}
\item[] \sphinxcode{cheatnext decrease, 1}
\item[] 
\end{DUlineblock}
\item[] Pesquise por todos os bytes que tenham diminuído por 1 desde a última pesquisa
\item[] 
\item[] Back to {\hyperref[debugger/cheats:debugger\string-cheats\string-list]{\sphinxcrossref{\DUrole{std,std-ref}{Comandos para o depurador de trapaça}}}}
\end{DUlineblock}
\begin{quote}
\phantomsection\label{debugger/cheats:debugger-command-cheatnextf}\end{quote}


\subsection{cheatnextf}
\label{debugger/cheats:cheatnextf}\label{debugger/cheats:debugger-command-cheatnextf}
\begin{DUlineblock}{0em}
\item[]
\begin{DUlineblock}{\DUlineblockindent}
\item[] \textbf{cheatnextf} \textless{}\emph{condition}\textgreater{}{[},\textless{}\emph{comparisonvalue}\textgreater{}{]}
\item[] 
\end{DUlineblock}
\item[] O comando ``\emph{cheatnextf}'' fará comparações com a pesquisa inicial.
\item[] 
\item[] Condição possível \textless{}\emph{condition}\textgreater{}:
\item[] 
\item[]
\begin{DUlineblock}{\DUlineblockindent}
\item[] \textbf{all} (\textbf{todas})
\item[] 
\end{DUlineblock}
\item[] Nenhum valor de comparação \textless{}\emph{comparisonvalue}\textgreater{} é necessário.
\item[] 
\item[] Use para atualizar o último valor sem mudar o os valores já encontrados.
\item[] 
\item[]
\begin{DUlineblock}{\DUlineblockindent}
\item[] \textbf{equal {[}eq{]}}
\item[] 
\end{DUlineblock}
\item[] Sem o valor de comparação \textless{}\emph{comparisonvalue}\textgreater{} pesquise por todos os bytes que são iguais ao valor pesquisa inicial
\item[] Com o valor de comparação \textless{}\emph{comparisonvalue}\textgreater{} onde todos os bytes sejam iguais com o valor de comparação \textless{}\emph{comparisonvalue}\textgreater{}.
\item[] 
\item[]
\begin{DUlineblock}{\DUlineblockindent}
\item[] \textbf{notequal {[}ne{]}}
\item[] 
\end{DUlineblock}
\item[] Sem o valor de comparação \textless{}\emph{comparisonvalue}\textgreater{} pesquise por todos os bytes que não são iguais ao valor pesquisa inicial
\item[] Com o valor de comparação \textless{}\emph{comparisonvalue}\textgreater{} pesquise por todos os bytes que não são iguais ao valor de comparação \textless{}\emph{comparisonvalue}\textgreater{}.
\item[] 
\item[]
\begin{DUlineblock}{\DUlineblockindent}
\item[] \textbf{decrease {[}de, +{]}}
\item[] 
\end{DUlineblock}
\item[] Sem o valor de comparação \textless{}\emph{comparisonvalue}\textgreater{} Pesquise por todos os bytes que tenham diminuído desde o último valor pesquisa inicial
\item[] Com o valor de comparação \textless{}\emph{comparisonvalue}\textgreater{} Pesquise por todos os bytes que tenham diminuído pelo valor de comparação \textless{}\emph{comparisonvalue}\textgreater{} desde o último valor pesquisa inicial.
\item[] 
\item[]
\begin{DUlineblock}{\DUlineblockindent}
\item[] \textbf{increase {[}in, -{]}}
\item[] 
\end{DUlineblock}
\item[] Sem o valor de comparação \textless{}\emph{comparisonvalue}\textgreater{} Pesquise por todos os bytes que tenham diminuído desde a pesquisa inicial.
\item[] 
\item[] Com o valor de comparação \textless{}\emph{comparisonvalue}\textgreater{} Pesquise por todos os bytes que tenham aumentado pelo valor de comparação \textless{}\emph{comparisonvalue}\textgreater{} desde a pesquisa inicial.
\item[] 
\item[]
\begin{DUlineblock}{\DUlineblockindent}
\item[] \textbf{decreaseorequal {[}deeq{]}}
\item[] 
\end{DUlineblock}
\item[] Nenhum valor de comparação \textless{}\emph{comparisonvalue}\textgreater{} é necessário.
\item[] 
\item[] Pesquise por todos os bytes que tenham diminuído ou tenha o mesmo valor da pesquisa inicial.
\item[] 
\item[]
\begin{DUlineblock}{\DUlineblockindent}
\item[] \textbf{increaseorequal {[}ineq{]}}
\item[] 
\end{DUlineblock}
\item[] Nenhum valor de comparação \textless{}\emph{comparisonvalue}\textgreater{} é necessário.
\item[] 
\item[] Pesquise por todos os bytes que tenham diminuído ou tenha o mesmo valor da pesquisa inicial.
\item[] 
\item[]
\begin{DUlineblock}{\DUlineblockindent}
\item[] \textbf{smallerof {[}lt{]}}
\item[] 
\end{DUlineblock}
\item[] Sem o valor de comparação \textless{}\emph{comparisonvalue}\textgreater{} essa condição é inválida.
\item[] Com o valor de comparação \textless{}\emph{comparisonvalue}\textgreater{} pesquise por todos os bytes que são menores que o valor de comparação \textless{}\emph{comparisonvalue}\textgreater{}.
\item[] 
\item[]
\begin{DUlineblock}{\DUlineblockindent}
\item[] \textbf{greaterof {[}gt{]}}
\item[] 
\end{DUlineblock}
\item[] Sem o valor de comparação \textless{}\emph{comparisonvalue}\textgreater{} essa condição é inválida.
\item[] Com o valor de comparação \textless{}\emph{comparisonvalue}\textgreater{} pesquise por todos os bytes que são maiores que o valor de comparação \textless{}\emph{comparisonvalue}\textgreater{}.
\item[] 
\item[]
\begin{DUlineblock}{\DUlineblockindent}
\item[] \textbf{changedby {[}ch, \textasciitilde{}{]}}
\item[] 
\end{DUlineblock}
\item[] Sem o valor de comparação \textless{}\emph{comparisonvalue}\textgreater{} essa condição é inválida
\item[] Com o valor de comparação \textless{}\emph{comparisonvalue}\textgreater{} Pesquise por todos os bytes que tenham mudado pelo valor de comparação \textless{}\emph{comparisonvalue}\textgreater{} desde a pesquisa inicial.
\item[] 
\item[] 
\item[] Exemplos:
\item[] 
\item[]
\begin{DUlineblock}{\DUlineblockindent}
\item[] \sphinxcode{cheatnextf increase}
\item[] 
\end{DUlineblock}
\item[] Pesquise por todos os bytes que tenham aumentado desde a pesquisa inicial.
\item[] 
\item[]
\begin{DUlineblock}{\DUlineblockindent}
\item[] \sphinxcode{cheatnextf decrease, 1}
\item[] 
\end{DUlineblock}
\item[] Pesquise por todos os bytes que tenham diminuído 1 byte desde a pesquisa inicial.
\item[] 
\item[] Back to {\hyperref[debugger/cheats:debugger\string-cheats\string-list]{\sphinxcrossref{\DUrole{std,std-ref}{Comandos para o depurador de trapaça}}}}
\end{DUlineblock}
\begin{quote}
\phantomsection\label{debugger/cheats:debugger-command-cheatlist}\end{quote}


\subsection{cheatlist}
\label{debugger/cheats:debugger-command-cheatlist}\label{debugger/cheats:cheatlist}
\begin{DUlineblock}{0em}
\item[]
\begin{DUlineblock}{\DUlineblockindent}
\item[] \textbf{cheatlist} {[}\textless{}\emph{filename}\textgreater{}{]}
\item[] 
\end{DUlineblock}
\item[] Sem o nome de arquivo \textless{}\emph{filename}\textgreater{} mostre a lista de coincidentes no console de depuração.
\item[] Com o nome de arquivo \textless{}\emph{filename}\textgreater{} salve a lista de coincidentes em formato XML básico para o nome do arquivo \textless{}\emph{filename}\textgreater{}.
\item[] 
\item[] Exemplos:
\item[] 
\item[]
\begin{DUlineblock}{\DUlineblockindent}
\item[] \sphinxcode{cheatlist}
\item[] 
\end{DUlineblock}
\item[] Mostra as coincidências atuais no console de depuração.
\item[] 
\item[]
\begin{DUlineblock}{\DUlineblockindent}
\item[] \sphinxcode{cheatlist cheat.txt}
\item[] 
\end{DUlineblock}
\item[] Salve todas as coincidências atuais em formato XML no arquivo \textbf{cheat.txt}.
\item[] 
\item[] Back to {\hyperref[debugger/cheats:debugger\string-cheats\string-list]{\sphinxcrossref{\DUrole{std,std-ref}{Comandos para o depurador de trapaça}}}}
\end{DUlineblock}
\begin{quote}
\phantomsection\label{debugger/cheats:debugger-command-cheatundo}\end{quote}


\subsection{cheatundo}
\label{debugger/cheats:cheatundo}\label{debugger/cheats:debugger-command-cheatundo}
\begin{DUlineblock}{0em}
\item[]
\begin{DUlineblock}{\DUlineblockindent}
\item[] \textbf{cheatundo}
\item[] 
\end{DUlineblock}
\item[] Desfaz os resultados da última pesquisa.
\item[] 
\item[] O comando desfazer não afeta o último valor.
\item[] 
\item[] 
\item[] Exemplos:
\item[] 
\item[]
\begin{DUlineblock}{\DUlineblockindent}
\item[] \sphinxcode{cheatundo}
\item[] 
\end{DUlineblock}
\item[] desfaz a última pesquisa (apenas do estado).
\item[] 
\item[] Back to {\hyperref[debugger/cheats:debugger\string-cheats\string-list]{\sphinxcrossref{\DUrole{std,std-ref}{Comandos para o depurador de trapaça}}}}
\end{DUlineblock}


\section{Comandos para depuração de imagem}
\label{debugger/image:debugger-image-list}\label{debugger/image::doc}\label{debugger/image:comandos-para-depuracao-de-imagem}
Na interface de depuração do MAME você pode digitar \textbf{help \textless{}command\textgreater{}}
para uma melhor descrição de cada comando.

\begin{DUlineblock}{0em}
\item[] {\hyperref[debugger/image:debugger\string-command\string-images]{\sphinxcrossref{\DUrole{std,std-ref}{images}}}} -- lista todos os dispositivos de imagens arquivos montados
\item[] {\hyperref[debugger/image:debugger\string-command\string-mount]{\sphinxcrossref{\DUrole{std,std-ref}{mount}}}} -- monta um arquivo para um dispositivo
\item[] {\hyperref[debugger/image:debugger\string-command\string-unmount]{\sphinxcrossref{\DUrole{std,std-ref}{unmount}}}} -- desmonta um aquivo de uma dispositivo específico
\end{DUlineblock}
\begin{quote}
\phantomsection\label{debugger/image:debugger-command-images}\end{quote}


\subsection{images}
\label{debugger/image:images}\label{debugger/image:debugger-command-images}
\begin{DUlineblock}{0em}
\item[]
\begin{DUlineblock}{\DUlineblockindent}
\item[] \textbf{images}
\item[] 
\end{DUlineblock}
\item[] Usado para exibir na tela uma lista de dispositivos de imagens disponíveis.
\item[] 
\item[] Exemplos:
\item[] 
\item[]
\begin{DUlineblock}{\DUlineblockindent}
\item[] \sphinxcode{images}
\item[] 
\end{DUlineblock}
\item[] Mosta uma lista de dispositivos e arquivos montados para o driver atual.
\end{DUlineblock}
\begin{quote}
\phantomsection\label{debugger/image:debugger-command-mount}\end{quote}


\subsection{mount}
\label{debugger/image:mount}\label{debugger/image:debugger-command-mount}
\begin{DUlineblock}{0em}
\item[]
\begin{DUlineblock}{\DUlineblockindent}
\item[] \textbf{mount} \textless{}\emph{device}\textgreater{},\textless{}\emph{filename}\textgreater{}
\item[] 
\end{DUlineblock}
\item[] Monta o dispositivo \textless{}\emph{device}\textgreater{} com o nome de arquivo \textless{}\emph{filename}\textgreater{} da imagem.
\item[] 
\item[] O nome do arquivo \textless{}\emph{filename}\textgreater{} pode ser o item na lista de software ou o caminho completo do arquivo.
\item[] 
\item[] Exemplos:
\item[] 
\item[]
\begin{DUlineblock}{\DUlineblockindent}
\item[] \sphinxcode{mount cart,aladdin}
\item[] 
\end{DUlineblock}
\item[] Monta a lista de software com o item aladdin no dispositivo de cartucho.
\end{DUlineblock}
\begin{quote}
\phantomsection\label{debugger/image:debugger-command-unmount}\end{quote}


\subsection{unmount}
\label{debugger/image:unmount}\label{debugger/image:debugger-command-unmount}
\begin{DUlineblock}{0em}
\item[]
\begin{DUlineblock}{\DUlineblockindent}
\item[] \textbf{unmount} \textless{}\emph{device}\textgreater{}
\item[] 
\end{DUlineblock}
\item[] Desmonta o arquivo de imagem do dispositivo \textless{}\emph{device}\textgreater{}.
\item[] 
\item[] Exemplos:
\item[] 
\item[]
\begin{DUlineblock}{\DUlineblockindent}
\item[] \sphinxcode{unmount cart}
\item[] 
\end{DUlineblock}
\item[] Desmonta qualquer arquivo montado no dispositivo cart.
\end{DUlineblock}


\section{Guia de expressões do depurador}
\label{debugger/expressions:guia-de-expressoes-do-depurador}\label{debugger/expressions:debugger-expressions-list}\label{debugger/expressions::doc}
As expressões podem ser usadas em qualquer lugar onde um parâmetro
numérico for esperado. A sintaxe das expressões está muito próxima da
sintaxe do estilo C padrão com ordenação completa do operador e
parênteses.
Existem alguns operadores faltando (principalmente o operador trinário
?:) e alguns novos ``acessadores de memória''. A tabela abaixo lista todos
os operadores em sua ordem, os operadores de primeira prioridade vem
primeiro.

\begin{DUlineblock}{0em}
\item[] 
\item[] 
\item[] 
\item[] \sphinxcode{( ) :} parênteses padrão
\item[] \sphinxcode{++ -{-} :} incremento/decremento do postfix
\item[] \sphinxcode{++ -{-} \textasciitilde{} ! - + b@ w@ d@ q@ :} prefixo inc/dec, binário NOT, lógico NOT, unário +/-, acesso à memória
\item[] \sphinxcode{* / \% :} multiplicar, dividir, módulo
\item[] \sphinxcode{+ - :} adicionar, subtrair
\item[] \sphinxcode{\textless{}\textless{} \textgreater{}\textgreater{} :} deslocar para a esquerda/direita
\item[] \sphinxcode{\textless{} \textless{}= \textgreater{} \textgreater{}= :} menor que, menor que ou igual, maior que, maior que ou igual
\item[] \sphinxcode{== != :} igual, não igual
\item[] \sphinxcode{\& :} binário AND
\item[] \sphinxcode{\textasciicircum{} :} binário XOR
\item[] \sphinxcode{\textbar{} :} bibinárionary OR
\item[] \sphinxcode{\&\& :} lógica AND
\item[] \sphinxcode{\textbar{}\textbar{} :} lógica OR
\item[] \sphinxcode{= \textbackslash{}*= /= \%= += -= \textless{}\textless{}= \textgreater{}\textgreater{}= \&= \textbackslash{}\textbar{}= \textasciicircum{}= :} atribuição
\item[] \sphinxcode{, :} termos separados, parâmetros de função
\end{DUlineblock}


\subsection{Números}
\label{debugger/expressions:numeros}
Os números são prefixados de acordo com suas bases:
\begin{itemize}
\item {} 
Números hexadecimais (base 16) são prefixados com \sphinxcode{\$} ou \sphinxcode{0x}.

\item {} 
Números decimais (base 10) são prefixados com \sphinxcode{\#}

\item {} 
Os números octais (base 8) são prefixados com \sphinxcode{0o}.

\item {} 
Números binários (base 2) são prefixados com \sphinxcode{0b}.

\item {} 
Números não pré-fixados são hexadecimais (base 16).

\end{itemize}

Exemplos:
\begin{itemize}
\item {} 
\sphinxcode{123} é um hexadecimal 123 (decimal 291).

\item {} 
\sphinxcode{\$123} é um hexadecimal 123 (decimal 291).

\item {} 
\sphinxcode{0x123} é um hexadecimal 123 hexadecimal (decimal 291).

\item {} 
\sphinxcode{\#123} é um decimal 123.

\item {} 
\sphinxcode{0o123} é um octal 123 (decimal 83).

\item {} 
\sphinxcode{0b1001} é um decimal 9.

\item {} 
\sphinxcode{0b123} é inválido.

\end{itemize}


\subsection{Diferenças dos comportamentos C}
\label{debugger/expressions:diferencas-dos-comportamentos-c}\begin{itemize}
\item {} 
Primeiro, toda matemática é executada em valores full 64-bit unsigned,
então coisas como \textbf{a \textless{} 0} não funcionam como esperado.

\item {} 
Segundo, os operadores lógicos \textbf{\&\&} e \textbf{\textbar{}\textbar{}} não possuem
propriedades short-circuit e as duas metades são sempre avaliadas.

\item {} 
Finalmente, os novos operadores de memória funcionam assim:

\item {} 
\textbf{b!\textless{}addr\textgreater{}} se refere ao byte \textless{}\emph{addr}\textgreater{} e \emph{NÃO} se suprime os efeitos
colaterais, como ler uma mailbox, remover o sinalizador pendente ou ao ler um FIFO, remover um item.

\item {} 
\textbf{b@} se refere ao byte \textless{}\emph{addr}\textgreater{} no momento em que se suprime os
efeitos colaterais.

\item {} 
Da mesma forma, \textbf{w@} e \textbf{w!} referem-se a um \emph{word} na memória,
\textbf{d@} e \textbf{d!} referem-se a um \emph{dword} na memória, e \textbf{q@} e \textbf{q!}
referem-se a um \emph{qword} na memória.

\item {} 
Os operadores de memória podem ser usados como \emph{lvalues} e \emph{rvalues},
então você pode escrever \textbf{b@100 = ff} para armazenar um byte na
memória. É predefinido que esses operadores leiam a partir do espaço
de memória do programa, mas você pode sobrescrevê-los prefixando-os
com um `d' ou um `i'.

\item {} 
Como tal, \textbf{dw@300} refere-se à palavra de memória de dados no
endereço 300 e \textbf{id@400} referem-se a memória de I/O.

\end{itemize}


\chapter{FERRAMENTAS ADICIONAIS DO MAME}
\label{tools/index:ferramentas-adicionais-do-mame}\label{tools/index::doc}
Esta seção abrange as várias ferramentas extras que vêm com a sua
distribuição MAME, como a ferramenta \emph{imgtool} por exemplo.


\section{Imgtool - Uma ferramenta genérica de manipulação de imagens para o MAME}
\label{tools/imgtool:imgtool-uma-ferramenta-generica-de-manipulacao-de-imagens-para-o-mame}\label{tools/imgtool::doc}
Imgtool é uma ferramenta usada para a manutenção e manipulação de
imagens de discos de diferentes tipos que os usuários precisam aprender
a usar. As funções incluem recuperar e armazenar os arquivos com
verificação/validação por CRC.

A ferramenta faz parte do projeto MAME compartilhando grande parte do
seu código e a mesma não existiria se não fosse pelo MAME.
Logo, os termos da sua distribuição seguem os mesmos termos existentes
para o MAME. Favor ler a toda {\hyperref[license:mame\string-license]{\sphinxcrossref{\DUrole{std,std-ref}{LICENÇA}}}} com atenção.

\textbf{Algumas porções do Imgtool contém direitos autorais dos Regentes da
Universidade da Califórnia (c) 1989, 1993.
Todos os direitos reservados.}


\section{Usando o Imgtool}
\label{tools/imgtool:usando-o-imgtool}
Imgtool é um programa de linha comando que contém alguns ``subcomandos''
que fazem todo o trabalho. A maioria dos comandos são invocados usando
uma cadência de instruções, exemplo:
\begin{quote}

\textbf{imgtool} \textless{}\emph{subcommand}\textgreater{} \textless{}\emph{format}\textgreater{} \textless{}\emph{image}\textgreater{} ...
\end{quote}
\begin{itemize}
\item {} 
\textbf{\textless{}subcommand\textgreater{}} é o nome do subcomando

\item {} 
\textbf{\textless{}format\textgreater{}} é o formato da imagem

\item {} 
\textbf{\textless{}image\textgreater{}} é o nome da imagem

\end{itemize}

Exemplo de uso:

\begin{DUlineblock}{0em}
\item[] \sphinxcode{imgtool dir coco\_jvc\_rsdos myimageinazip.zip}
\item[] \sphinxcode{imgtool get coco\_jvc\_rsdos myimage.dsk myfile.bin mynewfile.txt}
\item[] \sphinxcode{imgtool getall coco\_jvc\_rsdos myimage.dsk}
\end{DUlineblock}

As variações dos subcomandos serão dadas mais adiante. Observe que nem
todos os subcomandos são compatíveis ou aplicáveis em todos os
variados tipos diferentes de imagem.


\section{Subcomandos do Imgtool}
\label{tools/imgtool:subcomandos-do-imgtool}
\textbf{create}
\begin{quote}

\textbf{imgtool create} \textless{}\emph{format}\textgreater{} \textless{}\emph{imagename}\textgreater{} {[}--(\emph{createoption})=value{]}
\begin{itemize}
\item {} 
\textless{}\emph{format}\textgreater{} é o nome do formato da imagem, coco\_jvc\_rsdos por exemplo

\item {} 
\textless{}\emph{imagename}\textgreater{} é o nome de destino da imagem, é possível especificar um arquivo ZIP como nome da imagem

\end{itemize}

Cria uma imagem
\end{quote}

\textbf{dir}
\begin{quote}

\textbf{imgtool dir} \textless{}\emph{format}\textgreater{} \textless{}\emph{imagename}\textgreater{} {[}\emph{path}{]}
\begin{itemize}
\item {} 
\textless{}\emph{format}\textgreater{} é o nome do formato da imagem, coco\_jvc\_rsdos por exemplo

\item {} 
\textless{}\emph{imagename}\textgreater{} é o nome de destino da imagem; é possível especificar um arquivo ZIP como nome da imagem

\end{itemize}

Lista o conteúdo de uma imagem
\end{quote}

\textbf{get}
\begin{quote}

\textbf{imgtool get} \textless{}\emph{format}\textgreater{} \textless{}\emph{imagename}\textgreater{} \textless{}\emph{filename}\textgreater{} {[}\emph{newname}{]} {[}--filter=filter{]} {[}--fork=fork{]}
\begin{itemize}
\item {} 
\textless{}\emph{format}\textgreater{} é o nome do formato da imagem, coco\_jvc\_rsdos por exemplo

\item {} 
\textless{}\emph{imagename}\textgreater{} é o nome de destino da imagem; é possível especificar um arquivo ZIP como nome da imagem

\end{itemize}

Extrai um arquivo da imagem
\end{quote}

\textbf{put}
\begin{quote}

\textbf{imgtool put} \textless{}\emph{format}\textgreater{} \textless{}\emph{imagename}\textgreater{} \textless{}\emph{filename}\textgreater{}... \textless{}\emph{destname}\textgreater{} {[}--(\emph{fileoption})==value{]} {[}--filter=filter{]} {[}--fork=fork{]}
\begin{itemize}
\item {} 
\textless{}\emph{format}\textgreater{} é o nome do formato da imagem, coco\_jvc\_rsdos por exemplo

\item {} 
\textless{}\emph{imagename}\textgreater{} é o nome de destino da imagem; é possível especificar um arquivo ZIP como nome da imagem

\end{itemize}

Adiciona um arquivo na imagem (é compatível com coringas)
\end{quote}

\textbf{getall}
\begin{quote}

\textbf{imgtool getall} \textless{}\emph{format}\textgreater{} \textless{}\emph{imagename}\textgreater{} {[}\emph{path}{]} {[}--filter=filter{]}
\begin{itemize}
\item {} 
\textless{}\emph{format}\textgreater{} é o nome do formato da imagem, coco\_jvc\_rsdos por exemplo

\item {} 
\textless{}\emph{imagename}\textgreater{} é o nome de destino da imagem; é possível especificar um arquivo ZIP como nome da imagem

\end{itemize}

Extrai todos os arquivos de uma imagem
\end{quote}

\textbf{del}
\begin{quote}

\textbf{imgtool del} \textless{}\emph{format}\textgreater{} \textless{}\emph{imagename}\textgreater{} \textless{}\emph{filename}\textgreater{}...
\begin{itemize}
\item {} 
\textless{}\emph{format}\textgreater{} é o nome do formato da imagem, coco\_jvc\_rsdos por exemplo

\item {} 
\textless{}\emph{imagename}\textgreater{} é o nome de destino da imagem; é possível especificar um arquivo ZIP como nome da imagem

\end{itemize}

Apaga todos os arquivos de uma imagem
\end{quote}

\textbf{mkdir}
\begin{quote}

\textbf{imgtool mkdir} \textless{}\emph{format}\textgreater{} \textless{}\emph{imagename}\textgreater{} \textless{}\emph{dirname}\textgreater{}
\begin{itemize}
\item {} 
\textless{}\emph{format}\textgreater{} é o nome do formato da imagem, coco\_jvc\_rsdos por exemplo

\item {} 
\textless{}\emph{imagename}\textgreater{} é o nome de destino da imagem; é possível especificar um arquivo ZIP como nome da imagem

\end{itemize}

Cria um subdiretório em uma imagem
\end{quote}

\textbf{rmdir}
\begin{quote}

\textbf{imgtool rmdir} \textless{}\emph{format}\textgreater{} \textless{}\emph{imagename}\textgreater{} \textless{}\emph{dirname}\textgreater{}...
\begin{itemize}
\item {} 
\textless{}\emph{format}\textgreater{} é o nome do formato da imagem, coco\_jvc\_rsdos por exemplo

\item {} 
\textless{}\emph{imagename}\textgreater{} é o nome de destino da imagem; é possível especificar um arquivo ZIP como nome da imagem

\end{itemize}

Apaga um subdiretório em uma imagem
\end{quote}

\textbf{readsector}
\begin{quote}

\textbf{imgtool readsector} \textless{}\emph{format}\textgreater{} \textless{}\emph{imagename}\textgreater{} \textless{}\emph{track}\textgreater{} \textless{}\emph{head}\textgreater{} \textless{}\emph{sector}\textgreater{} \textless{}\emph{filename}\textgreater{}
\begin{itemize}
\item {} 
\textless{}\emph{format}\textgreater{} é o nome do formato da imagem, coco\_jvc\_rsdos por exemplo

\item {} 
\textless{}\emph{imagename}\textgreater{} é o nome de destino da imagem; é possível especificar um arquivo ZIP como nome da imagem

\end{itemize}

Lê o setor de uma imagem e grava em um nome de arquivo \textless{}\emph{filename}\textgreater{} específico.
\end{quote}

\textbf{writesector}
\begin{quote}

\textbf{imgtool writesector} \textless{}\emph{format}\textgreater{} \textless{}\emph{imagename}\textgreater{} \textless{}\emph{track}\textgreater{} \textless{}\emph{head}\textgreater{} \textless{}\emph{sector}\textgreater{} \textless{}\emph{filename}\textgreater{}
\begin{itemize}
\item {} 
\textless{}\emph{format}\textgreater{} é o nome do formato da imagem, coco\_jvc\_rsdos por exemplo

\item {} 
\textless{}\emph{imagename}\textgreater{} é o nome de destino da imagem; é possível especificar um arquivo ZIP como nome da imagem

\end{itemize}

Escreve no setor de uma imagem vinda de um arquivo \textless{}\emph{filename}\textgreater{} especificado
\end{quote}

\textbf{identify}
\begin{quote}
\begin{itemize}
\item {} 
\textless{}\emph{format}\textgreater{} é o nome do formato da imagem, coco\_jvc\_rsdos por exemplo

\item {} 
\textless{}\emph{imagename}\textgreater{} é o nome de destino da imagem; é possível especificar um arquivo ZIP como nome da imagem

\end{itemize}

\textbf{imgtool identify} \textless{}\emph{imagename}\textgreater{}
\end{quote}

\textbf{listformats}
\begin{quote}

Exibe uma lista com todos os formatos de imagem compatíveis com o imgtool
\end{quote}

\textbf{listfilters}
\begin{quote}

Exibe uma lista de todos os filtros compatíveis com o imgtool
\end{quote}

\textbf{listdriveroptions}
\begin{quote}

\textbf{imgtool listdriveroptions} \textless{}\emph{format}\textgreater{}
\begin{itemize}
\item {} 
\textless{}\emph{format}\textgreater{} é o nome do formato da imagem, coco\_jvc\_rsdos por exemplo

\end{itemize}

Exibe uma lista completa de todas as opções relacionadas a um formato em específico para os comandos `put' e `create'.
\end{quote}


\section{Filtros do Imgtool}
\label{tools/imgtool:filtros-do-imgtool}
Os filtros são uma maneira de processar a maneira que os dados estão
sendo escritos ou lidos em uma imagem. Os filtros podem ser usados nos
comandos \textbf{get}, \textbf{put} e \textbf{getall} ao usar a opção \sphinxcode{-{-}filter=xxxx}
na linha de comando. Atualmente, os seguintes filtros são compatíveis:

\textbf{ascii}
\begin{quote}

Converte o final de linha dos arquivos para o formato apropriado
\end{quote}

\textbf{cocobas}
\begin{quote}

Processa programas BASIC tokenizados para Computadores TRS-80 Color (CoCo)
\end{quote}

\textbf{dragonbas}
\begin{quote}

Processa programas BASIC tokenizados para o Tano/Dragon Data Dragon 32/64
\end{quote}

\textbf{macbinary}
\begin{quote}

Processa arquivos de imagem (merged forks) Apple em formato MacBinary
\end{quote}

\textbf{vzsnapshot}
\begin{quote}

{[}a fazer: VZ Snapshot? Descobrir o que que é isso...{]}
\end{quote}

\textbf{vzbas}
\begin{quote}

Processa programas BASIC tokenizados para o Laser/VZ
\end{quote}

\textbf{thombas5}
\begin{quote}

Processa programas BASIC tokenizados para o Thomson MO5 com BASIC 1.0 (apenas leitura, descriptografia automática)
\end{quote}

\textbf{thombas7}
\begin{quote}

Processa programas BASIC tokenizados para o Thomson TO7 com BASIC 1.0 (apenas leitura, descriptografia automática)
\end{quote}

\textbf{thombas128}
\begin{quote}

Processa programas BASIC tokenizados para o Thomson com BASIC 128/512 (apenas leitura, descriptografia automática)
\end{quote}

\textbf{thomcrypt}
\begin{quote}

Processa programas BASIC tokenizados para o Thomson BASIC, protegidos por criptografia (sem tokenização)
\end{quote}

\textbf{bm13bas}
\begin{quote}

Processa arquivos BASIC, Basic Master Level 3 tokenizados
\end{quote}


\section{Informação de formatação do Imgtool}
\label{tools/imgtool:informacao-de-formatacao-do-imgtool}

\subsection{Imagem de disquete do Amiga (formato OFS/FFS) - (\emph{amiga\_floppy})}
\label{tools/imgtool:imagem-de-disquete-do-amiga-formato-ofs-ffs-amiga-floppy}
Opções específicas de driver para o módulo `amiga\_floppy':

Nenhuma opção específica da imagem

Opções específicas para a criação da imagem (utilizável com o comando `create'):

\noindent\begin{tabulary}{\linewidth}{|L|L|L|}
\hline

Opção
&
Valores permitidos
&
Descrição
\\
\hline
--density
&
dd/hd
&
Densidade
\\
\hline
--filesystem
&
ofs/ffs
&
Sistema de Arquivos
\\
\hline
--mode
&
none/intl/dirc
&
Opções do sistema de arquivos
\\
\hline\end{tabulary}



\subsection{Apple {]}{[} imagem de disco DOS order (formato ProDOS) - (\emph{apple2\_do\_prodos\_525})}
\label{tools/imgtool:apple-imagem-de-disco-dos-order-formato-prodos-apple2-do-prodos-525}
Opções específicas de driver para o módulo `apple2\_do\_prodos\_525':

Nenhuma opção específica da imagem

Opções específicas para a criação da imagem (utilizável com o comando `create'):

\noindent\begin{tabulary}{\linewidth}{|L|L|L|}
\hline

Opções
&
Valores permitidos
&
Descrição
\\
\hline
--heads
&
1
&
Cabeças
\\
\hline
--tracks
&
35
&
Trilhas
\\
\hline
--sectors
&
16
&
Setores
\\
\hline
--sectorlength
&
256
&
Bytes por Setor
\\
\hline
--firstsectorid
&
0
&
Primeiro Setor
\\
\hline\end{tabulary}



\subsection{Apple {]}{[} imagem de disco Nibble order (formato ProDOS) - (\emph{apple2\_nib\_prodos\_525})}
\label{tools/imgtool:apple-imagem-de-disco-nibble-order-formato-prodos-apple2-nib-prodos-525}
Opções específicas de driver para o módulo `apple2\_nib\_prodos\_525':

Nenhuma opção específica da imagem

Opções específicas para a criação da imagem (utilizável com o comando `create'):

\noindent\begin{tabulary}{\linewidth}{|L|L|L|}
\hline

Opções
&
Valores permitidos
&
Descrição
\\
\hline
--heads
&
1
&
Cabeças
\\
\hline
--tracks
&
35
&
Trilhas
\\
\hline
--sectors
&
16
&
Setores
\\
\hline
--sectorlength
&
256
&
Bytes por Setor
\\
\hline
--firstsectorid
&
0
&
Primeiro Setor
\\
\hline\end{tabulary}



\subsection{Apple {]}{[} imagem de disco ProDOS order (formato ProDOS) - (\emph{apple2\_po\_prodos\_525})}
\label{tools/imgtool:apple-imagem-de-disco-prodos-order-formato-prodos-apple2-po-prodos-525}
Opções específicas de driver para o módulo `apple2\_po\_prodos\_525':

Nenhuma opção específica da imagem

Opções específicas para a criação da imagem (utilizável com o comando `create'):

\noindent\begin{tabulary}{\linewidth}{|L|L|L|}
\hline

Opções
&
Valores permitidos
&
Descrição
\\
\hline
--heads
&
1
&
Cabeças
\\
\hline
--tracks
&
35
&
Trilhas
\\
\hline
--sectors
&
16
&
Setores
\\
\hline
--sectorlength
&
256
&
Bytes por Setor
\\
\hline
--firstsectorid
&
0
&
Primeiro Setor
\\
\hline\end{tabulary}



\subsection{Apple {]}{[}imagem de disco gs 2IMG (formato ProDOS) - (\emph{apple35\_2img\_prodos\_35})}
\label{tools/imgtool:apple-imagem-de-disco-gs-2img-formato-prodos-apple35-2img-prodos-35}
Opções específicas de driver para o módulo `apple35\_2img\_prodos\_35':

Nenhuma opção específica da imagem

Opções específicas para a criação da imagem (utilizável com o comando `create'):

\noindent\begin{tabulary}{\linewidth}{|L|L|L|}
\hline

Opções
&
Valores permitidos
&
Descrição
\\
\hline
--heads
&
1-2
&
Cabeças
\\
\hline
--tracks
&
80
&
Trilhas
\\
\hline
--sectorlength
&
512
&
Bytes por Setor
\\
\hline
--firstsectorid
&
0
&
Primeiro Setor
\\
\hline\end{tabulary}



\subsection{Imagem de disco para o Apple DiskCopy (Disquete Mac HFS) - (\emph{apple35\_dc\_mac\_hfs})}
\label{tools/imgtool:imagem-de-disco-para-o-apple-diskcopy-disquete-mac-hfs-apple35-dc-mac-hfs}
Opções específicas de driver para o módulo `apple35\_dc\_mac\_hfs':

Nenhuma opção específica da imagem

Opções específicas para a criação da imagem (utilizável com o comando `create'):

\noindent\begin{tabulary}{\linewidth}{|L|L|L|}
\hline

Opções
&
Valores permitidos
&
Descrição
\\
\hline
--heads
&
1-2
&
Cabeças
\\
\hline
--tracks
&
80
&
Trilhas
\\
\hline
--sectorlength
&
512
&
Bytes por Setor
\\
\hline
--firstsectorid
&
0
&
Primeiro Setor
\\
\hline\end{tabulary}



\subsection{Imagem de disco para o Apple DiskCopy (Disquete Mac MFS) - (\emph{apple35\_dc\_mac\_hfs})}
\label{tools/imgtool:imagem-de-disco-para-o-apple-diskcopy-disquete-mac-mfs-apple35-dc-mac-hfs}
Opções específicas de driver para o módulo `apple35\_dc\_mac\_mfs':

Nenhuma opção específica da imagem

Opções específicas para a criação da imagem (utilizável com o comando `create'):

\noindent\begin{tabulary}{\linewidth}{|L|L|L|}
\hline

Opções
&
Valores permitidos
&
Descrição
\\
\hline
--heads
&
1-2
&
Cabeças
\\
\hline
--tracks
&
80
&
Trilhas
\\
\hline
--sectorlength
&
512
&
Bytes por Setor
\\
\hline
--firstsectorid
&
0
&
Primeiro Setor
\\
\hline\end{tabulary}



\subsection{Imagem de disco para o Apple DiskCopy (formato ProDOS) - (\emph{apple35\_dc\_prodos\_35})}
\label{tools/imgtool:imagem-de-disco-para-o-apple-diskcopy-formato-prodos-apple35-dc-prodos-35}
Opções específicas de driver para o módulo `apple35\_dc\_prodos\_35':

Nenhuma opção específica da imagem

Opções específicas para a criação da imagem (utilizável com o comando `create'):

\noindent\begin{tabulary}{\linewidth}{|L|L|L|}
\hline

Opções
&
Valores permitidos
&
Descrição
\\
\hline
--heads
&
1-2
&
Cabeças
\\
\hline
--tracks
&
80
&
Trilhas
\\
\hline
--sectorlength
&
512
&
Bytes por Setor
\\
\hline
--firstsectorid
&
0
&
Primeiro Setor
\\
\hline\end{tabulary}



\subsection{Imagem de disco para o Apple raw 3.5'' (Disquete Mac HFS) - (\emph{apple35\_raw\_mac\_hfs})}
\label{tools/imgtool:imagem-de-disco-para-o-apple-raw-3-5-disquete-mac-hfs-apple35-raw-mac-hfs}
Opções específicas de driver para o módulo `apple35\_raw\_mac\_hfs':

Nenhuma opção específica da imagem

Opções específicas para a criação da imagem (utilizável com o comando `create'):

\noindent\begin{tabulary}{\linewidth}{|L|L|L|}
\hline

Opções
&
Valores permitidos
&
Descrição
\\
\hline
--heads
&
1-2
&
Cabeças
\\
\hline
--tracks
&
80
&
Trilhas
\\
\hline
--sectorlength
&
512
&
Bytes por Setor
\\
\hline
--firstsectorid
&
0
&
Primeiro Setor
\\
\hline\end{tabulary}



\subsection{Imagem de disco para o Apple raw 3.5'' (Disquete Mac MFS) - (\emph{apple35\_raw\_mac\_mfs})}
\label{tools/imgtool:imagem-de-disco-para-o-apple-raw-3-5-disquete-mac-mfs-apple35-raw-mac-mfs}
Opções específicas de driver para o módulo `apple35\_raw\_mac\_mfs':

Nenhuma opção específica da imagem

Opções específicas para a criação da imagem (utilizável com o comando `create'):

\noindent\begin{tabulary}{\linewidth}{|L|L|L|}
\hline

Opções
&
Valores permitidos
&
Descrição
\\
\hline
--heads
&
1-2
&
Cabeças
\\
\hline
--tracks
&
80
&
Trilhas
\\
\hline
--sectorlength
&
512
&
Bytes por Setor
\\
\hline
--firstsectorid
&
0
&
Primeiro Setor
\\
\hline\end{tabulary}



\subsection{Imagem de disco para o Apple raw 3.5'' (formato ProDOS) - (\emph{apple35\_raw\_prodos\_35})}
\label{tools/imgtool:imagem-de-disco-para-o-apple-raw-3-5-formato-prodos-apple35-raw-prodos-35}
Opções específicas de driver para o módulo `apple35\_raw\_prodos\_35':

Nenhuma opção específica da imagem

Opções específicas para a criação da imagem (utilizável com o comando `create'):

\noindent\begin{tabulary}{\linewidth}{|L|L|L|}
\hline

Opções
&
Valores permitidos
&
Descrição
\\
\hline
--heads
&
1-2
&
Cabeças
\\
\hline
--tracks
&
80
&
Trilhas
\\
\hline
--sectorlength
&
512
&
Bytes por Setor
\\
\hline
--firstsectorid
&
0
&
Primeiro Setor
\\
\hline\end{tabulary}



\subsection{Imagem de disco para o CoCo DMK (formato OS-9) - (\emph{coco\_dmk\_os9})}
\label{tools/imgtool:imagem-de-disco-para-o-coco-dmk-formato-os-9-coco-dmk-os9}
Opções específicas de driver para o módulo `coco\_dmk\_os9':

Nenhuma opção específica da imagem

Opções específicas para a criação da imagem (utilizável com o comando `create'):

\noindent\begin{tabulary}{\linewidth}{|L|L|L|}
\hline

Opções
&
Valores Permitidos
&
Descrição
\\
\hline
--heads
&
1-2
&
Cabeças
\\
\hline
--tracks
&
35-255
&
Trilhas
\\
\hline
--sectors
&
1-18
&
Setores
\\
\hline
--sectorlength
&
128/256/512/1024/2048/4096/8192
&
Bytes por Setor
\\
\hline
--interleave
&
0-17
&
Intercalação
\\
\hline
--firstsectorid
&
0-1
&
Primeiro Setor
\\
\hline\end{tabulary}



\subsection{Imagem de disco para o CoCo DMK (formato RS-DOS) - (\emph{coco\_dmk\_rsdos})}
\label{tools/imgtool:imagem-de-disco-para-o-coco-dmk-formato-rs-dos-coco-dmk-rsdos}
Opções específicas de driver para o módulo `coco\_dmk\_rsdos':

Opções específicas para o arquivo (utilizável com o comando `put')

\noindent\begin{tabulary}{\linewidth}{|L|L|L|}
\hline

Opções
&
Valores permitidos
&
Descrição
\\
\hline
--ftype
&
basic/data/binary/assembler
&
Tipo do arquivo
\\
\hline
--ascii
&
ascii/binary
&
ASCII flag
\\
\hline\end{tabulary}


Opções específicas para a criação da imagem (utilizável com o comando `create'):

\noindent\begin{tabulary}{\linewidth}{|L|L|L|}
\hline

Opções
&
Valores Permitidos
&
Descrição
\\
\hline
--heads
&
1-2
&
Cabeças
\\
\hline
--tracks
&
35-255
&
Trilhas
\\
\hline
--sectors
&
1-18
&
Setores
\\
\hline
--sectorlength
&
128/256/512/1024/2048/4096/8192
&
Bytes por Setor
\\
\hline
--interleave
&
0-17
&
Intercalação
\\
\hline
--firstsectorid
&
0-1
&
Primeiro Setor
\\
\hline\end{tabulary}



\subsection{Imagem de disco para o CoCo JVC (formato OS-9) - (\emph{coco\_jvc\_os9})}
\label{tools/imgtool:imagem-de-disco-para-o-coco-jvc-formato-os-9-coco-jvc-os9}
Opções específicas de driver para o módulo `coco\_jvc\_os9':

Nenhuma opção específica da imagem

Opções específicas para a criação da imagem (utilizável com o comando `create'):

\noindent\begin{tabulary}{\linewidth}{|L|L|L|}
\hline

Opções
&
Valores permitidos
&
Descrição
\\
\hline
--heads
&
1-2
&
Cabeças
\\
\hline
--tracks
&
35-255
&
Trilhas
\\
\hline
--sectors
&
1-255
&
Setores
\\
\hline
--sectorlength
&
128/256/512/1024
&
Bytes por Setor
\\
\hline
--firstsectorid
&
0-1
&
Primeiro Setor
\\
\hline\end{tabulary}



\subsection{Imagem de disco para o CoCo JVC (formato RS-DOS) - (\emph{coco\_jvc\_rsdos})}
\label{tools/imgtool:imagem-de-disco-para-o-coco-jvc-formato-rs-dos-coco-jvc-rsdos}
Opções específicas de driver para o módulo `coco\_jvc\_rsdos':

Opções específicas para o arquivo (utilizável com o comando `put')

\noindent\begin{tabulary}{\linewidth}{|L|L|L|}
\hline

Opções
&
Valores permitidos
&
Descrição
\\
\hline
--ftype
&
basic/data/binary/assembler
&
Tipo do arquivo
\\
\hline
--ascii
&
ascii/binary
&
ASCII flag
\\
\hline\end{tabulary}


Opções específicas para a criação da imagem (utilizável com o comando `create'):

\noindent\begin{tabulary}{\linewidth}{|L|L|L|}
\hline

Opções
&
Valores permitidos
&
Descrição
\\
\hline
--heads
&
1-2
&
Cabeças
\\
\hline
--tracks
&
35-255
&
Trilhas
\\
\hline
--sectors
&
1-255
&
Setores
\\
\hline
--sectorlength
&
128/256/512/1024
&
Bytes por Setor
\\
\hline
--firstsectorid
&
0-1
&
Primeiro Setor
\\
\hline\end{tabulary}



\subsection{Imagem de disco para o CoCo OS-9 (formato OS-9) - (\emph{coco\_os9\_os9})}
\label{tools/imgtool:imagem-de-disco-para-o-coco-os-9-formato-os-9-coco-os9-os9}
Opções específicas de driver para o módulo `coco\_os9\_os9':

Nenhuma opção específica da imagem

Opções específicas para a criação da imagem (utilizável com o comando `create'):

\noindent\begin{tabulary}{\linewidth}{|L|L|L|}
\hline

Opções
&
Valores permitidos
&
Descrição
\\
\hline
--heads
&
1-2
&
Cabeças
\\
\hline
--tracks
&
35-255
&
Trilhas
\\
\hline
--sectors
&
1-255
&
Setores
\\
\hline
--sectorlength
&
128/256/512/1024
&
Bytes por Setor
\\
\hline
--firstsectorid
&
1
&
Primeiro Setor
\\
\hline\end{tabulary}



\subsection{Imagem de disco para o CoCo VDK (formato OS-9) - (\emph{coco\_vdk\_os9})}
\label{tools/imgtool:imagem-de-disco-para-o-coco-vdk-formato-os-9-coco-vdk-os9}
Opções específicas de driver para o módulo `coco\_vdk\_os9':

Nenhuma opção específica da imagem

Opções específicas para a criação da imagem (utilizável com o comando `create'):

\noindent\begin{tabulary}{\linewidth}{|L|L|L|}
\hline

Opções
&
Valores permitidos
&
Descrição
\\
\hline
--heads
&
1-2
&
Cabeças
\\
\hline
--tracks
&
35-255
&
Trilhas
\\
\hline
--sectors
&
18
&
Setores
\\
\hline
--sectorlength
&
256
&
Bytes por Setor
\\
\hline
--firstsectorid
&
1
&
Primeiro Setor
\\
\hline\end{tabulary}



\subsection{Imagem de disco para o CoCo VDK (formato RS-DOS) - (\emph{coco\_vdk\_rsdos})}
\label{tools/imgtool:imagem-de-disco-para-o-coco-vdk-formato-rs-dos-coco-vdk-rsdos}
Opções específicas de driver para o módulo `coco\_vdk\_rsdos':

Opções específicas para o arquivo (utilizável com o comando `put')

\noindent\begin{tabulary}{\linewidth}{|L|L|L|}
\hline

Opções
&
Valores permitidos
&
Descrição
\\
\hline
--ftype
&
basic/data/binary/assembler
&
Tipo do arquivo
\\
\hline
--ascii
&
ascii/binary
&
ASCII flag
\\
\hline\end{tabulary}


Opções específicas para a criação da imagem (utilizável com o comando `create'):

\noindent\begin{tabulary}{\linewidth}{|L|L|L|}
\hline

Opções
&
Valores permitidos
&
Descrição
\\
\hline
--heads
&
1-2
&
Cabeças
\\
\hline
--tracks
&
35-255
&
Trilhas
\\
\hline
--sectors
&
18
&
Setores
\\
\hline
--sectorlength
&
256
&
Bytes por Setor
\\
\hline
--firstsectorid
&
1
&
Primeiro Setor
\\
\hline\end{tabulary}



\subsection{Imagem de disquete para o Concept - (\emph{concept})}
\label{tools/imgtool:imagem-de-disquete-para-o-concept-concept}
Opções específicas de driver para o módulo `concept':

Nenhuma opção específica da imagem

Nenhuma opção específica para a criação da imagem


\subsection{Imagem de disquete para o CopyQM (formato Basic Master Level 3) - (\emph{cqm\_bml3})}
\label{tools/imgtool:imagem-de-disquete-para-o-copyqm-formato-basic-master-level-3-cqm-bml3}
Opções específicas de driver para o módulo `cqm\_bml3':

Opções específicas para o arquivo (utilizável com o comando `put')

\noindent\begin{tabulary}{\linewidth}{|L|L|L|}
\hline

Opções
&
Valores permitidos
&
Descrição
\\
\hline
--ftype
&
basic/data/binary/assembler
&
Tipo do arquivo
\\
\hline
--ascii
&
ascii/binary
&
ASCII flag
\\
\hline\end{tabulary}


Nenhuma opção específica para a criação da imagem


\subsection{Imagem de disquete para o CopyQM (formato FAT) - (\emph{cqm\_fat})}
\label{tools/imgtool:imagem-de-disquete-para-o-copyqm-formato-fat-cqm-fat}
Opções específicas de driver para o módulo `cqm\_fat':

Nenhuma opção específica da imagem

Nenhuma opção específica para a criação da imagem


\subsection{Imagem de disquete para o CopyQM (Mac HFS Floppy) - (\emph{cqm\_mac\_hfs})}
\label{tools/imgtool:imagem-de-disquete-para-o-copyqm-mac-hfs-floppy-cqm-mac-hfs}
Opções específicas de driver para o módulo `cqm\_mac\_hfs':

Nenhuma opção específica da imagem

Nenhuma opção específica para a criação da imagem


\subsection{Imagem de disquete para o CopyQM (Disquete Mac MFS) - (\emph{cqm\_mac\_mfs})}
\label{tools/imgtool:imagem-de-disquete-para-o-copyqm-disquete-mac-mfs-cqm-mac-mfs}
Opções específicas de driver para o módulo `cqm\_mac\_mfs':

Nenhuma opção específica da imagem

Nenhuma opção específica para a criação da imagem


\subsection{Imagem de disquete para o CopyQM (formato OS-9) - (\emph{cqm\_os9})}
\label{tools/imgtool:imagem-de-disquete-para-o-copyqm-formato-os-9-cqm-os9}
Opções específicas de driver para o módulo `cqm\_os9':

Nenhuma opção específica da imagem

Nenhuma opção específica para a criação da imagem


\subsection{Imagem de disquete para o CopyQM (formato ProDOS) - (\emph{cqm\_prodos\_35})}
\label{tools/imgtool:imagem-de-disquete-para-o-copyqm-formato-prodos-cqm-prodos-35}
Opções específicas de driver para o módulo `cqm\_prodos\_35':

Nenhuma opção específica da imagem

Nenhuma opção específica para a criação da imagem


\subsection{Imagem de disquete para o CopyQM (formato ProDOS) - (\emph{cqm\_prodos\_525})}
\label{tools/imgtool:imagem-de-disquete-para-o-copyqm-formato-prodos-cqm-prodos-525}
Opções específicas de driver para o módulo `cqm\_prodos\_525':

Nenhuma opção específica da imagem

Nenhuma opção específica para a criação da imagem


\subsection{Imagem de disquete para o CopyQM (formato RS-DOS) - (\emph{cqm\_rsdos})}
\label{tools/imgtool:imagem-de-disquete-para-o-copyqm-formato-rs-dos-cqm-rsdos}
Opções específicas de driver para o módulo `cqm\_rsdos':

Opções específicas para o arquivo (utilizável com o comando `put')

\noindent\begin{tabulary}{\linewidth}{|L|L|L|}
\hline

Opções
&
Valores permitidos
&
Descrição
\\
\hline
--ftype
&
basic/data/binary/assembler
&
Tipo do arquivo
\\
\hline
--ascii
&
ascii/binary
&
ASCII flag
\\
\hline\end{tabulary}


Nenhuma opção específica para a criação da imagem


\subsection{Imagem de disquete para o CopyQM (formato VZ-DOS) - (\emph{cqm\_vzdos})}
\label{tools/imgtool:imagem-de-disquete-para-o-copyqm-formato-vz-dos-cqm-vzdos}
Opções específicas de driver para o módulo `cqm\_vzdos':

Opções específicas para o arquivo (utilizável com o comando `put')

\noindent\begin{tabulary}{\linewidth}{|L|L|L|}
\hline

Opções
&
Valores permitidos
&
Descrição
\\
\hline
--ftype
&
basic/binary/data
&
Tipo do arquivo
\\
\hline
--fname
&
intern/extern
&
Nome do arquivo
\\
\hline\end{tabulary}


Nenhuma opção específica para a criação da imagem


\subsection{Sistema de arquivos para o Cybiko Classic - (\emph{cybiko})}
\label{tools/imgtool:sistema-de-arquivos-para-o-cybiko-classic-cybiko}
Opções específicas de driver para o módulo `cybiko':

Nenhuma opção específica da imagem

Opções específicas para a criação da imagem (utilizável com o comando `create'):

\noindent\begin{tabulary}{\linewidth}{|L|L|L|}
\hline

Opções
&
Valores permitidos
&
Descrição
\\
\hline
--flash
&
AT45DB041/AT45DB081/AT45DB161
&
Flash Tipo
\\
\hline\end{tabulary}



\subsection{Sistema de arquivos para o Cybiko Xtreme - (\emph{cybikoxt})}
\label{tools/imgtool:sistema-de-arquivos-para-o-cybiko-xtreme-cybikoxt}
Opções específicas de driver para o módulo `cybikoxt':

Nenhuma opção específica da imagem

Nenhuma opção específica para a criação da imagem


\subsection{Imagem de disquete para o D88 (formato Basic Master Level 3) - (\emph{d88\_bml3})}
\label{tools/imgtool:imagem-de-disquete-para-o-d88-formato-basic-master-level-3-d88-bml3}
Opções específicas de driver para o módulo `d88\_bml3':

Opções específicas para o arquivo (utilizável com o comando `put')

\noindent\begin{tabulary}{\linewidth}{|L|L|L|}
\hline

Opções
&
Valores permitidos
&
Descrição
\\
\hline
--ftype
&
basic/data/binary/assembler
&
Tipo do arquivo
\\
\hline
--ascii
&
ascii/binary
&
ASCII flag
\\
\hline\end{tabulary}


Nenhuma opção específica para a criação da imagem


\subsection{Imagem de disquete para o D88 (formato FAT) - (\emph{d88\_fat})}
\label{tools/imgtool:imagem-de-disquete-para-o-d88-formato-fat-d88-fat}
Opções específicas de driver para o módulo `d88\_fat':

Nenhuma opção específica da imagem

Nenhuma opção específica para a criação da imagem


\subsection{Imagem de disquete para o D88 (Disquete Mac HFS) - (\emph{d88\_mac\_hfs})}
\label{tools/imgtool:imagem-de-disquete-para-o-d88-disquete-mac-hfs-d88-mac-hfs}
Opções específicas de driver para o módulo `d88\_mac\_hfs':

Nenhuma opção específica da imagem

Nenhuma opção específica para a criação da imagem


\subsection{Imagem de disquete para o D88 (Disquete Mac MFS) - (\emph{d88\_mac\_mfs})}
\label{tools/imgtool:imagem-de-disquete-para-o-d88-disquete-mac-mfs-d88-mac-mfs}
Opções específicas de driver para o módulo `d88\_mac\_mfs':

Nenhuma opção específica da imagem

Nenhuma opção específica para a criação da imagem


\subsection{Imagem de disquete para o D88 (formato OS-9) - (\emph{d88\_os9})}
\label{tools/imgtool:imagem-de-disquete-para-o-d88-formato-os-9-d88-os9}
Opções específicas de driver para o módulo `d88\_os9':

Nenhuma opção específica da imagem

Nenhuma opção específica para a criação da imagem


\subsection{Imagem de disquete para o D88 (formato OS-9) - (\emph{d88\_os9})}
\label{tools/imgtool:id1}
Opções específicas de driver para o módulo `d88\_prodos\_35':

Nenhuma opção específica da imagem

Nenhuma opção específica para a criação da imagem


\subsection{Imagem de disquete para o D88 (formato ProDOS) - (\emph{d88\_prodos\_525})}
\label{tools/imgtool:imagem-de-disquete-para-o-d88-formato-prodos-d88-prodos-525}
Opções específicas de driver para o módulo `d88\_prodos\_525':

Nenhuma opção específica da imagem

Nenhuma opção específica para a criação da imagem


\subsection{Imagem de disquete para o D88 (formato RS-DOS) - (\emph{d88\_rsdos})}
\label{tools/imgtool:imagem-de-disquete-para-o-d88-formato-rs-dos-d88-rsdos}
Opções específicas de driver para o módulo `d88\_rsdos':

Opções específicas para o arquivo (utilizável com o comando `put')

\noindent\begin{tabulary}{\linewidth}{|L|L|L|}
\hline

Opções
&
Valores permitidos
&
Descrição
\\
\hline
--ftype
&
basic/data/binary/assembler
&
Tipo do arquivo
\\
\hline
--ascii
&
ascii/binary
&
ASCII flag
\\
\hline\end{tabulary}


Nenhuma opção específica para a criação da imagem


\subsection{Imagem de disquete para o D88 (formato VZ-DOS) - (\emph{d88\_vzdos})}
\label{tools/imgtool:imagem-de-disquete-para-o-d88-formato-vz-dos-d88-vzdos}
Opções específicas de driver para o módulo `d88\_vzdos':

Opções específicas para o arquivo (utilizável com o comando `put')

\noindent\begin{tabulary}{\linewidth}{|L|L|L|}
\hline

Opções
&
Valores permitidos
&
Descrição
\\
\hline
--ftype
&
basic/binary/data
&
Tipo do arquivo
\\
\hline
--fname
&
intern/extern
&
Nome do arquivo
\\
\hline\end{tabulary}


Nenhuma opção específica para a criação da imagem


\subsection{Imagem de disquete para o DSK (formato Basic Master Level 3) - (\emph{dsk\_bml3})}
\label{tools/imgtool:imagem-de-disquete-para-o-dsk-formato-basic-master-level-3-dsk-bml3}
Opções específicas de driver para o módulo `dsk\_bml3':

Opções específicas para o arquivo (utilizável com o comando `put')

\noindent\begin{tabulary}{\linewidth}{|L|L|L|}
\hline

Opções
&
Valores permitidos
&
Descrição
\\
\hline
--ftype
&
basic/data/binary/assembler
&
Tipo do arquivo
\\
\hline
--ascii
&
ascii/binary
&
ASCII flag
\\
\hline\end{tabulary}


Nenhuma opção específica para a criação da imagem


\subsection{Imagem de disquete para o DSK (formato FAT) - (\emph{dsk\_fat})}
\label{tools/imgtool:imagem-de-disquete-para-o-dsk-formato-fat-dsk-fat}
Opções específicas de driver para o módulo `dsk\_fat':

Nenhuma opção específica da imagem

Nenhuma opção específica para a criação da imagem


\subsection{Imagem de disquete para o DSK (disquete Mac HFS) - (\emph{dsk\_mac\_hfs})}
\label{tools/imgtool:imagem-de-disquete-para-o-dsk-disquete-mac-hfs-dsk-mac-hfs}
Opções específicas de driver para o módulo `dsk\_mac\_hfs':

Nenhuma opção específica da imagem

Nenhuma opção específica para a criação da imagem


\subsection{Imagem de disquete DSK (Disquete Mac MFS) - (\emph{dsk\_mac\_mfs})}
\label{tools/imgtool:imagem-de-disquete-dsk-disquete-mac-mfs-dsk-mac-mfs}
Opções específicas de driver para o módulo `dsk\_mac\_mfs':

Nenhuma opção específica da imagem

Nenhuma opção específica para a criação da imagem


\subsection{Imagem de disquete para o DSK (formato OS-9) - (\emph{dsk\_os9})}
\label{tools/imgtool:imagem-de-disquete-para-o-dsk-formato-os-9-dsk-os9}
Opções específicas de driver para o módulo `dsk\_os9':

Nenhuma opção específica da imagem

Nenhuma opção específica para a criação da imagem


\subsection{Imagem de disquete para o DSK (formato ProDOS) - (\emph{dsk\_prodos\_35})}
\label{tools/imgtool:imagem-de-disquete-para-o-dsk-formato-prodos-dsk-prodos-35}
Opções específicas de driver para o módulo `dsk\_prodos\_35':

Nenhuma opção específica da imagem

Nenhuma opção específica para a criação da imagem


\subsection{Imagem de disquete para o DSK (formato ProDOS) - (\emph{dsk\_prodos\_525})}
\label{tools/imgtool:imagem-de-disquete-para-o-dsk-formato-prodos-dsk-prodos-525}
Opções específicas de driver para o módulo `dsk\_prodos\_525':

Nenhuma opção específica da imagem

Nenhuma opção específica para a criação da imagem


\subsection{Imagem de disquete para o DSK (formato RS-DOS) - (\emph{dsk\_rsdos})}
\label{tools/imgtool:imagem-de-disquete-para-o-dsk-formato-rs-dos-dsk-rsdos}
Opções específicas de driver para o módulo `dsk\_rsdos':

Opções específicas para o arquivo (utilizável com o comando `put')

\noindent\begin{tabulary}{\linewidth}{|L|L|L|}
\hline

Opções
&
Valores permitidos
&
Descrição
\\
\hline
--ftype
&
basic/data/binary/assembler
&
Tipo do arquivo
\\
\hline
--ascii
&
ascii/binary
&
ASCII flag
\\
\hline\end{tabulary}


Nenhuma opção específica para a criação da imagem


\subsection{Imagem de disquete para o DSK (formato VZ-DOS) - (\emph{dsk\_vzdos})}
\label{tools/imgtool:imagem-de-disquete-para-o-dsk-formato-vz-dos-dsk-vzdos}
Opções específicas de driver para o módulo `dsk\_vzdos':

Opções específicas para o arquivo (utilizável com o comando `put')

\noindent\begin{tabulary}{\linewidth}{|L|L|L|}
\hline

Opções
&
Valores permitidos
&
Descrição
\\
\hline
--ftype
&
basic/binary/data
&
Tipo do arquivo
\\
\hline
--fname
&
intern/extern
&
Nome do arquivo
\\
\hline\end{tabulary}


Nenhuma opção específica para a criação da imagem


\subsection{Imagem de Disco Formatado (formato Basic Master Level 3) - (\emph{fdi\_bml3})}
\label{tools/imgtool:imagem-de-disco-formatado-formato-basic-master-level-3-fdi-bml3}
Opções específicas de driver para o módulo `fdi\_bml3':

Opções específicas para o arquivo (utilizável com o comando `put')

\noindent\begin{tabulary}{\linewidth}{|L|L|L|}
\hline

Opções
&
Valores permitidos
&
Descrição
\\
\hline
--ftype
&
basic/data/binary/assembler
&
Tipo do arquivo
\\
\hline
--ascii
&
ascii/binary
&
ASCII flag
\\
\hline\end{tabulary}


Nenhuma opção específica para a criação da imagem


\subsection{Imagem de Disco Formatado (formato FAT) - (\emph{fdi\_fat})}
\label{tools/imgtool:imagem-de-disco-formatado-formato-fat-fdi-fat}
Opções específicas de driver para o módulo `fdi\_fat':

Nenhuma opção específica da imagem

Nenhuma opção específica para a criação da imagem


\subsection{Imagem de Disco Formatado (Disquete Mac HFS) - (\emph{fdi\_mac\_hfs})}
\label{tools/imgtool:imagem-de-disco-formatado-disquete-mac-hfs-fdi-mac-hfs}
Opções específicas de driver para o módulo `fdi\_mac\_hfs':

Nenhuma opção específica da imagem

Nenhuma opção específica para a criação da imagem


\subsection{Imagem de Disco Formatado (Disquete Mac MFS) - (\emph{fdi\_mac\_mfs})}
\label{tools/imgtool:imagem-de-disco-formatado-disquete-mac-mfs-fdi-mac-mfs}
Opções específicas de driver para o módulo `fdi\_mac\_mfs':

Nenhuma opção específica da imagem

Nenhuma opção específica para a criação da imagem


\subsection{Imagem de Disco Formatado (formato OS-9) - (\emph{fdi\_os9})}
\label{tools/imgtool:imagem-de-disco-formatado-formato-os-9-fdi-os9}
Opções específicas de driver para o módulo `fdi\_os9':

Nenhuma opção específica da imagem

Nenhuma opção específica para a criação da imagem


\subsection{Imagem de Disco Formatado (formato ProDOS) - (\emph{fdi\_prodos\_35})}
\label{tools/imgtool:imagem-de-disco-formatado-formato-prodos-fdi-prodos-35}
Opções específicas de driver para o módulo `fdi\_prodos\_35':

Nenhuma opção específica da imagem

Nenhuma opção específica para a criação da imagem


\subsection{Imagem de Disco Formatado (formato ProDOS) - (\emph{fdi\_prodos\_525})}
\label{tools/imgtool:imagem-de-disco-formatado-formato-prodos-fdi-prodos-525}
Opções específicas de driver para o módulo `fdi\_prodos\_525':

Nenhuma opção específica da imagem

Nenhuma opção específica para a criação da imagem


\subsection{Imagem de Disco Formatado (formato RS-DOS) - (\emph{fdi\_rsdos})}
\label{tools/imgtool:imagem-de-disco-formatado-formato-rs-dos-fdi-rsdos}
Opções específicas de driver para o módulo `fdi\_rsdos':

Opções específicas para o arquivo (utilizável com o comando `put')

\noindent\begin{tabulary}{\linewidth}{|L|L|L|}
\hline

Opções
&
Valores permitidos
&
Descrição
\\
\hline
--ftype
&
basic/data/binary/assembler
&
Tipo do arquivo
\\
\hline
--ascii
&
ascii/binary
&
ASCII flag
\\
\hline\end{tabulary}


Nenhuma opção específica para a criação da imagem


\subsection{Imagem de Disco Formatado (formato VZ-DOS) - (\emph{fdi\_vzdos})}
\label{tools/imgtool:imagem-de-disco-formatado-formato-vz-dos-fdi-vzdos}
Opções específicas de driver para o módulo `fdi\_vzdos':

Opções específicas para o arquivo (utilizável com o comando `put')

\noindent\begin{tabulary}{\linewidth}{|L|L|L|}
\hline

Opções
&
Valores permitidos
&
Descrição
\\
\hline
--ftype
&
basic/binary/data
&
Tipo do arquivo
\\
\hline
--fname
&
intern/extern
&
Nome do arquivo
\\
\hline\end{tabulary}


Nenhuma opção específica para a criação da imagem


\subsection{Cartão de memória para o HP48 SX/GX - (\emph{hp48})}
\label{tools/imgtool:cartao-de-memoria-para-o-hp48-sx-gx-hp48}
Opções específicas de driver para o módulo `hp48':

Nenhuma opção específica da imagem

Opções específicas para a criação da imagem (utilizável com o comando `create'):

\noindent\begin{tabulary}{\linewidth}{|L|L|L|}
\hline

Opção
&
Valores permitidos
&
Descrição
\\
\hline
--flash
&
AT45DB041/AT45DB081/AT45DB161
&
Flash Tipo
\\
\hline\end{tabulary}



\subsection{Imagem de disquete IMD (formato Basic Master Level 3) - (\emph{imd\_bml3})}
\label{tools/imgtool:imagem-de-disquete-imd-formato-basic-master-level-3-imd-bml3}
Opções específicas de driver para o módulo `imd\_bml3':

Opções específicas para o arquivo (utilizável com o comando `put')

\noindent\begin{tabulary}{\linewidth}{|L|L|L|}
\hline

Opções
&
Valores permitidos
&
Descrição
\\
\hline
--ftype
&
basic/data/binary/assembler
&
Tipo do arquivo
\\
\hline
--ascii
&
ascii/binary
&
ASCII flag
\\
\hline\end{tabulary}


Nenhuma opção específica para a criação da imagem


\subsection{Imagem de disquete IMD (formato FAT) - (\emph{imd\_fat})}
\label{tools/imgtool:imagem-de-disquete-imd-formato-fat-imd-fat}
Opções específicas de driver para o módulo `imd\_fat':

Nenhuma opção específica da imagem

Nenhuma opção específica para a criação da imagem


\subsection{Imagem de disquete IMD (disquete Mac HFS) - (\emph{imd\_mac\_hfs})}
\label{tools/imgtool:imagem-de-disquete-imd-disquete-mac-hfs-imd-mac-hfs}
Opções específicas de driver para o módulo `imd\_mac\_hfs':

Nenhuma opção específica da imagem

Nenhuma opção específica para a criação da imagem


\subsection{Imagem de disquete IMD (Disquete Mac MFS) - (\emph{imd\_mac\_mfs})}
\label{tools/imgtool:imagem-de-disquete-imd-disquete-mac-mfs-imd-mac-mfs}
Opções específicas de driver para o módulo `imd\_mac\_mfs':

Nenhuma opção específica da imagem

Nenhuma opção específica para a criação da imagem


\subsection{Imagem de disquete IMD (formato OS-9) - (\emph{imd\_os9})}
\label{tools/imgtool:imagem-de-disquete-imd-formato-os-9-imd-os9}
Opções específicas de driver para o módulo `imd\_os9':

Nenhuma opção específica da imagem

Nenhuma opção específica para a criação da imagem


\subsection{Imagem de disquete IMD (formato ProDOS) - (\emph{imd\_prodos\_35})}
\label{tools/imgtool:imagem-de-disquete-imd-formato-prodos-imd-prodos-35}
Opções específicas de driver para o módulo `imd\_prodos\_35':

Nenhuma opção específica da imagem

Nenhuma opção específica para a criação da imagem


\subsection{Imagem de disquete IMD (formato ProDOS) - (\emph{imd\_prodos\_525})}
\label{tools/imgtool:imagem-de-disquete-imd-formato-prodos-imd-prodos-525}
Opções específicas de driver para o módulo `imd\_prodos\_525':

Nenhuma opção específica da imagem

Nenhuma opção específica para a criação da imagem


\subsection{Imagem de disquete IMD (formato RS-DOS) - (\emph{imd\_rsdos})}
\label{tools/imgtool:imagem-de-disquete-imd-formato-rs-dos-imd-rsdos}
Opções específicas de driver para o módulo `imd\_rsdos':

Opções específicas para o arquivo (utilizável com o comando `put')

\noindent\begin{tabulary}{\linewidth}{|L|L|L|}
\hline

Opções
&
Valores permitidos
&
Descrição
\\
\hline
--ftype
&
basic/data/binary/assembler
&
Tipo do arquivo
\\
\hline
--ascii
&
ascii/binary
&
ASCII flag
\\
\hline\end{tabulary}


Nenhuma opção específica para a criação da imagem


\subsection{Imagem de disquete IMD (formato VZ-DOS) - (\emph{imd\_vzdos})}
\label{tools/imgtool:imagem-de-disquete-imd-formato-vz-dos-imd-vzdos}
Opções específicas de driver para o módulo `imd\_vzdos':

Opções específicas para o arquivo (utilizável com o comando `put')

\noindent\begin{tabulary}{\linewidth}{|L|L|L|}
\hline

Opções
&
Valores permitidos
&
Descrição
\\
\hline
--ftype
&
basic/binary/data
&
Tipo do arquivo
\\
\hline
--fname
&
intern/extern
&
Nome do arquivo
\\
\hline\end{tabulary}


Nenhuma opção específica para a criação da imagem


\subsection{Imagem de disco rígido para o  MESS - (\emph{mess\_hd})}
\label{tools/imgtool:imagem-de-disco-rigido-para-o-mess-mess-hd}
Opções específicas de driver para o módulo `mess\_hd':

Nenhuma opção específica da imagem

Opções específicas para a criação da imagem (utilizável com o comando `create'):

\noindent\begin{tabulary}{\linewidth}{|L|L|L|}
\hline

Opções
&
Valores permitidos
&
Descrição
\\
\hline
--blocksize
&
1-2048
&
Setores por Bloco
\\
\hline
--cylinders
&
1-65536
&
Cilindros
\\
\hline
--heads
&
1-64
&
Cabeças
\\
\hline
--sectors
&
1-4096
&
Setores Totais
\\
\hline
--seclen
&
128/256/512/1024/2048/4096/8192/16384/32768/65536
&
Bytes por Setor
\\
\hline\end{tabulary}



\subsection{Disquete para o TI99 (formato PC99) - (\emph{pc99fm})}
\label{tools/imgtool:disquete-para-o-ti99-formato-pc99-pc99fm}
Opções específicas de driver para o módulo `pc99fm':

Nenhuma opção específica da imagem

Nenhuma opção específica para a criação da imagem


\subsection{Disquete para o TI99 (formato PC99 MFM) - (\emph{pc99mfm})}
\label{tools/imgtool:disquete-para-o-ti99-formato-pc99-mfm-pc99mfm}
Opções específicas de driver para o módulo `pc99mfm':

Nenhuma opção específica da imagem

Nenhuma opção específica para a criação da imagem


\subsection{Imagem de disco para o PC CHD - (\emph{pc\_chd})}
\label{tools/imgtool:imagem-de-disco-para-o-pc-chd-pc-chd}
Opções específicas de driver para o módulo `pc\_chd':

Nenhuma opção específica da imagem

Opções específicas para a criação da imagem (utilizável com o comando `create'):

\noindent\begin{tabulary}{\linewidth}{|L|L|L|}
\hline

Opções
&
Valores permitidos
&
Descrição
\\
\hline
--cylinders
&
10/20/30/40/50/60/70/80/90/100/110/120/130/140/150/160/170/180/190/200
&
Cilindros
\\
\hline
--heads
&
1-16
&
Cabeças
\\
\hline
--sectors
&
1-63
&
Setores
\\
\hline\end{tabulary}



\subsection{Imagem de disquete para o PC (formato FAT) - (\emph{pc\_dsk\_fat})}
\label{tools/imgtool:imagem-de-disquete-para-o-pc-formato-fat-pc-dsk-fat}
Opções específicas de driver para o módulo `pc\_dsk\_fat':

Nenhuma opção específica da imagem

Opções específicas para a criação da imagem (utilizável com o comando `create'):

\noindent\begin{tabulary}{\linewidth}{|L|L|L|}
\hline

Opções
&
Valores permitidos
&
Descrição
\\
\hline
--heads
&
1-2
&
Cabeças
\\
\hline
--tracks
&
40/80
&
Trilhas
\\
\hline
--sectors
&
8/9/10/15/18/36
&
Setores
\\
\hline\end{tabulary}



\subsection{Psion Organiser II Datapack - (\emph{psionpack})}
\label{tools/imgtool:psion-organiser-ii-datapack-psionpack}
Opções específicas de driver para o módulo `psionpack':

Opções específicas para o arquivo (utilizável com o comando `put')

\noindent\begin{tabulary}{\linewidth}{|L|L|L|}
\hline

Opções
&
Valores permitidos
&
Descrição
\\
\hline
--type
&
OB3/OPL/ODB
&
Tipo do arquivo
\\
\hline
--id
&
0/145-255
&
ID do arquivo
\\
\hline\end{tabulary}


Opções específicas para a criação da imagem (utilizável com o comando `create'):

\noindent\begin{tabulary}{\linewidth}{|L|L|L|}
\hline

opções
&
Valores permitidos
&
Descrição
\\
\hline
--size
&
8k/16k/32k/64k/128k
&
Tamanho do datapack
\\
\hline
--ram
&
0/1
&
EPROM/RAM datapack
\\
\hline
--paged
&
0/1
&
linear/paged datapack
\\
\hline
--protect
&
0/1
&
datapack com escrita protegida
\\
\hline
--boot
&
0/1
&
datapack inicializável
\\
\hline
--copy
&
0/1
&
datapack com permissão de cópia
\\
\hline\end{tabulary}



\subsection{Imagem de disquete para o Teledisk (formato Basic Master Level 3) - (\emph{td0\_bml3})}
\label{tools/imgtool:imagem-de-disquete-para-o-teledisk-formato-basic-master-level-3-td0-bml3}
Opções específicas de driver para o módulo `td0\_bml3':

Opções específicas para o arquivo (utilizável com o comando `put')

\noindent\begin{tabulary}{\linewidth}{|L|L|L|}
\hline

Opções
&
Valores permitidos
&
Descrição
\\
\hline
--ftype
&
basic/data/binary/assembler
&
Tipo do arquivo
\\
\hline
--ascii
&
ascii/binary
&
ASCII flag
\\
\hline\end{tabulary}


Nenhuma opção específica para a criação da imagem


\subsection{Imagem de disquete para o Teledisk (formato FAT) - (\emph{td0\_fat})}
\label{tools/imgtool:imagem-de-disquete-para-o-teledisk-formato-fat-td0-fat}
Opções específicas de driver para o módulo `td0\_fat':

Nenhuma opção específica da imagem

Nenhuma opção específica para a criação da imagem


\subsection{Imagem de disquete para o Teledisk (Disquete Mac HFS) - (\emph{td0\_mac\_hfs})}
\label{tools/imgtool:imagem-de-disquete-para-o-teledisk-disquete-mac-hfs-td0-mac-hfs}
Opções específicas de driver para o módulo `td0\_mac\_hfs':

Nenhuma opção específica da imagem

Nenhuma opção específica para a criação da imagem


\subsection{Imagem de disquete para o Teledisk (Disquete Mac MFS) - (\emph{td0\_mac\_mfs})}
\label{tools/imgtool:imagem-de-disquete-para-o-teledisk-disquete-mac-mfs-td0-mac-mfs}
Opções específicas de driver para o módulo `td0\_mac\_mfs':

Nenhuma opção específica da imagem

Nenhuma opção específica para a criação da imagem


\subsection{Imagem de disquete para o Teledisk (OS-9 format) - (\emph{td0\_os9})}
\label{tools/imgtool:imagem-de-disquete-para-o-teledisk-os-9-format-td0-os9}
Opções específicas de driver para o módulo `td0\_os9':

Nenhuma opção específica da imagem

Nenhuma opção específica para a criação da imagem


\subsection{Imagem de disquete para o Teledisk (formato ProDOS) - (\emph{td0\_prodos\_35})}
\label{tools/imgtool:imagem-de-disquete-para-o-teledisk-formato-prodos-td0-prodos-35}
Opções específicas de driver para o módulo `td0\_prodos\_35':

Nenhuma opção específica da imagem

Nenhuma opção específica para a criação da imagem


\subsection{Imagem de disquete para o Teledisk (formato ProDOS) - (\emph{td0\_prodos\_525})}
\label{tools/imgtool:imagem-de-disquete-para-o-teledisk-formato-prodos-td0-prodos-525}
Opções específicas de driver para o módulo `td0\_prodos\_525':

Nenhuma opção específica da imagem

Nenhuma opção específica para a criação da imagem


\subsection{Imagem de disquete para o Teledisk (RS-DOS format) - (\emph{td0\_rsdos})}
\label{tools/imgtool:imagem-de-disquete-para-o-teledisk-rs-dos-format-td0-rsdos}
Opções específicas de driver para o módulo `td0\_rsdos':

Opções específicas para o arquivo (utilizável com o comando `put')

\noindent\begin{tabulary}{\linewidth}{|L|L|L|}
\hline

Opções
&
Valores permitidos
&
Descrição
\\
\hline
--ftype
&
basic/data/binary/assembler
&
Tipo do arquivo
\\
\hline
--ascii
&
ascii/binary
&
ASCII flag
\\
\hline\end{tabulary}


Nenhuma opção específica para a criação da imagem


\subsection{Imagem de disquete para o Teledisk (VZ-DOS format) - (\emph{td0\_vzdos})}
\label{tools/imgtool:imagem-de-disquete-para-o-teledisk-vz-dos-format-td0-vzdos}
Opções específicas de driver para o módulo `td0\_vzdos':

Opções específicas para o arquivo (utilizável com o comando `put')

\noindent\begin{tabulary}{\linewidth}{|L|L|L|}
\hline

Opções
&
Valores permitidos
&
Descrição
\\
\hline
--ftype
&
basic/binary/data
&
Tipo do arquivo
\\
\hline
--fname
&
intern/extern
&
Nome do arquivo
\\
\hline\end{tabulary}


Nenhuma opção específica para a criação da imagem


\subsection{Imagem de disquete Thomson .fd, formato BASIC - (\emph{thom\_fd})}
\label{tools/imgtool:imagem-de-disquete-thomson-fd-formato-basic-thom-fd}
Opções específicas de driver para o módulo `thom\_fd':

Opções específicas para o arquivo (utilizável com o comando `put')

\noindent\begin{tabulary}{\linewidth}{|L|L|L|}
\hline

Opções
&
Valores permitidos
&
Descrição
\\
\hline
--ftype
&
auto/B/D/M/A
&
Tipo do arquivo
\\
\hline
--format
&
auto/B/A
&
Flag de formato
\\
\hline
--comment
&
(string)
&
Comentário
\\
\hline\end{tabulary}


Opções específicas para a criação da imagem (utilizável com o comando `create'):

\noindent\begin{tabulary}{\linewidth}{|L|L|L|}
\hline

Opções
&
Valores permitidos
&
Descrição
\\
\hline
--heads
&
1-2
&
Cabeças
\\
\hline
--tracks
&
40/80
&
Trilhas
\\
\hline
--density
&
SD/DD
&
Densidade
\\
\hline
--name
&
(string)
&
Nome do disquete
\\
\hline\end{tabulary}



\subsection{Imagem de disquete Thomson .fd, formato BASIC - (\emph{thom\_qd})}
\label{tools/imgtool:imagem-de-disquete-thomson-fd-formato-basic-thom-qd}
Opções específicas de driver para o módulo `thom\_qd':

Opções específicas para o arquivo (utilizável com o comando `put')

\noindent\begin{tabulary}{\linewidth}{|L|L|L|}
\hline

Opções
&
Valores permitidos
&
Descrição
\\
\hline
--ftype
&
auto/B/D/M/A
&
Tipo do arquivo
\\
\hline
--format
&
auto/B/A
&
Flag de formato
\\
\hline
--comment
&
(string)
&
Comentário
\\
\hline\end{tabulary}


Opções específicas para a criação da imagem (utilizável com o comando `create'):

\noindent\begin{tabulary}{\linewidth}{|L|L|L|}
\hline

Opções
&
Valores permitidos
&
Descrição
\\
\hline
--heads
&
1-2
&
Cabeças
\\
\hline
--tracks
&
25
&
Trilhas
\\
\hline
--density
&
SD/DD
&
Densidade
\\
\hline
--name
&
(string)
&
Nome do disquete
\\
\hline\end{tabulary}



\subsection{Imagem de disquete Thomson .fd, formato BASIC - (\emph{thom\_sap})}
\label{tools/imgtool:imagem-de-disquete-thomson-fd-formato-basic-thom-sap}
Opções específicas de driver para o módulo `thom\_sap':

Opções específicas para o arquivo (utilizável com o comando `put')

\noindent\begin{tabulary}{\linewidth}{|L|L|L|}
\hline

Opções
&
Valores permitidos
&
Descrição
\\
\hline
--ftype
&
auto/B/D/M/A
&
Tipo do arquivo
\\
\hline
--format
&
auto/B/A
&
Flag de formato
\\
\hline
--comment
&
(string)
&
Comentário
\\
\hline\end{tabulary}


Opções específicas para a criação da imagem (utilizável com o comando `create'):

\noindent\begin{tabulary}{\linewidth}{|L|L|L|}
\hline

Opções
&
Valores permitidos
&
Descrição
\\
\hline
--heads
&
1
&
Cabeças
\\
\hline
--tracks
&
40/80
&
Trilhas
\\
\hline
--density
&
SD/DD
&
Densidade
\\
\hline
--name
&
(string)
&
Nome do disquete
\\
\hline\end{tabulary}



\subsection{Imagem de Disco Rígido para o TI990 - (\emph{ti990hd})}
\label{tools/imgtool:imagem-de-disco-rigido-para-o-ti990-ti990hd}
Opções específicas de driver para o módulo `ti990hd':

Nenhuma opção específica da imagem

Opções específicas para a criação da imagem (utilizável com o comando `create'):

\noindent\begin{tabulary}{\linewidth}{|L|L|L|}
\hline

Opções
&
Valores permitidos
&
Descrição
\\
\hline
--cylinders
&
1-2047
&
Cilindros
\\
\hline
--heads
&
1-31
&
Cabeças
\\
\hline
--sectors
&
1-256
&
Setores
\\
\hline
--bytes per sector
&
(typically 25256-512 256-512
&
Bytes Por Setor {[}A fazer: O imgtool está com falhas nesta seção{]}
\\
\hline\end{tabulary}



\subsection{Disquete para o TI99 (formato antigo do MESS) - (\emph{ti99\_old})}
\label{tools/imgtool:disquete-para-o-ti99-formato-antigo-do-mess-ti99-old}
Opções específicas de driver para o módulo `ti99\_old':

Nenhuma opção específica da imagem

Opções específicas para a criação da imagem (utilizável com o comando `create'):

\noindent\begin{tabulary}{\linewidth}{|L|L|L|}
\hline

Opções
&
Valores permitidos
&
Descrição
\\
\hline
--sides
&
1-2
&
Lados
\\
\hline
--tracks
&
1-80
&
Trilhas
\\
\hline
--sectors
&
1-36
&
Setores (1-\textgreater{}9 para DS, 1-\textgreater{}18 para DD, 1-\textgreater{}36 para AD)
\\
\hline
--protection
&
0-1
&
Proteção (0 para normal, 1 para protegido)
\\
\hline
--density
&
Auto/DS/DD/AD
&
Densidade
\\
\hline\end{tabulary}



\subsection{Disco Rígido para o TI99 - (\emph{ti99hd})}
\label{tools/imgtool:disco-rigido-para-o-ti99-ti99hd}
Opções específicas de driver para o módulo `ti99hd':

Nenhuma opção específica da imagem

Nenhuma opção específica para a criação da imagem


\subsection{Disquete para o TI99 (formato V9T9) - (\emph{v9t9})}
\label{tools/imgtool:disquete-para-o-ti99-formato-v9t9-v9t9}
Opções específicas de driver para o módulo `v9t9':

Nenhuma opção específica da imagem

Opções específicas para a criação da imagem (utilizável com o comando `create'):

\noindent\begin{tabulary}{\linewidth}{|L|L|L|}
\hline

Opções
&
Valores permitidos
&
Descrição
\\
\hline
--sides
&
1-2
&
Lados
\\
\hline
--tracks
&
1-80
&
Trilhas
\\
\hline
--sectors
&
1-36
&
Setores (1-\textgreater{}9 para DS, 1-\textgreater{}18 para DD, 1-\textgreater{}36 para AD)
\\
\hline
--protection
&
0-1
&
Proteção (0 para normal, 1 para protegido)
\\
\hline
--density
&
Auto/DS/DD/AD
&
Densidade
\\
\hline\end{tabulary}



\subsection{Imagem de disco para o Laser/VZ (formato VZ-DOS) - (\emph{vtech1\_vzdos})}
\label{tools/imgtool:imagem-de-disco-para-o-laser-vz-formato-vz-dos-vtech1-vzdos}
Opções específicas de driver para o módulo `vtech1\_vzdos':

Opções específicas para o arquivo (utilizável com o comando `put')

\noindent\begin{tabulary}{\linewidth}{|L|L|L|}
\hline

Opções
&
Valores permitidos
&
Descrição
\\
\hline
--ftype
&
basic/binary/data
&
Tipo do arquivo
\\
\hline
--fname
&
intern/extern
&
Nome do arquivo
\\
\hline\end{tabulary}


Opções específicas para a criação da imagem (utilizável com o comando `create'):

\noindent\begin{tabulary}{\linewidth}{|L|L|L|}
\hline

Opções
&
Valores permitidos
&
Descrição
\\
\hline
--heads
&
1
&
Cabeças
\\
\hline
--tracks
&
40
&
Trilhas
\\
\hline
--sectors
&
16
&
Setores
\\
\hline
--sectorlength
&
154
&
Bytes por Setor
\\
\hline
--firstsectorid
&
0
&
Primeiro Setor
\\
\hline\end{tabulary}


{[}A fazer: preencher as estruturas e descrever melhor os comandos.
Essas descrições vieram do arquivo imgtool.txt e estão muito
simplificadas{]}


\section{Castool - Uma ferramenta genérica de manipulação de imagens de fita k7 para o MAME}
\label{tools/castool::doc}\label{tools/castool:castool-uma-ferramenta-generica-de-manipulacao-de-imagens-de-fita-k7-para-o-mame}
Castool é uma ferramenta para manutenção e manipulação de imagens de
fita k7 que os usuários precisarão aprender a usar. O MAME já é
compatível com formatos de áudio em .WAV, porém muitas das imagens
existentes, podem conter outros formatos como .TAP vinda de fitas
Comodore 64, .CAS para Tandy Color Computer e assim por diante.
A ferramenta Castool irá converter esses outros formatos para .WAV
caso seja usada no MAME.

A ferramenta faz parte do projeto MAME. Ele compartilha grande parte do
seu código com o MAME e a mesma não existiria se não fosse pelo MAME.
Logo os termos da sua distribuição seguem os mesmos termos existentes
para o MAME.
Favor ler a toda {\hyperref[license:mame\string-license]{\sphinxcrossref{\DUrole{std,std-ref}{LICENÇA}}}} com atenção.


\section{Usando o Castool}
\label{tools/castool:usando-o-castool}
Castool é um programa de linha comando que contém um conjunto simples de
instruções. Os comandos são invocados usando uma cadência de instruções,
exemplo:
\begin{quote}

\textbf{castool convert} \textless{}\emph{format}\textgreater{} \textless{}\emph{inputfile}\textgreater{} \textless{}\emph{outputfile}\textgreater{}
\end{quote}
\begin{itemize}
\item {} 
\textbf{\textless{}format\textgreater{}} é o formato da imagem

\item {} 
\textbf{\textless{}inputfile\textgreater{}} é o nome do arquivo que você estiver convertendo

\item {} 
\textbf{\textless{}outputfile\textgreater{}} é o nome de saída do arquivo WAV

\end{itemize}

Exemplo de uso:

\begin{DUlineblock}{0em}
\item[] \sphinxcode{castool convert coco zaxxon.cas zaxxon.wav}
\item[] \sphinxcode{castool convert cbm arkanoid.tap arkanoid.wav}
\item[] \sphinxcode{castool convert ddp mybasicprogram.ddp mybasicprogram.wav}
\end{DUlineblock}


\section{Formatos compatívels}
\label{tools/castool:formatos-compativels}
Esses são os formatos compatíveis com o Castool para a conversão para
o format .WAV.

\textbf{A26}
\begin{quote}

Imagem do Atari 2600 SuperCharger

Nome da Extensão: a26
\end{quote}

\textbf{APF}
\begin{quote}

APF Imagination Machine

Nome da Extensão: cas, cpf, apt
\end{quote}

\textbf{ATOM}
\begin{quote}

Acorn Atom

Nome da Extensão: tap, csw, uef
\end{quote}

\textbf{BBC}
\begin{quote}

Acorn BBC \& Electron

Nome da Extensão: csw, uef
\end{quote}

\textbf{CBM}
\begin{quote}

Série Comodore 8-bits

Nome da Extensão: tap
\end{quote}

\textbf{CDT}
\begin{quote}

Amstrad CPC

Nome da Extensão: cdt
\end{quote}

\textbf{CGENIE}
\begin{quote}

EACA Colour Genie

Nome da Extensão: cas
\end{quote}

\textbf{COCO}
\begin{quote}

Tandy Radio Shack Color Computer

Nome da Extensão: cas
\end{quote}

\textbf{CSW}
\begin{quote}

Compressed Square Wave

Nome da Extensão: csw
\end{quote}

\textbf{DDP}
\begin{quote}

Coleco ADAM

Nome da Extensão: ddp
\end{quote}

\textbf{FM7}
\begin{quote}

Fujitsu FM-7

Nome da Extensão: t77
\end{quote}

\textbf{FMSX}
\begin{quote}

MSX

Nome da Extensão: tap, cas
\end{quote}

\textbf{GTP}
\begin{quote}

Elektronika inzenjering Galaksija

Nome da Extensão: gtp
\end{quote}

\textbf{HECTOR}
\begin{quote}

Micronique Hector \& Interact Family Computer

Nome da Extensão: k7, cin, for
\end{quote}

\textbf{JUPITER}
\begin{quote}

Jupiter Cantab Jupiter Ace

Nome da Extensão: tap
\end{quote}

\textbf{KC85}
\begin{quote}

VEB Mikroelektronik KC 85

Nome da Extensão: kcc, kcb, tap, 853, 854, 855, tp2, kcm, sss
\end{quote}

\textbf{KIM1}
\begin{quote}

MOS KIM-1

Nome da Extensão: kim, kim1
\end{quote}

\textbf{LVIV}
\begin{quote}

PK-01 Lviv

Nome da Extensão: lvt, lvr, lv0, lv1, lv2, lv3
\end{quote}

\textbf{MO5}
\begin{quote}

Thomson MO-series

Nome da Extensão: k5, k7
\end{quote}

\textbf{MZ}
\begin{quote}

Sharp MZ-700

Nome da Extensão: m12, mzf, mzt
\end{quote}

\textbf{ORAO}
\begin{quote}

PEL Varazdin Orao

Nome da Extensão: tap
\end{quote}

\textbf{ORIC}
\begin{quote}

Tangerine Oric

Nome da Extensão: tap
\end{quote}

\textbf{PC6001}
\begin{quote}

NEC PC-6001

Nome da Extensão: cas
\end{quote}

\textbf{PHC25}
\begin{quote}

Sanyo PHC-25

Nome da Extensão: phc
\end{quote}

\textbf{PMD85}
\begin{quote}

Tesla PMD-85

Nome da Extensão: pmd, tap, ptp
\end{quote}

\textbf{PRIMO}
\begin{quote}

Microkey Primo

Nome da Extensão: ptp
\end{quote}

\textbf{RKU}
\begin{quote}

UT-88

Nome da Extensão: rku
\end{quote}

\textbf{RK8}
\begin{quote}

Mikro-80

Nome da Extensão: rk8
\end{quote}

\textbf{RKS}
\begin{quote}

Specialist

Nome da Extensão: rks
\end{quote}

\textbf{RKO}
\begin{quote}

Orion

Nome da Extensão: rko
\end{quote}

\textbf{RKR}
\begin{quote}

Radio-86RK

Nome da Extensão: rk, rkr, gam, g16, pki
\end{quote}

\textbf{RKA}
\begin{quote}

Zavod BRA Apogee BK-01

Nome da Extensão: rka
\end{quote}

\textbf{RKM}
\begin{quote}

Mikrosha

Nome da Extensão: rkm
\end{quote}

\textbf{RKP}
\begin{quote}

SAM SKB VM Partner-01.01

Nome da Extensão: rkp
\end{quote}

\textbf{SC3000}
\begin{quote}

Sega SC-3000

Nome da Extensão: bit
\end{quote}

\textbf{SOL20}
\begin{quote}

PTC SOL-20

Nome da Extensão: svt
\end{quote}

\textbf{SORCERER}
\begin{quote}

Exidy Sorcerer

Nome da Extensão: tape
\end{quote}

\textbf{SORDM5}
\begin{quote}

Sord M5

Nome da Extensão: cas
\end{quote}

\textbf{SPC1000}
\begin{quote}

Samsung SPC-1000

Nome da Extensão: tap, cas
\end{quote}

\textbf{SVI}
\begin{quote}

Spectravideo SVI-318 \& SVI-328

Nome da Extensão: cas
\end{quote}

\textbf{TO7}
\begin{quote}

Thomson TO-series

Nome da Extensão: k7
\end{quote}

\textbf{TRS8012}
\begin{quote}

TRS-80 Level 2

Nome da Extensão: cas
\end{quote}

\textbf{TVC64}
\begin{quote}

Videoton TVC 64

Nome da Extensão: cas
\end{quote}

\textbf{TZX}
\begin{quote}

Sinclair ZX Spectrum

Nome da Extensão: tzx, tap, blk
\end{quote}

\textbf{VG5K}
\begin{quote}

Philips VG 5000

Nome da Extensão: k7
\end{quote}

\textbf{VTECH1}
\begin{quote}

Video Technology Laser 110-310

Nome da Extensão: cas
\end{quote}

\textbf{VTECH2}
\begin{quote}

Video Technology Laser 350-700

Nome da Extensão: cas
\end{quote}

\textbf{X07}
\begin{quote}

Canon X-07

Nome da Extensão: k7, lst, cas
\end{quote}

\textbf{X1}
\begin{quote}

Sharp X1

Nome da Extensão: tap
\end{quote}

\textbf{ZX80\_O}
\begin{quote}

Sinclair ZX80

Nome da Extensão: o, 80
\end{quote}

\textbf{ZX81\_P}
\begin{quote}

Sinclair ZX81

Nome da Extensão: p, 81
\end{quote}


\section{Floptool - Uma ferramenta genérica de manipulação de imagens de disquete para o MAME}
\label{tools/floptool:floptool-uma-ferramenta-generica-de-manipulacao-de-imagens-de-disquete-para-o-mame}\label{tools/floptool::doc}
Floptool é uma ferramenta para manutenção e manipulação de imagens de
disquete que os usuários precisam aprender a usar. O MAME já é
compatível com formatos de áudio em .WAV, porém muitas das imagens
existentes, podem conter outros formatos como .TAP vinda de fitas
Comodore 64, .CAS para Tandy Color Computer e assim por diante.
A ferramenta Castool irá converter esses outros formatos para .WAV caso
seja usada no MAME.

A ferramenta faz parte do projeto MAME compartilhando grande parte do
seu código e a mesma não existiria se não fosse pelo MAME.
Logo, os termos da sua distribuição seguem os mesmos termos existentes
para o MAME. Favor ler a toda {\hyperref[license:mame\string-license]{\sphinxcrossref{\DUrole{std,std-ref}{LICENÇA}}}} com atenção.


\section{Usando o Floptool}
\label{tools/floptool:usando-o-floptool}
Floptool é um programa de linha comando que contém um conjunto simples
de instruções. Os comandos são invocados usando uma cadência de
instruções, exemplo:

\begin{DUlineblock}{0em}
\item[] \textbf{floptool identify} \textless{}\emph{inputfile}\textgreater{} {[}\textless{}\emph{inputfile}\textgreater{} ...{]}
\item[] \textbf{floptool convert} {[}\emph{input\_format} \textbar{} \emph{auto}{]} output\_format \textless{}\emph{inputfile}\textgreater{} \textless{}\emph{outputile}\textgreater{}
\end{DUlineblock}
\begin{itemize}
\item {} 
\textbf{\textless{}format\textgreater{}} é o formato da imagem

\item {} 
\textbf{\textless{}input\_format\textgreater{}} é o formato do arquivo de entrada, se for desconhecido, use auto

\item {} 
\textbf{\textless{}output\_format\textgreater{}} é o formato de destino do arquivo

\item {} 
\textbf{\textless{}inputfile\textgreater{}} é o nome do arquivo que você deseja converter ou identificar

\item {} 
\textbf{\textless{}outputfile\textgreater{}} é o nome do arquivo final que foi convertido

\end{itemize}

Exemplo de uso:

\begin{DUlineblock}{0em}
\item[] floptool convert coco zaxxon.cas zaxxon.wav
\item[] floptool convert cbm arkanoid.tap arkanoid.wav
\item[] floptool convert ddp mybasicprogram.ddp mybasicprogram.wav
\end{DUlineblock}
\clearpage

\section{Formatos compatíveis}
\label{tools/floptool:formatos-compativeis}
Esses são os formatos compatíveis com o Floptool para a conversão em
outros formatos.

\textbf{MFI}
\begin{quote}

Imagem de disquete do mame

Nome da Extensão: mfi
\end{quote}

\textbf{DFI}
\begin{quote}

Formato de extração DiscFerret flux

Nome da Extensão: dfi
\end{quote}

\textbf{IPF}
\begin{quote}

Imagem de disquete do SPS

Nome da Extensão: ipf
\end{quote}

\textbf{MFM}
\begin{quote}

Imagem de disquete do HxC Floppy Emulator

Nome da Extensão: mfm
\end{quote}

\textbf{ADF}
\begin{quote}

Imagem de disquete do Amiga ADF

Nome da Extensão: adf
\end{quote}

\textbf{ST}
\begin{quote}

Imagem de disquete do Atari ST

Nome da Extensão: st
\end{quote}

\textbf{MSA}
\begin{quote}

Imagem de disquete do Atari MSA

Nome da Extensão: msa
\end{quote}

\textbf{PASTI}
\begin{quote}

Imagem de disquete do Atari PASTI

Nome da Extensão: stx
\end{quote}

\textbf{DSK}
\begin{quote}

Formato CPC DSK

Nome da Extensão: dsk
\end{quote}

\textbf{D88}
\begin{quote}

Imagem de disco do D88

Nome da Extensão: d77, d88, 1dd
\end{quote}

\textbf{IMD}
\begin{quote}

Imagem de disco do IMD

Nome da Extensão: imd
\end{quote}

\textbf{TD0}
\begin{quote}

Imagem de disco do Teledisk

Nome da Extensão: td0
\end{quote}

\textbf{CQM}
\begin{quote}

Imagem de disco do CopyQM

Nome da Extensão: cqm, cqi, dsk
\end{quote}

\textbf{PC}
\begin{quote}

Imagem de disquete de PC

Nome da Extensão: dsk, ima, img, ufi, 360
\end{quote}

\textbf{NASLITE}
\begin{quote}

Imagem de disco do NASLite

Nome da Extensão: img
\end{quote}

\textbf{DC42}
\begin{quote}

Imagem DiskCopy 4.2

Nome da Extensão: dc42
\end{quote}

\textbf{A2\_16SECT}
\begin{quote}

Imagem de disco do Apple II com 16 setores

Nome da Extensão: dsk, do, po
\end{quote}

\textbf{A2\_RWTS18}
\begin{quote}

Imagem tipo RWTS18 do Apple II

Nome da Extensão: rti
\end{quote}

\textbf{A2\_EDD}
\begin{quote}

Imagem EDD do Apple II

Nome da Extensão: edd
\end{quote}

\textbf{ATOM}
\begin{quote}

Imagem de disco do Acorn Atom

Nome da Extensão: 40t, dsk
\end{quote}

\textbf{SSD}
\begin{quote}

Imagem de disco do Acorn SSD

Nome da Extensão: ssd, bbc, img
\end{quote}

\textbf{DSD}
\begin{quote}

Imagem de disco do Acorn DSD

Nome da Extensão: dsd
\end{quote}

\textbf{DOS}
\begin{quote}

Imagem de disco do Acorn DOS

Nome da Extensão: img
\end{quote}

\textbf{ADFS\_O}
\begin{quote}

Imagem de disco do Acorn ADFS (OldMap)

Nome da Extensão: adf, ads, adm, adl
\end{quote}

\textbf{ADFS\_N}
\begin{quote}

Imagem de disco do Acorn ADFS (NewMap)

Nome da Extensão: adf
\end{quote}

\textbf{ORIC\_DSK}
\begin{quote}

Imagem de disco do Oric

Nome da Extensão: dsk
\end{quote}

\textbf{APPLIX}
\begin{quote}

Imagem de disco do Applix

Nome da Extensão: raw
\end{quote}

\textbf{HPI}
\begin{quote}

Imagem de disquete do HP9845A

Nome da Extensão: hpi
\end{quote}


\section{Outras ferramentas que acompanham o MAME}
\label{tools/othertools:outras-ferramentas-que-acompanham-o-mame}\label{tools/othertools::doc}

\subsection{ledutil.exe/ledutil.sh}
\label{tools/othertools:ledutil-exe-ledutil-sh}
No Microsoft Windows o ledutil.exe pode ser usado para controlar as luzes
led do seu teclado para espelhar aquelas luzes presentes nos primeiros
jogos de arcade como o Asteroids por exemplo.

Para ativar essa funcionalidade inicie o \textbf{ledutil.exe} da linha de
comando. Rode o comando \textbf{ledutil.exe -kill} para interrompê-lo.

Nas plataformas SDLMAME como o Mac OS X e Linux, o \textbf{ledutil.sh} poderá
ser usado. Use o comando \textbf{ledutil.sh -a} para que ele seja fechado
automaticamente ao sair do SDLMAME.


\section{Ferramentas voltadas ao desenvolvimento}
\label{tools/othertools:ferramentas-voltadas-ao-desenvolvimento}

\subsection{pngcmp}
\label{tools/othertools:pngcmp}
Essa ferramenta é usada em teste de regressão ao comparar instantâneos
PNG vindos de um script teste \textbf{runtest.cmd} encontrado nos arquivos de
código fonte. Esse script só funciona no Windows.


\subsection{nltool}
\label{tools/othertools:nltool}
Componente de conversão discreto. A maioria dos usuários não precisam
lidar com ele.


\subsection{nlwav}
\label{tools/othertools:nlwav}
Componente discreto de conversão e ferramente de teste. A maioria dos
usuários não precisam lidar com ele.


\subsection{jedutil}
\label{tools/othertools:jedutil}
Ferramenta útil para extração de \textbf{PAL}/\textbf{PLA}/\textbf{PLD}/\textbf{GAL}.
Ele pode converter entre o formato JED padrão da indústria e o formato
binário compactado proprietário do MAME, pode mostrar também equações
lógicas para os tipos de dispositivos que conhecem tal lógica interna.
A maioria dos usuários não precisam lidar com ele.


\subsection{ldresample}
\label{tools/othertools:ldresample}
Essa ferramenta comprime novamente os dados de vídeo para laserdisc e
VHS. A maioria dos usuários não precisam lidar com ele.


\subsection{ldverify}
\label{tools/othertools:ldverify}
Essa ferramente é usada para comparar imagens de laserdisc ou VHS CHD
vinda de uma fonte AVI. A maioria dos usuários não precisam lidar com
ele.


\subsection{unidasm}
\label{tools/othertools:unidasm}
Disassembler universal para muitas das arquiteturas compatíveis com o
MAME. A maioria dos usuários não precisam lidar com ele.


\chapter{DEFINIÇÕES TÉCNICAS}
\label{techspecs/index::doc}\label{techspecs/index:definicoes-tecnicas}
Esta seção aborda as definições técnicas úteis para os programadores
que estão trabalhando no código fonte do MAME ou trabalhando em
scripts LUA que são executados na estrutura do MAME.


\section{Os arquivos de Layout do MAME}
\label{techspecs/layout_files::doc}\label{techspecs/layout_files:os-arquivos-de-layout-do-mame}
\begin{sphinxShadowBox}
\begin{itemize}
\item {} 
\phantomsection\label{techspecs/layout_files:id19}{\hyperref[techspecs/layout_files:introducao]{\sphinxcrossref{Introdução}}}

\item {} 
\phantomsection\label{techspecs/layout_files:id20}{\hyperref[techspecs/layout_files:conceitos\string-fundamentais]{\sphinxcrossref{Conceitos fundamentais}}}
\begin{itemize}
\item {} 
\phantomsection\label{techspecs/layout_files:id21}{\hyperref[techspecs/layout_files:numeros]{\sphinxcrossref{Números}}}

\item {} 
\phantomsection\label{techspecs/layout_files:id22}{\hyperref[techspecs/layout_files:coordenadas]{\sphinxcrossref{Coordenadas}}}

\item {} 
\phantomsection\label{techspecs/layout_files:id23}{\hyperref[techspecs/layout_files:cores]{\sphinxcrossref{Cores}}}

\item {} 
\phantomsection\label{techspecs/layout_files:id24}{\hyperref[techspecs/layout_files:parametros]{\sphinxcrossref{Parâmetros}}}

\item {} 
\phantomsection\label{techspecs/layout_files:id25}{\hyperref[techspecs/layout_files:parametros\string-predefinidos]{\sphinxcrossref{Parâmetros predefinidos}}}

\item {} 
\phantomsection\label{techspecs/layout_files:id26}{\hyperref[techspecs/layout_files:camadas]{\sphinxcrossref{Camadas}}}

\end{itemize}

\item {} 
\phantomsection\label{techspecs/layout_files:id27}{\hyperref[techspecs/layout_files:as\string-partes\string-de\string-um\string-layout]{\sphinxcrossref{As partes de um layout}}}
\begin{itemize}
\item {} 
\phantomsection\label{techspecs/layout_files:id28}{\hyperref[techspecs/layout_files:elementos]{\sphinxcrossref{Elementos}}}

\item {} 
\phantomsection\label{techspecs/layout_files:id29}{\hyperref[techspecs/layout_files:exibicoes]{\sphinxcrossref{Exibições}}}

\item {} 
\phantomsection\label{techspecs/layout_files:id30}{\hyperref[techspecs/layout_files:grupos\string-reutilizaveis]{\sphinxcrossref{Grupos reutilizáveis}}}

\item {} 
\phantomsection\label{techspecs/layout_files:id31}{\hyperref[techspecs/layout_files:repetindo\string-blocos]{\sphinxcrossref{Repetindo Blocos}}}

\end{itemize}

\item {} 
\phantomsection\label{techspecs/layout_files:id32}{\hyperref[techspecs/layout_files:o\string-tratamento\string-de\string-erros]{\sphinxcrossref{O Tratamento de erros}}}

\item {} 
\phantomsection\label{techspecs/layout_files:id33}{\hyperref[techspecs/layout_files:as\string-exibicoes\string-geradas\string-automaticamente]{\sphinxcrossref{As Exibições geradas automaticamente}}}

\item {} 
\phantomsection\label{techspecs/layout_files:id34}{\hyperref[techspecs/layout_files:usando\string-o\string-complay\string-py]{\sphinxcrossref{Usando o complay.py}}}

\end{itemize}
\end{sphinxShadowBox}


\subsection{Introdução}
\label{techspecs/layout_files:layout-intro}\label{techspecs/layout_files:introducao}
Os arquivos de layout \footnote[1]{\sphinxAtStartFootnote%
Arquivos de disposição dos elementos na tela. (Nota do tradutor)
} são usados para informar ao MAME o que exibir
enquanto um sistema emulado estiver rodando e como organizar estes
elementos na tela. O MAME pode renderizar as telas emuladas, imagens,
texto, formas e objetos especiais para dispositivos de saída comuns.
Os elementos podem ser estáticos, ou se atualizar de forma dinâmica para
refletir o estado das entradas e saídas.
Os layouts podem ser gerados automaticamente com base no número ou tipo
de tela emulada, construído e lincado internamente ao binário do MAME ou
disponibilizado externamente. Para o MAME os arquivos de layout são
interpretados como arquivos XML usando a extensão \sphinxcode{.lay}.


\subsection{Conceitos fundamentais}
\label{techspecs/layout_files:layout-concepts}\label{techspecs/layout_files:conceitos-fundamentais}

\subsubsection{Números}
\label{techspecs/layout_files:numeros}\label{techspecs/layout_files:layout-concepts-numbers}
Os layouts do MAME possuem dois tipos de números, inteiros e de ponto
flutuante.

Os números inteiros podem ser fornecidos em notação decimal ou
hexadecimal. Um número decimal inteiro consiste em um prefixo opcional
\textbf{\#} (hash \footnote[2]{\sphinxAtStartFootnote%
Em nosso idioma conhecido também como
cerquilha, jogo da velha, sustenido e atualmente como
\textbf{hashtag}. (Nota do tradutor)
}), um caractere opcional \textbf{+/-} (mais ou menos) e uma
sequência de dígitos entre \textbf{0-9}.

Um número hexadecimal consiste em que um dos prefixos
seja o \textbf{\$} (cifrão) ou \textbf{0x} (zero xis) seguido por uma sequência de
números hexadecimais entre \textbf{0-9} e \textbf{A-F}. O índice e os dígitos dos
números hexadecimais não diferenciam entre as letras maiúsculas e
minúsculas.

Os números de ponto flutuante podem ser fornecidos em decimal de ponto
fixo ou com notação científica. Observe que os prefixos de número
inteiro e os valores hexadecimais \emph{não} são aceitos onde um número de
ponto flutuante for esperado.

Para alguns atributos, ambos números inteiros e números de ponto
flutuante são permitidos. Nesses casos, a presença de um prefixo
\textbf{\#} (hash), \textbf{\$} (cifrão) ou \textbf{0x} (zero xis) faz com que o valor
seja interpretado como um número inteiro.
Caso nenhum prefixo de número inteiro, ponto decimal ou a letra E
(maiúsculo ou minusculo) seja encontrado para introduzir um expoente,
ele será interpretado como um número de ponto flutuante.
Caso nenhum prefixo de número inteiro, ponto decimal ou a letra E seja
encontrado, o número será interpretado como um número inteiro.

Os números são analisados usando uma acentuação de caracteres em C \footnote[3]{\sphinxAtStartFootnote%
\emph{C locale} em Inglês. (Nota do tradutor)
}
por questões de portabilidade.


\subsubsection{Coordenadas}
\label{techspecs/layout_files:layout-concepts-coordinates}\label{techspecs/layout_files:coordenadas}
As coordenadas de layout são representadas internamente através da norma
IEEE754 como um número binário de 32-bit de ponto flutuante (também
conhecido como ``\emph{precisão simples}''). O incremento das coordenadas
nas direções para a direita e para baixo. A origem (\textbf{0,0}) não tem um
significado em particular e valores negativos podem ser usados nos
layouts.
As coordenadas são fornecidas como números de ponto flutuante.

O MAME pressupõe que as coordenadas exibição tem a mesma proporção de
aspecto que o pixel de saída do dispositivo (janela ou nativa).
Considerando que sejam pixels quadrados e sem rotação, isso significa
que a distância seja igual nos eixos \textbf{X} e \textbf{Y} o que corresponde a
distâncias iguais na vertical e horizontal na saída que for renderizada.

Elementos, grupos e exibições, todos têm os seus sistemas internos
de coordenadas. Quando um elemento ou grupo é referenciado a partir de
uma visão ou um outro grupo, as suas coordenadas são dimensionadas
conforme necessário para ajustar os limites definidos.

Os objetos são posicionados e dimensionados usando os elementos limite
\sphinxcode{bounds}.
Os elementos limite podem definir a posição da parte do canto superior
esquerdo e usando o atributo de tamanho \sphinxcode{x}, \sphinxcode{y}, largura \sphinxcode{width}
e altura \sphinxcode{height} ou pode definir as coordenadas dos cantos com os
atributos esquerda \sphinxcode{left}, cima \sphinxcode{top}, direita \sphinxcode{right} e baixo
\sphinxcode{bottom}. Estes dois elementos \sphinxcode{bounds} são equivalentes:

\begin{Verbatim}[commandchars=\\\{\}]
\PYG{o}{\PYGZlt{}}\PYG{n}{bounds} \PYG{n}{x}\PYG{o}{=}\PYG{l+s+s2}{\PYGZdq{}}\PYG{l+s+s2}{455}\PYG{l+s+s2}{\PYGZdq{}} \PYG{n}{y}\PYG{o}{=}\PYG{l+s+s2}{\PYGZdq{}}\PYG{l+s+s2}{120}\PYG{l+s+s2}{\PYGZdq{}} \PYG{n}{width}\PYG{o}{=}\PYG{l+s+s2}{\PYGZdq{}}\PYG{l+s+s2}{11}\PYG{l+s+s2}{\PYGZdq{}} \PYG{n}{height}\PYG{o}{=}\PYG{l+s+s2}{\PYGZdq{}}\PYG{l+s+s2}{7}\PYG{l+s+s2}{\PYGZdq{}} \PYG{o}{/}\PYG{o}{\PYGZgt{}}
\PYG{o}{\PYGZlt{}}\PYG{n}{bounds} \PYG{n}{left}\PYG{o}{=}\PYG{l+s+s2}{\PYGZdq{}}\PYG{l+s+s2}{455}\PYG{l+s+s2}{\PYGZdq{}} \PYG{n}{top}\PYG{o}{=}\PYG{l+s+s2}{\PYGZdq{}}\PYG{l+s+s2}{120}\PYG{l+s+s2}{\PYGZdq{}} \PYG{n}{right}\PYG{o}{=}\PYG{l+s+s2}{\PYGZdq{}}\PYG{l+s+s2}{466}\PYG{l+s+s2}{\PYGZdq{}} \PYG{n}{bottom}\PYG{o}{=}\PYG{l+s+s2}{\PYGZdq{}}\PYG{l+s+s2}{127}\PYG{l+s+s2}{\PYGZdq{}} \PYG{o}{/}\PYG{o}{\PYGZgt{}}
\end{Verbatim}

Ambos os atributos \sphinxcode{x} ou \sphinxcode{left} devem estar presente para se
distinguir entre os dois elementos. O \sphinxcode{width} e \sphinxcode{height} ou
\sphinxcode{right} e \sphinxcode{bottom} o valor 1.0 é predefinido caso um valor não seja
informado.
Será considerado um erro caso os valores de \sphinxcode{width} ou \sphinxcode{height}
sejam negativos, caso \sphinxcode{right} sejam menor que \sphinxcode{left} ou se
\sphinxcode{bottom} seja menor que \sphinxcode{top}.


\subsubsection{Cores}
\label{techspecs/layout_files:cores}\label{techspecs/layout_files:layout-concepts-colours}
As cores são definidas no espaço RGBA. O MAME não tem conhecimento
do leque de perfis de gamma e cores, assim as cores normalmente serão
interpretadas como sRGB junto com a definição de gamma do seu
sistema com o valor de \textbf{2.2} geralmente. Os valores dos canais são
definidos como números de ponto flutuante. Os valores dos canais
vermelho, verde e azul variam entre \textbf{0.0} (desligado) até \textbf{1.0}
(intensidade plena).
Os valores alfa variam entre \textbf{0.0} (plena transparência) até \textbf{1.0}
(opaco). Os valores dos canais de cores não são previamente
multiplicados pelo valor alfa.

O componente e a cor do item de exibição são especificados usando os
elementos \sphinxcode{color}.
Os atributos relevantes são vermelho \sphinxcode{red}, verde \sphinxcode{green},
azul \sphinxcode{blue} e \sphinxcode{alpha}. Este exemplo de elemento \sphinxcode{color} determina
todos os valores dos canais:

\begin{Verbatim}[commandchars=\\\{\}]
\PYG{o}{\PYGZlt{}}\PYG{n}{color} \PYG{n}{red}\PYG{o}{=}\PYG{l+s+s2}{\PYGZdq{}}\PYG{l+s+s2}{0.85}\PYG{l+s+s2}{\PYGZdq{}} \PYG{n}{green}\PYG{o}{=}\PYG{l+s+s2}{\PYGZdq{}}\PYG{l+s+s2}{0.4}\PYG{l+s+s2}{\PYGZdq{}} \PYG{n}{blue}\PYG{o}{=}\PYG{l+s+s2}{\PYGZdq{}}\PYG{l+s+s2}{0.3}\PYG{l+s+s2}{\PYGZdq{}} \PYG{n}{alpha}\PYG{o}{=}\PYG{l+s+s2}{\PYGZdq{}}\PYG{l+s+s2}{1.0}\PYG{l+s+s2}{\PYGZdq{}} \PYG{o}{/}\PYG{o}{\PYGZgt{}}
\end{Verbatim}

Qualquer atributo de canal que for omitido o seu valor se torna 1.0,
valor já predefinido (intensidade plena ou opaco). Será considerado um
erro caso os valores do canal estejam fora do intervalo entre de \textbf{0.0}
até \textbf{1.0} (inclusive).


\subsubsection{Parâmetros}
\label{techspecs/layout_files:parametros}\label{techspecs/layout_files:layout-concepts-params}
Os parâmetros são variáveis nomeadas que podem ser usadas na maioria
dos atributos. Para usar um parâmetro em um atributo, cerque seu nome
com caracteres til \emph{(\textasciitilde{})}.
Caso um parâmetro não seja definido, nenhuma substituição será feita.
Aqui um exemplo mostrando os dois casos do parâmetro, use os valores dos
parâmetros de \sphinxcode{digitno} e \sphinxcode{x} que serão substituídos por
\sphinxcode{\textasciitilde{}digitno\textasciitilde{}} e \sphinxcode{\textasciitilde{}x\textasciitilde{}}:

\begin{Verbatim}[commandchars=\\\{\}]
\PYG{o}{\PYGZlt{}}\PYG{n}{bezel} \PYG{n}{name}\PYG{o}{=}\PYG{l+s+s2}{\PYGZdq{}}\PYG{l+s+s2}{digit\PYGZti{}digitno\PYGZti{}}\PYG{l+s+s2}{\PYGZdq{}} \PYG{n}{element}\PYG{o}{=}\PYG{l+s+s2}{\PYGZdq{}}\PYG{l+s+s2}{digit}\PYG{l+s+s2}{\PYGZdq{}}\PYG{o}{\PYGZgt{}}
    \PYG{o}{\PYGZlt{}}\PYG{n}{bounds} \PYG{n}{x}\PYG{o}{=}\PYG{l+s+s2}{\PYGZdq{}}\PYG{l+s+s2}{\PYGZti{}x\PYGZti{}}\PYG{l+s+s2}{\PYGZdq{}} \PYG{n}{y}\PYG{o}{=}\PYG{l+s+s2}{\PYGZdq{}}\PYG{l+s+s2}{80}\PYG{l+s+s2}{\PYGZdq{}} \PYG{n}{width}\PYG{o}{=}\PYG{l+s+s2}{\PYGZdq{}}\PYG{l+s+s2}{25}\PYG{l+s+s2}{\PYGZdq{}} \PYG{n}{height}\PYG{o}{=}\PYG{l+s+s2}{\PYGZdq{}}\PYG{l+s+s2}{40}\PYG{l+s+s2}{\PYGZdq{}} \PYG{o}{/}\PYG{o}{\PYGZgt{}}
\PYG{o}{\PYGZlt{}}\PYG{o}{/}\PYG{n}{bezel}\PYG{o}{\PYGZgt{}}
\end{Verbatim}

Um nome para o parâmetro é uma sequência de letras maiúsculas das letras
\textbf{A-Z}, das letras minusculas \textbf{a-z}, dígitos decimais \textbf{0-9}, ou
caracteres subtraço (\_).
Os nomes dos parâmetros levam em consideração as letras maiúsculas e
minúsculas. Quando a procura de um parâmetro, o motor do layout começa
no escopo atual trabalhando de dentro para fora. O nível do escopo mais periférico,
corresponde ao elemento de primeiro nível \sphinxcode{mamelayout}. Cada elemento
\sphinxcode{repeat}, \sphinxcode{group} ou \sphinxcode{view} cria um novo nível de escopo.

Internamente, um parâmetro pode conter uma carreira de caracteres,
números inteiros ou números de ponto flutuante, porém essa é mais
transparente.
Os números inteiros são armazenados como \emph{64-bit signed} com dois valores
complementares, os números de ponto flutuante são armazenados como
binários \emph{IEEE754 64-bit}, estes números de ponto flutuante também são
conhecido como ``precisão dupla''. Os números inteiros são substituídos em
notação decimal e números de ponto flutuante são substituídos em seu
formato padrão que pode ser decimal de ponto fixo ou notação científica
dependendo do valor. Não há nenhuma maneira de sobrescrever a formatação
padrão dos parâmetros de um número inteiro ou de ponto flutuante.

Existem dois tipos de parâmetros: \emph{value parameters} and \emph{generator
parameters}. O parâmetro ``value parameters'' mantém o seu valor atribuído
até que seja reatribuído.
O parâmetro ``\emph{generator parameters}'' tem um valor inicial, um incremento
e/ou uma transferência \footnote[4]{\sphinxAtStartFootnote%
O termo \emph{shift} é muito amplo, também pode ser
interpretado como desvio, mudança, turno, inversão, câmbio, etc.
(Nota do tradutor)
} aplicada para cada interação.

Os valores dos parâmetros são atribuídos usando um elemento \sphinxcode{param}
junto com elementos \sphinxcode{name} e \sphinxcode{value}. Os valores do parâmetro podem
aparecer de dentro de um elemento de primeiro nível \sphinxcode{mamelayout} e
dentro dos elementos \sphinxcode{repeat}, \sphinxcode{view} assim como dentro da definição
dos elementos \sphinxcode{group} (isso é, elementos \sphinxcode{group} dentro do nível
superior do elemento \sphinxcode{mamelayout}, ao contrário dos elementos
\sphinxcode{group} dentro de elementos \sphinxcode{view} definidos por outros elementos
\sphinxcode{group}.
O valor do parâmetro pode ser reatribuído a qualquer momento.

Aqui está um exemplo atribuindo o valor ``4'' para o parâmetro
``firstdigit'':

\begin{Verbatim}[commandchars=\\\{\}]
\PYG{o}{\PYGZlt{}}\PYG{n}{param} \PYG{n}{name}\PYG{o}{=}\PYG{l+s+s2}{\PYGZdq{}}\PYG{l+s+s2}{firstdigit}\PYG{l+s+s2}{\PYGZdq{}} \PYG{n}{value}\PYG{o}{=}\PYG{l+s+s2}{\PYGZdq{}}\PYG{l+s+s2}{4}\PYG{l+s+s2}{\PYGZdq{}} \PYG{o}{/}\PYG{o}{\PYGZgt{}}
\end{Verbatim}

Os geradores de parâmetros são atribuídos usando o elemento
\sphinxcode{param} com os atributos \sphinxcode{name}, \sphinxcode{start}, \sphinxcode{increment},
\sphinxcode{lshift} e \sphinxcode{rshift}.
Os geradores de parâmetros só podem aparecer de dentro de elementos
\sphinxcode{repeat} (veja {\hyperref[techspecs/layout_files:layout\string-parts\string-repeats]{\sphinxcrossref{\DUrole{std,std-ref}{Repetindo Blocos}}}} para mais informações).
Os geradores de parâmetros não deve ser reatribuídos no mesmo escopo
(um nome de parâmetro idêntico pode ser definido em um escopo filho.
Aqui alguns exemplos dos geradores de parâmetros:

\begin{Verbatim}[commandchars=\\\{\}]
\PYG{o}{\PYGZlt{}}\PYG{n}{param} \PYG{n}{name}\PYG{o}{=}\PYG{l+s+s2}{\PYGZdq{}}\PYG{l+s+s2}{nybble}\PYG{l+s+s2}{\PYGZdq{}} \PYG{n}{start}\PYG{o}{=}\PYG{l+s+s2}{\PYGZdq{}}\PYG{l+s+s2}{3}\PYG{l+s+s2}{\PYGZdq{}} \PYG{n}{increment}\PYG{o}{=}\PYG{l+s+s2}{\PYGZdq{}}\PYG{l+s+s2}{\PYGZhy{}1}\PYG{l+s+s2}{\PYGZdq{}} \PYG{o}{/}\PYG{o}{\PYGZgt{}}
\PYG{o}{\PYGZlt{}}\PYG{n}{param} \PYG{n}{name}\PYG{o}{=}\PYG{l+s+s2}{\PYGZdq{}}\PYG{l+s+s2}{switchpos}\PYG{l+s+s2}{\PYGZdq{}} \PYG{n}{start}\PYG{o}{=}\PYG{l+s+s2}{\PYGZdq{}}\PYG{l+s+s2}{74}\PYG{l+s+s2}{\PYGZdq{}} \PYG{n}{increment}\PYG{o}{=}\PYG{l+s+s2}{\PYGZdq{}}\PYG{l+s+s2}{156}\PYG{l+s+s2}{\PYGZdq{}} \PYG{o}{/}\PYG{o}{\PYGZgt{}}
\PYG{o}{\PYGZlt{}}\PYG{n}{param} \PYG{n}{name}\PYG{o}{=}\PYG{l+s+s2}{\PYGZdq{}}\PYG{l+s+s2}{mask}\PYG{l+s+s2}{\PYGZdq{}} \PYG{n}{start}\PYG{o}{=}\PYG{l+s+s2}{\PYGZdq{}}\PYG{l+s+s2}{0x0800}\PYG{l+s+s2}{\PYGZdq{}} \PYG{n}{rshift}\PYG{o}{=}\PYG{l+s+s2}{\PYGZdq{}}\PYG{l+s+s2}{4}\PYG{l+s+s2}{\PYGZdq{}} \PYG{o}{/}\PYG{o}{\PYGZgt{}}
\end{Verbatim}
\begin{itemize}
\item {} 
O parâmetro \sphinxcode{nybble} geram os valores 3, 2, 1...

\item {} 
O parâmetro \sphinxcode{switchpos} geram os valores 74, 230, 386...

\item {} 
O parâmetro \sphinxcode{mask} geram os valores 2048, 128, 8...

\end{itemize}

O atributo \sphinxcode{increment} deve ser um número inteiro ou de ponto
flutuante a ser adicionado ao valor do parâmetro. Os atributos
\sphinxcode{lshift} e \sphinxcode{rshift} devem ser números positivos inteiros definindo a
quantidade de bits que serão transferidos no valor dos parâmetros para a
esquerda e direita. A transferência e o incremento são aplicados no
final do bloco de repetição antes que a próxima iteração comece.
Se ambos os incrementos e a transferências forem fornecidas o valor
do incremento é aplicado antes do valor da transferência.

Caso o atributo \sphinxcode{incremento} esteja presente e for um número de
ponto flutuante, o valor do parâmetro será interpretado como um número
inteiro ou de ponto flutuante e depois convertido para um número de
ponto flutuante antes que o incremento seja adicionado. Caso o atributo
\sphinxcode{increment} esteja presente e for um número de ponto flutuante, o
valor do parâmetro será interpretado como um valor de número inteiro ou
de ponto flutuante antes que o valor incremental seja adicionado.
O valor do incremento será convertido em um número de ponto flutuante
antes da adição caso o valor seja um número de ponto flutuante.

Caso os atributos \sphinxcode{lshift} ou \sphinxcode{rshift} estejam presentes e não
forem iguais, o valor do parâmetro será interpretado como um número
inteiro ou de ponto flutuante convertido em um número inteiro conforme
seja necessário e transferido de acordo. A transferência para a esquerda
é definida como uma transferência feita para o bit mais importante.
Caso ambos os parâmetros \sphinxcode{lshift} e \sphinxcode{rshift} sejam fornecidos, estes
serão compensados antes dos valores serem aplicados. Isto significa que
você não pode, por exemplo, usar atributos iguais tanto para
{}`{}` lshift{}`{}` como para \sphinxcode{rshift} visando limpar os bits em um valor de
parâmetro extremo após a primeira interação.

Será considerado um erro caso o elemento \sphinxcode{param} não esteja em
qualquer um dos atributos \sphinxcode{value} ou \sphinxcode{start}, será também
considerado um erro caso ambos os elementos \sphinxcode{param} tiverem  os mesmos
atributos \sphinxcode{value} ou qualquer um dos mesmos atributos \sphinxcode{start},
\sphinxcode{increment}, \sphinxcode{lshift}, ou \sphinxcode{rshift}.

Um elemento \sphinxcode{param} define ou reatribui o seu valor em um parâmetro no
escopo atual mais interno. Não é possível definir ou reatribuir os
parâmetros em um escopo de contenção.
\clearpage

\subsubsection{Parâmetros predefinidos}
\label{techspecs/layout_files:layout-concepts-predef-params}\label{techspecs/layout_files:parametros-predefinidos}
Uma certa quantidade de valores predefinidos nos parâmetros já estão
disponíveis e fornecem informações sobre a máquina em execução:

\textbf{devicetag}
\begin{quote}

Um exemplo do caminho completo da tag do dispositivo que será
responsável pela leitura do layout, seria \sphinxcode{:} para o driver do
controlador do dispositivo raiz ou \sphinxcode{:tty:ie15} para o terminal
conectado em uma porta. Este parâmetro é uma sequência de caracteres
definida no escopo global de visualização do layout.
\end{quote}

\textbf{devicebasetag}
\begin{quote}

A base da tag do dispositivo que será responsável pela leitura do
layout, como por exemplo \sphinxcode{root} para o driver do dispositivo raiz
ou \sphinxcode{ie15} para o terminal que estiver conectado em uma porta.
Este parâmetro é uma sequência de caracteres definida no escopo
global do layout.
\end{quote}

\textbf{devicename}
\begin{quote}

O nome completo (descrição) do dispositivo que será responsável pela
leitura do layout, como por exemplo os terminais \sphinxcode{AIM-65/40} ou
\sphinxcode{IE15}. Este parâmetro é uma sequência de caracteres
definida no escopo global do layout.
\end{quote}

\textbf{deviceshortname}
\begin{quote}

Um nome curto do dispositivo que será responsável pela leitura do
layout, como por exemplo os terminais \sphinxcode{aim65\_40} ou \sphinxcode{ie15}.
Este parâmetro é uma sequência de caracteres definida no escopo
global do layout.
\end{quote}

\textbf{scr0physicalxaspect}
\begin{quote}

A parte horizontal da relação de aspecto físico da primeira tela
(caso esteja presente). A relação de aspecto físico é fornecida como
uma fração impropriamente reduzida. Observe que este é o componente
horizontal aplicado \emph{antes} da rotação. Este parâmetro é um número
inteiro definido no escopo global do layout.
\end{quote}

\textbf{scr0physicalyaspect}
\begin{quote}

A parte vertical da relação de aspecto físico da primeira tela
(caso esteja presente). A relação de aspecto físico é fornecida como
uma fração impropriamente reduzida. Observe que este é o componente
vertical aplicado \emph{antes} da rotação. Este parâmetro é um número
inteiro definido no escopo global do layout.
\end{quote}

\textbf{scr0nativexaspect}
\begin{quote}

A parte horizontal da relação de aspecto do pixel visível na área da
primeira tela (caso esteja presente). A relação de aspecto
do pixel é fornecida como uma fração impropriamente reduzida.
Observe que este é o componente horizontal aplicado \emph{antes} da
rotação. Este parâmetro é um número inteiro definido no escopo
global do layout.
\end{quote}

\textbf{scr0nativeyaspect}
\begin{quote}

A parte vertical da relação de aspecto do pixel visível na área da
primeira tela (caso esteja presente). A relação de aspecto do pixel
é fornecida como uma fração impropriamente reduzida. Observe que
este é o componente vertical aplicado \emph{antes} da rotação. Este
parâmetro é um número inteiro definido no escopo global do layout.
\end{quote}

\textbf{scr0width}
\begin{quote}

A largura da área visível da primeira tela (se houver) nos pixels
emulados. Observe que a largura é aplicada \emph{antes} da rotação.
Este parâmetro é um número inteiro definido no escopo global do
layout.
\end{quote}

\textbf{scr0height}
\begin{quote}

A altura da área visível da primeira tela (se houver) nos pixels
emulados. Observe que a altura é aplicada \emph{antes} da rotação.
Este parâmetro é um número inteiro definido no escopo global do
layout.
\end{quote}

\textbf{scr1physicalxaspect}
\begin{quote}

A parte horizontal da relação de aspecto físico da primeira tela
(caso esteja presente). Este parâmetro é um número inteiro definido
no escopo global do layout.
\end{quote}

\textbf{scr1physicalyaspect}
\begin{quote}

A parte vertical da relação de aspecto físico da segunda tela
(caso esteja presente). Este parâmetro é um número inteiro
definido no escopo global do layout.
\end{quote}

\textbf{scr1nativexaspect}
\begin{quote}

A parte horizontal da relação de aspecto do pixel visível na área da
segunda tela (caso esteja presente). Este parâmetro é um número
inteiro definido no escopo global de visualização do layout.
\end{quote}

\textbf{scr1nativeyaspect}
\begin{quote}

A parte vertical da relação de aspecto do pixel visível na área da
segunda tela (caso esteja presente). Este parâmetro é um número inteiro
definido no escopo global de visualização do layout.
\end{quote}

\textbf{scr1width}
\begin{quote}

A largura da área visível da segunda tela (se houver) nos pixels
emulados. Este parâmetro é um número inteiro definido no escopo
global do layout.
\end{quote}

\textbf{scr1height}
\begin{quote}

A altura da área visível da segunda tela (se houver) nos pixels
emulados. Este parâmetro é um número inteiro definido no escopo
global do layout.
\end{quote}

\textbf{scr*N*physicalxaspect}
\begin{quote}

A parte horizontal da relação de aspecto físico da tela (base-zero)
\emph{N}th (caso esteja presente). Este parâmetro é um número inteiro
definido no escopo global do layout.
\end{quote}

\textbf{scr*N*physicalyaspect}
\begin{quote}

A parte vertical da relação de aspecto físico da tela (base-zero)
\emph{N}th (caso esteja presente). Este parâmetro é um número inteiro
definido no escopo global do layout.
\end{quote}

\textbf{scr*N*nativexaspect}
\begin{quote}

A parte horizontal da relação de aspecto da parte visível da tela
(base-zero) \emph{N}th (caso esteja presente). Este parâmetro é um
número inteiro definido no escopo global do layout.
\end{quote}

\textbf{scr*N*nativeyaspect}
\begin{quote}

A parte vertical da relação de aspecto da parte visível da tela
(base-zero) \emph{N}th (caso esteja presente). Este parâmetro é um
número inteiro definido no escopo global do layout.
\end{quote}

\textbf{scr*N*width}
\begin{quote}

A largura da área visível da tela (base-zero) \emph{N}th (se presente)
nos pixels emulados. Este parâmetro é um número inteiro definido no
escopo de visualização do layout.
\end{quote}

\textbf{scr*N*height}
\begin{quote}

A largura da área visível da tela (base-zero) \emph{N}th (se presente)
nos pixels emulados. Este parâmetro é um número inteiro definido no
escopo de visualização do layout.
\end{quote}

\textbf{viewname}
\begin{quote}

O nome da exibição atual. Este parâmetro é uma sequências de
caracteres definido no escopo de visualização.
Não é definido fora do campo de visão.
\end{quote}

Para parâmetros relacionados à tela, elas são numeradas do zero na
ordem em que aparecem na configuração da máquina. Todas as telas estão
inclusas (não apenas nos sub-dispositivos do dispositivo que fizeram com
que o layout fosse carregado). \textbf{X/width} e \textbf{Y/height} referem-se as
dimensões horizontal e vertical da tela \emph{antes} da rotação ser aplicada.
Os valores baseados na área visível são calculados no final da
configuração. Caso o sistema não reconfigure a tela durante a execução
os valores dos parâmetros não serão atualizados assim como os layouts
não serão recalculados.


\subsubsection{Camadas}
\label{techspecs/layout_files:layout-concepts-layers}\label{techspecs/layout_files:camadas}
As exibições são renderizadas como uma pilha de camadas, ganharam
seus nomes com referência aos nomes de peças do arcade.
O layout fornece elementos a serem desenhados em todas as camadas além
da camada da tela, que é reservado para as telas emuladas. Com exceção
da camada de tela, os usuários podem ativar ou desativar as camadas
usando o cardápio interno do emulador ou a linha de comando.

As seguintes camadas estão disponíveis:

\textbf{backdrop}
\begin{quote}

Desenvolvido para o uso em situações onde a imagem da tela é projetada
sobre um pano de fundo usando um espelho semi reflexivo criando uma
ilusão de ótica (os fantasmas de Pepper \footnote[5]{\sphinxAtStartFootnote%
Pepper's ghosts, no Brasil ficou muito
conhecido como \href{https://www.youtube.com/watch?v=L5Lgn1vbeHA}{casa de Monga},
\emph{Monga} ou \emph{``Monga, a Mulher-Macaco''}, é uma \href{https://www.youtube.com/watch?v=xrAWgmfhOaM}{técnica de ilusão
de ótica} usada
em apresentações feitas em teatros no século XIX, inventado pelo
cientista Inglês \emph{John Henry Pepper} (1821–1900). Monga também é
relacionado à
\href{https://super.abril.com.br/ciencia/monga-a-verdadeira-mulher-macaco/}{Julia Pastrana}. (Nota do tradutor)
}). Esse arranjo famoso
ficou conhecido no gabinete de luxo do jogo \emph{Space Invaders}.
\end{quote}

\textbf{screen}
\begin{quote}

Esta camada é reservada para imagens emuladas da tela e não pode ser
desativada pelo usuário. É desenhado usando uma combinação
cumulativa \footnote[6]{\sphinxAtStartFootnote%
Additive blending. (Nota do tradutor)
}.
\end{quote}

\textbf{overlay}
\begin{quote}

Esta camada serve para o uso de sobreposições translúcidas usadas
antigamente para adicionar cores em jogos que usavam monitores CRT
monocromáticos, dentre eles o jogo \textbf{Circus}, \textbf{Gee Bee} e claro
o jogo \textbf{Space Invaders}.
É desenhado usando multiplicações RGB.
\end{quote}

\textbf{bezel}
\begin{quote}

Esta camada é para ser usada com elementos que iam ao redor da tela
e potencialmente podiam obscurecer a imagem na tela.
É desenhado usando um padrão de combinação do canal alfa.
\end{quote}

\textbf{cpanel}
\begin{quote}

Esta camada destina-se a exibir ilustrações de controles/dispositivos
de entrada (painéis de controle).
É desenhado usando um padrão de combinação do canal alfa.
\end{quote}

\textbf{marquee}
\begin{quote}

Esta camada é usada para exibir as imagens dos letreiros do gabinete
de arcade. Isto é, desenhado usando um padrão de combinação do canal
alfa.
\end{quote}

É predefinido que as camadas sejam desenhadas de trás para frente nesta
ordem:
\begin{itemize}
\item {} 
screen (adiciona)

\item {} 
overlay (multiplica)

\item {} 
backdrop (adiciona)

\item {} 
bezel (alfa)

\item {} 
cpanel (alfa)

\item {} 
marquee (alfa)

\end{itemize}

Caso uma visualização tenha vários elementos de pano de fundo e nenhum
elemento de sobreposição, uma ordem diferente de exibição é usada
(de trás para frente):
\begin{itemize}
\item {} 
backdrop (alfa)

\item {} 
screen (adiciona)

\item {} 
bezel (alfa)

\item {} 
cpanel (alfa)

\item {} 
marquee (alfa)

\end{itemize}

A alternância da ordem a ser desenhada torna-se mais simples para a
criação do pano de fundo vindo de diversos pedaços desenhados ou
escaneados de uma arte qualquer, assim como as partes opacas. Não pode
ser usado com elementos de sobreposição pois as cores sobrepostas são
convenientemente colocadas entre a tela e um espelho, por isso não tem
efeito algum no pano de fundo usado.


\subsection{As partes de um layout}
\label{techspecs/layout_files:as-partes-de-um-layout}\label{techspecs/layout_files:layout-parts}
Uma visualização define a disposição de um objeto gráfico a ser exibido.
O arquivo de layout do MAME pode conter diversas exibições. As
exibições são construídas a partir de elementos \emph{elements} e telas
\emph{screens}. Para simplificar os layouts complexos, os blocos repetidos e
os grupos reutilizáveis são compatíveis entre si.

O elemento de primeiro nível de um arquivo de layout do MAME deve ser um
elemento{}`{}`mamelayout{}`{}` junto com um atributo \sphinxcode{version}. O atributo
\sphinxcode{version} deve ser um valor inteiro. Atualmente, o MAME suporta apenas
a versão 2 e não carregará qualquer outra versão diferente.
Este é um exemplo de uma tag inicial para um elemento \sphinxcode{mamelayout}:

\begin{Verbatim}[commandchars=\\\{\}]
\PYG{o}{\PYGZlt{}}\PYG{n}{mamelayout} \PYG{n}{version}\PYG{o}{=}\PYG{l+s+s2}{\PYGZdq{}}\PYG{l+s+s2}{2}\PYG{l+s+s2}{\PYGZdq{}}\PYG{o}{\PYGZgt{}}
\end{Verbatim}

Em geral, os filhos de primeiro nível do elemento \sphinxcode{mamelayout} são
processados em ordem de chegada de cima para baixo. Uma exceção é que
por questões históricas, as exibições são processadas por último.
Isso significa que as exibições veem os valores finais de todos os
parâmetros do final do elemento \sphinxcode{mamelayout} e pode se referir a
elementos e grupos que possam aparecer depois deles.

Os seguintes elementos são permitidos dentro do elemento de primeiro
nível \sphinxcode{mamelayout}:

\textbf{param}
\begin{quote}

Define ou reatribui um valor para um parâmetro. Veja
{\hyperref[techspecs/layout_files:layout\string-concepts\string-params]{\sphinxcrossref{\DUrole{std,std-ref}{Parâmetros}}}} para mais informações.
\end{quote}

\textbf{element}
\begin{quote}

Define um elemento, um dos objetos básicos que podem ser organizados
em uma Visualização. Veja {\hyperref[techspecs/layout_files:layout\string-parts\string-elements]{\sphinxcrossref{\DUrole{std,std-ref}{Elementos}}}} para mais
informações.
\end{quote}

\textbf{group}
\begin{quote}

Define um grupo de elementos ou telas que possam ser reutilizáveis e
que também possam ser usados como referência em uma visualização
ou em outros grupos.

Veja {\hyperref[techspecs/layout_files:layout\string-parts\string-groups]{\sphinxcrossref{\DUrole{std,std-ref}{Grupos reutilizáveis}}}} para mais informações.
\end{quote}

\textbf{repeat}
\begin{quote}

Um grupo repetido de elementos que podem conter os elementos
\sphinxcode{param}, \sphinxcode{element}, \sphinxcode{group} e \sphinxcode{repeat}.
Veja {\hyperref[techspecs/layout_files:layout\string-parts\string-repeats]{\sphinxcrossref{\DUrole{std,std-ref}{Repetindo Blocos}}}} para mais informações.
\end{quote}

\textbf{view}
\begin{quote}

Um arranjo de elementos ou de telas que podem ser exibidos em um
dispositivo de saída (uma janela ou tela do host).
Veja {\hyperref[techspecs/layout_files:layout\string-parts\string-views]{\sphinxcrossref{\DUrole{std,std-ref}{Exibições}}}} para mais informações.
\end{quote}

\textbf{script}
\begin{quote}

Permite que scripts lua sejam usados para um layout aprimorado de
interação.
\end{quote}


\subsubsection{Elementos}
\label{techspecs/layout_files:layout-parts-elements}\label{techspecs/layout_files:elementos}
Os elementos são um dos objetos visuais mais básicos que podem ser
organizados junto com as telas para compor uma visualização. Os
elementos podem ser construídos com um ou mais componentes \emph{components}
porém um elemento é tratado como uma única superfície ao compor o
gráfico da cena e sua renderização. Um elemento pode ser usado em
diversas exibições e pode também ser usado diversas vezes dentro de
uma exibição.

A aparência de um elemento depende do seu estado \emph{state}. O estado é um
valor inteiro que geralmente vem de uma área da porta I/O ou de uma
saída emulada (veja a discussão em {\hyperref[techspecs/layout_files:layout\string-parts\string-views]{\sphinxcrossref{\DUrole{std,std-ref}{Exibições}}}} para
mais informações de como conectar um elemento a uma porta ou saída I/O).
Qualquer componente de um elemento pode ser restrito apenas ao desenho
quando o estado do elemento tiver um valor específico. Alguns
componentes (como mostradores de segmento múltiplo e mostradores
rotativos \footnote[7]{\sphinxAtStartFootnote%
Reels, \href{https://i.postimg.cc/FF2GYc9v/Reels.jpg}{mostradores mecânicos} usados em máquinas
caça niqueis. (Nota do tradutor)
}) que usam diretamente o estado para determinar a sua
aparência final.

Cada elemento possui o seu próprio sistema interno de coordenadas. Os
limites dos elementos dos sistema de coordenadas são computados de
maneira que cada parte individual dos componentes sejam unidos.

Todo elemento deve definir o seu nome usando o atributo \sphinxcode{name}. Os
elementos são consultados pelo nome quando são consultados em grupos ou
exibições. Será considerado um erro caso o arquivo de layout
contenha vários elementos com atributos \sphinxcode{name} idênticos.
Os elementos podem, opcionalmente, fornecer um valor de estado padrão
com um atributo \sphinxcode{defstate} para ser usado casp não esteja conectado em
uma saída emulada ou porta I/O. Se presente, o atributo \sphinxcode{defstate}
deve possuir um valor inteiro não negativo.

Os elementos filho do elemento \sphinxcode{element} representam os componentes
que são desenhados em ordem de leitura do primeiro ao último
(componentes desenhados em cima de componentes que vierem antes deles).
Suporte a todos os componentes com alguns recursos em comum:
\begin{itemize}
\item {} 
Cada componente pode ter um atributo \sphinxcode{state}. Se presente, o
componente só será desenhado quando o estado do elemento corresponder
ao seu valor (se ausente, o componente sempre será desenhado).
Se presente, o atributo \sphinxcode{state} deve ser um valor inteiro não
negativo.

\item {} 
Cada componente pode ter um elemento filho \sphinxcode{bounds} definindo a
sua posição e tamanho (veja {\hyperref[techspecs/layout_files:layout\string-concepts\string-coordinates]{\sphinxcrossref{\DUrole{std,std-ref}{Coordenadas}}}}). Caso
tal elemento não esteja presente, os limites serão predefinidos a uma
unidade quadrada, com o valor \textbf{1.0} para a largura e a altura e
\textbf{0.0} para o canto superior esquerdo.

\item {} 
Cada componente de cor pode ter um elemento filho \sphinxcode{color} definindo
uma cor RGBA (Veja {\hyperref[techspecs/layout_files:layout\string-concepts\string-colours]{\sphinxcrossref{\DUrole{std,std-ref}{Cores}}}} para mais
informações).
Isso pode ser usado para controlar a geometria da cor dos componentes,
desenhados de forma algorítmica ou textual, sendo ignorado pelos
componentes \sphinxcode{image}. Caso tal elemento não esteja presente,
será usada uma cor predefinida que é branca e opaca.

\end{itemize}

Há suporte para os seguintes componentes:

\textbf{rect}
\begin{quote}

Desenha um retângulo colorido uniforme preenchendo as suas bordas.
\end{quote}

\textbf{disk}
\begin{quote}

Desenha uma elipse colorida uniforme ajustada às suas bordas.
\end{quote}

\textbf{image}
\begin{quote}

Desenha uma imagem carregada de um arquivo PNG ou JPEG. O nome do
arquivo a ser carregado (incluindo o nome da extensão do arquivo) é
informado usando o atributo \sphinxcode{file}. Adicionalmente, um atributo
opcional \sphinxcode{alphafile} pode ser usado para determinar o nome de um
arquivo PNG (incluindo o nome da extensão do arquivo) para ser
carregado dentro do canal alfa da imagem. O(s) arquivo(s) de
imagem(s) devem ser colocados no mesmo diretório que o arquivo de
layout. Caso o atributo \sphinxcode{alphafile} esteja relacionado a um
arquivo, ele deve ter as mesmas dimensões que o arquivo definido no
atributo \sphinxcode{file} e a sua profundidade de bits por pixel não deve
ser maior que 8 bits por canal. A intensidade de brightness dessa
imagem, é copiada para o canal alfa, com intensidade plena
(branco em escala de cinza) o que corresponde a um opaco
pleno e o preto a uma transparência plena.
\end{quote}

\textbf{text}
\begin{quote}

Desenha o texto usando a fonte da interface e na cor definida pelo
usuário. O texto a ser desenhado deve ser informado usado um
atributo \sphinxcode{string}.  Um atributo \sphinxcode{align} pode ser usado para
definir o alinhamento do texto. Se presente, o atributo \sphinxcode{align}
deve ser um valor inteiro onde (zero) significa centralizado, 1 (um)
significa alinhado à esquerda e 2 (dois) significa alinhado à direita.
Caso o atributo \sphinxcode{align} esteja ausente a predefinição determina
que o texto seja centralizado.
\end{quote}

\textbf{dotmatrix}
\begin{quote}

Desenha um segmento horizontal de oito pixels em um mostrador em
formato de matriz de pontos, usando pixels circulares em uma cor
determinada. Os bits que determinam o estado do elemento definem
quais os pixels que estarão acesos, com o bit de menor importância
correspondendo ao pixel mais à esquerda. Os pixels apagados são
desenhados com uma menor intensidade (\textbf{0x20/0xff}).
\end{quote}

\textbf{dotmatrix5dot}
\begin{quote}

Desenha um segmento horizontal de cinco pixels em um mostrador em
formato de matriz de pontos, usando pixels circulares em uma cor
determinada. Os bits que determinam o estado do elemento definem
quais os pixels que estarão acesos, com o bit de menor importância
correspondendo ao pixel mais à esquerda. Os pixels apagados são
desenhados com uma menor intensidade (\textbf{0x20/0xff}).
\end{quote}

\textbf{dotmatrixdot}
\begin{quote}

Desenha um único elemento de um mostrador em formato de de matriz de
pontos com pixels circulares em uma cor determinada. O bit de menor
importância do estado do elemento determina se o pixel vai estar
aceso. Um pixel apagado é desenhado com uma menor intensidade
(\textbf{0x20/0xff}).
\end{quote}

\textbf{led7seg}
\begin{quote}

Desenha um mostrador LED ou fluorescente alfanumérico comum com
dezesseis segmentos e o mostrador em uma cor determinada. Os oito bits
baixos do estado do elemento controlam quais os segmentos estarão
acesos. Começando pelo bit de menor importância a sequência de
atualização dos bits correspondentes começam no segmento superior,
superior direito, depois continuando no sentido horário para o
segmento superior esquerdo, a barra central e o ponto decimal.
Os pixels apagados são desenhados com uma menor intensidade
(\textbf{0x20/0xff}).
\end{quote}

\textbf{led8seg\_gts1}
\begin{quote}

Desenha um mostrador fluorescente digital de oito segmentos do tipo
usado em máquinas de fliperama \emph{Gottlieb System 1} \footnote[8]{\sphinxAtStartFootnote%
\href{https://www.youtube.com/watch?v=-rrP4Prx1rc}{Aqui} um exemplo
destes mostradores. (Nota do tradutor)
} (na verdade uma
parte da Futaba). Comparado com um mostrador padrão
com sete segmentos, esses mostradores não têm ponto decimal, a barra
do meio horizontal está quebrada no centro, assim como no meio da
barra vertical controlada pelo bit que controlaria o ponto decimal
num mostrador comum com sete segmentos. Os pixels apagados são
desenhados com uma menor intensidade (\textbf{0x20/0xff}).
\end{quote}

\textbf{led14seg}
\begin{quote}

Desenha um mostrador LED ou fluorescente alfanumérico padrão com
catorze segmentos em uma cor determinada. Os 14 bits mais baixos do
controle de estado do elemento determinam quais os segmentos estarão
acesos.
Começando pelo bit com menor importância, os bits correspondentes ao
segmento superior, o segmento superior direito, continuando no
sentido horário para o segmento superior esquerdo, as metades
esquerda e direita da barra central horizontal, as metades superior
e inferior do meio vertical da barra, e as barras diagonais no
sentido horário da parte inferior esquerda para a direita inferior.
Os pixels apagados são desenhados com uma menor intensidade
(\textbf{0x20/0xff}).
\end{quote}

\textbf{led14segsc}
\begin{quote}

Desenha um mostrador LED ou fluorescente alfanumérico padrão com
catorze segmentos com ponto decimal/vírgula em uma cor determinada. Os
16 bits baixos do elemento controlam quais segmentos estarão acesos.
Os 14 bits baixos correspondem aos mesmos segmentos que no
componente \sphinxcode{led14seg}. Dois bits adicionais correspondem ao ponto
decimal e cauda de vírgula. Os pixels apagados são desenhados com
uma menor intensidade (\textbf{0x20/0xff}).
\end{quote}

\textbf{led16seg}
\begin{quote}

Desenha um mostrador LED ou fluorescente alfanumérico padrão com dezesseis
segmentos em uma cor determinada. Os 16 bit baixos do elemento controlam
quais os elementos que estarão acesos. Começando pelo bit de menor
importância a sequência de atualização dos bits correspondentes
começam na metade esquerda da barra superior, a metade direita da
barra superior, continuando no sentido horário para o segmento
superior esquerdo, as metades esquerda e direita da barra central e
horizontal, as metades superior e inferior da barra do meio
vertical, e as barras diagonais no sentido horário a partir do canto
inferior esquerdo até a parte inferior direito. Os pixels apagados
são desenhados com uma menor intensidade
(\textbf{0x20/0xff}).
\end{quote}

\textbf{led16segsc}
\begin{quote}

Desenha um mostrador LED ou fluorescente alfanumérico padrão com
dezesseis segmentos e o ponto decimal em uma cor determinada.
Os 16 bits baixos do elemento controlam quais segmentos estarão
acesos. Os 18 bits inferiores correspondem aos mesmos controles de
estado dos segmentos que em \sphinxcode{led16seg}. Dois bits adicionais
correspondem ao ponto decimal e cauda de vírgula. Os pixels apagados
são desenhados com uma menor intensidade (\textbf{0x20/0xff}).
\end{quote}

\textbf{simplecounter}
\begin{quote}

Exibe o valor numérico do estado do elemento usando a fonte do sistema
em uma cor determinada. O valor é formatado em notação decimal. Um
atributo \sphinxcode{digits} pode ser informado para definir a quantidade
mínima de dígitos a serem exibidos. Se presente, o atributo
\sphinxcode{digits} deve ser um número inteiro, se ausente, um mínimo de dois
dígitos será exibido.

O atributo \sphinxcode{maxstate} pode ser informado
para definir o valor máximo do estado a ser exibido. Se presente, o atributo
\sphinxcode{maxstate} deve ser um número positivo; caso esteja ausente o valor
predefinido é \textbf{999}.  Um atributo \sphinxcode{align} pode ser usado para
determinar o alinhamento do texto. Caso esteja presente, o atributo
\sphinxcode{align} deve ser um número inteiro onde \textbf{0} significa alinhar
ao centro, \textbf{1} alinhar à esquerda e \textbf{2} alinhar à direita.
Na sua ausência o texto será centralizado.
\end{quote}

\textbf{reel}
\begin{quote}

Usado para desenhar os cilindros usados por máquinas de caça
níquel.
Os atributos compatíveis são \sphinxcode{symbollist}, \sphinxcode{stateoffset},
\sphinxcode{numsymbolsvisible}, \sphinxcode{reelreversed} e \sphinxcode{beltreel}.
\end{quote}

Um exemplo de um elemento que desenha um texto estático do lado esquerdo
da tela:

\begin{Verbatim}[commandchars=\\\{\}]
\PYG{o}{\PYGZlt{}}\PYG{n}{element} \PYG{n}{name}\PYG{o}{=}\PYG{l+s+s2}{\PYGZdq{}}\PYG{l+s+s2}{label\PYGZus{}reset\PYGZus{}cpu}\PYG{l+s+s2}{\PYGZdq{}}\PYG{o}{\PYGZgt{}}
    \PYG{o}{\PYGZlt{}}\PYG{n}{text} \PYG{n}{string}\PYG{o}{=}\PYG{l+s+s2}{\PYGZdq{}}\PYG{l+s+s2}{CPU}\PYG{l+s+s2}{\PYGZdq{}} \PYG{n}{align}\PYG{o}{=}\PYG{l+s+s2}{\PYGZdq{}}\PYG{l+s+s2}{1}\PYG{l+s+s2}{\PYGZdq{}}\PYG{o}{\PYGZgt{}}\PYG{o}{\PYGZlt{}}\PYG{n}{color} \PYG{n}{red}\PYG{o}{=}\PYG{l+s+s2}{\PYGZdq{}}\PYG{l+s+s2}{1.0}\PYG{l+s+s2}{\PYGZdq{}} \PYG{n}{green}\PYG{o}{=}\PYG{l+s+s2}{\PYGZdq{}}\PYG{l+s+s2}{1.0}\PYG{l+s+s2}{\PYGZdq{}} \PYG{n}{blue}\PYG{o}{=}\PYG{l+s+s2}{\PYGZdq{}}\PYG{l+s+s2}{1.0}\PYG{l+s+s2}{\PYGZdq{}} \PYG{o}{/}\PYG{o}{\PYGZgt{}}\PYG{o}{\PYGZlt{}}\PYG{o}{/}\PYG{n}{text}\PYG{o}{\PYGZgt{}}
\PYG{o}{\PYGZlt{}}\PYG{o}{/}\PYG{n}{element}\PYG{o}{\PYGZgt{}}
\end{Verbatim}

Um exemplo de um elemento que mostra um LED redondo onde a intensidade do
seu brilho depende do estado alto da saída:

\begin{Verbatim}[commandchars=\\\{\}]
\PYG{o}{\PYGZlt{}}\PYG{n}{element} \PYG{n}{name}\PYG{o}{=}\PYG{l+s+s2}{\PYGZdq{}}\PYG{l+s+s2}{led}\PYG{l+s+s2}{\PYGZdq{}} \PYG{n}{defstate}\PYG{o}{=}\PYG{l+s+s2}{\PYGZdq{}}\PYG{l+s+s2}{0}\PYG{l+s+s2}{\PYGZdq{}}\PYG{o}{\PYGZgt{}}
    \PYG{o}{\PYGZlt{}}\PYG{n}{rect} \PYG{n}{state}\PYG{o}{=}\PYG{l+s+s2}{\PYGZdq{}}\PYG{l+s+s2}{0}\PYG{l+s+s2}{\PYGZdq{}}\PYG{o}{\PYGZgt{}}\PYG{o}{\PYGZlt{}}\PYG{n}{color} \PYG{n}{red}\PYG{o}{=}\PYG{l+s+s2}{\PYGZdq{}}\PYG{l+s+s2}{0.43}\PYG{l+s+s2}{\PYGZdq{}} \PYG{n}{green}\PYG{o}{=}\PYG{l+s+s2}{\PYGZdq{}}\PYG{l+s+s2}{0.35}\PYG{l+s+s2}{\PYGZdq{}} \PYG{n}{blue}\PYG{o}{=}\PYG{l+s+s2}{\PYGZdq{}}\PYG{l+s+s2}{0.39}\PYG{l+s+s2}{\PYGZdq{}} \PYG{o}{/}\PYG{o}{\PYGZgt{}}\PYG{o}{\PYGZlt{}}\PYG{o}{/}\PYG{n}{rect}\PYG{o}{\PYGZgt{}}
    \PYG{o}{\PYGZlt{}}\PYG{n}{rect} \PYG{n}{state}\PYG{o}{=}\PYG{l+s+s2}{\PYGZdq{}}\PYG{l+s+s2}{1}\PYG{l+s+s2}{\PYGZdq{}}\PYG{o}{\PYGZgt{}}\PYG{o}{\PYGZlt{}}\PYG{n}{color} \PYG{n}{red}\PYG{o}{=}\PYG{l+s+s2}{\PYGZdq{}}\PYG{l+s+s2}{1.0}\PYG{l+s+s2}{\PYGZdq{}} \PYG{n}{green}\PYG{o}{=}\PYG{l+s+s2}{\PYGZdq{}}\PYG{l+s+s2}{0.18}\PYG{l+s+s2}{\PYGZdq{}} \PYG{n}{blue}\PYG{o}{=}\PYG{l+s+s2}{\PYGZdq{}}\PYG{l+s+s2}{0.20}\PYG{l+s+s2}{\PYGZdq{}} \PYG{o}{/}\PYG{o}{\PYGZgt{}}\PYG{o}{\PYGZlt{}}\PYG{o}{/}\PYG{n}{rect}\PYG{o}{\PYGZgt{}}
\PYG{o}{\PYGZlt{}}\PYG{o}{/}\PYG{n}{element}\PYG{o}{\PYGZgt{}}
\end{Verbatim}

Um exemplo de elemento de um botão que retorna um efeito visual quando
pressionado:

\begin{Verbatim}[commandchars=\\\{\}]
\PYG{o}{\PYGZlt{}}\PYG{n}{element} \PYG{n}{name}\PYG{o}{=}\PYG{l+s+s2}{\PYGZdq{}}\PYG{l+s+s2}{btn\PYGZus{}rst}\PYG{l+s+s2}{\PYGZdq{}}\PYG{o}{\PYGZgt{}}
    \PYG{o}{\PYGZlt{}}\PYG{n}{rect} \PYG{n}{state}\PYG{o}{=}\PYG{l+s+s2}{\PYGZdq{}}\PYG{l+s+s2}{0}\PYG{l+s+s2}{\PYGZdq{}}\PYG{o}{\PYGZgt{}}\PYG{o}{\PYGZlt{}}\PYG{n}{bounds} \PYG{n}{x}\PYG{o}{=}\PYG{l+s+s2}{\PYGZdq{}}\PYG{l+s+s2}{0.0}\PYG{l+s+s2}{\PYGZdq{}} \PYG{n}{y}\PYG{o}{=}\PYG{l+s+s2}{\PYGZdq{}}\PYG{l+s+s2}{0.0}\PYG{l+s+s2}{\PYGZdq{}} \PYG{n}{width}\PYG{o}{=}\PYG{l+s+s2}{\PYGZdq{}}\PYG{l+s+s2}{1.0}\PYG{l+s+s2}{\PYGZdq{}} \PYG{n}{height}\PYG{o}{=}\PYG{l+s+s2}{\PYGZdq{}}\PYG{l+s+s2}{1.0}\PYG{l+s+s2}{\PYGZdq{}} \PYG{o}{/}\PYG{o}{\PYGZgt{}}\PYG{o}{\PYGZlt{}}\PYG{n}{color} \PYG{n}{red}\PYG{o}{=}\PYG{l+s+s2}{\PYGZdq{}}\PYG{l+s+s2}{0.2}\PYG{l+s+s2}{\PYGZdq{}} \PYG{n}{green}\PYG{o}{=}\PYG{l+s+s2}{\PYGZdq{}}\PYG{l+s+s2}{0.2}\PYG{l+s+s2}{\PYGZdq{}} \PYG{n}{blue}\PYG{o}{=}\PYG{l+s+s2}{\PYGZdq{}}\PYG{l+s+s2}{0.2}\PYG{l+s+s2}{\PYGZdq{}} \PYG{o}{/}\PYG{o}{\PYGZgt{}}\PYG{o}{\PYGZlt{}}\PYG{o}{/}\PYG{n}{rect}\PYG{o}{\PYGZgt{}}
    \PYG{o}{\PYGZlt{}}\PYG{n}{rect} \PYG{n}{state}\PYG{o}{=}\PYG{l+s+s2}{\PYGZdq{}}\PYG{l+s+s2}{1}\PYG{l+s+s2}{\PYGZdq{}}\PYG{o}{\PYGZgt{}}\PYG{o}{\PYGZlt{}}\PYG{n}{bounds} \PYG{n}{x}\PYG{o}{=}\PYG{l+s+s2}{\PYGZdq{}}\PYG{l+s+s2}{0.0}\PYG{l+s+s2}{\PYGZdq{}} \PYG{n}{y}\PYG{o}{=}\PYG{l+s+s2}{\PYGZdq{}}\PYG{l+s+s2}{0.0}\PYG{l+s+s2}{\PYGZdq{}} \PYG{n}{width}\PYG{o}{=}\PYG{l+s+s2}{\PYGZdq{}}\PYG{l+s+s2}{1.0}\PYG{l+s+s2}{\PYGZdq{}} \PYG{n}{height}\PYG{o}{=}\PYG{l+s+s2}{\PYGZdq{}}\PYG{l+s+s2}{1.0}\PYG{l+s+s2}{\PYGZdq{}} \PYG{o}{/}\PYG{o}{\PYGZgt{}}\PYG{o}{\PYGZlt{}}\PYG{n}{color} \PYG{n}{red}\PYG{o}{=}\PYG{l+s+s2}{\PYGZdq{}}\PYG{l+s+s2}{0.1}\PYG{l+s+s2}{\PYGZdq{}} \PYG{n}{green}\PYG{o}{=}\PYG{l+s+s2}{\PYGZdq{}}\PYG{l+s+s2}{0.1}\PYG{l+s+s2}{\PYGZdq{}} \PYG{n}{blue}\PYG{o}{=}\PYG{l+s+s2}{\PYGZdq{}}\PYG{l+s+s2}{0.1}\PYG{l+s+s2}{\PYGZdq{}} \PYG{o}{/}\PYG{o}{\PYGZgt{}}\PYG{o}{\PYGZlt{}}\PYG{o}{/}\PYG{n}{rect}\PYG{o}{\PYGZgt{}}
    \PYG{o}{\PYGZlt{}}\PYG{n}{rect} \PYG{n}{state}\PYG{o}{=}\PYG{l+s+s2}{\PYGZdq{}}\PYG{l+s+s2}{0}\PYG{l+s+s2}{\PYGZdq{}}\PYG{o}{\PYGZgt{}}\PYG{o}{\PYGZlt{}}\PYG{n}{bounds} \PYG{n}{x}\PYG{o}{=}\PYG{l+s+s2}{\PYGZdq{}}\PYG{l+s+s2}{0.1}\PYG{l+s+s2}{\PYGZdq{}} \PYG{n}{y}\PYG{o}{=}\PYG{l+s+s2}{\PYGZdq{}}\PYG{l+s+s2}{0.1}\PYG{l+s+s2}{\PYGZdq{}} \PYG{n}{width}\PYG{o}{=}\PYG{l+s+s2}{\PYGZdq{}}\PYG{l+s+s2}{0.9}\PYG{l+s+s2}{\PYGZdq{}} \PYG{n}{height}\PYG{o}{=}\PYG{l+s+s2}{\PYGZdq{}}\PYG{l+s+s2}{0.9}\PYG{l+s+s2}{\PYGZdq{}} \PYG{o}{/}\PYG{o}{\PYGZgt{}}\PYG{o}{\PYGZlt{}}\PYG{n}{color} \PYG{n}{red}\PYG{o}{=}\PYG{l+s+s2}{\PYGZdq{}}\PYG{l+s+s2}{0.1}\PYG{l+s+s2}{\PYGZdq{}} \PYG{n}{green}\PYG{o}{=}\PYG{l+s+s2}{\PYGZdq{}}\PYG{l+s+s2}{0.1}\PYG{l+s+s2}{\PYGZdq{}} \PYG{n}{blue}\PYG{o}{=}\PYG{l+s+s2}{\PYGZdq{}}\PYG{l+s+s2}{0.1}\PYG{l+s+s2}{\PYGZdq{}} \PYG{o}{/}\PYG{o}{\PYGZgt{}}\PYG{o}{\PYGZlt{}}\PYG{o}{/}\PYG{n}{rect}\PYG{o}{\PYGZgt{}}
    \PYG{o}{\PYGZlt{}}\PYG{n}{rect} \PYG{n}{state}\PYG{o}{=}\PYG{l+s+s2}{\PYGZdq{}}\PYG{l+s+s2}{1}\PYG{l+s+s2}{\PYGZdq{}}\PYG{o}{\PYGZgt{}}\PYG{o}{\PYGZlt{}}\PYG{n}{bounds} \PYG{n}{x}\PYG{o}{=}\PYG{l+s+s2}{\PYGZdq{}}\PYG{l+s+s2}{0.1}\PYG{l+s+s2}{\PYGZdq{}} \PYG{n}{y}\PYG{o}{=}\PYG{l+s+s2}{\PYGZdq{}}\PYG{l+s+s2}{0.1}\PYG{l+s+s2}{\PYGZdq{}} \PYG{n}{width}\PYG{o}{=}\PYG{l+s+s2}{\PYGZdq{}}\PYG{l+s+s2}{0.9}\PYG{l+s+s2}{\PYGZdq{}} \PYG{n}{height}\PYG{o}{=}\PYG{l+s+s2}{\PYGZdq{}}\PYG{l+s+s2}{0.9}\PYG{l+s+s2}{\PYGZdq{}} \PYG{o}{/}\PYG{o}{\PYGZgt{}}\PYG{o}{\PYGZlt{}}\PYG{n}{color} \PYG{n}{red}\PYG{o}{=}\PYG{l+s+s2}{\PYGZdq{}}\PYG{l+s+s2}{0.2}\PYG{l+s+s2}{\PYGZdq{}} \PYG{n}{green}\PYG{o}{=}\PYG{l+s+s2}{\PYGZdq{}}\PYG{l+s+s2}{0.2}\PYG{l+s+s2}{\PYGZdq{}} \PYG{n}{blue}\PYG{o}{=}\PYG{l+s+s2}{\PYGZdq{}}\PYG{l+s+s2}{0.2}\PYG{l+s+s2}{\PYGZdq{}} \PYG{o}{/}\PYG{o}{\PYGZgt{}}\PYG{o}{\PYGZlt{}}\PYG{o}{/}\PYG{n}{rect}\PYG{o}{\PYGZgt{}}
    \PYG{o}{\PYGZlt{}}\PYG{n}{rect}\PYG{o}{\PYGZgt{}}\PYG{o}{\PYGZlt{}}\PYG{n}{bounds} \PYG{n}{x}\PYG{o}{=}\PYG{l+s+s2}{\PYGZdq{}}\PYG{l+s+s2}{0.1}\PYG{l+s+s2}{\PYGZdq{}} \PYG{n}{y}\PYG{o}{=}\PYG{l+s+s2}{\PYGZdq{}}\PYG{l+s+s2}{0.1}\PYG{l+s+s2}{\PYGZdq{}} \PYG{n}{width}\PYG{o}{=}\PYG{l+s+s2}{\PYGZdq{}}\PYG{l+s+s2}{0.8}\PYG{l+s+s2}{\PYGZdq{}} \PYG{n}{height}\PYG{o}{=}\PYG{l+s+s2}{\PYGZdq{}}\PYG{l+s+s2}{0.8}\PYG{l+s+s2}{\PYGZdq{}} \PYG{o}{/}\PYG{o}{\PYGZgt{}}\PYG{o}{\PYGZlt{}}\PYG{n}{color} \PYG{n}{red}\PYG{o}{=}\PYG{l+s+s2}{\PYGZdq{}}\PYG{l+s+s2}{0.15}\PYG{l+s+s2}{\PYGZdq{}} \PYG{n}{green}\PYG{o}{=}\PYG{l+s+s2}{\PYGZdq{}}\PYG{l+s+s2}{0.15}\PYG{l+s+s2}{\PYGZdq{}} \PYG{n}{blue}\PYG{o}{=}\PYG{l+s+s2}{\PYGZdq{}}\PYG{l+s+s2}{0.15}\PYG{l+s+s2}{\PYGZdq{}} \PYG{o}{/}\PYG{o}{\PYGZgt{}}\PYG{o}{\PYGZlt{}}\PYG{o}{/}\PYG{n}{rect}\PYG{o}{\PYGZgt{}}
    \PYG{o}{\PYGZlt{}}\PYG{n}{text} \PYG{n}{string}\PYG{o}{=}\PYG{l+s+s2}{\PYGZdq{}}\PYG{l+s+s2}{RESET}\PYG{l+s+s2}{\PYGZdq{}}\PYG{o}{\PYGZgt{}}\PYG{o}{\PYGZlt{}}\PYG{n}{bounds} \PYG{n}{x}\PYG{o}{=}\PYG{l+s+s2}{\PYGZdq{}}\PYG{l+s+s2}{0.1}\PYG{l+s+s2}{\PYGZdq{}} \PYG{n}{y}\PYG{o}{=}\PYG{l+s+s2}{\PYGZdq{}}\PYG{l+s+s2}{0.4}\PYG{l+s+s2}{\PYGZdq{}} \PYG{n}{width}\PYG{o}{=}\PYG{l+s+s2}{\PYGZdq{}}\PYG{l+s+s2}{0.8}\PYG{l+s+s2}{\PYGZdq{}} \PYG{n}{height}\PYG{o}{=}\PYG{l+s+s2}{\PYGZdq{}}\PYG{l+s+s2}{0.2}\PYG{l+s+s2}{\PYGZdq{}} \PYG{o}{/}\PYG{o}{\PYGZgt{}}\PYG{o}{\PYGZlt{}}\PYG{n}{color} \PYG{n}{red}\PYG{o}{=}\PYG{l+s+s2}{\PYGZdq{}}\PYG{l+s+s2}{1.0}\PYG{l+s+s2}{\PYGZdq{}} \PYG{n}{green}\PYG{o}{=}\PYG{l+s+s2}{\PYGZdq{}}\PYG{l+s+s2}{1.0}\PYG{l+s+s2}{\PYGZdq{}} \PYG{n}{blue}\PYG{o}{=}\PYG{l+s+s2}{\PYGZdq{}}\PYG{l+s+s2}{1.0}\PYG{l+s+s2}{\PYGZdq{}} \PYG{o}{/}\PYG{o}{\PYGZgt{}}\PYG{o}{\PYGZlt{}}\PYG{o}{/}\PYG{n}{text}\PYG{o}{\PYGZgt{}}
\PYG{o}{\PYGZlt{}}\PYG{o}{/}\PYG{n}{element}\PYG{o}{\PYGZgt{}}
\end{Verbatim}


\subsubsection{Exibições}
\label{techspecs/layout_files:exibicoes}\label{techspecs/layout_files:layout-parts-views}
Uma exibição define um arranjo de elementos ou imagens na tela emulada
que podem ser exibidas em uma janela ou em uma tela.
As exibições também conectam elementos as entradas I/O e saídas
emuladas.
Um arquivo de layout podem conter vários modos de exibição. Caso uma
exibição corresponda a uma tela inexistente, ela se torna
\emph{inviável}.

O MAME exibirá uma mensagem de aviso, irá ignorar a exibição que for
inviável e continuará a carregar as exibições do arquivo de layout.
Isso é muito útil para sistemas onde uma tela é opcional, por exemplo,
computadores com controles do painel frontal e um terminal serial
opcional.

As exibições são identificadas pelo nome na interface do usuário
do MAME e na linha de comando. Para arquivos de layouts associados a
dispositivos outros que o dispositivo de driver raiz, os nomes das
exibições dos dispositivos são precedidos por uma tag (com os dois
pontos iniciais omitidos) por exemplo, para exibir um dispositivo
chamado ``\emph{Keyboard LEDs}'' vindo do dispositivo \sphinxcode{:tty:ie15}, ele deve ser
associado como \textbf{tty:ie15 Keyboard LEDs} na interface do usuário do
MAME.
As exibições são mostradas na ordem em que são carregadas.
Dentro de um arquivo de layout, as exibições são carregados em ordem
de chegada, começando de cima para baixo.

As exibições são criadas com elementos \sphinxcode{view} dentro de um atributo de
nível primário do elemento \sphinxcode{mamelayout}. Cada elemento \sphinxcode{view} deve
ter um nome usando o atributo \sphinxcode{name}, informando seu nome legível para
o uso na interface do usuário e nas opções de linha de comando.
Este é um exemplo de uma tag inicial válida para um elemento
\sphinxcode{view}:

\begin{Verbatim}[commandchars=\\\{\}]
\PYG{o}{\PYGZlt{}}\PYG{n}{view} \PYG{n}{name}\PYG{o}{=}\PYG{l+s+s2}{\PYGZdq{}}\PYG{l+s+s2}{Control panel}\PYG{l+s+s2}{\PYGZdq{}}\PYG{o}{\PYGZgt{}}
\end{Verbatim}

O elemento ``view'' cria um escopo emaranhado dentro do parâmetro de escopo
de primeiro nível \sphinxcode{mamelayout}. Por razões históricas, os elementos
\sphinxcode{view} são processados \emph{depois} de todos os outros elementos
herdados de \sphinxcode{mamelayout}. Isso significa que uma exibição pode
fazer referência a elementos e grupos que apareçam depois naquele
arquivo, os parâmetros anexados ao escopo terão seus valores ao final do
elemento \sphinxcode{mamelayout}.

Os seguintes elementos filho são permitidos dentro do elemento \sphinxcode{view}:

\textbf{bounds}
\begin{quote}

Define a origem e o tamanho da exibição interna do sistema de
coordenadas caso esteja presente.
Veja {\hyperref[techspecs/layout_files:layout\string-concepts\string-coordinates]{\sphinxcrossref{\DUrole{std,std-ref}{Coordenadas}}}} para maiores detalhes.
Se ausente, os limites de exibição serão computados unindo os
limites de todas as telas e elementos dentro da região sendo
exibida. Só faz sentido ter um elemento \sphinxcode{bounds} como um filho
direto de um elemento \sphinxcode{view}. Qualquer conteúdo fora dos limites
da exibição serão recortados e a visualização será redimensionada
proporcionalmente para se ajustar aos limites da tela ou janela.
\end{quote}

\textbf{param}
\begin{quote}

Define ou reatribui um parâmetro de valor no escopo da exibição. Veja
{\hyperref[techspecs/layout_files:layout\string-concepts\string-params]{\sphinxcrossref{\DUrole{std,std-ref}{Parâmetros}}}} para mais informações.
\end{quote}

\textbf{backdrop, overlay, bezel, cpanel e marquise}
\begin{quote}

Adiciona um elemento à camada relevante
(veja {\hyperref[techspecs/layout_files:layout\string-parts]{\sphinxcrossref{\DUrole{std,std-ref}{As partes de um layout}}}} e {\hyperref[techspecs/layout_files:layout\string-concepts\string-layers]{\sphinxcrossref{\DUrole{std,std-ref}{Camadas}}}}).
O nome do elemento a adicionar é definido usando o atributo
\sphinxcode{element}. Será considerado um erro caso nenhum elemento com este
nome seja definido no arquivo de layout. Opcionalmente, pode ser
conectado a uma porta I/O emulada usando os atributos \sphinxcode{inputtag},
\sphinxcode{inputmask} ou uma saída emulada usando o atributo \sphinxcode{name}.
Dentro de uma camada, os elementos são desenhados na ordem em que
forem aparecendo no arquivo de layout. A sua ordem de exibição
começa de frente para trás.
Veja abaixo para mais detalhes.
\end{quote}

\textbf{screen}
\begin{quote}

Adiciona uma imagem de tela emulada à exibição. A tela deve ser
identificada usando um atributo \sphinxcode{index} ou um atributo \sphinxcode{tag}
(um elemento \sphinxcode{screen} não pode ter ambos os atributos \sphinxcode{index} e
\sphinxcode{tag}).
Se presente, o atributo \sphinxcode{index} deve ser um valor inteiro e não
negativo. As telas são numeradas pela ordem em que aparecem na
configuração da máquina, começando com zero (\textbf{0}). Se presente, o
atributo \sphinxcode{tag} deve ser o caminho da tag para a tela em relação ao
dispositivo que provoque a leitura do layout. As telas são
desenhadas na ordem em que aparecem no arquivo de layout, A sua
ordem de exibição começa de frente para trás.
\end{quote}

\textbf{group}
\begin{quote}

Adiciona o conteúdo do grupo à exibição
(veja {\hyperref[techspecs/layout_files:layout\string-parts\string-groups]{\sphinxcrossref{\DUrole{std,std-ref}{Grupos reutilizáveis}}}}).
Para adicionar o nome do grupo use o atributo \sphinxcode{ref}. Será
considerado um erro caso nenhum grupo com este nome seja definido
no arquivo de layout. Veja abaixo para mais informações sobre a
questão de posicionamento.
\end{quote}

\textbf{repeat}
\begin{quote}

Repete o seu conteúdo definindo a sua quantidade pelo atributo
\sphinxcode{count}. O atributo \sphinxcode{count} deve ser um número inteiro e
positivo. Em uma exibição, o elemento \sphinxcode{repeat} pode conter os
elementos \sphinxcode{backdrop}, \sphinxcode{screen}, \sphinxcode{overlay}, \sphinxcode{bezel},
\sphinxcode{cpanel}, \sphinxcode{marquee}, \sphinxcode{group} e mais elementos \sphinxcode{repeat}, que
funcionam da mesma maneira que quando colocados em uma visualização
direta.
Veja {\hyperref[techspecs/layout_files:layout\string-parts\string-repeats]{\sphinxcrossref{\DUrole{std,std-ref}{Repetindo Blocos}}}} para uma discução de como usar os
elementos \sphinxcode{repeat}.
\end{quote}

As Telas com elementos \sphinxcode{screen},  elementos de layout \sphinxcode{backdrop},
\sphinxcode{overlay}, \sphinxcode{bezel}, \sphinxcode{cpanel} ou elementos \sphinxcode{marquee} e elementos
de grupos (\sphinxcode{group}) podem ter a sua orientação alterada usando um
elemento filho \sphinxcode{orientation}.
Para as telas, os modificadores de orientação são aplicados junto com os
modificadores de orientação definido no dispositivo de tela da máquina.
O elemento \sphinxcode{orientation} suporta os seguintes atributos, todos
eles são opcionais:

\textbf{rotate}
\begin{quote}

Se presente, aplica rotação no sentido horário em incrementos de
noventa graus. Deve ser um número inteiro igual a \textbf{0}, \textbf{90}, ou
\textbf{270}.
\end{quote}

\textbf{swapxy}
\begin{quote}

Permite que a tela, elemento ou grupo seja espelhado ao longo de uma
linha em quarenta e cinco graus para vertical, da esquerda para a
direita. Se presente deve ser entre \sphinxcode{yes} ou \sphinxcode{no}.
O espelhamento se aplica logicamente após a rotação.
\end{quote}

\textbf{flipx}
\begin{quote}

Permite que a tela, elemento ou grupo sejam espelhados à partir de
uma linha com 45 graus em torno de seu eixo vertical, vindo da quina
superior esquerda até a quina inferior direita. Se presente deve ser
entre \sphinxcode{yes} ou \sphinxcode{no}.
O espelhamento ocorre após a rotação.
\end{quote}

\textbf{flipy}
\begin{quote}

Permite que a tela, elemento ou grupo sejam espelhado ao redor do seu
eixo horizontal, de cima para baixo. Se presente, deve ser entre
\sphinxcode{yes} ou \sphinxcode{no}. O espelhamento ocorre após a rotação.
\end{quote}

As Telas (elementos \sphinxcode{screen}), elementos de layout
(\sphinxcode{backdrop}, \sphinxcode{overlay}, \sphinxcode{bezel}, \sphinxcode{cpanel} ou \sphinxcode{marquee}) e
elementos de grupo (\sphinxcode{group}) podem ser posicionados e redimensionados
usando um elemento \sphinxcode{bounds}
(veja {\hyperref[techspecs/layout_files:layout\string-concepts\string-coordinates]{\sphinxcrossref{\DUrole{std,std-ref}{Coordenadas}}}} para mais informações).
Na ausência do elemento \sphinxcode{bounds} os elementos ``screens'' e ``layout''
retornam aos valores predefinidos em unidades quadradas (origem em
\textbf{0,0} e ambos os valores de altura e largura serão igual a \textbf{1}).
Na ausência do elemento filho \sphinxcode{bounds}, os grupos serão expandidos sem
tradução ou escala (note que os grupos podem posicionar as telas ou
elementos fora dos seus limites. Este exemplo mostra uma exibição
com referência a posição da tela com um elemento de layout individual e
dois grupos de elementos:

\begin{Verbatim}[commandchars=\\\{\}]
\PYG{o}{\PYGZlt{}}\PYG{n}{view} \PYG{n}{name}\PYG{o}{=}\PYG{l+s+s2}{\PYGZdq{}}\PYG{l+s+s2}{LED Displays, Terminal and Keypad}\PYG{l+s+s2}{\PYGZdq{}}\PYG{o}{\PYGZgt{}}
    \PYG{o}{\PYGZlt{}}\PYG{n}{cpanel} \PYG{n}{element}\PYG{o}{=}\PYG{l+s+s2}{\PYGZdq{}}\PYG{l+s+s2}{beige}\PYG{l+s+s2}{\PYGZdq{}}\PYG{o}{\PYGZgt{}}\PYG{o}{\PYGZlt{}}\PYG{n}{bounds} \PYG{n}{x}\PYG{o}{=}\PYG{l+s+s2}{\PYGZdq{}}\PYG{l+s+s2}{320}\PYG{l+s+s2}{\PYGZdq{}} \PYG{n}{y}\PYG{o}{=}\PYG{l+s+s2}{\PYGZdq{}}\PYG{l+s+s2}{0}\PYG{l+s+s2}{\PYGZdq{}} \PYG{n}{width}\PYG{o}{=}\PYG{l+s+s2}{\PYGZdq{}}\PYG{l+s+s2}{172}\PYG{l+s+s2}{\PYGZdq{}} \PYG{n}{height}\PYG{o}{=}\PYG{l+s+s2}{\PYGZdq{}}\PYG{l+s+s2}{372}\PYG{l+s+s2}{\PYGZdq{}} \PYG{o}{/}\PYG{o}{\PYGZgt{}}\PYG{o}{\PYGZlt{}}\PYG{o}{/}\PYG{n}{cpanel}\PYG{o}{\PYGZgt{}}
    \PYG{o}{\PYGZlt{}}\PYG{n}{group} \PYG{n}{ref}\PYG{o}{=}\PYG{l+s+s2}{\PYGZdq{}}\PYG{l+s+s2}{displays}\PYG{l+s+s2}{\PYGZdq{}}\PYG{o}{\PYGZgt{}}\PYG{o}{\PYGZlt{}}\PYG{n}{bounds} \PYG{n}{x}\PYG{o}{=}\PYG{l+s+s2}{\PYGZdq{}}\PYG{l+s+s2}{0}\PYG{l+s+s2}{\PYGZdq{}} \PYG{n}{y}\PYG{o}{=}\PYG{l+s+s2}{\PYGZdq{}}\PYG{l+s+s2}{0}\PYG{l+s+s2}{\PYGZdq{}} \PYG{n}{width}\PYG{o}{=}\PYG{l+s+s2}{\PYGZdq{}}\PYG{l+s+s2}{320}\PYG{l+s+s2}{\PYGZdq{}} \PYG{n}{height}\PYG{o}{=}\PYG{l+s+s2}{\PYGZdq{}}\PYG{l+s+s2}{132}\PYG{l+s+s2}{\PYGZdq{}} \PYG{o}{/}\PYG{o}{\PYGZgt{}}\PYG{o}{\PYGZlt{}}\PYG{o}{/}\PYG{n}{group}\PYG{o}{\PYGZgt{}}
    \PYG{o}{\PYGZlt{}}\PYG{n}{group} \PYG{n}{ref}\PYG{o}{=}\PYG{l+s+s2}{\PYGZdq{}}\PYG{l+s+s2}{keypad}\PYG{l+s+s2}{\PYGZdq{}}\PYG{o}{\PYGZgt{}}\PYG{o}{\PYGZlt{}}\PYG{n}{bounds} \PYG{n}{x}\PYG{o}{=}\PYG{l+s+s2}{\PYGZdq{}}\PYG{l+s+s2}{336}\PYG{l+s+s2}{\PYGZdq{}} \PYG{n}{y}\PYG{o}{=}\PYG{l+s+s2}{\PYGZdq{}}\PYG{l+s+s2}{16}\PYG{l+s+s2}{\PYGZdq{}} \PYG{n}{width}\PYG{o}{=}\PYG{l+s+s2}{\PYGZdq{}}\PYG{l+s+s2}{140}\PYG{l+s+s2}{\PYGZdq{}} \PYG{n}{height}\PYG{o}{=}\PYG{l+s+s2}{\PYGZdq{}}\PYG{l+s+s2}{260}\PYG{l+s+s2}{\PYGZdq{}} \PYG{o}{/}\PYG{o}{\PYGZgt{}}\PYG{o}{\PYGZlt{}}\PYG{o}{/}\PYG{n}{group}\PYG{o}{\PYGZgt{}}
    \PYG{o}{\PYGZlt{}}\PYG{n}{screen} \PYG{n}{index}\PYG{o}{=}\PYG{l+s+s2}{\PYGZdq{}}\PYG{l+s+s2}{0}\PYG{l+s+s2}{\PYGZdq{}}\PYG{o}{\PYGZgt{}}\PYG{o}{\PYGZlt{}}\PYG{n}{bounds} \PYG{n}{x}\PYG{o}{=}\PYG{l+s+s2}{\PYGZdq{}}\PYG{l+s+s2}{0}\PYG{l+s+s2}{\PYGZdq{}} \PYG{n}{y}\PYG{o}{=}\PYG{l+s+s2}{\PYGZdq{}}\PYG{l+s+s2}{132}\PYG{l+s+s2}{\PYGZdq{}} \PYG{n}{width}\PYG{o}{=}\PYG{l+s+s2}{\PYGZdq{}}\PYG{l+s+s2}{320}\PYG{l+s+s2}{\PYGZdq{}} \PYG{n}{height}\PYG{o}{=}\PYG{l+s+s2}{\PYGZdq{}}\PYG{l+s+s2}{240}\PYG{l+s+s2}{\PYGZdq{}} \PYG{o}{/}\PYG{o}{\PYGZgt{}}\PYG{o}{\PYGZlt{}}\PYG{o}{/}\PYG{n}{screen}\PYG{o}{\PYGZgt{}}
\PYG{o}{\PYGZlt{}}\PYG{o}{/}\PYG{n}{view}\PYG{o}{\PYGZgt{}}
\end{Verbatim}

As Telas (elementos \sphinxcode{screen}), elementos de layout
(\sphinxcode{backdrop}, \sphinxcode{overlay}, \sphinxcode{bezel}, \sphinxcode{cpanel} ou \sphinxcode{marquee}) e
elementos de grupos (\sphinxcode{group}) podem ter um elemento filho \sphinxcode{color}
(veja {\hyperref[techspecs/layout_files:layout\string-concepts\string-colours]{\sphinxcrossref{\DUrole{std,std-ref}{Cores}}}}) ao definir uma cor
modificadora.
As cores componentes da tela ou elementos de layout são multiplicados
por essa cor.

Caso um elemento referencie um elemento de layout
(\sphinxcode{backdrop}, \sphinxcode{overlay}, \sphinxcode{bezel}, \sphinxcode{cpanel} ou \sphinxcode{marquee}) que
tenham os atributos \sphinxcode{inputtag} e \sphinxcode{inputmask}, ao clicar neles será o
mesmo que pressionar uma tecla ou botão correspondente mapeado para
essa(s) entrada(s).
O \sphinxcode{inputtag} define o caminho da tag de uma porta de I/O em relação
ao dispositivo que fez com que o arquivo de layout fosse carregado. O
atributo \sphinxcode{inputmask} deve ser um número inteiro definindo os bits
da porta de I/O que o elemento deve ativar. Este exemplo mostra a
inicialização dos botões pressionáveis:

\begin{Verbatim}[commandchars=\\\{\}]
\PYG{o}{\PYGZlt{}}\PYG{n}{cpanel} \PYG{n}{element}\PYG{o}{=}\PYG{l+s+s2}{\PYGZdq{}}\PYG{l+s+s2}{btn\PYGZus{}3}\PYG{l+s+s2}{\PYGZdq{}} \PYG{n}{inputtag}\PYG{o}{=}\PYG{l+s+s2}{\PYGZdq{}}\PYG{l+s+s2}{X2}\PYG{l+s+s2}{\PYGZdq{}} \PYG{n}{inputmask}\PYG{o}{=}\PYG{l+s+s2}{\PYGZdq{}}\PYG{l+s+s2}{0x10}\PYG{l+s+s2}{\PYGZdq{}}\PYG{o}{\PYGZgt{}}
    \PYG{o}{\PYGZlt{}}\PYG{n}{bounds} \PYG{n}{x}\PYG{o}{=}\PYG{l+s+s2}{\PYGZdq{}}\PYG{l+s+s2}{2.30}\PYG{l+s+s2}{\PYGZdq{}} \PYG{n}{y}\PYG{o}{=}\PYG{l+s+s2}{\PYGZdq{}}\PYG{l+s+s2}{4.325}\PYG{l+s+s2}{\PYGZdq{}} \PYG{n}{width}\PYG{o}{=}\PYG{l+s+s2}{\PYGZdq{}}\PYG{l+s+s2}{1.0}\PYG{l+s+s2}{\PYGZdq{}} \PYG{n}{height}\PYG{o}{=}\PYG{l+s+s2}{\PYGZdq{}}\PYG{l+s+s2}{1.0}\PYG{l+s+s2}{\PYGZdq{}} \PYG{o}{/}\PYG{o}{\PYGZgt{}}
\PYG{o}{\PYGZlt{}}\PYG{o}{/}\PYG{n}{cpanel}\PYG{o}{\PYGZgt{}}
\PYG{o}{\PYGZlt{}}\PYG{n}{cpanel} \PYG{n}{element}\PYG{o}{=}\PYG{l+s+s2}{\PYGZdq{}}\PYG{l+s+s2}{btn\PYGZus{}0}\PYG{l+s+s2}{\PYGZdq{}} \PYG{n}{inputtag}\PYG{o}{=}\PYG{l+s+s2}{\PYGZdq{}}\PYG{l+s+s2}{X0}\PYG{l+s+s2}{\PYGZdq{}} \PYG{n}{inputmask}\PYG{o}{=}\PYG{l+s+s2}{\PYGZdq{}}\PYG{l+s+s2}{0x20}\PYG{l+s+s2}{\PYGZdq{}}\PYG{o}{\PYGZgt{}}
    \PYG{o}{\PYGZlt{}}\PYG{n}{bounds} \PYG{n}{x}\PYG{o}{=}\PYG{l+s+s2}{\PYGZdq{}}\PYG{l+s+s2}{0.725}\PYG{l+s+s2}{\PYGZdq{}} \PYG{n}{y}\PYG{o}{=}\PYG{l+s+s2}{\PYGZdq{}}\PYG{l+s+s2}{5.375}\PYG{l+s+s2}{\PYGZdq{}} \PYG{n}{width}\PYG{o}{=}\PYG{l+s+s2}{\PYGZdq{}}\PYG{l+s+s2}{1.0}\PYG{l+s+s2}{\PYGZdq{}} \PYG{n}{height}\PYG{o}{=}\PYG{l+s+s2}{\PYGZdq{}}\PYG{l+s+s2}{1.0}\PYG{l+s+s2}{\PYGZdq{}} \PYG{o}{/}\PYG{o}{\PYGZgt{}}\PYG{o}{\PYGZlt{}}\PYG{o}{/}\PYG{n}{cpanel}\PYG{o}{\PYGZgt{}}
\PYG{o}{\PYGZlt{}}\PYG{n}{cpanel} \PYG{n}{element}\PYG{o}{=}\PYG{l+s+s2}{\PYGZdq{}}\PYG{l+s+s2}{btn\PYGZus{}rst}\PYG{l+s+s2}{\PYGZdq{}} \PYG{n}{inputtag}\PYG{o}{=}\PYG{l+s+s2}{\PYGZdq{}}\PYG{l+s+s2}{RESET}\PYG{l+s+s2}{\PYGZdq{}} \PYG{n}{inputmask}\PYG{o}{=}\PYG{l+s+s2}{\PYGZdq{}}\PYG{l+s+s2}{0x01}\PYG{l+s+s2}{\PYGZdq{}}\PYG{o}{\PYGZgt{}}
    \PYG{o}{\PYGZlt{}}\PYG{n}{bounds} \PYG{n}{x}\PYG{o}{=}\PYG{l+s+s2}{\PYGZdq{}}\PYG{l+s+s2}{1.775}\PYG{l+s+s2}{\PYGZdq{}} \PYG{n}{y}\PYG{o}{=}\PYG{l+s+s2}{\PYGZdq{}}\PYG{l+s+s2}{5.375}\PYG{l+s+s2}{\PYGZdq{}} \PYG{n}{width}\PYG{o}{=}\PYG{l+s+s2}{\PYGZdq{}}\PYG{l+s+s2}{1.0}\PYG{l+s+s2}{\PYGZdq{}} \PYG{n}{height}\PYG{o}{=}\PYG{l+s+s2}{\PYGZdq{}}\PYG{l+s+s2}{1.0}\PYG{l+s+s2}{\PYGZdq{}} \PYG{o}{/}\PYG{o}{\PYGZgt{}}
\PYG{o}{\PYGZlt{}}\PYG{o}{/}\PYG{n}{cpanel}\PYG{o}{\PYGZgt{}}
\end{Verbatim}

Caso um elemento referencie um elemento de layout
(\sphinxcode{backdrop}, \sphinxcode{overlay}, \sphinxcode{bezel}, \sphinxcode{cpanel} ou \sphinxcode{marquee}) e
tenha um atributo \sphinxcode{name}, ele usará seu estado com base no valor
correspondente da saída emulada com o mesmo nome. Observe que os nomes
de saída são globais, o que pode se tornar um problema quando uma
máquina usar diferentes categorias do mesmo tipo de dispositivo.
Veja {\hyperref[techspecs/layout_files:layout\string-parts\string-elements]{\sphinxcrossref{\DUrole{std,std-ref}{Elementos}}}} para mais informações de como um
estado do elemento afeta a sua aparência. Este exemplo mostra como os
mostradores digitais podem ser conectados nas saídas emuladas:

\begin{Verbatim}[commandchars=\\\{\}]
\PYG{o}{\PYGZlt{}}\PYG{n}{cpanel} \PYG{n}{name}\PYG{o}{=}\PYG{l+s+s2}{\PYGZdq{}}\PYG{l+s+s2}{digit6}\PYG{l+s+s2}{\PYGZdq{}} \PYG{n}{element}\PYG{o}{=}\PYG{l+s+s2}{\PYGZdq{}}\PYG{l+s+s2}{digit}\PYG{l+s+s2}{\PYGZdq{}}\PYG{o}{\PYGZgt{}}\PYG{o}{\PYGZlt{}}\PYG{n}{bounds} \PYG{n}{x}\PYG{o}{=}\PYG{l+s+s2}{\PYGZdq{}}\PYG{l+s+s2}{16}\PYG{l+s+s2}{\PYGZdq{}} \PYG{n}{y}\PYG{o}{=}\PYG{l+s+s2}{\PYGZdq{}}\PYG{l+s+s2}{16}\PYG{l+s+s2}{\PYGZdq{}} \PYG{n}{width}\PYG{o}{=}\PYG{l+s+s2}{\PYGZdq{}}\PYG{l+s+s2}{48}\PYG{l+s+s2}{\PYGZdq{}} \PYG{n}{height}\PYG{o}{=}\PYG{l+s+s2}{\PYGZdq{}}\PYG{l+s+s2}{80}\PYG{l+s+s2}{\PYGZdq{}} \PYG{o}{/}\PYG{o}{\PYGZgt{}}\PYG{o}{\PYGZlt{}}\PYG{o}{/}\PYG{n}{cpanel}\PYG{o}{\PYGZgt{}}
\PYG{o}{\PYGZlt{}}\PYG{n}{cpanel} \PYG{n}{name}\PYG{o}{=}\PYG{l+s+s2}{\PYGZdq{}}\PYG{l+s+s2}{digit5}\PYG{l+s+s2}{\PYGZdq{}} \PYG{n}{element}\PYG{o}{=}\PYG{l+s+s2}{\PYGZdq{}}\PYG{l+s+s2}{digit}\PYG{l+s+s2}{\PYGZdq{}}\PYG{o}{\PYGZgt{}}\PYG{o}{\PYGZlt{}}\PYG{n}{bounds} \PYG{n}{x}\PYG{o}{=}\PYG{l+s+s2}{\PYGZdq{}}\PYG{l+s+s2}{64}\PYG{l+s+s2}{\PYGZdq{}} \PYG{n}{y}\PYG{o}{=}\PYG{l+s+s2}{\PYGZdq{}}\PYG{l+s+s2}{16}\PYG{l+s+s2}{\PYGZdq{}} \PYG{n}{width}\PYG{o}{=}\PYG{l+s+s2}{\PYGZdq{}}\PYG{l+s+s2}{48}\PYG{l+s+s2}{\PYGZdq{}} \PYG{n}{height}\PYG{o}{=}\PYG{l+s+s2}{\PYGZdq{}}\PYG{l+s+s2}{80}\PYG{l+s+s2}{\PYGZdq{}} \PYG{o}{/}\PYG{o}{\PYGZgt{}}\PYG{o}{\PYGZlt{}}\PYG{o}{/}\PYG{n}{cpanel}\PYG{o}{\PYGZgt{}}
\PYG{o}{\PYGZlt{}}\PYG{n}{cpanel} \PYG{n}{name}\PYG{o}{=}\PYG{l+s+s2}{\PYGZdq{}}\PYG{l+s+s2}{digit4}\PYG{l+s+s2}{\PYGZdq{}} \PYG{n}{element}\PYG{o}{=}\PYG{l+s+s2}{\PYGZdq{}}\PYG{l+s+s2}{digit}\PYG{l+s+s2}{\PYGZdq{}}\PYG{o}{\PYGZgt{}}\PYG{o}{\PYGZlt{}}\PYG{n}{bounds} \PYG{n}{x}\PYG{o}{=}\PYG{l+s+s2}{\PYGZdq{}}\PYG{l+s+s2}{112}\PYG{l+s+s2}{\PYGZdq{}} \PYG{n}{y}\PYG{o}{=}\PYG{l+s+s2}{\PYGZdq{}}\PYG{l+s+s2}{16}\PYG{l+s+s2}{\PYGZdq{}} \PYG{n}{width}\PYG{o}{=}\PYG{l+s+s2}{\PYGZdq{}}\PYG{l+s+s2}{48}\PYG{l+s+s2}{\PYGZdq{}} \PYG{n}{height}\PYG{o}{=}\PYG{l+s+s2}{\PYGZdq{}}\PYG{l+s+s2}{80}\PYG{l+s+s2}{\PYGZdq{}} \PYG{o}{/}\PYG{o}{\PYGZgt{}}\PYG{o}{\PYGZlt{}}\PYG{o}{/}\PYG{n}{cpanel}\PYG{o}{\PYGZgt{}}
\PYG{o}{\PYGZlt{}}\PYG{n}{cpanel} \PYG{n}{name}\PYG{o}{=}\PYG{l+s+s2}{\PYGZdq{}}\PYG{l+s+s2}{digit3}\PYG{l+s+s2}{\PYGZdq{}} \PYG{n}{element}\PYG{o}{=}\PYG{l+s+s2}{\PYGZdq{}}\PYG{l+s+s2}{digit}\PYG{l+s+s2}{\PYGZdq{}}\PYG{o}{\PYGZgt{}}\PYG{o}{\PYGZlt{}}\PYG{n}{bounds} \PYG{n}{x}\PYG{o}{=}\PYG{l+s+s2}{\PYGZdq{}}\PYG{l+s+s2}{160}\PYG{l+s+s2}{\PYGZdq{}} \PYG{n}{y}\PYG{o}{=}\PYG{l+s+s2}{\PYGZdq{}}\PYG{l+s+s2}{16}\PYG{l+s+s2}{\PYGZdq{}} \PYG{n}{width}\PYG{o}{=}\PYG{l+s+s2}{\PYGZdq{}}\PYG{l+s+s2}{48}\PYG{l+s+s2}{\PYGZdq{}} \PYG{n}{height}\PYG{o}{=}\PYG{l+s+s2}{\PYGZdq{}}\PYG{l+s+s2}{80}\PYG{l+s+s2}{\PYGZdq{}} \PYG{o}{/}\PYG{o}{\PYGZgt{}}\PYG{o}{\PYGZlt{}}\PYG{o}{/}\PYG{n}{cpanel}\PYG{o}{\PYGZgt{}}
\PYG{o}{\PYGZlt{}}\PYG{n}{cpanel} \PYG{n}{name}\PYG{o}{=}\PYG{l+s+s2}{\PYGZdq{}}\PYG{l+s+s2}{digit2}\PYG{l+s+s2}{\PYGZdq{}} \PYG{n}{element}\PYG{o}{=}\PYG{l+s+s2}{\PYGZdq{}}\PYG{l+s+s2}{digit}\PYG{l+s+s2}{\PYGZdq{}}\PYG{o}{\PYGZgt{}}\PYG{o}{\PYGZlt{}}\PYG{n}{bounds} \PYG{n}{x}\PYG{o}{=}\PYG{l+s+s2}{\PYGZdq{}}\PYG{l+s+s2}{208}\PYG{l+s+s2}{\PYGZdq{}} \PYG{n}{y}\PYG{o}{=}\PYG{l+s+s2}{\PYGZdq{}}\PYG{l+s+s2}{16}\PYG{l+s+s2}{\PYGZdq{}} \PYG{n}{width}\PYG{o}{=}\PYG{l+s+s2}{\PYGZdq{}}\PYG{l+s+s2}{48}\PYG{l+s+s2}{\PYGZdq{}} \PYG{n}{height}\PYG{o}{=}\PYG{l+s+s2}{\PYGZdq{}}\PYG{l+s+s2}{80}\PYG{l+s+s2}{\PYGZdq{}} \PYG{o}{/}\PYG{o}{\PYGZgt{}}\PYG{o}{\PYGZlt{}}\PYG{o}{/}\PYG{n}{cpanel}\PYG{o}{\PYGZgt{}}
\PYG{o}{\PYGZlt{}}\PYG{n}{cpanel} \PYG{n}{name}\PYG{o}{=}\PYG{l+s+s2}{\PYGZdq{}}\PYG{l+s+s2}{digit1}\PYG{l+s+s2}{\PYGZdq{}} \PYG{n}{element}\PYG{o}{=}\PYG{l+s+s2}{\PYGZdq{}}\PYG{l+s+s2}{digit}\PYG{l+s+s2}{\PYGZdq{}}\PYG{o}{\PYGZgt{}}\PYG{o}{\PYGZlt{}}\PYG{n}{bounds} \PYG{n}{x}\PYG{o}{=}\PYG{l+s+s2}{\PYGZdq{}}\PYG{l+s+s2}{256}\PYG{l+s+s2}{\PYGZdq{}} \PYG{n}{y}\PYG{o}{=}\PYG{l+s+s2}{\PYGZdq{}}\PYG{l+s+s2}{16}\PYG{l+s+s2}{\PYGZdq{}} \PYG{n}{width}\PYG{o}{=}\PYG{l+s+s2}{\PYGZdq{}}\PYG{l+s+s2}{48}\PYG{l+s+s2}{\PYGZdq{}} \PYG{n}{height}\PYG{o}{=}\PYG{l+s+s2}{\PYGZdq{}}\PYG{l+s+s2}{80}\PYG{l+s+s2}{\PYGZdq{}} \PYG{o}{/}\PYG{o}{\PYGZgt{}}\PYG{o}{\PYGZlt{}}\PYG{o}{/}\PYG{n}{cpanel}\PYG{o}{\PYGZgt{}}
\end{Verbatim}

Caso um elemento justifique um elemento de layout e tenha ambos os
atributos \sphinxcode{inputtag} e \sphinxcode{inputmask} mas faltar um nome de atributo
\sphinxcode{name}, ele usará o seu estado com base no valor correspondente da
porta I/O mascarada com os valores do atributo \sphinxcode{inputmask},
transferindo-se para a direita para que o valor do bit com menos
importância da máscara se alinhe com o valor de menor importância (uma
máscara \textbf{0x05} não causará nenhuma transferência, já uma máscara
\textbf{0xb0} resultará em um deslocamento à direita e a transferência de
4 bits, por exemplo).
Isso costuma ser usado para permitir que os botões clicáveis e
chaves alavanca \footnote[9]{\sphinxAtStartFootnote%
Toggle switches, também é conhecido como chave interruptor.
(Nota do tradutor)
} retornem sinais visíveis.

O MAME trata todos os elementos do layout como sendo retangulares ao
lidar com a entrada do mouse habilitando apenas o elemento mais à frente
na região onde o ponteiro estiver presente.


\subsubsection{Grupos reutilizáveis}
\label{techspecs/layout_files:grupos-reutilizaveis}\label{techspecs/layout_files:layout-parts-groups}
Os grupos permitem que um arranjo de telas ou de elementos de layout
sejam usados várias vezes em uma exibição ou outros grupos. Os grupos
podem ser de grande ajuda mesmo que você use o arranjo apenas uma vez,
pois eles podem ser usados para agregar parte de um layout complexo.
Os grupos são definidos usando elementos \sphinxcode{group} dentro de elementos
\sphinxcode{mamelayout} de primeiro nível e representados ao usar elementos
\sphinxcode{group} dentro de elementos \sphinxcode{view} e outros elementos \sphinxcode{group}.

Cada definição de grupo deve ter um atributo \sphinxcode{name} informando um
identificador único. Será considerado um erro caso o arquivo de layout
tenha várias definições de grupos usando um atributo \sphinxcode{name} idêntico.
O valor do atributo \sphinxcode{name} é usado quando for justificar a exibição de
um grupo ou outro. Este é um exemplo da tag de abertura para a definição
de um elemento grupo dentro do elemento de primeiro nível
\sphinxcode{mamelayout}:

\begin{Verbatim}[commandchars=\\\{\}]
\PYG{o}{\PYGZlt{}}\PYG{n}{group} \PYG{n}{name}\PYG{o}{=}\PYG{l+s+s2}{\PYGZdq{}}\PYG{l+s+s2}{panel}\PYG{l+s+s2}{\PYGZdq{}}\PYG{o}{\PYGZgt{}}
\end{Verbatim}

Este grupo pode então ser justificado em uma exibição ou em outro
elemento \sphinxcode{group} usando um elemento de grupo como referência.
Opcionalmente os limites de destino, a orientação e as modificações
das cores poderão ser informados também.
O atributo \sphinxcode{ref} identifica o grupo a qual faz referência, neste
exemplo são fornecidos os valores de limite:

\begin{Verbatim}[commandchars=\\\{\}]
\PYG{o}{\PYGZlt{}}\PYG{n}{group} \PYG{n}{ref}\PYG{o}{=}\PYG{l+s+s2}{\PYGZdq{}}\PYG{l+s+s2}{panel}\PYG{l+s+s2}{\PYGZdq{}}\PYG{o}{\PYGZgt{}}\PYG{o}{\PYGZlt{}}\PYG{n}{bounds} \PYG{n}{x}\PYG{o}{=}\PYG{l+s+s2}{\PYGZdq{}}\PYG{l+s+s2}{87}\PYG{l+s+s2}{\PYGZdq{}} \PYG{n}{y}\PYG{o}{=}\PYG{l+s+s2}{\PYGZdq{}}\PYG{l+s+s2}{58}\PYG{l+s+s2}{\PYGZdq{}} \PYG{n}{width}\PYG{o}{=}\PYG{l+s+s2}{\PYGZdq{}}\PYG{l+s+s2}{23}\PYG{l+s+s2}{\PYGZdq{}} \PYG{n}{height}\PYG{o}{=}\PYG{l+s+s2}{\PYGZdq{}}\PYG{l+s+s2}{23.5}\PYG{l+s+s2}{\PYGZdq{}} \PYG{o}{/}\PYG{o}{\PYGZgt{}}\PYG{o}{\PYGZlt{}}\PYG{o}{/}\PYG{n}{group}\PYG{o}{\PYGZgt{}}
\end{Verbatim}

Os elementos de definição dos grupos permitem que todos os elementos
filhos que forem iguais, sejam exibidos. O posicionamento e as
orientações das tela, elementos de layout e arranjo desses grupos
funcionem da mesma maneira que as exibições.
Veja {\hyperref[techspecs/layout_files:layout\string-parts\string-views]{\sphinxcrossref{\DUrole{std,std-ref}{Exibições}}}} para mais informações.
Um grupo pode justificar outros grupos, porém loops recursivos não são
permitidos. Será considerado um erro caso um grupo representar a si
mesmo de forma direta ou indireta.

Os grupos possuem seus próprios sistemas de coordenadas internas.
Caso um elemento de definição de grupo não tenha um elemento limitador
\sphinxcode{bounds} como filho direto, os seus limites serão computados junto com
a união dos limites de todas as telas, elementos de layout ou grupos
relacionados.
Um elemento filho \sphinxcode{bounds} pode ser usado para definir
explicitamente grupos limitadores
(veja {\hyperref[techspecs/layout_files:layout\string-concepts\string-coordinates]{\sphinxcrossref{\DUrole{std,std-ref}{Coordenadas}}}} para mais informações).
Observe que os limites dos grupos são usados com a única justificativa
para calcular as coordenadas de transformação quando for relacionado
a um grupo.
Um grupo pode posicionar as telas ou elementos fora de seus limites e eles
não serão cortados.
\clearpage
Para demonstrar como o cálculo dos limites funcionam, considere este
exemplo:

\begin{Verbatim}[commandchars=\\\{\}]
\PYGZlt{}group name=\PYGZdq{}autobounds\PYGZdq{}\PYGZgt{}
    \PYGZlt{}!\PYGZhy{}\PYGZhy{} limites automaticamente calculados com sua origem em (5,10), largura 30, e altura 15 \PYGZhy{}\PYGZhy{}\PYGZgt{}
    \PYGZlt{}cpanel element=\PYGZdq{}topleft\PYGZdq{}\PYGZgt{}\PYGZlt{}bounds x=\PYGZdq{}5\PYGZdq{} y=\PYGZdq{}10\PYGZdq{} width=\PYGZdq{}10\PYGZdq{} height=\PYGZdq{}10\PYGZdq{} /\PYGZgt{}\PYGZlt{}/cpanel\PYGZgt{}
    \PYGZlt{}cpanel element=\PYGZdq{}bottomright\PYGZdq{}\PYGZgt{}\PYGZlt{}bounds x=\PYGZdq{}25\PYGZdq{} y=\PYGZdq{}15\PYGZdq{} width=\PYGZdq{}10\PYGZdq{} height=\PYGZdq{}10\PYGZdq{} /\PYGZgt{}\PYGZlt{}/cpanel\PYGZgt{}\PYGZlt{}/group\PYGZgt{}

\PYGZlt{}view name=\PYGZdq{}Teste\PYGZdq{}\PYGZgt{}
    \PYGZlt{}!\PYGZhy{}\PYGZhy{}
       Os grupos limitadores são traduzidos e escalonados para preencher 2/3 da escala horizontal e o dobro verticalmente
       Elemento superior esquerdo posicionado em  (0,0) com 6.67 de largura e 20 de altura
       Elemento inferior direito posicionado em (13.33,10) com 6.67 de largura e 20 de altura
       Os elementos de visualização calculado com origem em (0,0) 20 de largura e 30 de altura
    \PYGZhy{}\PYGZhy{}\PYGZgt{}
    \PYGZlt{}group ref=\PYGZdq{}autobounds\PYGZdq{}\PYGZgt{}\PYGZlt{}bounds x=\PYGZdq{}0\PYGZdq{} y=\PYGZdq{}0\PYGZdq{} width=\PYGZdq{}20\PYGZdq{} height=\PYGZdq{}30\PYGZdq{} /\PYGZgt{}\PYGZlt{}/group\PYGZgt{}
\PYGZlt{}/view\PYGZgt{}
\end{Verbatim}

Isto é relativamente simples, como todos os elementos inerentemente caem
dentro dos limites automaticamente calculados ao grupo. Agora, considere
o que acontece caso a posição dos elementos de um grupo estiver fora dos
seus limites:

\begin{Verbatim}[commandchars=\\\{\}]
\PYGZlt{}group name=\PYGZdq{}periphery\PYGZdq{}\PYGZgt{}
    \PYGZlt{}!\PYGZhy{}\PYGZhy{} os limites dos elementos estão acima da quina superior e à direita da quina direita \PYGZhy{}\PYGZhy{}\PYGZgt{}
    \PYGZlt{}bounds x=\PYGZdq{}10\PYGZdq{} y=\PYGZdq{}10\PYGZdq{} width=\PYGZdq{}20\PYGZdq{} height=\PYGZdq{}25\PYGZdq{} /\PYGZgt{}
    \PYGZlt{}cpanel element=\PYGZdq{}topleft\PYGZdq{}\PYGZgt{}\PYGZlt{}bounds x=\PYGZdq{}10\PYGZdq{} y=\PYGZdq{}0\PYGZdq{} width=\PYGZdq{}10\PYGZdq{} height=\PYGZdq{}10\PYGZdq{} /\PYGZgt{}\PYGZlt{}/cpanel\PYGZgt{}
    \PYGZlt{}cpanel element=\PYGZdq{}bottomright\PYGZdq{}\PYGZgt{}\PYGZlt{}bounds x=\PYGZdq{}30\PYGZdq{} y=\PYGZdq{}20\PYGZdq{} width=\PYGZdq{}10\PYGZdq{} height=\PYGZdq{}10\PYGZdq{} /\PYGZgt{}\PYGZlt{}/cpanel\PYGZgt{}\PYGZlt{}/group\PYGZgt{}

\PYGZlt{}view name=\PYGZdq{}Test\PYGZdq{}\PYGZgt{}
    \PYGZlt{}!\PYGZhy{}\PYGZhy{}
       Os grupos limitadores são traduzidos e escalonados para preencher 2/3 da escala horizontal unido verticalmente.
       Elemento superior esquerdo posicionado em (5,\PYGZhy{}5) com 15 de largura e 10 de altura
       Elemento inferior direito posicionado em (35,15) com 15 de largura e 10 de altura
       Os elementos de visualização calculado com origem em (5,\PYGZhy{}5) 45 de largura e 30 de altura
    \PYGZhy{}\PYGZhy{}\PYGZgt{}
    \PYGZlt{}group ref=\PYGZdq{}periphery\PYGZdq{}\PYGZgt{}\PYGZlt{}bounds x=\PYGZdq{}5\PYGZdq{} y=\PYGZdq{}5\PYGZdq{} width=\PYGZdq{}30\PYGZdq{} height=\PYGZdq{}25\PYGZdq{} /\PYGZgt{}\PYGZlt{}/group\PYGZgt{}
\PYGZlt{}/view\PYGZgt{}
\end{Verbatim}

Os elementos de grupo são traduzidos e escalonados conforme são
necessários para distorcer os limites internos dos grupos para o limite
de exibição final. O conteúdo dos grupos não fica limitado aos seus
limites. A exibição considera os limites dos elementos atuais ao
calcular seus próprios limites e não os limites de destino especificado
para o grupo.

Quando um grupo é interpretado, ele cria um escopo do parâmetro
agrupado.
A lógica do escopo pai é o escopo do parâmetro de visualização, grupo ou
bloco de repetição onde o grupo for interpretado (\emph{não} é um parente
léxico ao elemento de primeiro nível \sphinxcode{mamelayout}).
Qualquer elemento \sphinxcode{param} dentro do conjunto de definição, estabelece
os parâmetros dos elementos no escopo local para o grupo interpretado.
Os parâmetros locais não persistem através de várias interpretações.
Veja {\hyperref[techspecs/layout_files:layout\string-concepts\string-params]{\sphinxcrossref{\DUrole{std,std-ref}{Parâmetros}}}} para mais informações sobre os
parâmetros. (Observe que o nome dos grupos não fazem parte do seu
conteúdo e qualquer referência de parâmetro no próprio atributo \sphinxcode{name}
será substituído no ponto onde a definição do grupo aparecer no primeiro
nível do elemento de escopo \sphinxcode{mamelayout}.)


\subsubsection{Repetindo Blocos}
\label{techspecs/layout_files:repetindo-blocos}\label{techspecs/layout_files:layout-parts-repeats}
Os blocos repetidos fornecem uma maneira concisa de gerar ou organizar
um grande número de elementos similares. A repetição de blocos são
geralmente usadas em conjunto com o gerador de parâmetros
(veja {\hyperref[techspecs/layout_files:layout\string-concepts\string-params]{\sphinxcrossref{\DUrole{std,std-ref}{Parâmetros}}}}).
As repetições de blocos podem ser agrupados para criar arranjos mais
complexos.

Os blocos repetidos são criados com o elemento \sphinxcode{repeat}.
Cada elemento \sphinxcode{repeat} requer um atributo \sphinxcode{count} definindo um
número de iterações a serem geradas.
O atributo \sphinxcode{count} deve ser um número inteiro e positivo. A repetição
de blocos é permitida dentro do elemento de primeiro nível
\sphinxcode{mamelayout}, dentro dos elementos \sphinxcode{group} e \sphinxcode{view} assim como
dentro de outros elementos \sphinxcode{repeat}. O exato elemento filho permitido
dentro do elemento \sphinxcode{repeat} depende de onde ele aparecer:
\begin{itemize}
\item {} 
Um bloco repetido dentro do elemento de primeiro nível \sphinxcode{mamelayout}
podem conter os seguintes elementos
\sphinxcode{param}, \sphinxcode{element}, \sphinxcode{group} (definição), e \sphinxcode{repeat}.

\item {} 
Um bloco repetido dentro de um elemento \sphinxcode{group} ou \sphinxcode{view} podem
conter os seguintes elementos
\sphinxcode{param}, \sphinxcode{backdrop}, \sphinxcode{screen}, \sphinxcode{overlay}, \sphinxcode{bezel}, \sphinxcode{cpanel},
\sphinxcode{marquee}, \sphinxcode{group} (referência), e \sphinxcode{repeat}.

\end{itemize}

Um bloco de repetição faz a repetição efetiva do seu conteúdo diversas
vezes dependendo do valor definido no atributo \sphinxcode{count}.
Veja as seções relevantes para mais informações de como os elementos
filho são usados ({\hyperref[techspecs/layout_files:layout\string-parts]{\sphinxcrossref{\DUrole{std,std-ref}{As partes de um layout}}}}, {\hyperref[techspecs/layout_files:layout\string-parts\string-groups]{\sphinxcrossref{\DUrole{std,std-ref}{Grupos reutilizáveis}}}},
e {\hyperref[techspecs/layout_files:layout\string-parts\string-views]{\sphinxcrossref{\DUrole{std,std-ref}{Exibições}}}}). Um bloco que se repete cria um escopo de
parâmetros agrupados dentro do escopo do parâmetro de seu elemento pai
léxico (DOM).

Gerando rótulos numéricos em branco de zero a onze com o nome
\sphinxcode{label\_0}, \sphinxcode{label\_1}, e assim por diante (dentro do elemento de
primeiro nível \sphinxcode{mamelayout}):

\begin{Verbatim}[commandchars=\\\{\}]
\PYG{o}{\PYGZlt{}}\PYG{n}{repeat} \PYG{n}{count}\PYG{o}{=}\PYG{l+s+s2}{\PYGZdq{}}\PYG{l+s+s2}{12}\PYG{l+s+s2}{\PYGZdq{}}\PYG{o}{\PYGZgt{}}
    \PYG{o}{\PYGZlt{}}\PYG{n}{param} \PYG{n}{name}\PYG{o}{=}\PYG{l+s+s2}{\PYGZdq{}}\PYG{l+s+s2}{labelnum}\PYG{l+s+s2}{\PYGZdq{}} \PYG{n}{start}\PYG{o}{=}\PYG{l+s+s2}{\PYGZdq{}}\PYG{l+s+s2}{0}\PYG{l+s+s2}{\PYGZdq{}} \PYG{n}{increment}\PYG{o}{=}\PYG{l+s+s2}{\PYGZdq{}}\PYG{l+s+s2}{1}\PYG{l+s+s2}{\PYGZdq{}} \PYG{o}{/}\PYG{o}{\PYGZgt{}}
    \PYG{o}{\PYGZlt{}}\PYG{n}{element} \PYG{n}{name}\PYG{o}{=}\PYG{l+s+s2}{\PYGZdq{}}\PYG{l+s+s2}{label\PYGZus{}\PYGZti{}labelnum\PYGZti{}}\PYG{l+s+s2}{\PYGZdq{}}\PYG{o}{\PYGZgt{}}
        \PYG{o}{\PYGZlt{}}\PYG{n}{text} \PYG{n}{string}\PYG{o}{=}\PYG{l+s+s2}{\PYGZdq{}}\PYG{l+s+s2}{\PYGZti{}labelnum\PYGZti{}}\PYG{l+s+s2}{\PYGZdq{}}\PYG{o}{\PYGZgt{}}\PYG{o}{\PYGZlt{}}\PYG{n}{color} \PYG{n}{red}\PYG{o}{=}\PYG{l+s+s2}{\PYGZdq{}}\PYG{l+s+s2}{1.0}\PYG{l+s+s2}{\PYGZdq{}} \PYG{n}{green}\PYG{o}{=}\PYG{l+s+s2}{\PYGZdq{}}\PYG{l+s+s2}{1.0}\PYG{l+s+s2}{\PYGZdq{}} \PYG{n}{blue}\PYG{o}{=}\PYG{l+s+s2}{\PYGZdq{}}\PYG{l+s+s2}{1.0}\PYG{l+s+s2}{\PYGZdq{}} \PYG{o}{/}\PYG{o}{\PYGZgt{}}\PYG{o}{\PYGZlt{}}\PYG{o}{/}\PYG{n}{text}\PYG{o}{\PYGZgt{}}
    \PYG{o}{\PYGZlt{}}\PYG{o}{/}\PYG{n}{element}\PYG{o}{\PYGZgt{}}
\PYG{o}{\PYGZlt{}}\PYG{o}{/}\PYG{n}{repeat}\PYG{o}{\PYGZgt{}}
\end{Verbatim}

Uma fileira horizontal com 40 mostradores digitais, com cinco unidades
de espaço entre elas, controladas pelas saídas \sphinxcode{digit0} até
\sphinxcode{digit39} (dentro de um elemento \sphinxcode{group} ou \sphinxcode{view}):

\begin{Verbatim}[commandchars=\\\{\}]
\PYG{o}{\PYGZlt{}}\PYG{n}{repeat} \PYG{n}{count}\PYG{o}{=}\PYG{l+s+s2}{\PYGZdq{}}\PYG{l+s+s2}{40}\PYG{l+s+s2}{\PYGZdq{}}\PYG{o}{\PYGZgt{}}
    \PYG{o}{\PYGZlt{}}\PYG{n}{param} \PYG{n}{name}\PYG{o}{=}\PYG{l+s+s2}{\PYGZdq{}}\PYG{l+s+s2}{i}\PYG{l+s+s2}{\PYGZdq{}} \PYG{n}{start}\PYG{o}{=}\PYG{l+s+s2}{\PYGZdq{}}\PYG{l+s+s2}{0}\PYG{l+s+s2}{\PYGZdq{}} \PYG{n}{increment}\PYG{o}{=}\PYG{l+s+s2}{\PYGZdq{}}\PYG{l+s+s2}{1}\PYG{l+s+s2}{\PYGZdq{}} \PYG{o}{/}\PYG{o}{\PYGZgt{}}
    \PYG{o}{\PYGZlt{}}\PYG{n}{param} \PYG{n}{name}\PYG{o}{=}\PYG{l+s+s2}{\PYGZdq{}}\PYG{l+s+s2}{x}\PYG{l+s+s2}{\PYGZdq{}} \PYG{n}{start}\PYG{o}{=}\PYG{l+s+s2}{\PYGZdq{}}\PYG{l+s+s2}{5}\PYG{l+s+s2}{\PYGZdq{}} \PYG{n}{increment}\PYG{o}{=}\PYG{l+s+s2}{\PYGZdq{}}\PYG{l+s+s2}{30}\PYG{l+s+s2}{\PYGZdq{}} \PYG{o}{/}\PYG{o}{\PYGZgt{}}
    \PYG{o}{\PYGZlt{}}\PYG{n}{bezel} \PYG{n}{name}\PYG{o}{=}\PYG{l+s+s2}{\PYGZdq{}}\PYG{l+s+s2}{digit\PYGZti{}i\PYGZti{}}\PYG{l+s+s2}{\PYGZdq{}} \PYG{n}{element}\PYG{o}{=}\PYG{l+s+s2}{\PYGZdq{}}\PYG{l+s+s2}{digit}\PYG{l+s+s2}{\PYGZdq{}}\PYG{o}{\PYGZgt{}}
        \PYG{o}{\PYGZlt{}}\PYG{n}{bounds} \PYG{n}{x}\PYG{o}{=}\PYG{l+s+s2}{\PYGZdq{}}\PYG{l+s+s2}{\PYGZti{}x\PYGZti{}}\PYG{l+s+s2}{\PYGZdq{}} \PYG{n}{y}\PYG{o}{=}\PYG{l+s+s2}{\PYGZdq{}}\PYG{l+s+s2}{5}\PYG{l+s+s2}{\PYGZdq{}} \PYG{n}{width}\PYG{o}{=}\PYG{l+s+s2}{\PYGZdq{}}\PYG{l+s+s2}{25}\PYG{l+s+s2}{\PYGZdq{}} \PYG{n}{height}\PYG{o}{=}\PYG{l+s+s2}{\PYGZdq{}}\PYG{l+s+s2}{50}\PYG{l+s+s2}{\PYGZdq{}} \PYG{o}{/}\PYG{o}{\PYGZgt{}}
    \PYG{o}{\PYGZlt{}}\PYG{o}{/}\PYG{n}{bezel}\PYG{o}{\PYGZgt{}}
\PYG{o}{\PYGZlt{}}\PYG{o}{/}\PYG{n}{repeat}\PYG{o}{\PYGZgt{}}
\end{Verbatim}

Oito mostradores com matrix de ponto medindo cinco por sete em uma
linha, com pixels controlados por \sphinxcode{Dot\_000} até \sphinxcode{Dot\_764}
(dentro de um elemento \sphinxcode{group} ou \sphinxcode{view}):

\begin{Verbatim}[commandchars=\\\{\}]
\PYGZlt{}repeat count=\PYGZdq{}8\PYGZdq{}\PYGZgt{} \PYGZlt{}!\PYGZhy{}\PYGZhy{} 8 digits \PYGZhy{}\PYGZhy{}\PYGZgt{}
    \PYGZlt{}param name=\PYGZdq{}digitno\PYGZdq{} start=\PYGZdq{}1\PYGZdq{} increment=\PYGZdq{}1\PYGZdq{} /\PYGZgt{}
    \PYGZlt{}param name=\PYGZdq{}digitx\PYGZdq{} start=\PYGZdq{}0\PYGZdq{} increment=\PYGZdq{}935\PYGZdq{} /\PYGZgt{} \PYGZlt{}!\PYGZhy{}\PYGZhy{} distância entre dígitos ((111 * 5) + 380) \PYGZhy{}\PYGZhy{}\PYGZgt{}
    \PYGZlt{}repeat count=\PYGZdq{}7\PYGZdq{}\PYGZgt{} \PYGZlt{}!\PYGZhy{}\PYGZhy{} 7 rows in each digit \PYGZhy{}\PYGZhy{}\PYGZgt{}
        \PYGZlt{}param name=\PYGZdq{}rowno\PYGZdq{} start=\PYGZdq{}1\PYGZdq{} increment=\PYGZdq{}1\PYGZdq{} /\PYGZgt{}
        \PYGZlt{}param name=\PYGZdq{}rowy\PYGZdq{} start=\PYGZdq{}0\PYGZdq{} increment=\PYGZdq{}114\PYGZdq{} /\PYGZgt{} \PYGZlt{}!\PYGZhy{}\PYGZhy{} distância vertical entre LEDs \PYGZhy{}\PYGZhy{}\PYGZgt{}
        \PYGZlt{}repeat count=\PYGZdq{}5\PYGZdq{}\PYGZgt{} \PYGZlt{}!\PYGZhy{}\PYGZhy{} 5 columns in each digit \PYGZhy{}\PYGZhy{}\PYGZgt{}
            \PYGZlt{}param name=\PYGZdq{}colno\PYGZdq{} start=\PYGZdq{}1\PYGZdq{} increment=\PYGZdq{}1\PYGZdq{} /\PYGZgt{}
            \PYGZlt{}param name=\PYGZdq{}colx\PYGZdq{} start=\PYGZdq{}\PYGZti{}digitx\PYGZti{}\PYGZdq{} increment=\PYGZdq{}111\PYGZdq{} /\PYGZgt{} \PYGZlt{}!\PYGZhy{}\PYGZhy{} distância horizontal entre LEDs \PYGZhy{}\PYGZhy{}\PYGZgt{}
            \PYGZlt{}bezel name=\PYGZdq{}Dot\PYGZus{}\PYGZti{}digitno\PYGZti{}\PYGZti{}rowno\PYGZti{}\PYGZti{}colno\PYGZti{}\PYGZdq{} element=\PYGZdq{}Pixel\PYGZdq{} state=\PYGZdq{}0\PYGZdq{}\PYGZgt{}
                \PYGZlt{}bounds x=\PYGZdq{}\PYGZti{}colx\PYGZti{}\PYGZdq{} y=\PYGZdq{}\PYGZti{}rowy\PYGZti{}\PYGZdq{} width=\PYGZdq{}100\PYGZdq{} height=\PYGZdq{}100\PYGZdq{} /\PYGZgt{} \PYGZlt{}!\PYGZhy{}\PYGZhy{} tamanho de cada LED \PYGZhy{}\PYGZhy{}\PYGZgt{}
            \PYGZlt{}/bezel\PYGZgt{}
        \PYGZlt{}/repeat\PYGZgt{}
    \PYGZlt{}/repeat\PYGZgt{}
\PYGZlt{}/repeat\PYGZgt{}
\end{Verbatim}

Dois teclados ``clicáveis'', separados horizontalmente por um teclado
numérico quatro por quatro (dentro de um elemento \sphinxcode{group} ou
\sphinxcode{view}):

\begin{Verbatim}[commandchars=\\\{\}]
\PYG{o}{\PYGZlt{}}\PYG{n}{repeat} \PYG{n}{count}\PYG{o}{=}\PYG{l+s+s2}{\PYGZdq{}}\PYG{l+s+s2}{2}\PYG{l+s+s2}{\PYGZdq{}}\PYG{o}{\PYGZgt{}}
    \PYG{o}{\PYGZlt{}}\PYG{n}{param} \PYG{n}{name}\PYG{o}{=}\PYG{l+s+s2}{\PYGZdq{}}\PYG{l+s+s2}{group}\PYG{l+s+s2}{\PYGZdq{}} \PYG{n}{start}\PYG{o}{=}\PYG{l+s+s2}{\PYGZdq{}}\PYG{l+s+s2}{0}\PYG{l+s+s2}{\PYGZdq{}} \PYG{n}{increment}\PYG{o}{=}\PYG{l+s+s2}{\PYGZdq{}}\PYG{l+s+s2}{4}\PYG{l+s+s2}{\PYGZdq{}} \PYG{o}{/}\PYG{o}{\PYGZgt{}}
    \PYG{o}{\PYGZlt{}}\PYG{n}{param} \PYG{n}{name}\PYG{o}{=}\PYG{l+s+s2}{\PYGZdq{}}\PYG{l+s+s2}{padx}\PYG{l+s+s2}{\PYGZdq{}} \PYG{n}{start}\PYG{o}{=}\PYG{l+s+s2}{\PYGZdq{}}\PYG{l+s+s2}{10}\PYG{l+s+s2}{\PYGZdq{}} \PYG{n}{increment}\PYG{o}{=}\PYG{l+s+s2}{\PYGZdq{}}\PYG{l+s+s2}{530}\PYG{l+s+s2}{\PYGZdq{}} \PYG{o}{/}\PYG{o}{\PYGZgt{}}
    \PYG{o}{\PYGZlt{}}\PYG{n}{param} \PYG{n}{name}\PYG{o}{=}\PYG{l+s+s2}{\PYGZdq{}}\PYG{l+s+s2}{mask}\PYG{l+s+s2}{\PYGZdq{}} \PYG{n}{start}\PYG{o}{=}\PYG{l+s+s2}{\PYGZdq{}}\PYG{l+s+s2}{0x01}\PYG{l+s+s2}{\PYGZdq{}} \PYG{n}{lshift}\PYG{o}{=}\PYG{l+s+s2}{\PYGZdq{}}\PYG{l+s+s2}{4}\PYG{l+s+s2}{\PYGZdq{}} \PYG{o}{/}\PYG{o}{\PYGZgt{}}
    \PYG{o}{\PYGZlt{}}\PYG{n}{repeat} \PYG{n}{count}\PYG{o}{=}\PYG{l+s+s2}{\PYGZdq{}}\PYG{l+s+s2}{4}\PYG{l+s+s2}{\PYGZdq{}}\PYG{o}{\PYGZgt{}}
        \PYG{o}{\PYGZlt{}}\PYG{n}{param} \PYG{n}{name}\PYG{o}{=}\PYG{l+s+s2}{\PYGZdq{}}\PYG{l+s+s2}{row}\PYG{l+s+s2}{\PYGZdq{}} \PYG{n}{start}\PYG{o}{=}\PYG{l+s+s2}{\PYGZdq{}}\PYG{l+s+s2}{0}\PYG{l+s+s2}{\PYGZdq{}} \PYG{n}{increment}\PYG{o}{=}\PYG{l+s+s2}{\PYGZdq{}}\PYG{l+s+s2}{1}\PYG{l+s+s2}{\PYGZdq{}} \PYG{o}{/}\PYG{o}{\PYGZgt{}}
        \PYG{o}{\PYGZlt{}}\PYG{n}{param} \PYG{n}{name}\PYG{o}{=}\PYG{l+s+s2}{\PYGZdq{}}\PYG{l+s+s2}{y}\PYG{l+s+s2}{\PYGZdq{}} \PYG{n}{start}\PYG{o}{=}\PYG{l+s+s2}{\PYGZdq{}}\PYG{l+s+s2}{100}\PYG{l+s+s2}{\PYGZdq{}} \PYG{n}{increment}\PYG{o}{=}\PYG{l+s+s2}{\PYGZdq{}}\PYG{l+s+s2}{110}\PYG{l+s+s2}{\PYGZdq{}} \PYG{o}{/}\PYG{o}{\PYGZgt{}}
        \PYG{o}{\PYGZlt{}}\PYG{n}{repeat} \PYG{n}{count}\PYG{o}{=}\PYG{l+s+s2}{\PYGZdq{}}\PYG{l+s+s2}{4}\PYG{l+s+s2}{\PYGZdq{}}\PYG{o}{\PYGZgt{}}
            \PYG{o}{\PYGZlt{}}\PYG{n}{param} \PYG{n}{name}\PYG{o}{=}\PYG{l+s+s2}{\PYGZdq{}}\PYG{l+s+s2}{col}\PYG{l+s+s2}{\PYGZdq{}} \PYG{n}{start}\PYG{o}{=}\PYG{l+s+s2}{\PYGZdq{}}\PYG{l+s+s2}{\PYGZti{}group\PYGZti{}}\PYG{l+s+s2}{\PYGZdq{}} \PYG{n}{increment}\PYG{o}{=}\PYG{l+s+s2}{\PYGZdq{}}\PYG{l+s+s2}{1}\PYG{l+s+s2}{\PYGZdq{}} \PYG{o}{/}\PYG{o}{\PYGZgt{}}
            \PYG{o}{\PYGZlt{}}\PYG{n}{param} \PYG{n}{name}\PYG{o}{=}\PYG{l+s+s2}{\PYGZdq{}}\PYG{l+s+s2}{btnx}\PYG{l+s+s2}{\PYGZdq{}} \PYG{n}{start}\PYG{o}{=}\PYG{l+s+s2}{\PYGZdq{}}\PYG{l+s+s2}{\PYGZti{}padx\PYGZti{}}\PYG{l+s+s2}{\PYGZdq{}} \PYG{n}{increment}\PYG{o}{=}\PYG{l+s+s2}{\PYGZdq{}}\PYG{l+s+s2}{110}\PYG{l+s+s2}{\PYGZdq{}} \PYG{o}{/}\PYG{o}{\PYGZgt{}}
            \PYG{o}{\PYGZlt{}}\PYG{n}{param} \PYG{n}{name}\PYG{o}{=}\PYG{l+s+s2}{\PYGZdq{}}\PYG{l+s+s2}{mask}\PYG{l+s+s2}{\PYGZdq{}} \PYG{n}{start}\PYG{o}{=}\PYG{l+s+s2}{\PYGZdq{}}\PYG{l+s+s2}{\PYGZti{}mask\PYGZti{}}\PYG{l+s+s2}{\PYGZdq{}} \PYG{n}{lshift}\PYG{o}{=}\PYG{l+s+s2}{\PYGZdq{}}\PYG{l+s+s2}{1}\PYG{l+s+s2}{\PYGZdq{}} \PYG{o}{/}\PYG{o}{\PYGZgt{}}
            \PYG{o}{\PYGZlt{}}\PYG{n}{bezel} \PYG{n}{element}\PYG{o}{=}\PYG{l+s+s2}{\PYGZdq{}}\PYG{l+s+s2}{btn\PYGZti{}row\PYGZti{}\PYGZti{}col\PYGZti{}}\PYG{l+s+s2}{\PYGZdq{}} \PYG{n}{inputtag}\PYG{o}{=}\PYG{l+s+s2}{\PYGZdq{}}\PYG{l+s+s2}{row\PYGZti{}row\PYGZti{}}\PYG{l+s+s2}{\PYGZdq{}} \PYG{n}{inputmask}\PYG{o}{=}\PYG{l+s+s2}{\PYGZdq{}}\PYG{l+s+s2}{\PYGZti{}mask\PYGZti{}}\PYG{l+s+s2}{\PYGZdq{}}\PYG{o}{\PYGZgt{}}
                \PYG{o}{\PYGZlt{}}\PYG{n}{bounds} \PYG{n}{x}\PYG{o}{=}\PYG{l+s+s2}{\PYGZdq{}}\PYG{l+s+s2}{\PYGZti{}btnx\PYGZti{}}\PYG{l+s+s2}{\PYGZdq{}} \PYG{n}{y}\PYG{o}{=}\PYG{l+s+s2}{\PYGZdq{}}\PYG{l+s+s2}{\PYGZti{}y\PYGZti{}}\PYG{l+s+s2}{\PYGZdq{}} \PYG{n}{width}\PYG{o}{=}\PYG{l+s+s2}{\PYGZdq{}}\PYG{l+s+s2}{80}\PYG{l+s+s2}{\PYGZdq{}} \PYG{n}{height}\PYG{o}{=}\PYG{l+s+s2}{\PYGZdq{}}\PYG{l+s+s2}{80}\PYG{l+s+s2}{\PYGZdq{}} \PYG{o}{/}\PYG{o}{\PYGZgt{}}
            \PYG{o}{\PYGZlt{}}\PYG{o}{/}\PYG{n}{bezel}\PYG{o}{\PYGZgt{}}
        \PYG{o}{\PYGZlt{}}\PYG{o}{/}\PYG{n}{repeat}\PYG{o}{\PYGZgt{}}
    \PYG{o}{\PYGZlt{}}\PYG{o}{/}\PYG{n}{repeat}\PYG{o}{\PYGZgt{}}
\PYG{o}{\PYGZlt{}}\PYG{o}{/}\PYG{n}{repeat}\PYG{o}{\PYGZgt{}}
\end{Verbatim}

Os botões são desenhados usando os elementos \sphinxcode{btn00} na parte superior
esquerda, \sphinxcode{btn07} na parte superior direita, \sphinxcode{btn30} na parte
inferior esquerda e \sphinxcode{btn37} na parte inferior direita contando entre
eles. As quatro colunas são conectadas às portas I/O \sphinxcode{row0}, \sphinxcode{row1},
\sphinxcode{row2}, and \sphinxcode{row3}, de cima para baixo.
As colunas consecutivas são conectadas aos bits das portas I/O, começando
com o bit de menor importância do lado esquerdo. Observe que o parâmetro
\sphinxcode{mask} no elemento mais interno \sphinxcode{repeat}, recebe o seu valor inicial
vindo do parâmetro correspondentemente nomeado no delimitador de escopo,
mas não o modifica.

Gerando um tabuleiro de xadrez com valores alfa alternados entre 0.4 e
0.2 (dentro de um elemento \sphinxcode{group} ou \sphinxcode{view}):

\begin{Verbatim}[commandchars=\\\{\}]
\PYG{o}{\PYGZlt{}}\PYG{n}{repeat} \PYG{n}{count}\PYG{o}{=}\PYG{l+s+s2}{\PYGZdq{}}\PYG{l+s+s2}{4}\PYG{l+s+s2}{\PYGZdq{}}\PYG{o}{\PYGZgt{}}
    \PYG{o}{\PYGZlt{}}\PYG{n}{param} \PYG{n}{name}\PYG{o}{=}\PYG{l+s+s2}{\PYGZdq{}}\PYG{l+s+s2}{pairy}\PYG{l+s+s2}{\PYGZdq{}} \PYG{n}{start}\PYG{o}{=}\PYG{l+s+s2}{\PYGZdq{}}\PYG{l+s+s2}{3}\PYG{l+s+s2}{\PYGZdq{}} \PYG{n}{increment}\PYG{o}{=}\PYG{l+s+s2}{\PYGZdq{}}\PYG{l+s+s2}{20}\PYG{l+s+s2}{\PYGZdq{}} \PYG{o}{/}\PYG{o}{\PYGZgt{}}
    \PYG{o}{\PYGZlt{}}\PYG{n}{param} \PYG{n}{name}\PYG{o}{=}\PYG{l+s+s2}{\PYGZdq{}}\PYG{l+s+s2}{pairno}\PYG{l+s+s2}{\PYGZdq{}} \PYG{n}{start}\PYG{o}{=}\PYG{l+s+s2}{\PYGZdq{}}\PYG{l+s+s2}{7}\PYG{l+s+s2}{\PYGZdq{}} \PYG{n}{increment}\PYG{o}{=}\PYG{l+s+s2}{\PYGZdq{}}\PYG{l+s+s2}{\PYGZhy{}2}\PYG{l+s+s2}{\PYGZdq{}} \PYG{o}{/}\PYG{o}{\PYGZgt{}}
    \PYG{o}{\PYGZlt{}}\PYG{n}{repeat} \PYG{n}{count}\PYG{o}{=}\PYG{l+s+s2}{\PYGZdq{}}\PYG{l+s+s2}{2}\PYG{l+s+s2}{\PYGZdq{}}\PYG{o}{\PYGZgt{}}
        \PYG{o}{\PYGZlt{}}\PYG{n}{param} \PYG{n}{name}\PYG{o}{=}\PYG{l+s+s2}{\PYGZdq{}}\PYG{l+s+s2}{rowy}\PYG{l+s+s2}{\PYGZdq{}} \PYG{n}{start}\PYG{o}{=}\PYG{l+s+s2}{\PYGZdq{}}\PYG{l+s+s2}{\PYGZti{}pairy\PYGZti{}}\PYG{l+s+s2}{\PYGZdq{}} \PYG{n}{increment}\PYG{o}{=}\PYG{l+s+s2}{\PYGZdq{}}\PYG{l+s+s2}{10}\PYG{l+s+s2}{\PYGZdq{}} \PYG{o}{/}\PYG{o}{\PYGZgt{}}
        \PYG{o}{\PYGZlt{}}\PYG{n}{param} \PYG{n}{name}\PYG{o}{=}\PYG{l+s+s2}{\PYGZdq{}}\PYG{l+s+s2}{rowno}\PYG{l+s+s2}{\PYGZdq{}} \PYG{n}{start}\PYG{o}{=}\PYG{l+s+s2}{\PYGZdq{}}\PYG{l+s+s2}{\PYGZti{}pairno\PYGZti{}}\PYG{l+s+s2}{\PYGZdq{}} \PYG{n}{increment}\PYG{o}{=}\PYG{l+s+s2}{\PYGZdq{}}\PYG{l+s+s2}{\PYGZhy{}1}\PYG{l+s+s2}{\PYGZdq{}} \PYG{o}{/}\PYG{o}{\PYGZgt{}}
        \PYG{o}{\PYGZlt{}}\PYG{n}{param} \PYG{n}{name}\PYG{o}{=}\PYG{l+s+s2}{\PYGZdq{}}\PYG{l+s+s2}{lalpha}\PYG{l+s+s2}{\PYGZdq{}} \PYG{n}{start}\PYG{o}{=}\PYG{l+s+s2}{\PYGZdq{}}\PYG{l+s+s2}{0.4}\PYG{l+s+s2}{\PYGZdq{}} \PYG{n}{increment}\PYG{o}{=}\PYG{l+s+s2}{\PYGZdq{}}\PYG{l+s+s2}{\PYGZhy{}0.2}\PYG{l+s+s2}{\PYGZdq{}} \PYG{o}{/}\PYG{o}{\PYGZgt{}}
        \PYG{o}{\PYGZlt{}}\PYG{n}{param} \PYG{n}{name}\PYG{o}{=}\PYG{l+s+s2}{\PYGZdq{}}\PYG{l+s+s2}{ralpha}\PYG{l+s+s2}{\PYGZdq{}} \PYG{n}{start}\PYG{o}{=}\PYG{l+s+s2}{\PYGZdq{}}\PYG{l+s+s2}{0.2}\PYG{l+s+s2}{\PYGZdq{}} \PYG{n}{increment}\PYG{o}{=}\PYG{l+s+s2}{\PYGZdq{}}\PYG{l+s+s2}{0.2}\PYG{l+s+s2}{\PYGZdq{}} \PYG{o}{/}\PYG{o}{\PYGZgt{}}
        \PYG{o}{\PYGZlt{}}\PYG{n}{repeat} \PYG{n}{count}\PYG{o}{=}\PYG{l+s+s2}{\PYGZdq{}}\PYG{l+s+s2}{4}\PYG{l+s+s2}{\PYGZdq{}}\PYG{o}{\PYGZgt{}}
            \PYG{o}{\PYGZlt{}}\PYG{n}{param} \PYG{n}{name}\PYG{o}{=}\PYG{l+s+s2}{\PYGZdq{}}\PYG{l+s+s2}{lx}\PYG{l+s+s2}{\PYGZdq{}} \PYG{n}{start}\PYG{o}{=}\PYG{l+s+s2}{\PYGZdq{}}\PYG{l+s+s2}{3}\PYG{l+s+s2}{\PYGZdq{}} \PYG{n}{increment}\PYG{o}{=}\PYG{l+s+s2}{\PYGZdq{}}\PYG{l+s+s2}{20}\PYG{l+s+s2}{\PYGZdq{}} \PYG{o}{/}\PYG{o}{\PYGZgt{}}
            \PYG{o}{\PYGZlt{}}\PYG{n}{param} \PYG{n}{name}\PYG{o}{=}\PYG{l+s+s2}{\PYGZdq{}}\PYG{l+s+s2}{rx}\PYG{l+s+s2}{\PYGZdq{}} \PYG{n}{start}\PYG{o}{=}\PYG{l+s+s2}{\PYGZdq{}}\PYG{l+s+s2}{13}\PYG{l+s+s2}{\PYGZdq{}} \PYG{n}{increment}\PYG{o}{=}\PYG{l+s+s2}{\PYGZdq{}}\PYG{l+s+s2}{20}\PYG{l+s+s2}{\PYGZdq{}} \PYG{o}{/}\PYG{o}{\PYGZgt{}}
            \PYG{o}{\PYGZlt{}}\PYG{n}{param} \PYG{n}{name}\PYG{o}{=}\PYG{l+s+s2}{\PYGZdq{}}\PYG{l+s+s2}{lmask}\PYG{l+s+s2}{\PYGZdq{}} \PYG{n}{start}\PYG{o}{=}\PYG{l+s+s2}{\PYGZdq{}}\PYG{l+s+s2}{0x01}\PYG{l+s+s2}{\PYGZdq{}} \PYG{n}{lshift}\PYG{o}{=}\PYG{l+s+s2}{\PYGZdq{}}\PYG{l+s+s2}{2}\PYG{l+s+s2}{\PYGZdq{}} \PYG{o}{/}\PYG{o}{\PYGZgt{}}
            \PYG{o}{\PYGZlt{}}\PYG{n}{param} \PYG{n}{name}\PYG{o}{=}\PYG{l+s+s2}{\PYGZdq{}}\PYG{l+s+s2}{rmask}\PYG{l+s+s2}{\PYGZdq{}} \PYG{n}{start}\PYG{o}{=}\PYG{l+s+s2}{\PYGZdq{}}\PYG{l+s+s2}{0x02}\PYG{l+s+s2}{\PYGZdq{}} \PYG{n}{lshift}\PYG{o}{=}\PYG{l+s+s2}{\PYGZdq{}}\PYG{l+s+s2}{2}\PYG{l+s+s2}{\PYGZdq{}} \PYG{o}{/}\PYG{o}{\PYGZgt{}}
            \PYG{o}{\PYGZlt{}}\PYG{n}{bezel} \PYG{n}{element}\PYG{o}{=}\PYG{l+s+s2}{\PYGZdq{}}\PYG{l+s+s2}{hl}\PYG{l+s+s2}{\PYGZdq{}} \PYG{n}{inputtag}\PYG{o}{=}\PYG{l+s+s2}{\PYGZdq{}}\PYG{l+s+s2}{board:IN.\PYGZti{}rowno\PYGZti{}}\PYG{l+s+s2}{\PYGZdq{}} \PYG{n}{inputmask}\PYG{o}{=}\PYG{l+s+s2}{\PYGZdq{}}\PYG{l+s+s2}{\PYGZti{}lmask\PYGZti{}}\PYG{l+s+s2}{\PYGZdq{}}\PYG{o}{\PYGZgt{}}
                \PYG{o}{\PYGZlt{}}\PYG{n}{bounds} \PYG{n}{x}\PYG{o}{=}\PYG{l+s+s2}{\PYGZdq{}}\PYG{l+s+s2}{\PYGZti{}lx\PYGZti{}}\PYG{l+s+s2}{\PYGZdq{}} \PYG{n}{y}\PYG{o}{=}\PYG{l+s+s2}{\PYGZdq{}}\PYG{l+s+s2}{\PYGZti{}rowy\PYGZti{}}\PYG{l+s+s2}{\PYGZdq{}} \PYG{n}{width}\PYG{o}{=}\PYG{l+s+s2}{\PYGZdq{}}\PYG{l+s+s2}{10}\PYG{l+s+s2}{\PYGZdq{}} \PYG{n}{height}\PYG{o}{=}\PYG{l+s+s2}{\PYGZdq{}}\PYG{l+s+s2}{10}\PYG{l+s+s2}{\PYGZdq{}} \PYG{o}{/}\PYG{o}{\PYGZgt{}}
                \PYG{o}{\PYGZlt{}}\PYG{n}{color} \PYG{n}{alpha}\PYG{o}{=}\PYG{l+s+s2}{\PYGZdq{}}\PYG{l+s+s2}{\PYGZti{}lalpha\PYGZti{}}\PYG{l+s+s2}{\PYGZdq{}} \PYG{o}{/}\PYG{o}{\PYGZgt{}}
            \PYG{o}{\PYGZlt{}}\PYG{o}{/}\PYG{n}{bezel}\PYG{o}{\PYGZgt{}}
            \PYG{o}{\PYGZlt{}}\PYG{n}{bezel} \PYG{n}{element}\PYG{o}{=}\PYG{l+s+s2}{\PYGZdq{}}\PYG{l+s+s2}{hl}\PYG{l+s+s2}{\PYGZdq{}} \PYG{n}{inputtag}\PYG{o}{=}\PYG{l+s+s2}{\PYGZdq{}}\PYG{l+s+s2}{board:IN.\PYGZti{}rowno\PYGZti{}}\PYG{l+s+s2}{\PYGZdq{}} \PYG{n}{inputmask}\PYG{o}{=}\PYG{l+s+s2}{\PYGZdq{}}\PYG{l+s+s2}{\PYGZti{}rmask\PYGZti{}}\PYG{l+s+s2}{\PYGZdq{}}\PYG{o}{\PYGZgt{}}
                \PYG{o}{\PYGZlt{}}\PYG{n}{bounds} \PYG{n}{x}\PYG{o}{=}\PYG{l+s+s2}{\PYGZdq{}}\PYG{l+s+s2}{\PYGZti{}rx\PYGZti{}}\PYG{l+s+s2}{\PYGZdq{}} \PYG{n}{y}\PYG{o}{=}\PYG{l+s+s2}{\PYGZdq{}}\PYG{l+s+s2}{\PYGZti{}rowy\PYGZti{}}\PYG{l+s+s2}{\PYGZdq{}} \PYG{n}{width}\PYG{o}{=}\PYG{l+s+s2}{\PYGZdq{}}\PYG{l+s+s2}{10}\PYG{l+s+s2}{\PYGZdq{}} \PYG{n}{height}\PYG{o}{=}\PYG{l+s+s2}{\PYGZdq{}}\PYG{l+s+s2}{10}\PYG{l+s+s2}{\PYGZdq{}} \PYG{o}{/}\PYG{o}{\PYGZgt{}}
                \PYG{o}{\PYGZlt{}}\PYG{n}{color} \PYG{n}{alpha}\PYG{o}{=}\PYG{l+s+s2}{\PYGZdq{}}\PYG{l+s+s2}{\PYGZti{}ralpha\PYGZti{}}\PYG{l+s+s2}{\PYGZdq{}} \PYG{o}{/}\PYG{o}{\PYGZgt{}}
            \PYG{o}{\PYGZlt{}}\PYG{o}{/}\PYG{n}{bezel}\PYG{o}{\PYGZgt{}}
        \PYG{o}{\PYGZlt{}}\PYG{o}{/}\PYG{n}{repeat}\PYG{o}{\PYGZgt{}}
    \PYG{o}{\PYGZlt{}}\PYG{o}{/}\PYG{n}{repeat}\PYG{o}{\PYGZgt{}}
\PYG{o}{\PYGZlt{}}\PYG{o}{/}\PYG{n}{repeat}\PYG{o}{\PYGZgt{}}
\end{Verbatim}

O elemento \sphinxcode{repeat} mais externo gera um grupo de duas colunas em cada
interação; o próximo elemento \sphinxcode{repeat} gera uma coluna individual em
cada interação; o elemento \sphinxcode{repeat} interno produz dois recortes
horizontais adjacentes em cada interação.
As colunas são conectadas às portas I/O através do \sphinxcode{board:IN.7}
no topo do \sphinxcode{board.IN.0} na parte inferior.


\subsection{O Tratamento de erros}
\label{techspecs/layout_files:o-tratamento-de-erros}\label{techspecs/layout_files:layout-errors}\begin{itemize}
\item {} 
Para os arquivos de layout internos (fornecidos pelo desenvolvedor),
os erros são detectados pelo script \sphinxcode{complay.py} durante uma falha
de compilação.

\item {} 
O MAME irá parar de carregar um arquivo de layout caso haja um erro
de sintaxe e nenhuma exibição de layout estará disponível.
Alguns exemplos de erros de sintaxe são referências para elementos ou
grupos indefinidos, limites inválidos, cores inválidas, grupos
recursivamente emaranhados e a redefinição do gerador de parâmetros.

\item {} 
O MAME mostrará uma mensagem de aviso e continuará caso uma exibição
faça referência à uma tela inexistente durante o carregamento de um
layout.
Exibições apontando para telas não existentes, não são exibidas, são
consideradas inviáveis e tão pouco estarão disponíveis para o usuário.

\end{itemize}


\subsection{As Exibições geradas automaticamente}
\label{techspecs/layout_files:as-exibicoes-geradas-automaticamente}\label{techspecs/layout_files:layout-autogen}
Após o carregamento interno de layouts (fornecido pelo desenvolvedor) e
do layout externo (fornecido pelo usuário). As seguintes exibições são
geradas de forma automática:
\begin{itemize}
\item {} 
Será exibido a mensagem ``\emph{No screens Attached to the system}'' ou
``\emph{Sem telas anexadas ao sistema}'' caso o sistema não possua telas e
tão pouco sejam encontradas exibições viáveis no sistema interno ou
externo de layout.

\item {} 
A tela será exibida em sua proporção física e com a rotação aplicada
em cada tela emulada.

\item {} 
A tela será exibida em uma proporção onde os pixels sejam quadrados e
com a rotação aplicada para cada tela emulada onde a proporção de
pixel configurada não corresponda a proporção física.

\item {} 
Serão exibidos duas cópias da imagem da tela uma em cima da outra com
um pequeno espaço entre elas caso o sistema emule apenas uma tela.
A cópia da parte de cima será rotacionada em 180 graus. Esta visão
pode ser usada em uma cabine tipo cocktail, que disponibiliza uma mesa
onde os jogadores se sentam frente a frente, ou alternando os jogos
que não girem automaticamente a tela para o segundo jogador.

\item {} 
As telas serão organizadas horizontalmente da esquerda para a direita
e verticalmente de cima para baixo, ambos com e sem pequenas lacunas
entre elas caso o sistema tenha exatamente duas telas emuladas e
nenhuma exibição no layout interno ou externo mostrando todas as
telas, ou caso o sistema tenha mais de duas telas emuladas.

\item {} 
As telas serão exibidas em formato de grade, em ambas as fileiras
principais (da esquerda para a direita e de cima para baixo) e o pilar
principal (de cima para baixo e depois da esquerda para a direita).
As exibições são geradas com e sem intervalos entre as telas.

\end{itemize}
\clearpage

\subsection{Usando o complay.py}
\label{techspecs/layout_files:layout-complay}\label{techspecs/layout_files:usando-o-complay-py}
No código fonte do MAME existe um script Python chamado \sphinxcode{complay.py},
encontrado no subdiretório \sphinxcode{scripts/build}. Como parte do processo de
compilação do MAME esse script é usado para reduzir o tamanho dos dados
dos layouts internos e para convertê-los de maneira que possam ser
anexados dentro do executável.
O script pode também detectar muitos erros comuns de formatação nos
arquivos de layout fornecendo melhores mensagens de erro do que o MAME
durante a carga de tais arquivos.
Observe que o script não executa todo o mecanismo de layout, por isso
não pode detectar erros nos parâmetros usados como referências para os
elementos indefinidos ou agrupamentos dos grupos organizados de forma
recursiva.
O script \sphinxcode{complay.py} é compatível com os interpretadores Python 2.7
e Python 3.

O script \sphinxcode{complay.py} usa três parâmetros, um nome de arquivo de
entrada, um nome do arquivo de saída e um nome base para as variáveis na
saída:

\begin{Verbatim}[commandchars=\\\{\}]
\PYG{n}{python} \PYG{n}{scripts}\PYG{o}{/}\PYG{n}{build}\PYG{o}{/}\PYG{n}{complay}\PYG{o}{.}\PYG{n}{py} \PYG{n+nb}{input} \PYG{p}{[}\PYG{n}{output} \PYG{p}{[}\PYG{n}{varname}\PYG{p}{]}\PYG{p}{]}
\end{Verbatim}

O nome do arquivo de entrada é obrigatório. Caso nenhum nome de arquivo
de saída seja fornecido, o \sphinxcode{complay.py} irá analisar e verificar a
entrada, informando qualquer erros encontrado, sem gerar qualquer
arquivo na saída.
Caso nenhum nome de variável base seja fornecido, o \sphinxcode{complay.py} irá
gerar um com base no nome do arquivo de entrada. Isso não garante a
produção de identificadores válidos. O status de saída é \textbf{0}
(zero) quando for concluído com sucesso, \textbf{1} quando houver um erro
durante a invocação por linha de comando, \textbf{2} caso haja erro no
arquivo de entrada ou \textbf{3} caso seja um erro de I/O.
Ao definir um arquivo de saída o arquivo será criado ou substituído caso
seja concluído com sucesso ou removido no caso de falha.

Para aferir um arquivo de layout visando identificar se há algum tipo de
erro, execute o script apontando o caminho completo para o arquivo, como
mostra o exemplo abaixo:

\begin{Verbatim}[commandchars=\\\{\}]
\PYG{n}{python} \PYG{n}{scripts}\PYG{o}{/}\PYG{n}{build}\PYG{o}{/}\PYG{n}{complay}\PYG{o}{.}\PYG{n}{py} \PYG{n}{artwork}\PYG{o}{/}\PYG{n}{dino}\PYG{o}{/}\PYG{n}{default}\PYG{o}{.}\PYG{n}{lay}
\end{Verbatim}


\section{O dispositivo da interface de memória}
\label{techspecs/device_memory_interface:o-dispositivo-da-interface-de-memoria}\label{techspecs/device_memory_interface::doc}

\subsection{1. Das capacidades}
\label{techspecs/device_memory_interface:das-capacidades}
A interface do dispositivo de memória provê aos dispositivos a
capacidade de criar espaços de endereços mapeados aos quais estes possam
ser associados. É usado por qualquer dispositivo que forneça um
endereço/barramento de dados (lógico) para que os outros dispositivos
possam se conectar à ela. É em essência, mas não apenas, as CPUs.

A interface permite um conjunto ilimitado de espaços de endereços,
numerados com valores positivos pequenos. Os IDs devem permanecer
pequenos pois eles indexam os vetores visando manter a rápida pesquisa.
Os espaços com os números entre 0-3 tem uma constante com um nome
associado à ela:

\noindent\begin{tabulary}{\linewidth}{|L|L|}
\hline
\textsf{\relax 
ID
\unskip}\relax &\textsf{\relax 
Nome
\unskip}\relax \\
\hline
0
&
AS\_PROGRAM
\\
\hline
1
&
AS\_DATA
\\
\hline
2
&
AS\_IO
\\
\hline
3
&
AS\_OPCODES
\\
\hline\end{tabulary}


Os espaços 0 e 3 como ``\emph{AS\_PROGRAM}'' e ``\emph{AS\_OPCODE}'' são especiais para
o depurador e algumas CPU's por exemplo. AS\_PROGRAM é usado pelo
depurador e CPUs como um espaço de onde a CPU lê as suas instruções para
o desmontador. Quando presente, AS\_OPCODE é usado pelo depurador e
algumas CPUs para ler parte do `opcode' da instrução. O opcode significa
que ele é dependente do dispositivo. Por exemplo, para o z80 é o byte
inicial que é lido junto com o sinal M1 declarado.
Para o 68000 significa que cada instrução `word' mais os acessos
relativos ao PC. O principal, mas não o único uso do AS\_OPCODE, serve
para implementar a descriptografia de instruções através de um hardware
de forma separada dos dados.


\subsection{2. Configuração}
\label{techspecs/device_memory_interface:configuracao}
\begin{DUlineblock}{0em}
\item[] std::vector\textless{}std::pair\textless{}int, const address\_space\_config *\textgreater{}\textgreater{}\textbf{memory\_space\_config}\emph{(int spacenum) const}
\end{DUlineblock}

O dispositivo deve sobrescrever esse método fornecendo um vetor de pares
compreendendo um espaço numerado e seu descritor de configuração
associado \textbf{address\_space\_config}. Alguns exemplos para pesquisar
quando precisar:
\begin{itemize}
\item {} 
Vetor padrão two-space: v60\_device

\item {} 
Condicional AS\_OPCODE: z80\_device

\item {} 
Configuração herdada e com um espaço adicionado: m6801\_device

\item {} 
Configuração herdada e com um patch no espaço: tmpz84c011\_device

\end{itemize}

\begin{DUlineblock}{0em}
\item[] bool \textbf{has\_configured\_map}\emph{() const}
\item[] bool \textbf{has\_configured\_map}\emph{(int index) const}
\end{DUlineblock}

O método \textbf{has\_configured\_map} permite um teste no método
\textbf{memory\_space\_config} caso um \textbf{address\_map} seja associado com o
espaço dado. Isso permite a implementação opcional de espaços de memória
como as AS\_OPCODES em determinados núcleos de CPUs, em versões de teste
sem o uso de parâmetros para o espaço zero (0).


\subsection{3. Associando mapas aos espaços}
\label{techspecs/device_memory_interface:associando-mapas-aos-espacos}
A associação de mapas aos espaços é feito a nível de configuração da
máquina, após a declaração de dispositivo:

\begin{DUlineblock}{0em}
\item[] \textbf{MCFG\_DEVICE\_ADDRESS\_MAP}\emph{(\_space, \_map)}
\item[] \textbf{MCFG\_DEVICE\_PROGRAM\_MAP}\emph{(\_map)}
\item[] \textbf{MCFG\_DEVICE\_DATA\_MAP}\emph{(\_map)}
\item[] \textbf{MCFG\_DEVICE\_IO\_MAP}\emph{(\_map)}
\item[] \textbf{MCFG\_DEVICE\_DECRYPTED\_OPCODES\_MAP}\emph{(\_map)}
\end{DUlineblock}

A macro genérica e as quatro associações específicas associadas a um
mapa para um espaço dado. Endereços mapeados associados com espaços não
existentes são ignorados sem qualquer aviso. O \emph{devcpu.h} definem os
apelidos \textbf{MCFG\_CPU\_*\_MAP} para macros específicos.

\begin{DUlineblock}{0em}
\item[] \textbf{MCFG\_DEVICE\_REMOVE\_ADDRESS\_MAP}\emph{(\_space)}
\end{DUlineblock}

Essa macro remove a memória associada a um mapa em um determinado
espaço. Útil para remover um mapa de um espaço opcional, quando for
derivado de uma configuração de máquina.


\subsection{4. Acessando os espaços}
\label{techspecs/device_memory_interface:acessando-os-espacos}
\begin{DUlineblock}{0em}
\item[] address\_space \&\textbf{space}\emph{() const}
\item[] address\_space \&\textbf{space}\emph{(int index) const}
\end{DUlineblock}

Retorna um determinado espaço de endereços depois da inicialização.
É uma versão de testes sem parâmetros para AS\_PROGRAM/AS\_0.
Aborta na inexistência do espaço.

\begin{DUlineblock}{0em}
\item[] bool \textbf{has\_space}\emph{() const}
\item[] bool \textbf{has\_space}\emph{(int index) const}
\end{DUlineblock}

Indica se um determinado espaço fornecido realmente existe. É uma versão
de testes sem parâmetros para AS\_PROGRAM/AS\_0.


\subsection{5. Compatibilidade do MMU para o desmontador}
\label{techspecs/device_memory_interface:compatibilidade-do-mmu-para-o-desmontador}
\begin{DUlineblock}{0em}
\item[] bool \textbf{translate}\emph{(int spacenum, int intention, offs\_t \&address)}
\end{DUlineblock}

Faz uma tradução lógica para o endereço físico através do dispositivo
MMU \footnote[1]{\sphinxAtStartFootnote%
Memory management unit ou Unidade de gerenciamento de memória.
(Nota do tradutor)
}. O ``\emph{spacenum}'' dá o número do espaço, intenção do tipo do
acesso futuro \emph{(TRANSLATE\_(READ\textbar{}WRITE\textbar{}FETCH)(\textbar{}\_USER\textbar{}\_DEBUG))} e o
endereço é um parâmetro de entrada e saída (in/out) com o endereço para
traduzir e a sua versão traduzida. Deve retornar \textbf{true} caso a tradução
seja correta e \textbf{false} caso o endereço não tenha sido mapeado.

Observe que, por algum motivo histórico, o próprio dispositivo
deve substituir o método virtual \textbf{memory\_translate} com a
mesma assinatura.
\clearpage

\section{O dispositivo da interface da ROM}
\label{techspecs/device_rom_interface::doc}\label{techspecs/device_rom_interface:o-dispositivo-da-interface-da-rom}

\subsection{1. Das capacidades}
\label{techspecs/device_rom_interface:das-capacidades}
Esta interface foi concebida para dispositivos que esperam ter uma
ROM conectada a ela através de um barramento dedicado sendo
principalmente desenvolvido para CIs de áudio. Pode haver interesse de
outros tipos de dispositivos, no entanto há outros pontos a serem
levados em consideração, que podem torná-lo impraticável (como a cache
de decodificação de gráficos, por exemplo). A interface provê a
possibilidade de conexão entre um \textbf{ROM\_REGION} com um \textbf{ADDRESS\_MAP}
ou dinamicamente configurando um bloco de memória como se fosse uma ROM.
Nos casos da região e blocos, esse banco de memória é tratado de forma
automática.


\subsection{2. Configuração}
\label{techspecs/device_rom_interface:configuracao}
\begin{DUlineblock}{0em}
\item[] \textbf{device\_rom\_interface}(\emph{const machine\_config \&mconfig, device\_t \&device, u8 addrwidth, endianness\_t endian = ENDIANNESS\_LITTLE, u8 datawidth = 8})
\end{DUlineblock}

Além disso, a ordenação dos bits (\emph{endianness} \footnote[1]{\sphinxAtStartFootnote%
Para maiores explicações sobre os diferentes tipos de endianness, acesse \href{http://carlosdelfino.eti.br/programacao/cplusplus/Diferencas\_entre\_BigEndian\_Little\_Endian\_e\_Bit\_Endianness/}{este link}. (Nota do tradutor)
}), podem ser
fornecidos casos eles não sejam ``\emph{little endian}'' ou um barramento com
tamanho e largura de byte.

\begin{DUlineblock}{0em}
\item[] \textbf{MCFG\_DEVICE\_ADDRESS\_MAP}(\emph{AS\_0, map})
\end{DUlineblock}

Use esse método na configuração de tempo da máquina para que seja
providenciado um mapa de endereçamento para que seja possível a conexão
ao barramento.
Tem prioridade sobre uma região da rom, caso uma esteja presente.

\begin{DUlineblock}{0em}
\item[] \textbf{MCFG\_DEVICE\_ROM}(\emph{tag})
\end{DUlineblock}

Usado para selecionar uma região da rom a ser usada caso o mapa
de endereços de um dispositivo não seja informado. Predefinido para
\textbf{DEVICE\_SELF}, por exemplo, uma tag do dispositivo.

\begin{DUlineblock}{0em}
\item[] \textbf{ROM\_REGION}(\emph{length, tag, flags})
\end{DUlineblock}

Caso uma etiqueta (tag) esteja presente e seja idêntica a etiqueta do
dispositivo, assim como a descrição da ROM seja a mesma para o sistema,
a região da ROM definida com \textbf{MCFG\_DEVICE\_ROM}, será selecionada e
conectada. Um mapa de endereço tem prioridade sobre uma região da ROM
caso uma esteja presente na configuração da máquina.

\begin{DUlineblock}{0em}
\item[] void \textbf{set\_rom\_endianness}(\emph{endianness\_t endian})
\item[] void \textbf{set\_rom\_data\_width}(\emph{u8 width})
\item[] void \textbf{set\_rom\_addr\_width}(\emph{u8 width})
\end{DUlineblock}

Esses métodos são voltados para dispositivos genéricos com
especificações de hardware indefinidas, sobrescreve a ordenação dos
bits, a largura do barramento e o endereçamento de dados atribuída por
meio de um construtor. Eles devem ser chamados de dentro do dispositivo
antes que o \textbf{config\_complete} termine.

\begin{DUlineblock}{0em}
\item[] void \textbf{set\_rom}(\emph{const void *base, u32 size});
\end{DUlineblock}

A qualquer momento publique \textbf{interface\_pre\_start}, com este método,
um bloco de memória pode ser configurado como se uma rom estivesse
conectada. Sobrescreve qualquer configuração prévia que possa ter sido
fornecida. Pode ser feito mais de uma vez.


\subsection{3. Acesso a ROM}
\label{techspecs/device_rom_interface:acesso-a-rom}
\begin{DUlineblock}{0em}
\item[] u8 \textbf{read\_byte}(\emph{offs\_t byteaddress})
\item[] u16 \textbf{read\_word}(\emph{offs\_t byteaddress})
\item[] u32 \textbf{read\_dword}(\emph{offs\_t byteaddress})
\item[] u64 \textbf{read\_qword}(\emph{offs\_t byteaddress})
\end{DUlineblock}

Esses métodos fornecem o acesso de leitura para uma rom que esteja
conectada. O acesso fora dos limites retorna mensagens não mapeadas de
erro (\emph{logerror}).


\subsection{4. Banco da Rom}
\label{techspecs/device_rom_interface:banco-da-rom}
Caso a região da rom ou o bloco da memória no \textbf{set\_rom} seja maior
que o barramento de endereços, o banco da ROM \footnote[2]{\sphinxAtStartFootnote%
Rom banking no texto original. (Nota do tradutor)
} é configurado
automaticamente.

\begin{DUlineblock}{0em}
\item[] void \textbf{set\_rom\_bank}(\emph{int bank})
\end{DUlineblock}

Esse método seleciona o número atual do banco da rom.


\subsection{5. Ressalvas}
\label{techspecs/device_rom_interface:ressalvas}
Ao usar aquela interface, faz com que o dispositivo derive de
\textbf{device\_memory\_interface}. Caso o dispositivo queira realmente usar a
memória da interface para si mesmo, lembre-se que \textbf{AS\_0/AS\_PROGRAM} é
usado pela interface da ROM, por isso não se esqueça de chamar
\textbf{memory\_space\_config}.

Para dispositivos com saídas que possam ser usadas para endereçar
ROMs, porém restrito apenas ao encaminhamento de dados para outro
dispositivo com a única finalidade de processamento, pode ser que seja
de grande ajuda desativar a interface quando não a estiver usando.
Isso pode ser feito sobrescrevendo o \textbf{memory\_space\_config} para
retornar um vetor vazio.


\section{O device\_disasm\_interface e os desmontadores}
\label{techspecs/device_disasm_interface:o-device-disasm-interface-e-os-desmontadores}\label{techspecs/device_disasm_interface::doc}

\subsection{1. Das capacidades}
\label{techspecs/device_disasm_interface:das-capacidades}
Os desmontadores são classes que fornecem desmontagem e opcode
meta informações para os núcleos da CPU e \textbf{unidasm}. O
\textbf{device\_disasm\_interface} conecta um núcleo de CPU com seu
desmontador.


\subsection{2. Os desmontadores}
\label{techspecs/device_disasm_interface:os-desmontadores}

\subsubsection{2.1. Definição}
\label{techspecs/device_disasm_interface:definicao}
Um desmontador é uma classe que deriva de \textbf{util::disasm\_interface}.
Em seguida, ele tem dois métodos necessários de implementação,
\textbf{opcode\_alignment} e \textbf{disassemble} assim como 6 opcionais,
\textbf{interface\_flags}, \textbf{page\_address\_bits}, \textbf{pc\_linear\_to\_real},
\textbf{pc\_real\_to\_linear} e uma com quatro variantes possíveis,
\textbf{decrypt8/16/32/64}.


\subsubsection{2.2. opcode\_alignment}
\label{techspecs/device_disasm_interface:opcode-alignment}
\begin{DUlineblock}{0em}
\item[] u32 \textbf{opcode\_alignment}() const
\end{DUlineblock}

Retorna o alinhamento de opcode requisitado pela CPU nas unidades PC.
Em outras palavras o alinhamento necessário para os registros PC
da CPU.
Tende a ser 1 (quase todos), 2 (68000...), 4 (mips, ppc...),
com um excepcional 8 (processador paralelo tms 32082) e 16
(tms32010, instruções são 16-bits aligned e o PC targets bits).
Deve ser a potência de dois para evitar que as coias se quebrem.

Note que processadores como o tms32031 que têm instruções em 32-bits
mas onde os valores PC targets em 32-bits têm um alinhamento de 1.


\subsubsection{2.3. disassemble}
\label{techspecs/device_disasm_interface:disassemble}
\begin{DUlineblock}{0em}
\item[] offs\_t \textbf{disassemble}(std::ostream \&stream, offs\_t pc, const data\_buffer \&opcodes, const data\_buffer \&params)
\end{DUlineblock}

Este é o método onde o trabalho de fato é acontece. Esse comando
desmonta uma instrução no endereço \emph{PC} e escreve o resultado para
\emph{stream}. Os valores a serem decodificados são recuperados
da memória intermediária \emph{opcode}. Um objeto \textbf{data\_buffer} oferecem
quatro métodos de acesso:

\begin{DUlineblock}{0em}
\item[] u8  util::disasm\_interface::data\_buffer::\textbf{r8} (offs\_t pc) const
\item[] u16 util::disasm\_interface::data\_buffer::\textbf{r16}(offs\_t pc) const
\item[] u32 util::disasm\_interface::data\_buffer::\textbf{r32}(offs\_t pc) const
\item[] u64 util::disasm\_interface::data\_buffer::\textbf{r64}(offs\_t pc) const
\end{DUlineblock}

Eles leem os dados em um determinado endereço e pegam o endianness e os
PCs não lineares por acessos maiores que a largura do barramento.
A variante do depurador também armazena em cache os dados lidos em um
bloco, então por essa razão um não deve ler os dados muito longe da base
pc (ficar entre de 16K ou então, ter cuidado ao tentar seguir acessos
indiretos, por exemplo).

Uma quantidade de CPUs tem um sinal externo que divide as buscas em
parte um opcode e parte um parâmetro. Este é, por exemplo o sinal M1
do z80 ou o sinal SYNC do 6502. Alguns sistemas apresentam
diferentes valores para a CPU dependendo se esse sinal for
ativo, em geral usado para fins de proteção. Nestes CPUs a parte do opcode
deve ser lida a partir da memória intermediária do \emph{opcode} e o
parâmetro part vindo da memória intermediária \emph{params}. Eles serão ou
não a mesma memória intermediária, tudo vai depender do próprio sistema.

O método retorna o tamanho da instrução em unidades de PC, com um valor
máximo de 65535. Além disso, caso seja possível o desmontador deve
dar algumas informações meta sobre o opcode por ``OR-ing'' no resultado:
\begin{itemize}
\item {} \begin{description}
\item[{\textbf{STEP\_OVER} para chamadas de sub-rotina ou auto-decrementos de}] \leavevmode
loops. Caso haja alguns slots com atraso, faça também OR com
\textbf{step\_over\_extra}(n) onde n é o número da instrução.

\end{description}

\item {} 
\textbf{STEP\_OUT} para o retorno das instruções da sub-rotina

\end{itemize}

Além disso, para indicar que esses sinalizadores são compatíveis, OU o
resultado com \textbf{SUPPORTED}. Uma quantidade chata de desmontadores mentem
sobre essa compatibilidade (eles fazem um OR com \textbf{SUPPORTED} mesmo sem
gerar o \textbf{STEP\_OVER} ou \textbf{STEP\_OUT}, por exemplo). Não faça
isso, pois quebra a funcionalidade do \emph{step over/step out} do depurador.


\subsubsection{2.4. interface\_flags}
\label{techspecs/device_disasm_interface:interface-flags}
\begin{DUlineblock}{0em}
\item[] u32 \textbf{interface\_flags}() const
\end{DUlineblock}

Esse método opcional mostra detalhes do desmontador. O valor zero
predefinido é o correto na maioria das vezes. As bandeiras possíveis e
que precisam ser ``OR-ed'' juntas, são:
\begin{itemize}
\item {} 
\textbf{NONLINEAR\_PC}: passar para o próximo opcode ou o próximo byte do opcode se não adicionar um ao pc. Usado para antigos PCs com base em LFSR.

\item {} 
\textbf{PAGED}: o PC é envolvido com um limite de página

\item {} 
\textbf{PAGED2LEVEL}: não apenas o PC envolve em algum tipo de limite de página, mas há dois níveis de paginação

\item {} 
\textbf{INTERNAL\_DECRYPTION}: há alguma descriptografia escondida entre a leitura de AS\_PROGRAM e o desmontador atual

\item {} 
\textbf{SPLIT\_DECRYPTION}: há alguma descriptografia escondida entre a leitura do AS\_PROGRAM e o desmontador atual, assim como essa descriptografia é diferente para os opcodes e os parâmetros

\end{itemize}

Note que, na prática, os sistemas de PC não lineares também são paginados,
o \textbf{PAGED2LEVEL} implica em \textbf{PAGED} e que \textbf{SPLIT\_DECRYPTION}
implica em \textbf{DECRYPTION}.


\subsubsection{2.5. pc\_linear\_to\_real and pc\_real\_to\_linear}
\label{techspecs/device_disasm_interface:pc-linear-to-real-and-pc-real-to-linear}
\begin{DUlineblock}{0em}
\item[] offs\_t \textbf{pc\_linear\_to\_real}(offs\_t pc) const
\item[] offs\_t \textbf{pc\_real\_to\_linear}(offs\_t pc) const
\end{DUlineblock}

Esses métodos devem estar presentes apenas quando \textbf{NONLINEAR\_PC}
estiver definido nos sinalizadores da interface. Eles devem converter o
PC de e para um valor com destino a um domínio linear onde os parâmetros
de instrução e a próxima instrução sejam alcançadas ao incrementar o
valor. O \textbf{pc\_real\_to\_linear} converte para aquele domínio, já o
\textbf{pc\_linear\_to\_real} é convertido de volta daquele domínio.


\subsubsection{2.6. page\_address\_bits}
\label{techspecs/device_disasm_interface:page-address-bits}
\begin{DUlineblock}{0em}
\item[] u32 \textbf{page\_address\_bits}() const
\end{DUlineblock}

Presente quando \textbf{PAGED} ou \textbf{PAGED2LEVEL} for definido, retorna a
quantidade de endereços de bits na pagina inferior.


\subsubsection{2.7. page2\_address\_bits}
\label{techspecs/device_disasm_interface:page2-address-bits}
\begin{DUlineblock}{0em}
\item[] u32 \textbf{page2\_address\_bits}() const
\end{DUlineblock}

Presente quando \textbf{PAGED2LEVEL} for definido, retorna a quantidade
de endereços de bits na página superior.


\subsubsection{2.8. decryptnn}
\label{techspecs/device_disasm_interface:decryptnn}
\begin{DUlineblock}{0em}
\item[] u8  \textbf{decrypt8} (u8  value, offs\_t pc, bool opcode) const
\item[] u16 \textbf{decrypt16}(u16 value, offs\_t pc, bool opcode) const
\item[] u32 \textbf{decrypt32}(u32 value, offs\_t pc, bool opcode) const
\item[] u64 \textbf{decrypt64}(u64 value, offs\_t pc, bool opcode) const
\end{DUlineblock}

Um destes deve ser definido quando \textbf{INTERNAL\_DECRYPTION} ou
\textbf{SPLIT\_DECRYPTION} for configurado. O escolhido será aquele que leva
o que \textbf{opcode\_alignment} representa em bytes.

Esse método descriptografa um determinado valor do endereço PC (a partir
de AS\_PROGRAM) e retorna o que será passado para o desmontador.
No caso da descriptografia dividida, o opcode indica se estamos no
opcode (true) ou no parâmetro (false) parte da instrução.


\subsection{3. Interface do desmontador, device\_disasm\_interface}
\label{techspecs/device_disasm_interface:interface-do-desmontador-device-disasm-interface}

\subsubsection{3.1. Definição}
\label{techspecs/device_disasm_interface:id1}
Um núcleo de CPU deriva de \textbf{device\_disasm\_interface} através do
\textbf{cpu\_device}. Um método deve ser implementado,
\textbf{create\_disassembler}.


\subsubsection{3.2. create\_disassembler}
\label{techspecs/device_disasm_interface:create-disassembler}
\begin{DUlineblock}{0em}
\item[] util::disasm\_interface *\textbf{create\_disassembler}()
\end{DUlineblock}

Esse método deve retornar um ponteiro para um novo objeto desmontado que
foi recém-alocado. O solicitante apropria-se do objeto e lida com o seu
tempo de vida.

Esse método será chamado no máximo uma vez durante a vida útil
do objeto da CPU.


\subsection{4. A comunicação e a configuração do Desmontador}
\label{techspecs/device_disasm_interface:a-comunicacao-e-a-configuracao-do-desmontador}
Alguns desmontadores precisam ser configurados. A configuração pode ser
imutável (estático) duração da execução (como o modelo da CPU por
exemplo) ou dinâmico (o estado de um sinalizador ou uma preferência de
usuário). A configuração estática que pode ser feita seja por parâmetro(s)
para o construtor do desmontador ou através da derivação da classe do
desmontador principal. Caso a informação seja curta e sua semântica seja
óbvia (como o nome do modelo), fique à vontade para usar um parâmetro.
Caso contrário, deriva a classe.

A configuração dinâmica deve ser feita definindo primeiro uma
estrutura de grupo público chamado ``config'' no desmontador,
com o destruidor virtual e métodos virtuais puros para extrair
as informações necessárias. Um ponteiro para essa estrutura deve ser
passada para o construtor do desmontador. O núcleo da CPU deve então
adicionar uma derivação dessa estrutura de configuração e implementar os
métodos. O Unidasm terá que separar pequena classe da configuração de
classes para que possa passar a informação.


\subsection{5. Coisas que faltam}
\label{techspecs/device_disasm_interface:coisas-que-faltam}
Atualmente, não há como a GUI do depurador adicionar
uma configuração para cada núcleo. Ela se faz necessária para o s2650 e
os núcleos do saturn. É necessário também passar pela própria classe do
núcleo da CPU uma vez que é retirado da estrutura de configuração.

Falta compatibilidade do unidasm para uma configuração individual dos
núcleos da CPU. Isso se faz útil para muitas coisas, veja o código-fonte
do unidasm para a um lista atual (comentários ``Configuration missing'').
\clearpage

\section{O novo subsistema de disquete}
\label{techspecs/floppy:o-novo-subsistema-de-disquete}\label{techspecs/floppy::doc}

\subsection{Introdução}
\label{techspecs/floppy:introducao}
O novo subsistema de disquete visa emular o comportamento de disquetes e
controladores de disquetes em nível baixo o suficiente a ponto de fazer
com que as proteções também funcionem de forma transparente. O objetivo
é alcançado ao seguir a configuração de um hardware real:
\begin{itemize}
\item {} 
uma classe de imagem de disquete que mantém na memória o estado
magnético da superfície flexível e as suas características físicas.

\item {} 
uma classe manipuladora de imagem fala com a classe de imagem de
disquete visando simular o drive de disquete, fornecendo todos os
sinais existentes em um conector de disquete.

\item {} 
dispositivos controladores que conversam com o manipulador de imagem e
fornecem as interfaces de registo para o host que todos nós conhecemos
e amamos.

\item {} 
nas classes de manipulação de formato, lhes são dadas a tarefa de
converter a origem e destino de forma neutra de uma imagem de disco
físico para um estado de formato magnético do disco na memória,
de forma que a classe gerenciadora do disquete possa geri-la.

\end{itemize}


\subsection{O armazenamento de disquete para leigos}
\label{techspecs/floppy:o-armazenamento-de-disquete-para-leigos}

\subsubsection{O disquete}
\label{techspecs/floppy:o-disquete}
O disquete é um disco que armazena as orientações magnéticas em sua
superfície, dispostas em uma série de círculos concêntricos chamado de
faixas ou cilindros \footnote[1]{\sphinxAtStartFootnote%
O cilindro é um termo de disco rígido usado de forma inadequada
para disquetes. Ele vem do fato que os discos rígidos são
semelhantes aos disquetes, mas incluem uma série de discos
empilhados com uma cabeça de leitura/gravação em cada um deles.
As cabeças estão fisicamente ligadas e todas apontam para o
mesmo círculo em cada disco em um determinado momento, fazendo
com que a área acessada pareça com um cilindro.
Daí o nome. (Nota do tradutor)
}. As suas principais características são o seu
tamanho que vai de um diâmetro em torno de 2.8 polegadas
(63.5 milímetros) até 8 polegadas (200 milímetros), seu número de lados
graváveis (1 ou 2) e sua resistividade magnética. A resistividade
magnética indica o quão perto uma mudança na orientação magnética pode
ocorrer e a informação mantida.
Isso é um terço do que define o termo ``densidade'' que é usado com tanta
frequência para disquetes (os outros dois são o tamanho da cabeça do
drive de disquetes e a codificação a nível de bit (\emph{bit-level encoding}
no Inglês).

As orientações magnéticas são sempre binárias, elas sempre apontam para
um lado ou para o outro, não há nenhum estado intermediário. A sua
direção pode estar na tangente da pista, na mesma direção, oposta a
rotação ou no caso de uma gravação no sentido perpendicular, a direção é
perpendicular (por isso o nome). A gravação no sentido perpendicular
permite que os dados de gravação ocupem menos espaço permitindo uma
maior densidade de gravação, porém chegou no final do tempo de vida da
tecnologia. Os discos com 2.88 Mb e derivados dos disquetes como Zip
Drives (etc), usavam gravação perpendicular. Para fins de emulação, a
direção não importa, o que importa é o fato que duas orientações são
possíveis. Além dessas orientações mais duas são possíveis: uma parte da
trilha pode ser desmagnetizada (sem orientação) ou danificada (sem
orientação ou não pode ser gravada).

Uma posição específica na rotação disco dispara um pulso de índice.
Essa posição pode ser detectada através de um buraco na superfície
(muito visível em disquetes 5.25 e 3 polegadas por exemplo) ou através
de uma posição específica do centro de rotação (disquetes com 3.5
polegadas, talvez outros). Esse pulso de índice é usado para determinar
o início da faixa, porém não é usado por todos os sistemas. Os disquetes
mais antigos de 8 polegadas têm múltiplos buracos marcando o índice
determinando o início dos setores (chamados de setor duro), no entanto
um deles está numa posição diferente para ser reconhecido como um início
de trilha, e os outros estão em posições fixas relativas à origem.


\subsubsection{Unidade de Disquete}
\label{techspecs/floppy:unidade-de-disquete}
Uma unidade de disquete é o aparelho que lê e grava um disquete. Inclui
um conjunto capaz de girar o disco a uma velocidade fixa e uma ou duas
cabeças magnéticas ligadas a um motor de posicionamento para acessar as
trilhas.

A largura da cabeça e o tamanho do passo do motor de posicionamento
determinam quantas trilhas estão escritas no disquete. O número total de
trilhas varia entre 32 até 84 de acordo com o disquete e o drive, a
trilha 0 ficando mais ao externo (mais longo) dos círculos concêntricos,
e o maior com o menor círculo interno. Como resultado, as faixas com os
números mais baixos têm a menor densidade física de orientação
magnética, portanto, uma melhor confiabilidade. É por isso que
estruturas importantes e/ou frequentemente alteradas, como o bloco de
inicialização ou a tabela de alocação FAT, estão na trilha 0. É também
aí que vem a terminologia ``stepping in'' para aumentar o número da faixa
e ``stepping out'' para diminuí-lo. O número de faixas disponíveis é a
segunda parte do que geralmente está por trás do termo ``densidade''.

Um sensor detecta quando a cabeça está na faixa 0 e o controlador não
deve passar por ela. Além disso, bloqueios físicos impedem que a cabeça
saia do alcance correto da pista. Alguns sistemas (Apple II, alguns C64)
não levam em conta o sensor da trilha 0 fazendo com que a cabeça vá
contra o limite físico do bloco, fazendo um ruído de impacto bem
conhecido e eventualmente danificando o alinhamento da cabeça.

Além disso, alguns sistemas (Apple II e C64) têm acesso direto às fases
do motor de posicionamento da cabeça, permitindo que a cabeça se
posicione entre as pistas, no meio ou mesmo em posições intermediárias.
Isso não era útil para escrever mais faixas, uma vez que a largura da
cabeça não mudava, mas como a leitura confiável só era possível com a
posição correta, ela era usada como proteção contra cópia por alguns
sistemas.

O disco gira a uma velocidade fixa para uma determinada faixa.
A velocidade mais comum é de 300 RPM para cada faixa, com 360 rpm
encontrado para os disquetes de alta densidade com 5.25 polegadas e a
maioria dos disquetes com 8 polegadas. A velocidade dos primeiros
disquetes giravam em torno de 90 RPM ou até mesmo 150 RPM para um
disquete de alta densidade em um Amiga. Ter uma velocidade rotacional
fixa para todo o disco é chamada de Velocidade Angular Constante
(CAV em inglês) usada por quase todos ou Velocidade Angular Constante
Zoneada (ZCAV em inglês, usado no C64), dependendo se a taxa de bits de
leitura/gravação é constante ou depende da faixa. Alguns sistemas como
Apple II e Mac variam a velocidade de rotação dependendo da faixa (algo
como até 394 RPM) para terminar como uma Velocidade Linear Constante
(\emph{Constant Linear Velocity} ou CLV em Inglês). A ideia por trás do
ZCAV/CLV é extrair mais bits da mídia mantendo o espaçamento mínimo
entre transições de orientação magnética, oferecendo a melhor
performance possível entre o espaço ocupado e a velocidade de transição
da cabeça. Parece que a complexidade não foi considerada válida já que
quase nenhum sistema faz.

Finalmente, após o disco girar e a cabeça estiver sob a posição
adequada, a leitura correta da faixa acontece. A leitura é feita através
de uma cabeça indutiva, que lhe dá a característica interessante de não
ler a orientação magnética de forma direta, ao invés disso, ser sensível
o suficiente às inversões de orientação, chamadas de transições de
fluxo. Esta detecção é fraca e pouco precisa, de modo que um
amplificador com Ajuste de Ganho Automático (\emph{Automatic Gain Control}
ou AGC em Inglês) e um detector de pico são colocados de forma a
trabalhar em conjunto com da cabeça para fornecer pulsos limpos.
O AGC aumenta lentamente o nível de amplificação até que um sinal
ultrapasse um limite pré determinado, em seguida ajusta seu ganho para
que o dito sinal esteja estável em um nível fixo dentro deste limite.
Conforme a oscilação vai acontecendo o AGC entra em ação novamente.
Isso faz com que o amplificador se calibre para os sinais lidos no
disquete, desde que as transições de fluxo aconteçam com uma certa
frequência. Em uma zona muito longa, ocorre a captação de ruídos
aleatórios do ambiente, fazendo com que a amplificação deste sinal
ultrapasse o limite pré estabelecido, criando pulsos falsos onde não
existem nenhum. Muito longa neste caso são aquelas que acontecem entre
16-20us sem nenhuma transição.

Isso significa que uma zona suficientemente longa com uma orientação
magnética fixa ou nenhuma orientação (desmagnetizada ou danificada) será
lida como uma série de pulsos aleatórios após um breve atraso. Isso é
usado por proteções e é conhecido como ``weak bits'', que ao serem lidos
os dados são diferente cada vez que são acessados.

Um segundo nível de filtragem ocorre após o detector de pico. Quando
duas transições estão um pouco próximas (mas ainda acima do limiar da
mídia), um efeito saltante acontece entre elas, dando dois pulsos muito
próximos no meio, além dos dois pulsos normais. O drive de disquete
consegue detectar quando os pulsos estão muito próximos e os elimina,
deixando os pulsos normais novamente. Como resultado, se alguém escrever
uma cadeia de pulsos de alta frequência para o disquete, eles serão
lidos como um trem de pulsos muito próximos (fracos porque estão acima
da tolerância da mídia, mas capturados pelo AGC de qualquer forma,
apenas de forma pouco confiável) eles serão todos filtrados, dando uma
grande quantidade de tempo sem qualquer pulso no sinal de saída. Isso é
usado por algumas proteções uma vez que não é gravável usando o relógio
normal do controlador.

A escrita é simétrica, com uma série de pulsos enviados que fazem a
cabeça de gravação inverter a orientação do campo magnético cada vez que
um pulso é recebido.

Então, para concluir, a unidade de disquete fornece insumos para disco
de controle de rotação e a posição da cabeça (assim como a escolha
quando é de dupla-face), os dados são enviados de duas maneiras como um
trem de pulsos que representam inversões de orientação magnética.
O valor absoluto da orientação em si nunca é conhecido.


\subsubsection{Controlador de Disquete}
\label{techspecs/floppy:controlador-de-disquete}
A tarefa do controlador de disquete é transformar a comunicação da
unidade de disquete em algo a CPU principal possa compreender.
O nível de compatibilidade entre um controlador e outro varia aos
extremos, vai de praticamente nada nos Apple II e C64, com alguma coisa
no Amiga e para completar Circuitos Integrados da \emph{Western Digital},
família \textbf{uPD765}).
Funções comuns incluem a seleção da unidade, controle do motor, busca
das trilhas e claro a leitura e gravação de dados. Destes somente os
dois últimos precisam ser descritos pois o resto é óbvio.

Os dados são estruturados em dois níveis: como bits individuais (meio
byte ou bytes) que são codificados na superfície e como estes são
agrupados em setores endereçados individualmente. Existem dois padrões
para eles chamados \emph{Frequency Modulation} (sigla FM no inglês) e
\emph{Modified Frequency Modulation} (sigla MFM no inglês), além de uma
série de outros sistemas e suas variantes. Além disso, alguns sistemas
tais como o Amiga usa um padrão de codificação \emph{bit-level encoding}
(MFM) com uma organização de nível setorial local.


\subsection{Codificação a nível de bit}
\label{techspecs/floppy:codificacao-a-nivel-de-bit}

\subsubsection{Organização Celular}
\label{techspecs/floppy:organizacao-celular}
Todos os controladores de disquetes, até os mais esquisitos como o
Apple II, começa dividindo a pista em células de igual tamanho. Eles são
seções angulares no meio de onde uma inversão de orientação magnética
pode estar presente. Do ponto de vista do hardware, as células são
vistas como durações que combinada com a rotação do disquete determina
a seção. Por exemplo o tamanho padrão de uma célula MFM para um disquete
de dupla densidade com 3 polegadas é de 2us, também combinada com uma
velocidade de rotação com 300 RPM, dá um tamanho angular de 1/100.000
por volta. Outra maneira de dizer a mesma coisa é que há 100K (cem mil)
células em uma pista de dupla densidade de um disquete de 3 polegadas.

Em cada célula pode ou não haver uma transição de orientação magnética,
por exemplo, uma pulsação vindo de uma leitura ou ir para a escrita da
unidade de disquete. Uma célula com um pulso é tradicionalmente
conhecida como `1', e um sem `0'. Embora, duas restrições aplicam-se
para o conteúdo da célula. Primeiro, os pulsos não devem ser muito
juntos ou eles irão causar um borrão um ao outro, e/ou serão filtrados.

O limite é ligeiramente melhor do que 1/50.000 de uma volta para
disquete com densidade simples e dupla, metade disso para disquetes
de alta densidade e metade disso novamente para disquetes com densidade
estendida (ED) com gravação perpendicular. Segundo, eles não devem ser
muito longe um do outro, ou seja o AGC vai ficar instável e introduzir
pulsos fantasmas ou o controlador vai perder sincronização e obter um
sincronismo errado sobre as células durante a leitura.
Para via de regra geral, é melhor não ter mais de 3 células `0'
consecutivas.

Certas proteções usam isso para tornar os formatos não reconhecíveis
pelo controlador do sistema, quebrando a regra de três zeros ou brincar
com as durações e tamanhos das células.

Bit endocing é a arte de transformar dados brutos em uma célula de
configuração 0/1 que respeite as os dois limites.


\subsubsection{Codificação FM}
\label{techspecs/floppy:codificacao-fm}
O primeiro método de codificação desenvolvido para disquetes é chamado
de Frequência Modulada (\emph{Frequency Modulation} ou FM), o tamanho da
célula é definida um pouco além do limite físico, como 4us por exemplo.
Isso significa que é possível ter `1' célula consecutiva de confiança.
Cada bit é codificado em duas células:
\begin{itemize}
\item {} 
a primeira célula, chamada o clock bit é `1'

\item {} 
a segunda célula, chamada de data bit, é o bit em si

\end{itemize}

Uma vez que todas as outras células seja pelo menos `1' não há nenhum
risco de ir além de três zeros.

O nome Frequência Modulada simplesmente deriva do fato de que um 0 é
codificado com um período de trem de pulsos em 125 Khz enquanto um 1
são dois períodos do trem de pulso em 250 Khz.


\subsubsection{Codificação MFM}
\label{techspecs/floppy:codificacao-mfm}
A codificação de FM foi substituída pela codificação \emph{Modified Frequency
Modulation (MFM)}, que pode empilhar exatamente o dobro de dados na
mesma superfície, daí seu outro nome de ``dupla densidade''.
O tamanho da célula é definido com um pouco mais de metade do limite
físico, 2us normalmente. A restrição significa que duas células `1'
devem ser separadas por pelo menos uma célula `0'. Cada bit é novamente
codificado em duas células:
\begin{itemize}
\item {} 
a primeira célula, chamada de clock bit, é `1' se ambos os bits de
dados anteriores e atuais forem 0, então será `0'

\item {} 
a segunda célula, chamada de data bit, é o bit em si

\end{itemize}

A regra de espaço mínimo é respeitada uma vez que um `1' de clock bit é,
por definição, rodeado por dois `0' de data bits e um `1' data bit é
rodeado por dois `0' clock bits. A maior cadeia de célula 0 possível é
quando ao codificar 101 que retorna x10001, respeitando o limite máximo
de três zeros.


\subsubsection{Codificação GCR}
\label{techspecs/floppy:codificacao-gcr}
As codificações \emph{Group Coded Recording}, ou GCR, são uma classe de
codificações onde cadeias de bits com pelo menos tamanho de meio byte ou
4 bit são codificadas em um determinado fluxo de células dado por uma
tabela. Ele foi usado particularmente pelo Apple II, o Mac e o C64, e
cada sistema tem sua própria tabela ou tabelas.


\subsubsection{Outras codificações}
\label{techspecs/floppy:outras-codificacoes}
Existem outras codificações como o M2FM, mas elas são muito raras e
específicas para um determinado sistema.


\subsubsection{Lendo os dados codificados}
\label{techspecs/floppy:lendo-os-dados-codificados}
Escrever dados codificados é fácil, você só precisa de um relógio na
frequência apropriada e enviar ou não uma cadeia de pulsos ao redor do
relógio. A diversão está em ler esses dados.
As células são uma construção lógica e não uma entidade física
mensurável.

As velocidades rotacionais variam ao redor dos valores definidos (+/- 2\%
não é raro) e perturbações locais (turbulência do ar, distância da
superfície...) no geral, tornam a velocidade instantânea muito variável.
Portanto, para extrair o fluxo de valores da célula, o controlador deve
sincronizar dinamicamente com o trem de pulso que a cabeça do disquete
seleciona. O princípio é simples: uma janela de duração do tamanho da
célula é construída dentro da qual a presença de pelo menos um pulso
indica que a célula é um `1' e a ausência de qualquer um `0'.
Depois de chegar ao final da janela, a hora de início é movida
apropriadamente para tentar manter o pulso observado no meio exato dessa
janela. Isso permite corrigir a fase em cada célula `1', fazendo a
sincronização funcionar se a velocidade de rotação não estiver muito
fora.

Gerações subsequentes de controladores usaram um \emph{Phase Locked Loop}
(PLL) que varia a duração da fase e da janela para se adaptar melhor as
velocidades erradas de rotação, geralmente com uma tolerância de +/-
15\%.

Depois que o fluxo de dados da célula é extraído, a decodificação
depende da codificação. No caso de FM e MFM, a única questão é
reconhecer os bits de dados dos bits de clock, enquanto no GCR a posição
inicial do primeiro grupo deve ser encontrada. O segundo nível de
sincronização é tratado em um nível mais alto usando padrões não
encontrados em um fluxo normal.


\subsection{Organização de nível no setor}
\label{techspecs/floppy:organizacao-de-nivel-no-setor}
Os disquetes foram concebidos para a leitura e gravação com acesso
aleatório para blocos de dados de tamanhos razoáveis. Permite a seleção
de faixas para um primeiro nível de acesso aleatório e dimensionamento,
mas os 6 K de uma faixa de densidade dupla seria muito grande para ser
lidado por um bloco. 256/512 bytes são considerados um valor mais
apropriado. Para o efeito, dados em uma faixa são organizados como uma
série de (cabeçalho do setor, dados do setor) pares onde o cabeçalho do
setor indicam informações importantes, como o número do setor, tamanho,
e os dados do setor que contém os dados. Os setores tem que ser
quebrados em duas partes, porque enquanto a leitura é fácil, é lido o
cabeçalho, depois os dados sem assim for necessário, para escrever
requer a leitura do cabeçalho para encontrar o lugar correto, para só
então ligar a cabeça de escrita para os dados. A escrita inicial não é
instantânea e a fase não está perfeitamente alinhada com a cabeça de
leitura, portanto, um espaço para a sincronização é necessária entre o
cabeçalho e dados.

Somando a isso, em algum lugar no setor do cabeçalho e no sector dos
dados, geralmente são adicionados algum tipo de checksum para permitir
a verificação da integridade destes dados.

O FM e o MFM (nem sempre utilizaram) métodos de layout padrão do setor.


\subsubsection{Layout do setor de FM}
\label{techspecs/floppy:layout-do-setor-de-fm}
O layout padrão em FM de trilha/setor para um ``PC'' é assim:
\begin{itemize}
\item {} 
Uma quantidade de 0xff codificados em FM (40 geralmente)

\item {} 
6 0x00 codificados em FM (dando uma cadeia de pulso estável em 125 Khz)

\item {} 
Um fluxo 1111011101111010 com 16 células (f77a, clock 0xd7, data 0xfc)

\item {} 
Uma quantidade de 0xff codificados em FM (geralmente 26, muito volátil)

\end{itemize}

Então para cada setor:
- 6 0x00 codificados em FM (dando uma cadeia de pulso estável em 125 Khz)
\begin{itemize}
\item {} 
Um fluxo 1111010101111110 com 16 células (f57a, clock 0xc7, data 0xfe)

\end{itemize}

Cabeçalho do sector, faixa codificada em FM, cabeça, setor, código de
tamanho e dois bytes de crc por exemplo
\begin{itemize}
\item {} 
11 0xff codificados em FM

\item {} 
6 0x00 codificados em FM (dando uma cadeia de pulso estável em 125 Khz)

\item {} 
Um fluxo 1111010101101111 com 16 células (f56f, clock 0xc7, data 0xfb)

\item {} 
Dados do setor codificado em FM seguido por dois bytes CRC

\item {} 
Uma quantidade de 0xff codificados em FM (geralmente 48, muito volátil)

\end{itemize}

A trilha é terminada com um fluxo de células `1'.

Os trens de pulsos com 125 KHz são utilizados para travar o PLL ao
sinal corretamente. Os fluxos específicos com 16 células permitem
distinguir entre o clock e os data bits fornecendo um arranjo que não é
comum ocorrer em dados codificados em FM. No cabeçalho do sector da
trilha, os números começam em 0, cabeças são 0/1 dependendo do tamanho,
os números do setor geralmente começam em 1 e o tamanho do código é 0
para 128 bytes, 1 para 256, 2 para 512, etc.

O CRC é uma verificação de redundância cíclica dos bits de dados,
começando com uma marca logo após o trem de pulso usando o polinômio
0x11021.

Os controladores com base na Western Digital geralmente livram-se de
tudo deixando alguns 0xff no primeiro setor e permitem um melhor uso do
espaço como resultado.


\subsubsection{Layout do setor de FM}
\label{techspecs/floppy:id2}
O layout padrão de trilha/sector para MFM num ``PC'' é assim:
\begin{itemize}
\item {} 
Uma quantidade de 0x4e codificados em MFM (80 geralmente)

\item {} 
12 0x00 codificados em FM (dando uma cadeia de pulso estável em
125 Khz)

\item {} 
Um fluxo 0101001000100100 com 16 células (5224, clock 0x14, data 0xc2)

\item {} 
O valor 0xfc codificado em MFM

\item {} 
Uma quantidade de 0x4e codificados em MFM (geralmente 50, muito
volátil)

\end{itemize}

Então para cada setor:
\begin{itemize}
\item {} 
12 0x00 codificados em FM (dando uma cadeia de pulso estável em
125 Khz)

\item {} 
Três vezes um fluxo 0100010010001001 com 16 células (5224, clock 0x14,
data 0xc2)

\item {} 
Cabeçalho do sector, 0xfe codificado em MFM, trilha, cabeça, setor,
código de tamanho e dois bytes de CRC por exemplo

\item {} 
22 0x4e codificado em MFM

\item {} 
12 0x00 codificados em MFM (dando uma cadeia de pulso estável em
125 Khz)

\item {} 
Três vezes um fluxo 0100010010001001 com 16 células (5224, clock 0x14,
data 0xc2)

\item {} 
0xfb codificado em MFM, dados do setor seguido por dois bytes CRC

\item {} 
Uma quantidade de 0x4e codificados em MFM (geralmente 84, muito
volátil)

\end{itemize}

A trilha é finalizada com um fluxo 0x4e codificado em MFM.

Os trens de pulsos com 125 KHz são utilizados para travar o PLL ao
sinal de forma correta. A célula com o arranjo 4489 não aparece numa
codificação de dados MFM normal e é usada para a separação de
clock/dados.

Já para FM, os controladores com base Western Digital geralmente
livrarm-se de tudo menos alguns 0x4e antes do primeiro setor e permite
um melhor uso do espaço como resultado.


\subsubsection{Formatação e escrita}
\label{techspecs/floppy:formatacao-e-escrita}
Para ser utilizável, um disquete deve ter os cabeçalhos do setor e os
dados padrão escritos em cada trilha. O controlador começa a escrita em
um determinado lugar, muitas vezes pelo pulso de índice, mas em alguns
sistemas sempre que o comando é enviado ele grava até que seja feita uma
volta completa. Isso é conhecido como formatação de disquete. No ponto
onde a escrita termina, há uma perda de sincronização uma vez que não
há nenhuma chance do relógio de fluxo da célula terminar a escrita de
forma correta. Esta mudança de fase brutal é chamada uma gravação da
tala, especificamente a faixa escrever da tala. É o ponto onde a
escrita deve começar se você quiser uma cópia raw da faixa para um novo
disquete.

Igualmente duas junções de gravação são criadas quando um setor é
escrito no início e no final da parte do bloco de dados. Não deveria
acontecer num disco masterizado, mesmo que haja algumas raras exceções.


\subsection{A nova implementação}
\label{techspecs/floppy:a-nova-implementacao}

\subsubsection{Representação do disquete}
\label{techspecs/floppy:representacao-do-disquete}
O conteúdo do disquete é representado pela classe \emph{floppy\_image}.
Contém informações do tipo de mídia e uma representação do estado
magnético da superfície.

O tipo de mídia é dividido em duas partes. A primeira metade indica o
fator de forma física, ou seja, todas as mídias com esse fator podem ser
fisicamente inseridas em um leitor que puder manuseá-lo.
A segunda metade indicam as variantes que são geralmente detectáveis
pelo leitor, tais como a densidade e o número de lados.

Trilha de dados consiste em uma série valores lsb primários em 32-bits
representando as células magnéticas. Os bits 0-27 indicam a posição
absoluta do início da célula (não o tamanho) e os bits 28-31 indicam os
tipos. Os tipos podem ser:
\begin{itemize}
\item {} 
0, MG\_A -\textgreater{} Orientação Magnética A

\item {} 
1, MG\_B -\textgreater{} Orientação Magnética B

\item {} 
2, MG\_N -\textgreater{} Zona não magnetizada (neutra)

\item {} 
3, MG\_D -\textgreater{} Zona danificada, lê como neutra mas não pode ser alterada
por escrita

\end{itemize}

A posição está em unidades angulares de 1/200,000,000 de uma volta.
Corresponde a um nanossegundo quando a unidade gira a 300 RPM.

A última posição implícita da célula é 200,000,000.

As trilhas não formatadas são codificadas com um tamanho zero.

A informação de ``junção de trilha'' indica onde começar a escrever caso
você tente reescrever um disco físico com dados. Alguns formatos de
preservação codificam essa informação, ela é adivinhada para os outros.
A função de gravação da trilha do fdcs deve configurá-la.
A representação é a posição angular relativa ao índice.


\subsection{Convertendo de e para uma representação interna}
\label{techspecs/floppy:convertendo-de-e-para-uma-representacao-interna}

\subsubsection{Classe e interface}
\label{techspecs/floppy:classe-e-interface}
Precisamos ser capazes de converter para a representação interna os
formatos de dados contidos no disquete. Isso é feito através de classes
derivadas de \emph{floppy\_image\_format\_t}. A interface a ser implementada
deve conter:
\begin{itemize}
\item {} 
\textbf{name()} fornece um nome abreviado ao formato no disco

\item {} 
\textbf{description()} fornece uma breve descrição do formato

\item {} 
\textbf{extensions()} fornece uma lista separada por vírgula das extensões
dos nomes de arquivos encontrados para esse formato

\item {} 
\textbf{supports\_save()} retorna verdadeiro se houver compatibilidade com o
formato externo

\item {} 
\textbf{identify(file, form factor)} retorna uma pontuação entre 0-100 para
o arquivo que for daquele formato:
\begin{itemize}
\item {} 
\textbf{0}       = esse formato não

\item {} 
\textbf{100}     = provavelmente esse formato

\item {} 
\textbf{50}      = formato identificado apenas pelo tamanho do arquivo

\end{itemize}

\item {} 
\textbf{load(file, form factor, floppy\_image)} carrega uma imagem e a
converte para a representação interna

\item {} 
\textbf{save(file, floppy\_image)} (se implementado) convertido da
representação interna e salva em uma imagem

\end{itemize}

Todos estes métodos são previstos para serem sem estado.


\subsubsection{Métodos auxiliares de conversão}
\label{techspecs/floppy:metodos-auxiliares-de-conversao}
Vários métodos são fornecidos para simplificar a gravação das classes do
conversor.


\subsubsection{Métodos de conversão orientados à leitura}
\label{techspecs/floppy:metodos-de-conversao-orientados-a-leitura}
\begin{DUlineblock}{0em}
\item[] \textbf{generate\_track\_from\_bitstream(track number,}
\item[]
\begin{DUlineblock}{\DUlineblockindent}
\item[] \textbf{head number,}
\item[] \textbf{UINT8 *cell stream,}
\item[] \textbf{int cell count,}
\item[] \textbf{floppy image)}
\item[] 
\end{DUlineblock}
\end{DUlineblock}
\begin{quote}

Obtém um fluxo de tipos de células (0/1), primeiro o MSB, converte-o
para o formato interno e armazena-o na devida trilha e cabeça de uma
determinada imagem.
\end{quote}

\begin{DUlineblock}{0em}
\item[] \textbf{generate\_track\_from\_levels(track number,}
\item[]
\begin{DUlineblock}{\DUlineblockindent}
\item[] \textbf{head number,}
\item[] \textbf{UINT32 *cell levels,}
\item[] \textbf{int cell count,}
\item[] \textbf{splice position,}
\item[] \textbf{floppy image)}
\end{DUlineblock}
\end{DUlineblock}
\begin{quote}

Pega uma variante do formato interno onde cada valor representa uma
célula, a parte da posição dos valores é o tamanho da célula e a parte
do nível é MG\_0, MG\_1 para os tipos de células normais, MG\_N, MG\_D
para as células não formatadas ou danificadas e MG\_W para os bits mais
fracos no estilo \emph{Dungeon-Master}.
Converte para o formato interno. Os tamanhos são normalizados para que
eles tenham uma volta completa no total.
\end{quote}

\begin{DUlineblock}{0em}
\item[] \textbf{normalize\_times(UINT32 *levels,}
\item[]
\begin{DUlineblock}{\DUlineblockindent}
\item[] \textbf{int level\_count)}
\end{DUlineblock}
\end{DUlineblock}
\begin{quote}

Pega um buffer de formato interno onde a parte da posição representa o
ângulo até a próxima mudança e o transforma em um fluxo normal de
posição, primeiro garantindo que o tamanho total seja normalizado para
uma volta completa.
\end{quote}


\subsubsection{Métodos de conversão orientados a gravação}
\label{techspecs/floppy:metodos-de-conversao-orientados-a-gravacao}
\begin{DUlineblock}{0em}
\item[] \textbf{generate\_bitstream\_from\_track(track number,}
\item[]
\begin{DUlineblock}{\DUlineblockindent}
\item[] \textbf{head number,}
\item[] \textbf{base cell size},
\item[] \textbf{UINT8 *cell stream,}
\item[] \textbf{int \&cell\_stream\_size,}
\item[] \textbf{floppy image)}
\end{DUlineblock}
\end{DUlineblock}
\begin{quote}

Extrai um fluxo da célula 0/1 do formato interno usando uma
configuração PPL com um tamanho de célula inicial definida para
`\emph{base cell size}` e uma tolerância de +/- 25\%.
\end{quote}

\begin{DUlineblock}{0em}
\item[] \textbf{struct desc\_xs \{ int track, head, size; const UINT8 *data \}}
\item[] \textbf{extract\_sectors\_from\_bitstream\_mfm\_pc(...)}
\item[] \textbf{extract\_sectors\_from\_bitstream\_fm\_pc(const UINT8 *cell stream,}
\item[]
\begin{DUlineblock}{\DUlineblockindent}
\item[] \textbf{int cell\_stream\_size,}
\item[] \textbf{desc\_xs *sectors,}
\item[] \textbf{UINT8 *sectdata,}
\item[] \textbf{int sectdata\_size)}
\end{DUlineblock}
\end{DUlineblock}
\begin{quote}

Extrai os setores padrão MFM ou FM de um fluxo de células regeneradas.
Os setores devem apontar para uma matriz com 256 ofdesc\_xs.

Um setor existente é reconhecível por ter -\textgreater{} dados não nulos.
Os dados do setor são escritos em sectdata até os bytes sectdata\_size.
\end{quote}

\begin{DUlineblock}{0em}
\item[] \textbf{get\_geometry\_mfm\_pc(...)}
\item[] \textbf{get\_geometry\_fm\_pc(floppy image,}
\item[]
\begin{DUlineblock}{\DUlineblockindent}
\item[] \textbf{base cell size,}
\item[] \textbf{int \&track\_count,}
\item[] \textbf{int \&head\_count,}
\item[] \textbf{int \&sector\_count)}
\end{DUlineblock}
\end{DUlineblock}
\begin{quote}

Extrai a geometria (cabeças, trilhas, setores) de uma imagem de
disquete tipo pc, verificando a trilha 20.
\end{quote}

\begin{DUlineblock}{0em}
\item[] \textbf{get\_track\_data\_mfm\_pc(...)}
\item[] \textbf{get\_track\_data\_fm\_pc(track number,}
\item[]
\begin{DUlineblock}{\DUlineblockindent}
\item[] \textbf{head number,}
\item[] \textbf{floppy image,}
\item[] \textbf{base cell size,}
\item[] \textbf{sector size,}
\item[] \textbf{sector count,}
\item[] \textbf{UINT8 *sector data)}
\end{DUlineblock}
\end{DUlineblock}
\begin{quote}

Extrai o que você obteria ao ler na ordem dos setores `\emph{sector size}`
do número 1 para o contador do setor e registra o resultado no setor
de dados.
\end{quote}
\clearpage

\subsection{Unidade de Disquete}
\label{techspecs/floppy:id3}
A classe \emph{floppy\_image\_interface} simula a unidade de disquete.
Isso inclui uma série de sinais de controle, leitura e escrita.
Os sinais de controle de alterações devem ser sincronizadas, disparo
do temporizador para assegurar que a hora atual seja a mesma para
todos os dispositivos, por exemplo.


\subsubsection{Sinais de controle}
\label{techspecs/floppy:sinais-de-controle}
Devido à maneira de como estão ligados na CPUs (diretamente numa porta
I/O por exemplo), o controlador de sinais trabalha com valores físicos
ao invés de lógicos. Em geral, o 0 significa ativo e 1 inativo.
Alguns sinais têm também um retorno de chamada associado a eles quando
mudam.

\textbf{mon\_w(state) / mon\_r()}
\begin{quote}

Sinal para ligar o motor, gira no 0
\end{quote}

\textbf{idx\_r() / setup\_index\_pulse\_cb(cb)}
\begin{quote}

Sinal de indexação, vai a 0 no início da pista por aproximadamente
2ms. O retorno de chamada é sincronizado. Só acontece quando um disco
está em funcionamento e o motor está funcionando.
\end{quote}

\textbf{ready\_r() / setup\_ready\_cb(cb)}
\begin{quote}

Sinal de pronto (\emph{Ready}), vai a 1 quando o disco é removido ou o motor
é parado. Vai a 0 depois de dois pulsos indexados.
\end{quote}

\textbf{wpt\_r() / setup\_wpt\_cb(cb)}
\begin{quote}

Sinal de proteção contra gravação (1 = somente leitura).
O retorno de chamada não é sincronizado.
\end{quote}

\textbf{dskchg\_r()}
\begin{quote}

Sinal de mudança de disco, vai a 1 quando um disco é alterado, vai a 0
para a mudança de trilha.
\end{quote}

\textbf{dir\_w(dir)}
\begin{quote}

Seleciona a direção do passo da trilha (1 = fora = diminui o número da
trilha).
\end{quote}

\textbf{stp\_w(state)}
\begin{quote}

Sinal de passo, move-se por uma trilha na transição 1-\textgreater{}0.
\end{quote}

\textbf{trk00\_r()}
\begin{quote}

Sensor de trilha 0, retorna 0 quando estiver na trilha 0
\end{quote}

\textbf{ss\_w(ss) / ss\_r()}
\begin{quote}

Seleciona um lado
\end{quote}


\subsection{Interface de leitura e gravação}
\label{techspecs/floppy:interface-de-leitura-e-gravacao}
A interface de leitura e gravação é projetada para trabalhar de forma
assíncrona, de maneira independentemente da hora atual, por exemplo.
\clearpage

\section{O novo subsistema SCSI}
\label{techspecs/nscsi:o-novo-subsistema-scsi}\label{techspecs/nscsi::doc}

\subsection{Introdução}
\label{techspecs/nscsi:introducao}
O subsistema \textbf{nscsi} foi criado para permitir que uma implementação
fique o mais próximo possível do hardware físico real, (na esperança de)
facilitar a implementação de novos CIs controladores a partir de
documentações.


\subsection{Estrutura global}
\label{techspecs/nscsi:estrutura-global}
O SCSI paralelo é construído em torno de um barramento simétrico ao qual
vários dispositivos estão conectados. O barramento é composto de 9
linhas de controle (no momento, as versões posteriores do SCSI podem ter
mais) e até 32 linhas de dados (mas os chips atualmente implementados
suportam apenas 8). Todas as linhas são coletores abertos, o que
significa que um ou vários chips conectam a linha ao terra e a linha,
óbvio, vai para o terra ou nenhum CI conduz nada e a linha continua no
Vcc. Além disso, o barramento usa um lógica invertida, significa que o
sinal ao ser aterrado equivale a 1.
Os controladores SCSI tradicionalmente funcionam em níveis lógicos e não
físicos, então o subsistema nscsi também funciona em níveis lógicos,
assim todas as suas saídas para os dispositivos são lógicos.

Estruturalmente, a implementação é feita em torno de duas classes
principais:
\begin{itemize}
\item {} 
\textbf{nscsi\_bus\_devices} representa o barramento

\item {} 
\textbf{nscsi\_device} representa um dispositivo individual

\end{itemize}

Um dispositivo só se comunica com o barramento e o barramento cuida da
manipulação transparente da descoberta e a comunicação do dispositivo.
Além disso a classe \textbf{nscsi\_full\_device} propõe um dispositivo SCSI com
o protocolo SCSI implementado, facilitando a criação de dispositivos
SCSI genéricos como se fossem discos rígidos ou leitores de CD-ROM.


\subsection{Conectando um barramento SCSI em um driver}
\label{techspecs/nscsi:conectando-um-barramento-scsi-em-um-driver}
O subsistema nscsi aproveita as interfaces de slot e a nomenclatura do
dispositivo para permitir uma implementação e configuração de barramento
de forma simples.

Primeiro você precisa criar uma lista de dispositivos aceitáveis para
conectar ao barramento. Isso geralmente inclui \textbf{cdrom}, \textbf{hasrdisk} e
o CI do controlador.
Por exemplo:

\begin{DUlineblock}{0em}
\item[] 
\item[] static SLOT\_INTERFACE\_START( next\_scsi\_devices )
\item[]
\begin{DUlineblock}{\DUlineblockindent}
\item[] SLOT\_INTERFACE(``cdrom'', NSCSI\_CDROM)
\item[] SLOT\_INTERFACE(``harddisk'', NSCSI\_HARDDISK)
\item[] SLOT\_INTERFACE\_INTERNAL(``ncr5390'', NCR5390)
\end{DUlineblock}
\item[] SLOT\_INTERFACE\_END
\end{DUlineblock}

A interface \textbf{\_INTERNAL} indica um dispositivo que não é selecionável
pelo usuário, o que é útil para o controlador.

Então na configuração da máquina (ou em uma configuração de fragmento)
você precisa primeiro adicionar o barramento e em seguida os
dispositivos (potenciais) como dispositivos de sub-dispositivos do
barramento com o SCSI ID como seu nome. Você pode usar como exemplo:

\begin{DUlineblock}{0em}
\item[] 
\item[]
\begin{DUlineblock}{\DUlineblockindent}
\item[] MCFG\_NSCSI\_BUS\_ADD(``scsibus'')
\item[] MCFG\_NSCSI\_ADD(``scsibus:0'', next\_scsi\_devices, ``cdrom'', 0, 0, 0, false)
\item[] MCFG\_NSCSI\_ADD(``scsibus:1'', next\_scsi\_devices, ``harddisk'', 0, 0, 0, false)
\item[] MCFG\_NSCSI\_ADD(``scsibus:2'', next\_scsi\_devices, 0, 0, 0, 0, false)
\item[] MCFG\_NSCSI\_ADD(``scsibus:3'', next\_scsi\_devices, 0, 0, 0, 0, false)
\item[] MCFG\_NSCSI\_ADD(``scsibus:4'', next\_scsi\_devices, 0, 0, 0, 0, false)
\item[] MCFG\_NSCSI\_ADD(``scsibus:5'', next\_scsi\_devices, 0, 0, 0, 0, false)
\item[] MCFG\_NSCSI\_ADD(``scsibus:6'', next\_scsi\_devices, 0, 0, 0, 0, false)
\item[] MCFG\_NSCSI\_ADD(``scsibus:7'', next\_scsi\_devices, ``ncr5390'', 0, \&next\_ncr5390\_interface, 10000000, true)
\end{DUlineblock}
\end{DUlineblock}

Essa configuração coloca como um leitor de CD-ROM padrão no SCSI ID 0 e
um disco rígido com SCSI ID 1forçando o controlador no ID 7.
Os parâmetros para adição são:
\begin{itemize}
\item {} 
device tag, composto por \sphinxcode{bus-tag:scsi-id}

\item {} 
uma lista com os dispositivos aceitos

\item {} 
um dispositivo conforme disposto na lista, se um já não estiver por padrão

\item {} 
a configuração do dispositivo de entrada, caso haja (e geralmente não há)

\item {} 
a estrutura de configuração do dispositivo, geralmente usado apenas para o controlador

\item {} 
a frequência, geralmente usado apenas pelo controlador

\end{itemize}

O nome completo do dispositivo, para fins de mapeamento seria
\sphinxcode{bus-tag:scsi-id:device-type}, \sphinxcode{scsibus:7:ncr5390} para o nosso
controlador aqui.


\subsection{Criando um novo dispositivo SCSI usando nscsi\_device}
\label{techspecs/nscsi:criando-um-novo-dispositivo-scsi-usando-nscsi-device}
A classe base ``\textbf{nscsi\_device}'' deve ser usado para os CIs do
controlador SCSI. A classe fornece três variáveis e um método:
\begin{itemize}
\item {} 
A primeira variável, \textbf{scsi\_bus}, é um ponteiro para o
\textbf{nscsi\_bus\_device}.

\item {} 
A segunda, \textbf{scsi\_refid}, é uma referência opaca para passar algumas
operação ao barramento.

\item {} 
Finalmente, o \textbf{scsi\_id} dá um SCSI ID individual por tag de
dispositivo. É escrito uma vez na inicialização e nunca é lido ou
gravado depois, o dispositivo pode fazer o que quiser com o valor ou a
variável.

\item {} 
O método virtual \textbf{scsi\_ctrl\_changed} é chamado quando for assistir
as mudanças das linhas de controle. É através do barramento que são
definidas quais as linhas serão monitoradas.

\end{itemize}

Para acessar as linhas existe uma proposta com cinco métodos:
\begin{itemize}
\item {} 
\textbf{ctrl\_r()} e \textbf{data\_r()} são os métodos de leitura.
Os controles de bits são definidos dentro do enum \footnote[1]{\sphinxAtStartFootnote%
Assumo que o termo abreviado ``\emph{enum}'' seja um enumerador.
(Nota do tradutor)
} \textbf{s\_*} de
\textbf{nscsi\_device}.

\item {} 
Os três bits abaixo (\textbf{INP}, \textbf{CTL} e \textbf{MSG}) são configurações
para que o ``masking'' com 7(\textbf{S\_PHASE\_MASK}) retorne os números para
as fases, que também estão disponíveis com o enum \textbf{S\_PHASE\_*}.

\item {} 
A escrita nas linhas de dados é feito com \sphinxcode{data\_w(scsi\_refid,
value)}.

\item {} 
A escrita nas linhs de controle é feito com
\sphinxcode{ctrl\_w(scsi\_refid, value, mask-of-lines-to-change)}.
Para alterar todas as linhas de controle com uma chamada use a máscara
\textbf{S\_ALL}.

\end{itemize}

Claro que o que é lido é a lógica de tudo o que é conduzido por todos os
dispositivos.
\begin{itemize}
\item {} 
Finalmente, o método
\sphinxcode{ctrl\_wait\_w(scsi\_id, value, mask of wait lines to change)}, permite
selecionar quais as linhas de controle que são monitoradas. A máscara
de monitoramento é individual para cada dispositivo, o
\textbf{scsi\_ctrl\_changed} é chamado sempre que uma linha de controle da
máscara for alterado, devido a uma ação de um outro dispositivo (não
em si, para evitar uma recursão irritante e um tanto inútil).

\end{itemize}

A implementação do controle é apenas uma questão de seguir o estado
descritivos das máquinas, pelo menos se eles estiverem disponíveis.
A única parte não descrita é a arbitragem/seleção que está documentada
na norma do SCSI. Para um iniciador (o que é que o controlador sempre é
essencialmente), funciona assim:
\begin{itemize}
\item {} 
espera o barramento ficar ocioso

\item {} 
garante em qual número o seu \textbf{scsi\_id} está na linha de dados
(\sphinxcode{1 \textless{}\textless{} scsi\_id})

\item {} 
espera o tempo de atribuição

\item {} 
verifica se as linhas de dados ativas com o número maior é a sua
\begin{itemize}
\item {} 
caso não seja, a atribuição é perdida, pare a condução de tudo e
reinicie

\end{itemize}

\item {} 
garante a linha selecionada (nesse ponto o barramento é seu)

\item {} 
espera um pouco

\item {} 
mantém a sua linha de dados garantida, garante que o número da linha
de dados é o SCSI ID de destino

\item {} 
espera um pouco

\item {} 
garante que caso a linha \textbf{atn} seja necessária, retorne como sinal
ocupado

\item {} 
espera que o sinal ocupado seja garantido ou que acabe o tempo limite
(timeout)
\begin{itemize}
\item {} 
O tempo limite significa que ninguém está respondendo naquele ID,
desocupe tudo e pare

\end{itemize}

\item {} 
aguarda por um curto período até o \textbf{de-skewing}

\item {} 
desocupa o barramento de dados e seleciona uma linha

\item {} 
espera mais um pouco

\end{itemize}

E tudo pronto, você está conectado com o dispositivo de destino até que
o alvo desocupe a linha ocupada, seja porque você pediu ou apenas para
te aborrecer. O \textbf{de-assert} (desocupar) é chamado de desconexão.

O \textbf{ncr5390} é um exemplo de como usar um estado de máquina com dois
níveis de estado para lidar com todos os eventos.


\subsection{Criando um novo dispositivo SCSI usando o \textbf{nscsi\_full\_device}}
\label{techspecs/nscsi:criando-um-novo-dispositivo-scsi-usando-o-nscsi-full-device}
A classe base ``\textbf{nscsi\_full\_device}'' é usada para criar dispositivos
SCSI HLE-d destinados para uso genérico, como discos rígidos, CD-ROMs,
scanners talvez, etc. A classe fornece a manipulação de protocolo SCSI,
deixando somente a manipulação de comando e (opcionalmente) o tratamento
de mensagens para a implementação.

A classe atualmente suporta apenas dispositivos de destino.

O primeiro método para implementar é \textbf{scsi\_command()}. Esse método é
chamado quando um comando chegar por completo. O comando está disponível
em \textbf{scsi\_cmdbuf{[}{]}} e seu comprimento fica em \textbf{scsi\_cmdsize} (porém o
comprimento em geral é inútil ao primeiro byte de comando dado).
A matriz de 4096-bytes \textbf{scsi\_cmdbuf} pode então ser modificada
livremente.

Em \textbf{scsi\_command()}, o dispositivo pode lidar com o comando ou
passá-lo com \textbf{nscsi\_full\_device::scsi\_command()}.

Para lidar com o comando, vários métodos estão disponíveis:
\begin{itemize}
\item {} 
\textbf{get\_lun(lua set in command)} lhe dará o LUN a ser trabalhado (o
\textbf{in-command} um pode ser substituído por um nível de mensagem um).

\item {} 
\textbf{bad\_lun()} respostas para o host que o LUN específico não tiver
suporte.

\item {} 
\textbf{scsi\_data\_in(buffer id, size)} envia bytes com tamanho vindo da
memória intermédia \textbf{buffer-id}

\item {} 
\textbf{scsi\_data\_in(buffer id, size)} recebe bytes com o tamanho para a
memória intermédia \textbf{buffer-id}

\item {} 
\textbf{scsi\_status\_complete(status)} termina o comando com um determinado
status.

\item {} 
\textbf{sense(deferred, key)} prepara o senso da memória intermédia para um
comando subsequente de solicitação, que é útil ao retornar um status
de verificação da condição.

\end{itemize}

Os comandos \textbf{scsi\_data\_*} e \textbf{scsi\_status\_complete} são
enfileirados, o manipulador de comandos deve chamá-los todos sem
tempo de espera.

O \textbf{buffer-id} identifica a memória intermediária. 0 também conhecido
como \textbf{SBUF\_MAIN}, direciona a memória intermédia \textbf{scsi\_cmdbuf}.
Os outros valores aceitáveis são 2 ou mais. 2+ ids são manipulados pelo
método \textbf{scsi\_get\_data} para leitura e \textbf{scsi\_put\_data} para gravação.

\textbf{UINT8 device::scsi\_get\_data(int id, int pos)} deve retornar o id da
posição do byte na memória intermediária, chamando em
\textbf{nscsi\_full\_device} por \emph{id \textless{} 2}.

\textbf{void device::scsi\_put\_data(int id, int pos, UINT8 data)} deve
escrever o id da posição do byte na memória intermediária, chamando em
\textbf{nscsi\_full\_device} por \emph{id \textless{} 2}.

O \textbf{scsi\_get\_data} e o \textbf{scsi\_put\_data} devem fazer as leituras e
gravações externas quando for necessário.

O dispositivo também pode sobrescrever o \textbf{scsi\_message} para lidar com
mensagens SCSI diferentes daquelas tratadas de forma genérica e também
pode substituir alguns dos tempos (mas muitos deles não são usados,
cuidado).

Para facilitar as coisas uma certa quantidade de ``\emph{enums}'' é definida:
\begin{itemize}
\item {} 
O enum \textbf{SS\_*} dá retornos de status (como \textbf{SS\_GOOD} para todos
que em condições boas).

\item {} 
O enum \textbf{SC\_*} fornece os comandos SCSI.

\item {} 
O enum \textbf{SM\_*} fornece as mensagens SCSI, com exceção do
identificador (que é \sphinxcode{80-ff}, realmente não se encaixa em um enum).

\end{itemize}


\subsection{O que falta no \textbf{scsi\_full\_device}}
\label{techspecs/nscsi:o-que-falta-no-scsi-full-device}\begin{itemize}
\item {} 
\textbf{Suporte ao iniciador} Nesse momento, não temos nenhum dispositivo
iniciador para o HLE.

\item {} 
\textbf{Delays} Um comando \emph{scsi\_delay} ajudaria a dar tempos (\emph{timings})
mais realistas, particularmente ao leitor de CD-ROM.

\item {} 
\textbf{Operações desconectadas} Primeiro exigiria atrasos e além disso,
um sistema operacional emulado que pudesse manipulá-lo.

\item {} 
\textbf{Operação ampla em 16-bits} Precisa de um SO e de um iniciador que
possam manipulá-lo.

\end{itemize}


\subsection{O que falta no ncr5390 (e provavelmente em outros controladores futuros)}
\label{techspecs/nscsi:o-que-falta-no-ncr5390-e-provavelmente-em-outros-controladores-futuros}\begin{itemize}
\item {} 
\textbf{A detecção de um barramento livre} No momento, o barramento é
considerado livre caso o controlador não esteja ocupado, o que é
verdade. Isso pode mudar uma vez que a operação de desconexão esteja
em ação.

\item {} 
\textbf{Comandos alvo} Ainda não são emulados ainda (vs. HLE).

\end{itemize}
\clearpage

\section{Usando Scripts LUA com o MAME}
\label{techspecs/luaengine:usando-scripts-lua-com-o-mame}\label{techspecs/luaengine::doc}

\subsection{Introdução}
\label{techspecs/luaengine:introducao}
Agora é possível controlar o MAME externamente usando scripts LUA \footnote[1]{\sphinxAtStartFootnote%
Acesse o \href{https://www.lua.org/portugues.html}{site do projeto LUA} para maiores informações.
(Nota do tradutor)
}.
Essa funcionalidade apareceu inicialmente na versão 0.148, quando o
\sphinxcode{luaengine} foi implementado. Hoje em dia, a interface LUA é rica o
suficiente para deixar você inspecionar e manipular os estados dos
dispositivos, acesso aos registros do CPU, ler e escrever a memória,
desenhar um painel customizado na tela.

Internamente, o MAME faz o uso intensivo de \sphinxcode{luabridge} para
implementar esse recurso: a ideia é expor muitos dos recursos internos
de forma mais transparente possível.

Aqui fica o alerta: A API LUA ainda não é considerada estável havendo a
possibilidade de ser alterada sem nenhum aviso prévio. No entanto,
podemos demonstrar metodologias para que você saiba qual a versão do
API está rodando e quais os objetos são os mais usados durante a
execução.


\subsection{Características}
\label{techspecs/luaengine:caracteristicas}
Pelo fato da API estar incompleta, abaixo uma lista parcial de recursos
disponíveis atualmente com os scripts LUA:
\begin{itemize}
\item {} 
metadata de máquina (versão do app, rom atual, descrição da rom)

\item {} 
controle da máquina (iniciar, pausar, resetar, parar)

\item {} 
ganchos da máquina (pinta em cima do frame e nos eventos de usuário)

\item {} 
introspeção dos dispositivos (enumeração da árvore dos dispositivos, memória e registros)

\item {} 
introspeção das telas (listagem de telas, descritivos, contagem de quadros)

\item {} 
desenho de um painel (HUD) na tela (texto, linhas, caixas em múltiplas telas)

\item {} 
leitura/escrita de memória (8/16/32/64 bits, signed e unsigned)

\item {} 
controle de estados e registros (enumeração dos estados, obter e definir)

\end{itemize}


\subsection{Uso}
\label{techspecs/luaengine:uso}
O MAME suporta o carregamento de scripts LUA (\textgreater{}= 5.3), seja ele escrito
no console interativo ou se for carregado como um arquivo externo. Para
usar o console, rode o mame usando o comando \textbf{-console}, você será
apresentado a um prompt de comando com um \sphinxcode{\textgreater{}}, onde será possível
redigir o seu script.

Use o comando \textbf{-autoboot\_script} para carregar um script. Por
predefinição o carregamento do script pode ser atrasado em alguns poucos
segundos, essa predefinição pode ser substituída com o comando
\textbf{-autoboot\_delay}.

Para controlar a execução do seu código, você pode usar uma abordagem do
tipo \emph{loop-bases} ou \emph{event-based}. Não encorajamos o uso deste último
devido ao alto consumo de recursos e faz a continuidade de controle
desnecessariamente complicada. Em vez disso, sugerimos o registro de
ganchos personalizados que poderão ser invocados em um evento específico
(como a cada renderização de quadro por exemplo).


\subsection{Demonstração passo a passo}
\label{techspecs/luaengine:demonstracao-passo-a-passo}
Rode o MAME num terminal para ter acesso ao console Lua:

\begin{Verbatim}[commandchars=\\\{\}]
\PYGZdl{} mame \PYGZhy{}console YOUR\PYGZus{}ROM
     \PYGZus{}/      \PYGZus{}/    \PYGZus{}/\PYGZus{}/    \PYGZus{}/      \PYGZus{}/  \PYGZus{}/\PYGZus{}/\PYGZus{}/\PYGZus{}/
   \PYGZus{}/\PYGZus{}/  \PYGZus{}/\PYGZus{}/  \PYGZus{}/    \PYGZus{}/  \PYGZus{}/\PYGZus{}/  \PYGZus{}/\PYGZus{}/  \PYGZus{}/
  \PYGZus{}/  \PYGZus{}/  \PYGZus{}/  \PYGZus{}/\PYGZus{}/\PYGZus{}/\PYGZus{}/  \PYGZus{}/  \PYGZus{}/  \PYGZus{}/  \PYGZus{}/\PYGZus{}/\PYGZus{}/
 \PYGZus{}/      \PYGZus{}/  \PYGZus{}/    \PYGZus{}/  \PYGZus{}/      \PYGZus{}/  \PYGZus{}/
\PYGZus{}/      \PYGZus{}/  \PYGZus{}/    \PYGZus{}/  \PYGZus{}/      \PYGZus{}/  \PYGZus{}/\PYGZus{}/\PYGZus{}/\PYGZus{}/
mame v0.195
Copyright (C) Nicola Salmoria and the MAME team

Lua 5.3
Copyright (C) Lua.org, PUC\PYGZhy{}Rio

[MAME]\PYGZgt{}
\end{Verbatim}

Neste ponto, o seu jogo provavelmente pode estar sendo executado,
use o comando abaixo para pausá-lo:

\begin{Verbatim}[commandchars=\\\{\}]
\PYG{p}{[}\PYG{n}{MAME}\PYG{p}{]}\PYG{o}{\PYGZgt{}} \PYG{n}{emu}\PYG{o}{.}\PYG{n}{pause}\PYG{p}{(}\PYG{p}{)}
\PYG{p}{[}\PYG{n}{MAME}\PYG{p}{]}\PYG{o}{\PYGZgt{}}
\end{Verbatim}

Mesmo sem qualquer tipo de retorno no console, você deve ter notado que
o jogo parou. Em geral, os comandos não retornam informação de
confirmação o terminal retorna mensagens de erro apenas.

Você pode verificar durante a execução, qual a versão do MAME que
você está rodando com o comando abaixo:

\begin{Verbatim}[commandchars=\\\{\}]
\PYG{p}{[}\PYG{n}{MAME}\PYG{p}{]}\PYG{o}{\PYGZgt{}} \PYG{n+nb}{print}\PYG{p}{(}\PYG{n}{emu}\PYG{o}{.}\PYG{n}{app\PYGZus{}name}\PYG{p}{(}\PYG{p}{)} \PYG{o}{.}\PYG{o}{.} \PYG{l+s+s2}{\PYGZdq{}}\PYG{l+s+s2}{ }\PYG{l+s+s2}{\PYGZdq{}} \PYG{o}{.}\PYG{o}{.} \PYG{n}{emu}\PYG{o}{.}\PYG{n}{app\PYGZus{}version}\PYG{p}{(}\PYG{p}{)}\PYG{p}{)}
\PYG{n}{mame} \PYG{l+m+mf}{0.195}
\end{Verbatim}

Nós agora começaremos a explorar os métodos relacionadas à tela.
Primeiro, vamos enumerar as telas disponíveis:

\begin{Verbatim}[commandchars=\\\{\}]
\PYG{p}{[}\PYG{n}{MAME}\PYG{p}{]}\PYG{o}{\PYGZgt{}} \PYG{k}{for} \PYG{n}{i}\PYG{p}{,}\PYG{n}{v} \PYG{o+ow}{in} \PYG{n}{pairs}\PYG{p}{(}\PYG{n}{manager}\PYG{p}{:}\PYG{n}{machine}\PYG{p}{(}\PYG{p}{)}\PYG{o}{.}\PYG{n}{screens}\PYG{p}{)} \PYG{n}{do} \PYG{n+nb}{print}\PYG{p}{(}\PYG{n}{i}\PYG{p}{)} \PYG{n}{end}
\PYG{p}{:}\PYG{n}{screen}
\end{Verbatim}

\textbf{manager:machine()} este é o objeto raiz da sua máquina atualmente em
execução: será usada com bastante frequência. \textbf{screens} é uma tabela
com todas as telas disponíveis; a maioria das máquinas tem apenas uma
tela principal. No nosso caso, a tela principal e única é marcada como
\textbf{:screen}, e podemos inspecioná-la mais a fundo:

\begin{Verbatim}[commandchars=\\\{\}]
\PYG{p}{[}\PYG{n}{MAME}\PYG{p}{]}\PYG{o}{\PYGZgt{}} \PYG{o}{\PYGZhy{}}\PYG{o}{\PYGZhy{}} \PYG{n}{vamos} \PYG{n}{definir} \PYG{n}{um} \PYG{n}{atalho} \PYG{n}{para} \PYG{n}{a} \PYG{n}{tela} \PYG{n}{principal}
\PYG{p}{[}\PYG{n}{MAME}\PYG{p}{]}\PYG{o}{\PYGZgt{}} \PYG{n}{s} \PYG{o}{=} \PYG{n}{manager}\PYG{p}{:}\PYG{n}{machine}\PYG{p}{(}\PYG{p}{)}\PYG{o}{.}\PYG{n}{screens}\PYG{p}{[}\PYG{l+s+s2}{\PYGZdq{}}\PYG{l+s+s2}{:screen}\PYG{l+s+s2}{\PYGZdq{}}\PYG{p}{]}
\PYG{p}{[}\PYG{n}{MAME}\PYG{p}{]}\PYG{o}{\PYGZgt{}} \PYG{n+nb}{print}\PYG{p}{(}\PYG{n}{s}\PYG{p}{:}\PYG{n}{width}\PYG{p}{(}\PYG{p}{)} \PYG{o}{.}\PYG{o}{.} \PYG{l+s+s2}{\PYGZdq{}}\PYG{l+s+s2}{x}\PYG{l+s+s2}{\PYGZdq{}} \PYG{o}{.}\PYG{o}{.} \PYG{n}{s}\PYG{p}{:}\PYG{n}{height}\PYG{p}{(}\PYG{p}{)}\PYG{p}{)}
\PYG{l+m+mi}{320}\PYG{n}{x224}
\end{Verbatim}

Temos diferentes métodos para desenhar um painel (HUD) na tela composta
de linhas, caixas e textos:

\begin{Verbatim}[commandchars=\\\{\}]
\PYG{p}{[}\PYG{n}{MAME}\PYG{p}{]}\PYG{o}{\PYGZgt{}} \PYG{o}{\PYGZhy{}}\PYG{o}{\PYGZhy{}} \PYG{n}{definimos} \PYG{n}{a} \PYG{n}{função} \PYG{n}{para} \PYG{n}{desenhar} \PYG{n}{a} \PYG{n}{interface} \PYG{n}{e} \PYG{n}{a} \PYG{n}{chamamos}
\PYG{p}{[}\PYG{n}{MAME}\PYG{p}{]}\PYG{o}{\PYGZgt{}} \PYG{n}{function} \PYG{n}{draw\PYGZus{}hud}\PYG{p}{(}\PYG{p}{)}
\PYG{p}{[}\PYG{n}{MAME}\PYG{p}{]}\PYG{o}{\PYGZgt{}\PYGZgt{}} \PYG{n}{s}\PYG{p}{:}\PYG{n}{draw\PYGZus{}text}\PYG{p}{(}\PYG{l+m+mi}{40}\PYG{p}{,} \PYG{l+m+mi}{40}\PYG{p}{,} \PYG{l+s+s2}{\PYGZdq{}}\PYG{l+s+s2}{foo}\PYG{l+s+s2}{\PYGZdq{}}\PYG{p}{)}\PYG{p}{;} \PYG{o}{\PYGZhy{}}\PYG{o}{\PYGZhy{}} \PYG{p}{(}\PYG{n}{x0}\PYG{p}{,} \PYG{n}{y0}\PYG{p}{,} \PYG{n}{msg}\PYG{p}{)}
\PYG{p}{[}\PYG{n}{MAME}\PYG{p}{]}\PYG{o}{\PYGZgt{}\PYGZgt{}} \PYG{n}{s}\PYG{p}{:}\PYG{n}{draw\PYGZus{}box}\PYG{p}{(}\PYG{l+m+mi}{20}\PYG{p}{,} \PYG{l+m+mi}{20}\PYG{p}{,} \PYG{l+m+mi}{80}\PYG{p}{,} \PYG{l+m+mi}{80}\PYG{p}{,} \PYG{l+m+mi}{0}\PYG{p}{,} \PYG{l+m+mh}{0xff00ffff}\PYG{p}{)}\PYG{p}{;} \PYG{o}{\PYGZhy{}}\PYG{o}{\PYGZhy{}} \PYG{p}{(}\PYG{n}{x0}\PYG{p}{,} \PYG{n}{y0}\PYG{p}{,} \PYG{n}{x1}\PYG{p}{,} \PYG{n}{y1}\PYG{p}{,} \PYG{n}{fill}\PYG{o}{\PYGZhy{}}\PYG{n}{color}\PYG{p}{,} \PYG{n}{line}\PYG{o}{\PYGZhy{}}\PYG{n}{color}\PYG{p}{)}
\PYG{p}{[}\PYG{n}{MAME}\PYG{p}{]}\PYG{o}{\PYGZgt{}\PYGZgt{}} \PYG{n}{s}\PYG{p}{:}\PYG{n}{draw\PYGZus{}line}\PYG{p}{(}\PYG{l+m+mi}{20}\PYG{p}{,} \PYG{l+m+mi}{20}\PYG{p}{,} \PYG{l+m+mi}{80}\PYG{p}{,} \PYG{l+m+mi}{80}\PYG{p}{,} \PYG{l+m+mh}{0xff00ffff}\PYG{p}{)}\PYG{p}{;} \PYG{o}{\PYGZhy{}}\PYG{o}{\PYGZhy{}} \PYG{p}{(}\PYG{n}{x0}\PYG{p}{,} \PYG{n}{y0}\PYG{p}{,} \PYG{n}{x1}\PYG{p}{,} \PYG{n}{y1}\PYG{p}{,} \PYG{n}{line}\PYG{o}{\PYGZhy{}}\PYG{n}{color}\PYG{p}{)}
\PYG{p}{[}\PYG{n}{MAME}\PYG{p}{]}\PYG{o}{\PYGZgt{}\PYGZgt{}} \PYG{n}{end}
\PYG{p}{[}\PYG{n}{MAME}\PYG{p}{]}\PYG{o}{\PYGZgt{}} \PYG{n}{draw\PYGZus{}hud}\PYG{p}{(}\PYG{p}{)}\PYG{p}{;}
\end{Verbatim}

Isso desenha alguns desenhos inúteis na tela. No entanto, seu painel
desaparecerá caso não seja atualizado ao sair da pausa. Para evitar
isso, registre o gancho a ser chamado em cada quadro desenhado:

\begin{Verbatim}[commandchars=\\\{\}]
\PYG{p}{[}\PYG{n}{MAME}\PYG{p}{]}\PYG{o}{\PYGZgt{}} \PYG{n}{emu}\PYG{o}{.}\PYG{n}{register\PYGZus{}frame\PYGZus{}done}\PYG{p}{(}\PYG{n}{draw\PYGZus{}hud}\PYG{p}{,} \PYG{l+s+s2}{\PYGZdq{}}\PYG{l+s+s2}{frame}\PYG{l+s+s2}{\PYGZdq{}}\PYG{p}{)}
\end{Verbatim}

Todas as cores são no formato ARGB (32b unsigned), enquanto a origem da
tela geralmente corresponde ao canto superior esquerdo da tela (0,0).

Da mesma forma para telas, você pode inspecionar todos os dispositivos
conectados em uma máquina:

\begin{Verbatim}[commandchars=\\\{\}]
\PYG{p}{[}\PYG{n}{MAME}\PYG{p}{]}\PYG{o}{\PYGZgt{}} \PYG{k}{for} \PYG{n}{k}\PYG{p}{,}\PYG{n}{v} \PYG{o+ow}{in} \PYG{n}{pairs}\PYG{p}{(}\PYG{n}{manager}\PYG{p}{:}\PYG{n}{machine}\PYG{p}{(}\PYG{p}{)}\PYG{o}{.}\PYG{n}{devices}\PYG{p}{)} \PYG{n}{do} \PYG{n+nb}{print}\PYG{p}{(}\PYG{n}{k}\PYG{p}{)} \PYG{n}{end}
\PYG{p}{:}\PYG{n}{audiocpu}
\PYG{p}{:}\PYG{n}{maincpu}
\PYG{p}{:}\PYG{n}{saveram}
\PYG{p}{:}\PYG{n}{screen}
\PYG{p}{:}\PYG{n}{palette}
\PYG{p}{[}\PYG{o}{.}\PYG{o}{.}\PYG{o}{.}\PYG{p}{]}
\end{Verbatim}

Em alguns casos, você também pode inspecionar e manipular a memória
e o estado:

\begin{Verbatim}[commandchars=\\\{\}]
\PYG{p}{[}\PYG{n}{MAME}\PYG{p}{]}\PYG{o}{\PYGZgt{}} \PYG{n}{cpu} \PYG{o}{=} \PYG{n}{manager}\PYG{p}{:}\PYG{n}{machine}\PYG{p}{(}\PYG{p}{)}\PYG{o}{.}\PYG{n}{devices}\PYG{p}{[}\PYG{l+s+s2}{\PYGZdq{}}\PYG{l+s+s2}{:maincpu}\PYG{l+s+s2}{\PYGZdq{}}\PYG{p}{]}
\PYG{p}{[}\PYG{n}{MAME}\PYG{p}{]}\PYG{o}{\PYGZgt{}} \PYG{o}{\PYGZhy{}}\PYG{o}{\PYGZhy{}} \PYG{n}{enumera}\PYG{p}{,} \PYG{n}{lê} \PYG{n}{e} \PYG{n}{escreve} \PYG{n}{registros} \PYG{n}{de} \PYG{n}{estado}
\PYG{p}{[}\PYG{n}{MAME}\PYG{p}{]}\PYG{o}{\PYGZgt{}} \PYG{k}{for} \PYG{n}{k}\PYG{p}{,}\PYG{n}{v} \PYG{o+ow}{in} \PYG{n}{pairs}\PYG{p}{(}\PYG{n}{cpu}\PYG{o}{.}\PYG{n}{state}\PYG{p}{)} \PYG{n}{do} \PYG{n+nb}{print}\PYG{p}{(}\PYG{n}{k}\PYG{p}{)} \PYG{n}{end}
\PYG{n}{D5}
\PYG{n}{SP}
\PYG{n}{A4}
\PYG{n}{A3}
\PYG{n}{D0}
\PYG{n}{PC}
\PYG{p}{[}\PYG{o}{.}\PYG{o}{.}\PYG{o}{.}\PYG{p}{]}
\PYG{p}{[}\PYG{n}{MAME}\PYG{p}{]}\PYG{o}{\PYGZgt{}} \PYG{n+nb}{print}\PYG{p}{(}\PYG{n}{cpu}\PYG{o}{.}\PYG{n}{state}\PYG{p}{[}\PYG{l+s+s2}{\PYGZdq{}}\PYG{l+s+s2}{D0}\PYG{l+s+s2}{\PYGZdq{}}\PYG{p}{]}\PYG{o}{.}\PYG{n}{value}\PYG{p}{)}
\PYG{l+m+mi}{303}
\PYG{p}{[}\PYG{n}{MAME}\PYG{p}{]}\PYG{o}{\PYGZgt{}} \PYG{n}{cpu}\PYG{o}{.}\PYG{n}{state}\PYG{p}{[}\PYG{l+s+s2}{\PYGZdq{}}\PYG{l+s+s2}{D0}\PYG{l+s+s2}{\PYGZdq{}}\PYG{p}{]}\PYG{o}{.}\PYG{n}{value} \PYG{o}{=} \PYG{l+m+mi}{255}
\PYG{p}{[}\PYG{n}{MAME}\PYG{p}{]}\PYG{o}{\PYGZgt{}} \PYG{n+nb}{print}\PYG{p}{(}\PYG{n}{cpu}\PYG{o}{.}\PYG{n}{state}\PYG{p}{[}\PYG{l+s+s2}{\PYGZdq{}}\PYG{l+s+s2}{D0}\PYG{l+s+s2}{\PYGZdq{}}\PYG{p}{]}\PYG{o}{.}\PYG{n}{value}\PYG{p}{)}
\PYG{l+m+mi}{255}
\end{Verbatim}

\begin{Verbatim}[commandchars=\\\{\}]
\PYG{p}{[}\PYG{n}{MAME}\PYG{p}{]}\PYG{o}{\PYGZgt{}} \PYG{o}{\PYGZhy{}}\PYG{o}{\PYGZhy{}} \PYG{n}{inspeciona} \PYG{n}{a} \PYG{n}{mamória}
\PYG{p}{[}\PYG{n}{MAME}\PYG{p}{]}\PYG{o}{\PYGZgt{}} \PYG{k}{for} \PYG{n}{k}\PYG{p}{,}\PYG{n}{v} \PYG{o+ow}{in} \PYG{n}{pairs}\PYG{p}{(}\PYG{n}{cpu}\PYG{o}{.}\PYG{n}{spaces}\PYG{p}{)} \PYG{n}{do} \PYG{n+nb}{print}\PYG{p}{(}\PYG{n}{k}\PYG{p}{)} \PYG{n}{end}
\PYG{n}{program}
\PYG{p}{[}\PYG{n}{MAME}\PYG{p}{]}\PYG{o}{\PYGZgt{}} \PYG{n}{mem} \PYG{o}{=} \PYG{n}{cpu}\PYG{o}{.}\PYG{n}{spaces}\PYG{p}{[}\PYG{l+s+s2}{\PYGZdq{}}\PYG{l+s+s2}{program}\PYG{l+s+s2}{\PYGZdq{}}\PYG{p}{]}
\PYG{p}{[}\PYG{n}{MAME}\PYG{p}{]}\PYG{o}{\PYGZgt{}} \PYG{n+nb}{print}\PYG{p}{(}\PYG{n}{mem}\PYG{p}{:}\PYG{n}{read\PYGZus{}i8}\PYG{p}{(}\PYG{l+m+mh}{0xC000}\PYG{p}{)}\PYG{p}{)}
\PYG{l+m+mi}{41}
\end{Verbatim}
\clearpage

\section{A implementação da nova família 6502}
\label{techspecs/m6502:a-implementacao-da-nova-familia-6502}\label{techspecs/m6502::doc}

\subsection{Introdução}
\label{techspecs/m6502:introducao}
A implementação da nova família 6502 foi criada de maneira que as suas
sub-instruções sejam observáveis de maneira precisa.
Foi projetado visando 3 coisas:
\begin{itemize}
\item {} 
cada ciclo do barramento deve acontecer no exato momento que
aconteceria em uma CPU real assim como cada acesso.

\item {} 
as instruções podem ser interrompidas a qualquer momento e depois
reiniciado deste ponto de forma transparente

\item {} 
para fins de emulação, as instruções podem ser interrompidas mesmo de
dentro de um manipulador de memória para a contenção/espera do
barramento.

\end{itemize}

O Ponto 1 foi garantido através de bi-simulação do \emph{perfect6502} a nível
de gate. O Ponto 2 foi garantido estruturalmente através de um gerador
de código que será explicado com mais detalhes na seção 8. O Ponto 2
ainda não está pronto devido a falta de suporte nos subsistemas de
memória, no entanto a seção 9 mostra como isso será tratado.


\subsection{A família 6502}
\label{techspecs/m6502:a-familia-6502}
A família do MOS 6502 tem sido grande e produtiva. Existe um grande
número de variantes, tamanhos de barramentos variados, I/O e até mesmo
opcodes. Alguns coadjuvantes (g65c816, hu6280) até existem e estão
perdidos em algum lugar dentro do código fonte do MAME. A classe
hierárquica final ficou assim:

\begin{Verbatim}[commandchars=\\\{\}]
                          \PYG{l+m+mi}{6502}
                           \PYG{o}{\textbar{}}
        \PYG{o}{+}\PYG{o}{\PYGZhy{}}\PYG{o}{\PYGZhy{}}\PYG{o}{\PYGZhy{}}\PYG{o}{\PYGZhy{}}\PYG{o}{\PYGZhy{}}\PYG{o}{\PYGZhy{}}\PYG{o}{+}\PYG{o}{\PYGZhy{}}\PYG{o}{\PYGZhy{}}\PYG{o}{\PYGZhy{}}\PYG{o}{\PYGZhy{}}\PYG{o}{\PYGZhy{}}\PYG{o}{\PYGZhy{}}\PYG{o}{\PYGZhy{}}\PYG{o}{\PYGZhy{}}\PYG{o}{+}\PYG{o}{\PYGZhy{}}\PYG{o}{\PYGZhy{}}\PYG{o}{+}\PYG{o}{\PYGZhy{}}\PYG{o}{\PYGZhy{}}\PYG{o}{+}\PYG{o}{\PYGZhy{}}\PYG{o}{\PYGZhy{}}\PYG{o}{\PYGZhy{}}\PYG{o}{\PYGZhy{}}\PYG{o}{\PYGZhy{}}\PYG{o}{\PYGZhy{}}\PYG{o}{\PYGZhy{}}\PYG{o}{+}\PYG{o}{\PYGZhy{}}\PYG{o}{\PYGZhy{}}\PYG{o}{\PYGZhy{}}\PYG{o}{\PYGZhy{}}\PYG{o}{\PYGZhy{}}\PYG{o}{\PYGZhy{}}\PYG{o}{\PYGZhy{}}\PYG{o}{+}
        \PYG{o}{\textbar{}}      \PYG{o}{\textbar{}}        \PYG{o}{\textbar{}}     \PYG{o}{\textbar{}}       \PYG{o}{\textbar{}}       \PYG{o}{\textbar{}}
      \PYG{l+m+mi}{6510}   \PYG{n}{deco16}   \PYG{l+m+mi}{6504}   \PYG{l+m+mi}{6509}   \PYG{n}{n2a03}   \PYG{l+m+mi}{65}\PYG{n}{c02}
        \PYG{o}{\textbar{}}                                     \PYG{o}{\textbar{}}
  \PYG{o}{+}\PYG{o}{\PYGZhy{}}\PYG{o}{\PYGZhy{}}\PYG{o}{\PYGZhy{}}\PYG{o}{\PYGZhy{}}\PYG{o}{\PYGZhy{}}\PYG{o}{+}\PYG{o}{\PYGZhy{}}\PYG{o}{\PYGZhy{}}\PYG{o}{\PYGZhy{}}\PYG{o}{\PYGZhy{}}\PYG{o}{\PYGZhy{}}\PYG{o}{+}                            \PYG{n}{r65c02}
  \PYG{o}{\textbar{}}     \PYG{o}{\textbar{}}     \PYG{o}{\textbar{}}                               \PYG{o}{\textbar{}}
\PYG{l+m+mi}{6510}\PYG{n}{t}  \PYG{l+m+mi}{7501}  \PYG{l+m+mi}{8502}                         \PYG{o}{+}\PYG{o}{\PYGZhy{}}\PYG{o}{\PYGZhy{}}\PYG{o}{\PYGZhy{}}\PYG{o}{+}\PYG{o}{\PYGZhy{}}\PYG{o}{\PYGZhy{}}\PYG{o}{\PYGZhy{}}\PYG{o}{+}
                                          \PYG{o}{\textbar{}}       \PYG{o}{\textbar{}}
                                       \PYG{l+m+mi}{65}\PYG{n}{ce02}   \PYG{l+m+mi}{65}\PYG{n}{sc02}
                                          \PYG{o}{\textbar{}}
                                        \PYG{l+m+mi}{4510}
\end{Verbatim}

O 6510 adiciona 8 bits na porta I/O, com o 6510t, o 7501 e o 8502 são
variantes, compatíveis entre si a nível de software com uma quantidade
de pinos diferente (quantidade de I/O), processo da die (NMOS, HNMOS,
etc) e suporte a clock.

O deco16 é uma variante do Deco, com um pequeno número de instruções
adicionais ainda não compreendidas e alguns I/O.

O 6504 é uma versão reduzida de pinos e barramento de endereços.

O 6509 adiciona um suporte interno para paginação.

O n2a03 é uma variante NES com a bandeira D desativada e uma
funcionalidade de som integrada.

O 65c02 é a primeira variante CMOS com algumas instruções adicionais,
algumas correções e a maioria das instruções não documentadas se
transformaram em \emph{nops}. A variante R (\emph{Rockwell}, mas eventualmente
produzida pela WDC também dentre outras) adiciona várias instruções
\emph{bitwise} e também \emph{stp} e \emph{wai}. A variante SC, usada pelo console
portátil Lynx, parece idêntica à variante R. O `S' provavelmente indica
um processo estático de células de memória RAM, permitindo total
controle de clock \emph{DC-to-max}.

O 65ce02 é a evolução final do ISA nesta hierarquia, com instruções
adicionais, registros e remoções de muitos acessos inertes que
desacelerava o 6502 original em pelo menos 25\%. O 4510 é o 65ce02 com
suporte a MMU e GPIO integrados.


\subsection{O uso das classes}
\label{techspecs/m6502:o-uso-das-classes}
Todas as CPUs são dispositivos de CPU modernos com toda a interação
normal junto com a infraestrutura do dispositivo. Para incluir uma
destas CPUs no seu driver, você precisa incluir ``\textbf{CPU/m6502/\textless{}CPU\textgreater{}.h}''
e então fazer um \textbf{MCFG\_CPU\_ADD(``tag'', \textless{}CPU\textgreater{}, clock)}.

Os calbacks da porta I/O das variantes do 6510 são configuradas através
de:
\begin{quote}

\sphinxcode{MCFG\_\textless{}CPU\textgreater{}\_PORT\_CALLBACKS(READ8(type, read\_method), WRITE8(type, write\_method))}
\end{quote}

E as linhas das máscaras \textbf{pullup} e \textbf{floating} são fornecidas
através de:
\begin{quote}

\sphinxcode{MCFG\_\textless{}CPU\textgreater{}\_PORT\_PULLS(pullups, floating)}
\end{quote}

Para ver todos os acessos de barramento nos manipuladores de memória,
é necessário desativar os acessos através do mapa direto (ao custo de
um processamento extra de CPU, é claro) com:
\begin{quote}

\sphinxcode{MCFG\_M6502\_DISABLE\_DIRECT()}
\end{quote}

Nesse caso, o suporte à descriptografia transparente também é
desabilitada, tudo passa através de chamadas comuns de leitura/gravação
no mapa de memória. O estado da linha de sincronização é dado pelo
método da CPU \textbf{get\_sync()}, possibilitando a implementação da
descriptografia no manipulador.

A cada dispositivo executável, o método de CPU \textbf{total\_cycles()} dá o
tempo atual em ciclos desde o início da máquina do ponto de vista da
CPU. Ou em outras palavras, o que normalmente é chamado o número de
ciclo da CPU quando alguém fala sobre contenção do barramento ou a
espera do estado. A chamada é projetada para ser rápida (sem
sincronização ampla do sistema, sem apelo à \textbf{machine.time()}) e é
preciso. A quantidade de ciclos para cada acesso é exata a nível de
sub-instruções.

A linha especial do nomap 4510 é acessível usando \textbf{get\_nomap()}.

Além destes detalhes específicos, estas são classes normais de CPU.


\subsection{Estrutura geral das emulações}
\label{techspecs/m6502:estrutura-geral-das-emulacoes}
Cada variante é emulada através de 4 arquivos:
\begin{itemize}
\item {} 
\textless{}CPU\textgreater{}.h    = cabeçalho para a classe de CPU

\item {} 
\textless{}CPU\textgreater{}.c    = implementação para a maioria das classes de CPU

\item {} 
d\textless{}CPU\textgreater{}.lst = tabelas de despacho para a CPU

\item {} 
o\textless{}CPU\textgreater{}.lst = implementações opcode para a CPU

\end{itemize}

As duas últimas são opcionais. Eles são usados para gerar um arquivo
\textbf{\textless{}CPU\textgreater{}.inc} no diretório de objeto que está incluso no arquivo fonte
.c.

A classe deve incluir, no mínimo, um construtor e um enum, captando as
identificações de linha de entrada correta. Veja o \emph{m65sc02} para um
exemplo minimalista. O cabeçalho também pode incluir macros de
configuração específica (consulte o \emph{m8502}) e também a classe pode
incluir assessores específicos de memória (mais sobre estes mais tarde,
exemplo simples no \emph{m6504}).

Se a CPU tiver a sua própria tabela de expedição, a classe também deve
incluir uma declaração (mas não uma definição) de \textbf{disasm\_entries},
\textbf{do\_exec\_full} e \textbf{do\_exec\_partial}, a declaração e definição de
\textbf{disasm\_disassemble} (idêntico para todas as classes, mas refere-se a
uma matriz classe específica \textbf{disasm\_entries}) e incluí o arquivo .inc
(que fornece as definições que faltarem). Suporte para a geração também
deve ser adicionada ao CPU.mak.

Se a CPU possuir algo a mais do que seus opcodes, a sua declaração deve
ser feita por meio de uma macro, veja por exemplo o m65c02. O arquivo
.inc irá fornecer as definições.


\subsection{Tabelas de despacho}
\label{techspecs/m6502:tabelas-de-despacho}
Cada arquivo d\textless{}CPU\textgreater{}.lst é uma tabelas de despacho para a CPU. As linhas
que começam com `\#' são comentários. O arquivo deve conter 257 entrada,
sendo as primeiras 256 sendo opcodes e o 257º dever ser a instrução que
a CPU deve fazer durante um reset. Dentro do IRQ e mni do 6502 há uma
chamada ``mágica'' para o opcode ``brk'', dai a falta de descrição
especifica para eles.

As entradas entre 0 e 255 por exemplo, os opcodes devem ter uma dessas
estruturas:
\begin{itemize}
\item {} 
opcode\_addressing-mode

\item {} 
opcode\_middle\_addressing-mode

\end{itemize}

O opcode tradicionalmente é um valor com três caracteres. O modo de
endereçamento devem ser um valor de 3 cartas correspondente a um dos
DASM\_* macros no m6502.h. O Opcode e modo de endereçamento são
utilizados para gerar a tabela de desmontagem. O texto completo de
entrada é usado na descrição do arquivo de opcode, os métodos de
expedição permitem opcodes variantes por CPU que sejam aparentemente
idênticos.

Uma entrada de ''.'' era utilizável para opcodes não implementados ou
desconhecidos, pois gera códigos ''???'' na desmontagem, não é uma boa
ideia neste momento uma vez que vai realizar um \emph{infloop} numa função
\textbf{execute()} caso seja encontrado.


\subsection{Descrições de Opcode}
\label{techspecs/m6502:descricoes-de-opcode}
Cada arquivo \textbf{o\textless{}CPU\textgreater{}.lst} incluí descrições de opcodes específicas
para uma CPU. Uma descrição de opcode é uma série de linhas que começam
por uma entrada de opcode por si mesmo e seguido por uma série de linhas
recuadas com o código opcode a ser executando.
Por exemplo, o opcode asl \textless{}\emph{absolute address}\textgreater{} ficaria assim:

\begin{DUlineblock}{0em}
\item[] asl\_aba
\item[]
\begin{DUlineblock}{\DUlineblockindent}
\item[] TMP = read\_pc();
\item[] TMP = set\_h(TMP, read\_pc());
\item[] TMP2 = read(TMP);
\item[] write(TMP, TMP2);
\item[] TMP2 = do\_asl(TMP2);
\item[] write(TMP, TMP2);
\item[] prefetch();
\end{DUlineblock}
\end{DUlineblock}

A primeira parte baixa do endereço é a leitura, em seguida a parte alta
(\textbf{read\_pc} é incrementada automaticamente). Assim, agora que o
endereço está disponível o valor a ser deslocado é lido, depois
reescrito (sim, o 6502 faz isso), deslocado novamente e o resultado
final é escrito (o \textbf{do\_asl} cuida das bandeiras). A instrução termina
com um prefetch da próxima instrução, assim como todas as instruções que
não quebram a CPU \footnote[1]{\sphinxAtStartFootnote%
\emph{non-CPU-crashing instructions} no original. (Nota do tradutor)
} fazem.

As funções de acesso ao barramento são:

\noindent\begin{tabulary}{\linewidth}{|L|L|}
\hline

read(adr)
&
leitura comum
\\
\hline
read\_direct(adr)
&
lê do espaço do programa
\\
\hline
read\_pc()
&
lê no endereço do PC e incrementa
\\
\hline
read\_pc\_noinc()
&
lê no endereço do PC
\\
\hline
read\_9()
&
indexador y do depósito de leitura do 6509
\\
\hline
write(adr, val)
&
escrita comum
\\
\hline
prefetch()
&
instrução prefetch
\\
\hline
prefetch\_noirq()
&
instrução prefetch sem verificação de IRQ
\\
\hline\end{tabulary}


A contagem dos ciclos é feita pelo gerador de código que detecta através
de strings correspondentes os acessos e gera o código apropriado.
Além das funções de acesso ao barramento, uma linha especial pode ser
usada para aguardar o próximo evento (irq ou qualquer outro). o
``\textbf{eat-all-cycles;}'' numa linha fará essa espera para que só então
continue. Para o m65c02 é usado um \emph{wai\_imp} e um \emph{stp\_imp}.

Devido às restrições da geração do código, algumas regras devem ser
seguidas:
\begin{itemize}
\item {} 
no geral, fique com uma instrução ou expressão por linha

\item {} 
não deve haver efeitos colaterais nos parâmetros de uma função de
acesso ao barramento

\item {} 
a vida útil das variáveis locais não deve ultrapassar a de um acesso
ao barramento Em geral é melhor deixá-los para ajudar em métodos
auxiliares (como o \textbf{do\_asl}) que não fazem acesso ao barramento.
Note que ``TMP'' e ``TMP'' não são variáveis locais, são variáveis da
classe.

\item {} 
então uma linha única ou então as construções devem ter chaves ao
redor delas caso elas estejam chamando uma função de acesso ao
barramento

\end{itemize}

O código gerado para cada opcode são métodos da classe da CPU. Como tal
eles têm acesso completo a outros métodos da classe, variáveis, tudo.


\subsection{Interface da Memória}
\label{techspecs/m6502:interface-da-memoria}
Para uma melhor reutilização do opcode com as variantes MMU/banking, foi
criada uma subclasse de acesso à memória.
É chamado de \textbf{memory\_interface}, que declarado em um dispositivo
\emph{m6502\_device} e provê os seguintes auxiliares:

\noindent\begin{tabulary}{\linewidth}{|L|L|}
\hline

UINT8 read(UINT16 adr)
&
leitura normal
\\
\hline
UINT8 read\_sync(UINT16 adr)
&
leiura com sync ativo para opcode (primeiro byte do opcode)
\\
\hline
UINT8 read\_arg(UINT16 adr)
&
leiura com sync inativo para opcode (resto do opcode)
\\
\hline
void write(UINT16 adr, UINT8 val)
&
escrita normal
\\
\hline\end{tabulary}


\noindent\begin{tabulary}{\linewidth}{|L|L|}
\hline

UINT8 read\_9(UINT16 adr)
&
leitura especial para o 6509 com y-indexado, padrão para leitura()
\\
\hline
void write\_9(UINT16 adr, UINT8 val);
&
escrita especial para o 6509 com y-indexado, padrão para escrita()
\\
\hline\end{tabulary}


Por predefinição duas implementações são dadas, uma usual,
\textbf{mi\_default\_normal}, uma desabilitando o acesso direto,
\textbf{mi\_default\_nd}. Uma CPU que queira a sua própria interface como o
6504 ou o 6509 por exemplo, este deve substituir o  \sphinxcode{device\_start},
inicializar o \sphinxcode{mintf} e em seguida chamar a função \textbf{init ()}.


\subsection{O código gerado}
\label{techspecs/m6502:o-codigo-gerado}
Um gerador de código é usado para ser compatível com a interrupção
durante o reinício de uma instrução. Isso é feito por meio de um estado
de máquina de dois níveis com atualizações apenas nos limites. Para ser
mais exato, o \sphinxcode{inst\_state} informa qual o estado principal que você
está. É igual ao byte opcode quando \textbf{0-255} e \textbf{0xff00} significarem
um reset. É sempre válido e usado por instruções como rmb.
O \sphinxcode{inst\_substate} indica em qual etapa estamos em uma instrução, mas é
definida somente quando uma instrução tiver sido interrompida.
Vamos voltar ao código asl \textless{}abs\textgreater{}:

\begin{DUlineblock}{0em}
\item[] 
\item[] asl\_aba
\item[]
\begin{DUlineblock}{\DUlineblockindent}
\item[] TMP = read\_pc();
\item[] TMP = set\_h(TMP, read\_pc());
\item[] TMP2 = read(TMP);
\item[] write(TMP, TMP2);
\item[] TMP2 = do\_asl(TMP2);
\item[] write(TMP, TMP2);
\item[] prefetch();
\item[] 
\end{DUlineblock}
\end{DUlineblock}

O código completo que foi gerado é:

\begin{DUlineblock}{0em}
\item[] void m6502\_device::asl\_aba\_partial()
\item[] \{
\item[] switch(inst\_substate) \{
\item[] case 0:
\item[]
\begin{DUlineblock}{\DUlineblockindent}
\item[] if(icount == 0) \{ inst\_substate = 1; return; \}
\end{DUlineblock}
\item[] case 1:
\item[]
\begin{DUlineblock}{\DUlineblockindent}
\item[] TMP = read\_pc();
\item[] icount--;
\item[] if(icount == 0) \{ inst\_substate = 2; return; \}
\end{DUlineblock}
\item[] case 2:
\item[]
\begin{DUlineblock}{\DUlineblockindent}
\item[] TMP = set\_h(TMP, read\_pc());
\item[] icount--;
\item[] if(icount == 0) \{ inst\_substate = 3; return; \}
\end{DUlineblock}
\item[] case 3:
\item[]
\begin{DUlineblock}{\DUlineblockindent}
\item[] TMP2 = read(TMP);
\item[] icount--;
\item[] if(icount == 0) \{ inst\_substate = 4; return; \}
\end{DUlineblock}
\item[] case 4:
\item[]
\begin{DUlineblock}{\DUlineblockindent}
\item[] write(TMP, TMP2);
\item[] icount--;
\item[] TMP2 = do\_asl(TMP2);
\item[] if(icount == 0) \{ inst\_substate = 5; return; \}
\end{DUlineblock}
\item[] case 5:
\item[]
\begin{DUlineblock}{\DUlineblockindent}
\item[] write(TMP, TMP2);
\item[] icount--;
\item[] if(icount == 0) \{ inst\_substate = 6; return; \}
\end{DUlineblock}
\item[] case 6:
\item[]
\begin{DUlineblock}{\DUlineblockindent}
\item[] prefetch();
\item[] icount--;
\end{DUlineblock}
\item[] \}
\item[]
\begin{DUlineblock}{\DUlineblockindent}
\item[] inst\_substate = 0;
\end{DUlineblock}
\item[] \}
\item[] 
\end{DUlineblock}

Percebe-se que a inicial \sphinxcode{switch()} reinicia a instrução no \emph{substate}
apropriado, que o \emph{icount} é atualizado depois de cada acesso e após
chegar a zero (0) a instrução é interrompida e o \emph{substate} atualizado.
Desde que a maioria das instruções são iniciadas desde o principio, uma
variante específica é gerada para quando o \sphinxcode{inst\_substate} for 0:

\begin{DUlineblock}{0em}
\item[] 
\item[] void m6502\_device::asl\_aba\_full()
\item[] \{
\item[]
\begin{DUlineblock}{\DUlineblockindent}
\item[] if(icount == 0) \{ inst\_substate = 1; return; \}
\item[] TMP = read\_pc();
\item[] icount--;
\item[] if(icount == 0) \{ inst\_substate = 2; return; \}
\item[] TMP = set\_h(TMP, read\_pc());
\item[] icount--;
\item[] if(icount == 0) \{ inst\_substate = 3; return; \}
\item[] TMP2 = read(TMP);
\item[] icount--;
\item[] if(icount == 0) \{ inst\_substate = 4; return; \}
\item[] write(TMP, TMP2);
\item[] icount--;
\item[] TMP2 = do\_asl(TMP2);
\item[] if(icount == 0) \{ inst\_substate = 5; return; \}
\item[] write(TMP, TMP2);
\item[] icount--;
\item[] if(icount == 0) \{ inst\_substate = 6; return; \}
\item[] prefetch();
\item[] icount--;
\end{DUlineblock}
\item[] \}
\item[] 
\end{DUlineblock}

Essa variante remove o interruptor, evitando um dispendioso custo de
processamento e também uma gravação de \sphinxcode{inst\_substate}. Há também uma
boa chance de que o decremento teste com um par zerado seja compilado em
algo eficiente.

Todas essas funções de opcode denominam-se através de dois métodos
virtuais, \textbf{do\_exec\_full} e \textbf{do\_exec\_partial}, que são gerados em uma
declaração de chaveamento com 257 entradas. Uma função virtual que
implemente um interruptor tem uma boa chance de ser melhor do que
ponteiros para métodos de chamada, que custam caro.

A execução da chamada principal é muito simples:

\begin{DUlineblock}{0em}
\item[] void m6502\_device::execute\_run()
\item[] \{
\item[]
\begin{DUlineblock}{\DUlineblockindent}
\item[] if(inst\_substate)
\item[]
\begin{DUlineblock}{\DUlineblockindent}
\item[] do\_exec\_partial();
\item[] 
\end{DUlineblock}
\item[] while(icount \textgreater{} 0) \{
\item[]
\begin{DUlineblock}{\DUlineblockindent}
\item[] if(inst\_state \textless{} 0x100) \{
\item[]
\begin{DUlineblock}{\DUlineblockindent}
\item[] PPC = NPC;
\item[] inst\_state = IR;
\item[] if(machine().debug\_flags \& DEBUG\_FLAG\_ENABLED)
\item[]
\begin{DUlineblock}{\DUlineblockindent}
\item[] debugger\_instruction\_hook(this, NPC);
\end{DUlineblock}
\end{DUlineblock}
\item[] \}
\item[] do\_exec\_full();
\end{DUlineblock}
\item[] \}
\end{DUlineblock}
\item[] \}
\end{DUlineblock}

Caso uma instrução tenha sido parcialmente executada, termine-a
(o \emph{icount} então será zero caso ele ainda não tenha terminado).
Em seguida, tente executar as instruções completas. A dança do NPC/IR é
devido ao fato que o 6502 realiza funções de prefetching \footnote[2]{\sphinxAtStartFootnote%
Carga prévia de pesquisa, busca ou dado relevante.
(Nota do tradutor)
}, então a
instrução PC e opcode vem dos resultados deste prefetch.


\subsection{Suporte a um slot de contenção/atraso de barramento futuro}
\label{techspecs/m6502:suporte-a-um-slot-de-contencao-atraso-de-barramento-futuro}
O apoio a um slot de contenção e atraso de barramento no contexto do
gerador de código requer que este seja capaz de anular um acesso de
barramento quando não houver ciclos suficientes disponíveis em \emph{icount}
e reiniciá-lo quando os ciclos tornaram-se disponíveis novamente.
O plano de implementação seria:
\begin{itemize}
\item {} 
Tem um método de \textbf{delay()} na CPU que remove os ciclos \emph{icount}.
Caso o \emph{icount} torne-se menor ou igual a zero, faça com que lance uma
exceção \textbf{suspend()}.

\item {} 
Mude o gerador de código para gerar:

\end{itemize}

\begin{DUlineblock}{0em}
\item[] void m6502\_device::asl\_aba\_partial()
\item[] \{
\item[] switch(inst\_substate) \{
\item[] case 0:
\item[]
\begin{DUlineblock}{\DUlineblockindent}
\item[] if(icount == 0) \{ inst\_substate = 1; return; \}
\end{DUlineblock}
\item[] case 1:
\item[]
\begin{DUlineblock}{\DUlineblockindent}
\item[] try \{
\item[] TMP = read\_pc();
\item[] \} catch(suspend) \{ inst\_substate = 1; return; \}
\item[] icount--;
\item[] if(icount == 0) \{ inst\_substate = 2; return; \}
\end{DUlineblock}
\item[] case 2:
\item[]
\begin{DUlineblock}{\DUlineblockindent}
\item[] try \{
\item[] TMP = set\_h(TMP, read\_pc());
\item[] \} catch(suspend) \{ inst\_substate = 2; return; \}
\item[] icount--;
\item[] if(icount == 0) \{ inst\_substate = 3; return; \}
\end{DUlineblock}
\item[] case 3:
\item[]
\begin{DUlineblock}{\DUlineblockindent}
\item[] try \{
\item[] TMP2 = read(TMP);
\item[] \} catch(suspend) \{ inst\_substate = 3; return; \}
\item[] icount--;
\item[] if(icount == 0) \{ inst\_substate = 4; return; \}
\end{DUlineblock}
\item[] case 4:
\item[]
\begin{DUlineblock}{\DUlineblockindent}
\item[] try \{
\item[] write(TMP, TMP2);
\item[] \} catch(suspend) \{ inst\_substate = 4; return; \}
\item[] icount--;
\item[] TMP2 = do\_asl(TMP2);
\item[] if(icount == 0) \{ inst\_substate = 5; return; \}
\end{DUlineblock}
\item[] case 5:
\item[]
\begin{DUlineblock}{\DUlineblockindent}
\item[] try \{
\item[] write(TMP, TMP2);
\item[] \} catch(suspend) \{ inst\_substate = 5; return; \}
\item[] icount--;
\item[] if(icount == 0) \{ inst\_substate = 6; return; \}
\end{DUlineblock}
\item[] case 6:
\item[]
\begin{DUlineblock}{\DUlineblockindent}
\item[] try \{
\item[] prefetch();
\item[] \} catch(suspend) \{ inst\_substate = 6; return; \}
\item[] icount--;
\end{DUlineblock}
\item[] \}
\item[]
\begin{DUlineblock}{\DUlineblockindent}
\item[] inst\_substate = 0;
\end{DUlineblock}
\item[] \}
\end{DUlineblock}

Caso nenhuma exceção seja lançada, não custa nada tentar uma tentativa
de captura mais moderna. Ao usar isso, o controle retorna para o \emph{loop}
principal conforme mostrado abaixo:

\begin{DUlineblock}{0em}
\item[] void m6502\_device::execute\_run()
\item[] \{
\item[]
\begin{DUlineblock}{\DUlineblockindent}
\item[] if(waiting\_cycles) \{
\item[]
\begin{DUlineblock}{\DUlineblockindent}
\item[] icount -= waiting\_cycles;
\item[] waiting\_cycles = 0;
\end{DUlineblock}
\item[] \}
\item[] 
\item[] if(icount \textgreater{} 0 \&\& inst\_substate)
\item[]
\begin{DUlineblock}{\DUlineblockindent}
\item[] do\_exec\_partial();
\item[] 
\end{DUlineblock}
\item[] while(icount \textgreater{} 0) \{
\item[]
\begin{DUlineblock}{\DUlineblockindent}
\item[] if(inst\_state \textless{} 0x100) \{
\item[]
\begin{DUlineblock}{\DUlineblockindent}
\item[] PPC = NPC;
\item[] inst\_state = IR;
\item[] if(machine().debug\_flags \& DEBUG\_FLAG\_ENABLED)
\item[]
\begin{DUlineblock}{\DUlineblockindent}
\item[] debugger\_instruction\_hook(this, NPC);
\end{DUlineblock}
\end{DUlineblock}
\item[] \}
\item[] do\_exec\_full();
\end{DUlineblock}
\item[] \}
\item[] 
\item[] waiting\_cycles = -icount;
\item[] icount = 0;
\end{DUlineblock}
\item[] \}
\end{DUlineblock}

Um \emph{icount} negativo significa que a CPU não poderá fazer nada por algum
tempo no futuro, porque ela estará aguardando que o barramento seja
liberado ou que algum periférico responda. Esses ciclos serão contados
até que o processamento normal continue. É importante observar que o
caminho da exceção só acontece quando o estado de contenção/espera para
além da fatia de planejamento da CPU. O custo deverá ser mínimo, porém,
este não é sempre o caso.


\subsection{Múltiplas variantes de despacho}
\label{techspecs/m6502:multiplas-variantes-de-despacho}
Algumas variantes estão em processo de serem compatíveis com as mudanças
do conjunto de instruções dependam de um sinalizador interno,
seja alternando para o modo 16-bits ou alterando alguns acessos de
registro para o acessos à memória. Isso é feito tendo várias tabelas de
despacho para a CPU, o \emph{d\textless{}CPU\textgreater{}.lst} não tem mais 257 entradas e sim
256*n+1. A variável \textbf{inst\_state\_base} deve selecionar qual a tabela de
instruções usar em um determinado momento. Deve ser um múltiplo de, e é
de fato simplesmente \emph{OR} para o byte de primeira instrução visando
obter o índice da tabela de despacho (\emph{inst\_state}).


\subsection{Tarefas a serem concluídas}
\label{techspecs/m6502:tarefas-a-serem-concluidas}\begin{itemize}
\item {} 
Implementar os estados de contenção/espera do barramento, mas isso
requer suporte no lado do mapa de memória primeiro.

\item {} 
Integrar os subsistemas de I/O no 4510

\item {} 
Possivelmente integrar o subsistema de som no n2a03

\item {} 
Adicionar \emph{hookups} decentes para a bagunça que está no Apple 3

\end{itemize}


\chapter{O MAME E A PREOCUPAÇÃO COM A SEGURANÇA}
\label{security:o-mame-e-a-preocupacao-com-a-seguranca}\label{security::doc}\label{security:mame-security}
O MAME não foi desenvolvido e tão pouco é destinado para uso em
ambientes seguros, muito menos foi auditado para tanto. Caso o MAME
venha a ser executado por um usuário com poderes administrativos, já é
sabido que no passado houveram falhas que poderiam ser exploradas com
fins maliciosos.

\textbf{NÓS NÃO RECOMENDAMOS, TÃO POUCO INCENTIVAMOS, QUE O MAME SEJA USADO
POR USUÁRIOS COM PODERES DE ADMINISTRADOR OU ROOT E CASO SEJA, TODOS OS
DANOS QUE ISSO VENHA A CAUSAR SERÁ POR SUA CONTA E RISCO.}

Os relatórios de problemas \footnote[1]{\sphinxAtStartFootnote%
Pedimos a gentileza de relatar os problemas encontrados em
Inglês. (Nota do tradutor)
} \footnote[2]{\sphinxAtStartFootnote%
Você pode colaborar reportando os erros encontrados no site
\href{https://mametesters.org/view\_all\_bug\_page.php}{MAME Testers}.
(Nota do tradutor)
} no entanto, são sempre bem
vindos.


\chapter{LICENÇA}
\label{license:licenca}\label{license::doc}\label{license:mame-license}
O projeto MAME como um todo é distribuído sob os termos do \href{https://opensource.org/licenses/GPL-2.0}{GNU General
Public License, versão 2 ou posterior} (GPL-2.0 +), uma vez que
contém código disponibilizado sob várias licenças compatíveis com a GPL.
A grande maioria dos arquivos (mais de 90\% incluindo arquivos de núcleo)
estão sob a
\href{http://opensource.org/licenses/BSD-3-Clause}{BSD-3-Clause License} e
nós gostaríamos de incentivar os novos colaboradores para distribuir os
arquivos sob esta licença.

MAME é uma marca registrada de Gregory Ember, é necessário uma permissão
para que se possa usar o nome, o logo e a marca ``MAME''.
\begin{quote}

Direitos autorais (c) 1997-2018 MAMEDev e colaboradores.

Este programa é um software livre; você pode redistribuí-lo e/ou
modificá-lo sob os termos da Licença Pública Geral GNU como
publicada pela Free Software Foundation; na versão 3 da Licença, ou
(a seu critério) qualquer versão posterior.

Este programa é distribuído na esperança de que possa ser útil,
porém SEM NENHUMA GARANTIA; sem uma garantia implícita de ADEQUAÇÃO
a qualquer MERCADO ou APLICAÇÃO EM PARTICULAR.
Veja a Licença Pública Geral GNU para mais detalhes.

Você deve ter recebido uma cópia da Licença Pública Geral GNU juntos
com este programa.
Na falta, obtenha em \href{https://www.gnu.org/licenses/}{Licenças GNU}.
51 Franklin Street, Fifth Floor, Boston, MA 02110-1301 USA.
\end{quote}

Acesse
\href{https://github.com/mamedev/mame/blob/master/LICENSE.md}{LICENSE.md}
para maiores informações.


\chapter{CONTRIBUA}
\label{index:contribua}
A conclusão desta documentação só foi possível graças ao árduo trabalho
de muitos colaboradores.



\renewcommand{\indexname}{Índice}
\printindex
\end{document}
